\PlaceImage{prokhoris01.JPG}{Sabine Prokhoris en Simon Hecquet tijdens V/J10}

\AuthorStyle{Sabine Prokhoris, Simon Hecquet} 

\licenseStyle{Copyright Presses Universitaires de France, 2007}
\Remark{Text translated from French original: \quote{D'une Atopie} in Prokhoris, Sabine, and Simon Hecquet, Fabriques de la danse. (Presses Universitaires de France - PUF,  2007)

}

\Remark{Translation from French: \Translator{Steven Tallon}{French to Dutch}}

\Ned{\Title{Over een atopie} 
\blank
{\tfx\setupinterlinespace Noot vertaler:\crlf
Letterlijke citaten (in het Frans) van Sabine Prokhoris en Simon Hecquet worden als volgt weergegeven: ({\em schuingedrukte woorden tussen haakjes}); ze staan hier en daar in deze vertaling, ter verduidelijking, dan kan men op voorhand elke poging tot transcriptie van een gedanste beweging (en trouwens van eender welke beweging) diskwalificeren als een dwaling, die slechts aanleiding geeft tot \quote{een nutteloos en vreemd supplement} \footnote{F. Pouillaude, art. cit., p. 99.} (aan de ziel van de dans, zou men moeten toevoegen). Op zijn best, en voor wat gullere blikken, zou dit wanhopig misprijzen, dat door Laurence Louppe in de catalogus van de tentoonstelling {\em Danses trac\'ees} \footnote{Paris, Editions Dis Voir, 1991.} (waarvan zij de commissaris was) zo mooi geschetst wordt als een \quote{bescheiden drang voorbij het mogelijke} \footnote{L. Louppe, \quote{les imperfections du papier}, in{\em Danses Trac\'ees}, p. 33.} , getuigen {--} een beetje \`a la Icarus, maar wat nederiger {--} van een elan dat op het papier verzwakt en wat steelse sporen achterlaat. Die sporen zijn \quote{bemiddelaars tussen het niets en het leven} \footnote{L. Louppe, idem.} , tekens van een bloedloze beweging, zolang er geen {\em lichaam} is om haar toegang te verlenen tot het zijn \footnote{L. Louppe zegt met betrekking tot het lot van deze, voor haar spookachtige, schriften, exact het volgende: \quotation{{\dots} jullie bestaan slechts half, bij afwezigheid van het lichaam dat jullie als enige zal kunnen lezen.} Maar wat is deze \quote{lezing} anders dan een incarnatie? Is het niet via een metafoor dat er sprake zal zijn van \quote{lezen}? In ieder geval wordt de specifieke beweging waaruit de leeshandeling bestaat noch beschreven, noch geanalyseerd. Dat kan ook niet, aangezien de auteur een exacte kennis van de werking van deze systemen ontbeert. Het gaat er in elk geval om te erkennen dat, hoe pertinent sommige van haar intu\"ities ook zijn, de antwoorden die L. Louppe op deze vragen formuleert tot heel wat verwarring leiden. Ze berusten immers op talrijke onnauwkeurigheden, en zelfs fouten, wat betreft de principes die aan de basis liggen van de transcriptiesystemen die zij vermeldt. L. Louppe behandelde het onderwerp bijvoorbeeld in {\em Danses trac\'ees} en in haar boek{\em Po\'etique de la danse contemporaine} (Bruxelles, Contredanse, 1997), een boek dat in danskringen gezaghebbend is; en in talrijke artikelen, zoals bijvoorbeeld \quote{Les notations en danse, gardiennes de l'invention}, verschenen in {\em R\'esonnance}, het tijdschrift van de IRCAM (oct. 1994, n{\textdegree} 7).} {--} dat wil zeggen: geen lichaam om bij de onmiddellijkheid aan te sluiten, bij de {\em aura}, verloren in het papier. In het papier, of in de woorden, aangezien volgens deze visie van automediatie, deze laatste (net als de grafische tekens, meer nog misschien) ons slechts kunnen vervreemden van het ding zelf. Men kan het dansende lichaam blijkbaar niet {\em zeggen}, en nog minder wanneer men zelf geen danser is, wanneer men de intieme ervaring van die automediatie\footnote{Mevrouw Monnier is, tijdens haar gesprekken met J.L. Nancy (in {\em Allit\'erations} op. cit.{\em } p. 14), aangenaam verrast: \quotation{Ik vond het verbazend dat een filosoof zich zo nauwkeurig in de beschrijving van een dansend lichaam, van een danser, kon verplaatsen.} Een verbazende verbazing, eerlijk gezegd. Wat te denken van een schrijver of een schilder die erover verbaasd zou zijn dat men over een boek of een schilderij kan spreken, zonder zelf schrijver of schilder te zijn?} ontbeert.\par}
\godown[3em]

\par
Het is een weinig bekend, en erg zelden erkend, feit uit de geschiedenis van de westerse kunst: er bestaan, zover men weet reeds sinds het eind van de vijftiende eeuw, grafische systemen voor de transcriptie van beweging, die het mogelijk maken om partituren voor dans \footnote{Zie Ann Hutchinson Guest, {\em Dance notation}, London, Dance \ Books, 1984. Op dit moment worden er twee systemen gebruikt, namelijk het Laban{}- (1928) en het Beneshsysteem (1955).} uit te werken. Toch is, wat dans betreft, het gebruik van een schriftsysteem een marginale (vandaag erg minoritaire) praktijk gebleven. Dit in tegenstelling tot wat er is gebeurd in de wereld van de muziek, waar het gebruik van een muzikale standaardnotatie {--} bij het componeren van \quote{klassieke} muziek, maar ook daarbuiten {--} ondertussen een normale zaak is.

Deze schriftsystemen verschijnen, gegeven het feit ook dat ze zo'n vreemde verspreiding kennen, als evenzoveel eenzame pogingen, al quasi opgelost zodra ze opduiken, en dus {--} bij gebrek aan gebruikers {--} gedoemd te mislukken. Twee of drie onder hen hebben het nochtans overleefd, maar in het veld van de choreografische kunst is hun gebruik helemaal niet wijdverbreid.

Geconfronteerd met deze systemen, met deze niet{}-ge\"identificeerde grafische objecten, moet de fascinatie voor hun plastische schoonheid in het algemeen wedijveren met het vooroordeel volgens hetwelk ze \quote{de wezenlijke kern laten ontsnappen van dat wat moet neergeschreven worden} \footnote{Fr\'ed\'eric Pouillaude, \quote{D'une graphie qui ne dit rien}, {\em Po\'etique}, f\'ev. 2004, n{\textdegree}137, p. 99.} {--} met name het onuitsprekelijke \quote{dit is mijn lichaam} van de danser. Want, in tegenstelling tot alle andere kunsten, zou dans \quote{dit gedicht, losgemaakt van elk kopiistenapparaat} \footnote{St\'ephane Mallarm\'e in \quote{Crayonn\'e au th\'e\^atre} (1886), in {\em Oeuvres Compl\`etes, ll}, Paris, Gallimard, \quote{Biblioth\`eque de la Pl\'eiade}, 2003, p. 171.} zijn, vanuit het idee dat zijn medium \quote{het lichaam zelf van de artiest} \footnote{Volgens de formulering van Jean{}-Luc Nancy in Mathilde Monnier, Jean{}-Luc Nancy,{\em Allit\'erations}, Paris, Galil\'ee, 2005, p. 29.} is, en dat zijn essentie dus te vinden is in de onmiddellijkheid {(\quote{on{}-middellijkheid})}\footnote{M. Monnier, J.L. Nancy, ibid, p. 30.} , in het hier en nu, van een gebeurtenis zonder gelijke. Zo lijkt het dan alsof dans vandaag verschijnt als de enige kunst waar de {\em aura} volhardt, de enige waar, in de woorden van Walter Benjamin, \quote{de cultuswaarde} niet door \quote{de tentoonstellingswaarde} \footnote{Walter Benjamin, \quote{Het kunstwerk in het tijdperk van zijn technische reproduceerbaarheid}, in {\em Het kunstwerk in het tijdperk van zijn technische reproduceerbaarheid; Kleine geschiedenis van de fotografie; Eduard Fuchs, verzamelaar en historicus}, Boom (SUN), 1985 (Duits: 1963).} zou vervangen zijn. De enige ook die door de esthetische mutatie van de moderne wereld {--} die van elke toeschouwer \quote{een verstrooide expert} \footnote{Die uitdrukking duidt voor W. Benjamin de moderne toeschouwer aan, zoals hij door de cinematografische kunst wordt geconstitueerd. Een toeschouwer die op kunstwerken een kritische blik kan werpen (Benjamin spreekt van \quote{examinator}), maar met een blik die niet die van een specialist is {--} en die functioneert zonder dat er een bijzondere inspanning wordt vereist. Benjamin beschrijft die manier van receptie als \quote{receptie in de verstrooiing}.} maakt\footnote{Noot S.T.: Benjamin heeft het bij de bespreking van de hedendaagse toeschouwer effectief over een \quote{verstrooide examinator}, en ook de \quote{receptie in de verstrooiing} (en niet {\em \quote{par}}, zoals er in de Franse vertaling staat) is aan de orde. Maar op een aantal plaatsen in de tekst wordt de toeschouwer tegelijk ook een deskundige, respectievelijk een expert genoemd. Een paar keer gaat het om een \quote{deskundige} (p. 23, over de filmtoeschouwer, of p. 28, over de dagbladlezer), een andere keer om een \quote{halve expert} (p. 27), of een \quote{beoordelende expert} (p. 31). Deze laatste term (\quote{beoordelend}) zwakt ogenschijnlijk Benjamins gebruik van \quote{verstrooid} en \quote{verstrooiing} af. Een oplossing van de paradox is te vinden in de verschillende houding van het algemene (jaren dertig) publiek (\quote{de massa}'; \quote{de massa's}) dat Benjamin t.o.v. verschillende kunstvormen ook verschillende houdingen ziet aannemen: \quote{reactionair} en \quote{achterlijk} t.o.v. schilderkunst (Picasso, surrealisme), \quote{progressief} (en effectief functionerend als beoordelend expert) t.o.v de filmkunst, bijvoorbeeld bij het zien van een Chaplin. \quotation{In de bioscoop vallen kritische en genietende houding van het publiek samen} (p.31) {--} waarbij trouwens (ook historisch gezien) een publiek dat verstrooiing zoekt in het geheel niet minderwaardig is aan een publiek dat contempleert. Zo is de hedendaagse toeschouwer iemand waarbij ook andere dan puur visuele zintuigen aangesproken worden: het {\em gebruik} van een kunstwerk leidt sowieso tot de {\em tactiele} (en niet de zuiver optische) receptie ervan (p.39). In {\em Het kunstwerk} gaat het m.a.w. niet alleen om wat er met het kunstwerk is gebeurd \quote{in het tijdperk van zijn technische reproduceerbaarheid}, maar tegelijk om hoe de toeschouwer (\quote{de massa}) zich tegenover het kunstwerk gedraagt. Daarbij is trouwens ook het onderscheid auteur/publiek aan het vervagen: \quotation{De lezer staat ten allen tijde klaar een schrijver te worden.} (p. 28) De moderne tijd heeft volgens Benjamin dan ook nood aan een nieuwe esthetiek (we kunnen toevoegen: van kunstreceptie \'en {}-productie) die de wereld van religie en sacraliteit voorgoed verlaat {--} en zich openstelt voor het collectief. Daarbij zal haar (revolutionaire) opdracht eruit bestaan {--} niet de {\em politiek te esthetiseren}, zoals volgens hem in het fascisme gebeurt, maar om de {\em kunst} (en dus de esthetiek) te {\em politiseren} (p. 42). Het basisidee van de auteurs echter, dat namelijk volgens Benjamin in de moderne tijd de tentoonstellingswaarde het van de cultuswaarde (en het sacrale) heeft gehaald {--} waarbij de uniciteit (de \quote{aura}) van het kunstwerk tegelijk verloren is gegaan {--} blijft met dit alles natuurlijk wel behouden. (Alle citaten uit {\em Het kunstwerk}..., op. cit.)}, in plaats van een communicant van het miraculeuze evenement, van de sacrale dimensie van het unieke oeuvre {--} niet ontaard zou zijn. Zo zou de fameuze \quote{moderniteit in dans} \footnote{Deze uitdrukking functioneert, in bijna alle actuele teksten over moderne en eigentijdse dans (in elk geval in Frankrijk), als een soort verplicht wachtwoord. Maar waar het dan precies om gaat, blijft steeds uiterst vaag.} , en dit in een nog zuiverdere vorm bij performance, paradoxaal genoeg het laatste en onherleidbare toevluchtsoord van de {\em aura} zijn, nog meer dan alle oude {--} naar de smaak van onze moderne halfgoden te gecodificeerde {--} dansen.

Is de danser zodoende, volgens de formule van Jean{}-Luc Nancy, \quotation{een in het bijzonder (op zich)zelfverwezen artiest} \footnote{M. Monnier, J.L. Nancy, op. cit., p. 30.} ({\em particuli\`erement autor\'ef\'erenci\'e}) 

Laat het ons meteen al onderstrepen: dergelijke vertogen komen voort uit een visie op het lichaam, en op het teken, die op zijn minst ge\"expliciteerd mag worden, en waarvan de inzet allesbehalve onschuldig is {--} behalve als men haar aanvaardt als de uitdrukking van een fundamentele en onbetwistbare waarheid. Ze getuigen ook, zoals we al kort schetsten, van een nostalgie naar het sacrale, dat sacrale dat door de dans {--} uitzondering onder de kunsten voor zover zijn identiteit gedefinieerd zou kunnen worden als \quote{de dichterlijke emissie van een uniek evenement} \footnote{L. Louppe, op. cit., p. 9.} {--} onophoudelijk zou hernieuwd worden. En wel omdat, zoals Laurence Louppe ons in de hierboven aangehaalde tekst in herinnering brengt, zijn roeping dat zou zijn wat Trisha Brown \quote{de onschuld van de eerste daad} \footnote{L. Louppe, idem.} noemt. Die twee dimensies {--} een specifieke visie op lichaam en teken; het idee van een choreografische uitzondering {--} zijn aan elkaar gelieerd, want het is het lichaam, het {\em dansende lichaam}, dat hier de plaats zelf van het sacrale zal zijn. Het is dan ook \'e\'en van de opgaven van onze tekst om op deze punten enkele bressen te slaan.

Wat de \quote{noteerders}, zoals ze zichzelf noemen, betreft: gezien de \quote{notatie}systemen (van beweging/dans) de miraculeuze macht wordt toegedicht \quote{de essentie van de beweging} \footnote{Symptomatisch voor deze toestand: in een verzameling uittreksels van dansen van moderne choreografen, die door verschillende noteerders werden getranscribeerd, komen deze laatste alleen in het dankwoord voor, en helemaal niet als auteurs van transcripties. Een nog meer ondergeschikte positie dus dan die van een literair vertaler. Zie {\em Readings in modern dance,} vol. 2, New York, Dance Notation Bureau Press, 1977.} op te vangen, leidt een al even priesterlijke verleiding hen er toe {--} misschien ter compensatie van het proces van uitwissing waaruit het concrete {\em werk} van het opstellen van een partituur \footnote{Bijvoorbeeld in een werk dat verscheen onder de titel {\em Labanotation \ {--} The system of analysing and recording movement}, New York, Routledge, 1977. A. Hutchinson Guest schrijft op p. 4: \quotation{Het systeem dat veranderingen m.b.t. de hoek van de ledematen, de trajecten in de ruimte, en de energieflux, objectief kan noteren, evenals de motivatie van de beweging, \'en de subtiliteit en de kwaliteit van haar expressie, verdient onze bijzondere aandacht. De Labanotatie is zo'n systeem.} Laten we het drogbeeld van de \quote{objectiviteit} voor wat het is, maar je blijft toch in het ongewisse over wat dan wel het teken voor \quote{motivatie} zou kunnen zijn{\dots} De hele passage verdient het trouwens om bekritiseerd te worden, in zoverre ze een systeem, dat de auteur nochtans kent, op approximatieve en onjuiste wijze voorstelt.} bestaat {--} zichzelf te beschouwen als de depothouders of bewaarders van de dans op zich. Hoe dan wel? Via het mysterie van een of andere transsubstantiatie {--} zodat hetzelfde fantasma van de Aanwezigheid ook aan deze zijde van het geschil over de genoemde \quote{notaties} aan het werk is. Ook hier zullen we, in wat volgt, nadenken over welk soort beweging lezen, of schrijven, precies is. Want beide zijn misschien toch iets anders dan de zuivere {--} en on{}-gemedieerde {--} impregnatie van een lichaam door bezielde tekens (hoe precies is niet zo duidelijk), of omgekeerd: iets anders dan een metempsychose van de ziel van de dans (dat wil zeggen van het lichaam van de danser) richting partituur.

Het valt in elk geval op dat deze schriftsystemen, die soms gezien worden als po\"etische curiositeiten, dan weer als simpele instrumenten ten dienste van de Kunst (machteloos dan wel magisch, al naargelang) tot hiertoe nooit het voorwerp waren van een wat grondigere reflectie, die zich de moeite getroost ze te analyseren zonder zich te onderwerpen aan een onbetwiste visie op wat dans {--} echt, en definitief {--} dan wel zou zijn. Er bestaan weliswaar massa's artikelen, zonder dan nog de handboeken te tellen. Maar zelfs de fijnste, meest suggestieve en meest nauwgezette onder hen schijnen de daadwerkelijke analyse van hun objecten al zeer snel op te geven, om zich neer te leggen bij dat wat inzake dans \quote{evident is} \footnote{F. Pouillaude, art. cit., p. 99.} , zoals een auteur zo professoraal schrijft {--} voor wie het evident is dat die evidenties er {\em niet} zijn om in vraag gesteld te worden. Zeldzaam zijn ze, de studies die de taak op zich nemen de \quote{proeve van het vreemde} (om de titel van een boek van Antoine Berman, gewijd aan vertalen, aan te halen \footnote{Als voorbeeld van een werk dat de uitdaging op zich neemt, kunnen we de subtiele en pertinente tekst van Maria Daniella Strouthou vermelden: \quote{Notes sur une exp\'erience personnelle} in {\em Funambules}, revue du D\'epartement Danse Universit\'e de Paris Vlll, mars 1997, n{\textdegree}5, p. 27{}-30. Het boek van A. Berman getiteld {\em L'\'epreuve de l'\'etranger} is in 1984 verschenen bij Gallimard.}) te doorstaan {--} doorheen de objecten zoals ze zich aandienen, en beslagen in de analyse van de processen die door die objecten ingezet worden. Alsof het volstond te melden dat ze marginaal zijn, in een poging een keer te meer te affirmeren dat de hoofdzaak, ge\"identificeerd, zoals het hoort, met het heilige woord \quote{aanwezigheid}, definitief elders te zoeken is. En zoals iedereen weet maken marges het mogelijk om het onaantastbare unieke te waarborgen van precies dat waarvan zij de rand vormen. Ze zijn verder van geen tel. Toch niet echt.

Maar wat gebeurt er precies wanneer men besluit om deze marges, deze \quote{supplementen}, te beschouwen als volwaardige creaties {--} noch slaaf, noch bijvoegsel {--}, hoe beperkt hun gebruik in dans ook moge zijn? Wat als we ze, in plaats van ze te beschouwen als vreemde bijkomstigheden die men kost wat kost wil beoordelen in relatie tot dat waarvan men gelooft dat het de substantie zelf van de choreografische kunst uitmaakt {--} met name het {\em dansende lichaam}, nogmaals: dat \quote{zelfverwezen} wezen, verheerlijkt in een soort van autocommunie; \quotation{dat wat achter de dans verschijnt, dat is \quote{ik}}, schrijft Mathilde Monnier \footnote{M. Monnier, J.L. Nancy, op. cit., p. 32. Vermelden we terloops ook dat die automediatie, wanneer ze zich presenteert als een soort trance (volgens J.L Nancy, op p. 58 van hetzelfde werk, \quotation{de trance, waarvan het rijm op dans zich, op een vreemde manier, als een cadans oplegt {--} die elke dans zou ritmeren}), nochtans niet wordt gezien als overgave aan iets vreemds, voorbij het voorondersteld oorspronkelijke \quote{ik}. Het gaat hier niet om een denken van de alteratie.} {--}, wat dus als we ze eerder onderzoeken in relatie tot wat ze teweegbrengen? Dat zijn dan met name processen die betrekking hebben op lezen, op het schrift; waarbij handelingen als interpreteren/zeggen/kijken aan de orde zijn; die zich met andere woorden afspelen in het complexe, instabiele, open netwerk van bedreven bewegingen. Want transcriptie zal dan niet langer een vergeefs streven zijn om een communie met het \quote{zelfverwezen} {\em dansende lichaam} aan te gaan, maar doodeenvoudig een poging om het maaswerk van interpretatieve bewegingen te lezen/zien/interpreteren {--} een patroon geweven door om het even welke beweging die zich in de ruimte afspeelt, en {\em a fortiori} door een gedanste beweging. Een maaswerk, zo zullen we argumenteren, waaruit precies dat wat men een lichaam noemt bestaat: als dat lichaam tenminste gedacht wordt {--} niet als een oorspronkelijk en onaantastbaar gegeven, maar (meer freudiaans misschien), als de plaats van verrichtingen die de beweging doen totstandkomen, in een onophoudelijke modificatie van de gemeenschappelijke ruimte.

Natuurlijk zou niemand er ook maar aan denken te betwisten dat het {\em medium} van dans het lichaam, en zelfs het \quote{eigen lichaam }van de danser, is. Maar dat \quote{eigen lichaam}, is dat, als dansend lichaam, een gewijde hostie {--} een autohostie, zou je kunnen zeggen, aangezien alleen \quote{ik} het lichaam bewoon? Of eerder een betekenende matrix {--} hetgeen aan dat dansende \quote{ik} een heel andere dimensie zou geven. Het is duidelijk dat, indien we ervoor kiezen om vanuit dit tweede perspectief te vertrekken, de vraag naar transcripties, en naar de tekensystemen die eraan ten grondslag liggen, zich anders zal stellen {--} in vruchtbaarder termen, en zeker niet doordrongen van spijt voor wat hun zogenaamd onvermijdelijke mislukking betreft.

Een passage uit {\em Naamloos} van Samuel Beckett komt ons hier voor de geest, meer bepaald enkele duizelingwekkende lijnen waarin het gaat om de intieme ervaring, nochtans hopeloos {\em niet} \quote{zelfverwezen}, van aanwezigheid, en van de ruimte. Hier is ze:

\QuoteStyle{{\dots}ik ben in woorden, ik ben van woorden gemaakt, woorden van anderen, welke anderen, de plaats ook, de lucht ook, de muren, de grond, het plafond, woorden, het hele universum is hier, bij mij, ik ben de lucht, de muur, de ommuurde, alles geeft mee, opent zich, wijkt af, vloeit terug, vlokken, ik ben al deze vlokken, zich kruisend, zich verenigend, zich scheidend, waar ik ook ga vind ik me terug, laat ik me achter, ga naar mij, kom van mij, nooit alleen ik, slechts een stuk van mij, hernomen, gemist, verloren, woorden, ik ben al deze woorden, al deze vreemdelingen{\dots}} 

En op een andere plaats, dit:

\QuoteStyle{de woorden zijn daar, ergens, zonder ook maar het minste geluid te maken, (...), zij zullen me zeggen wie ik ben, ik zal het niet begrijpen, maar het zal gezegd zijn, ze zullen gezegd hebben wie ik ben, en ik, ik zal het gehoord hebben, zonder oor zal ik het gehoord hebben, en ik zal het gezegd hebben, zonder mond zal ik het gezegd hebben, ik zal het buiten mij gehoord hebben, dan plots binnenin mij, misschien is het dat wat ik voel, dat er een buiten is en een binnen en ik tussenin, misschien is het dat wat ik ben, het ding dat de wereld in twee deelt, enerzijds het buiten, anderzijds het binnen, dat kan dun zijn als een snede, ik ben niet aan \'e\'en kant noch aan de andere, ik ben in het midden, ik ben de wand, ik heb twee gezichten en geen dikte, misschien is het dat wat ik voel, ik voel me die trilt, ik ben het trommelvlies, (...) 
\footnote{S. Beckett, {\em L'innommable}, Paris, 1992, Les \'Editions de Minuit, resp. p. 166 en 159{}-160 (Vertaling S.T. \ Uitgegeven in het Nederlands (eerste druk) onder de titel {\em Naamloos} in {\em Molly/Mallone sterft/Naamloos}, Amsterdam, 1970, De Bezige Bij.)}
\footnote{De aanwezigheid {--} de aanwezigheid van een lichaam, van mijn lichaam {--} de simpele aanwezigheid, waarvan een kunst zoals dans weliswaar alle specifieke dimensies toont, is {\em alleen} die van de ander. \quote{De woorden van de anderen}, zoals Beckett zegt, waarmee hij voortreffelijk de verontrusting weergeeft die het \quote{extieme}}
\footnote{Jacques Lacan bedenkt dit (geniale) woord om de wijze waarop Freud de genese van het subject {--} met name de genese van de veronderstelde \quote{innerlijkheid} {--} beschrijft, te vertalen. Lacan vertrekt van de mutuele band, zodat het \quote{intieme} helemaal is opgemaakt uit wat Beckett nu juist \quote{de woorden van de anderen}, die \quote{vreemdelingen}, noemt. Dit heeft niets te zien met de figuur van het \quote{ik}, waarover M. Monnier het heeft in haar dialoog met J.L. Nancy.}}

(om een woord van Lacan te hernemen) bestaan uitmaakt, het bestaan van dat wat men \quote{zelf} wil noemen. Het \quote{zelf}? Niets meer dan de vreemde niet{}-consistentie van een tussenruimte, opgebouwd uit tussenruimte, altijd verplaatst door de een of de andere zelfverwijzing ({\em auto{}-r\'ef\'erence}), als de woorden {--} die gebaren {--} slechts gericht bestaan, en slechts op die manier de wederzijdse, vertalende ruimte ({\em l'espace mutuel, traductif}) van aanwezigheden defini\"eren. Vibrerende ruimte van permanente interpretatie, opgebouwd uit verschillen en leegtes {--} geen ruimte van on{}-middellijkheid. \quote{Ik}, dit \quote{ik} van de aanwezigheid van het dansende lichaam, dit intieme eigendom van zelf voor zelf, zal dan niets anders zijn dan het gevolg van een oneindige vertaalspanning ({\em une tension traductive).} Vertalen, steeds opnieuw en onvermoeibaar, van een ont{}-verwijzing ({\em une d\'e{}-r\'ef\'erenciation}) waarvan het raadsel blijft aandringen, en achterhalen hoe het gehalte aan desillusie van onmiddellijkheid kan aangehouden worden. Misschien zou het doodeenvoudig dat kunnen zijn: dansen.

Met betrekking tot deze vragen, die een directe weerslag hebben op de relatie die men vandaag tot de choreografische kunst kan hebben, leggen we, op de drempel van deze bladzijden, een weddenschap af. Die luidt meer bepaald: het perspectief dat door de specifieke inzet van transcriptie van {--} gedanste of andere {--} beweging wordt geboden (transcriptie die aan de basis ligt van de uitvinding van schriftsystemen voor beweging, die vreemde en zeer miskende grafische objecten), is van dien aard dat het een crisis veroorzaakt binnen de dominerende visies op dans, een volwaardige kunst; hetgeen ook wil zeggen: kunst onder de kunsten. Het gaat om pogingen die wij decisief als minoritair {--} eerder dan als marginaal {--} zullen bestempelen, in zoverre ze de kracht kunnen hebben om als kritische werktuigen te fungeren. Werktuigen die daarenboven de zekerheid van dat wat vanzelfsprekend is kunnen doorbreken, om uiteindelijk misschien het solipsisme achter zich te laten waarin {--} zo lijkt ons {--} de hedendaagse dans, en het merendeel van de vertogen die in zijn naam worden gehouden, zich al te vaak heeft opgesloten.

Het zal er ons dus ten eerste om te doen zijn om {--} doordat we onszelf de taak opleggen om zo precies mogelijk aan te geven hoe systemen voor bewegingstranscriptie werken, welke specifieke logica ze volgen {--} de aanzet te geven tot een werk, in de zin die Foucault eraan geeft: 

\QuoteStyle{Werk: dat wat een significant verschil kan introduceren in het veld van het weten, waarbij auteur en lezer een zekere moeite moeten doen, met eventueel de beloning van een bepaald plezier, dat wil zeggen: van een toegang tot een andere figuur van de waarheid. \footnote{Definitie uitgewerkt door Michel Foucault, Jean{}-Claude Milner, Paul Veyne, Fran\c{c}ois Wahl, als {\em incipit} voor de collectie \quote{Des travaux}, opgericht in 1982, bij Seuil. Tekst hernomen {\em in Dits et \'Ecrits, IV,} dir. Daniel Defet \& Fran\c{c}ois Ewald, Paris, Gallimard, 1994, p. 367. (Vertaling citaat Foucault: S.T.)}}

Een werk voor iedereen, voor en met dans. Doorheen dans.}
