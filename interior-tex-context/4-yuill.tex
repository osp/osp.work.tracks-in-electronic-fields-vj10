\AuthorStyle{Simon Yuill}

\licenseStyle{The text is under a GPL. The images are a little trickier as none of them belong to me.\\The images from ap and David Griffiths can be GPL as well, the Scratch Orchestra images (the graphic music scores) were always published \quote{without copyright} so I guess are public domain. The photograph of the Scratch Orchestra performance can be GPL or public domain and should be credited to Stefan Szczelkun. The other images, Sun Ra, Black Arts Group and Lester Bowie would need to mention \quote{contact the photographers}. Sorry the images are complicated but they largely come from a time before copyleft was widespread.\par}

\Remark{Text first published in English in Mute: \Url{http://www.metamute.org/en/All-Problems-of-Notation-Will-be-Solved-by-the-Masses.}\par}

\Remark{Translation from English: \Translator{Anne Smolar}{English to French}\par}

\Fra{\Title{Tous les probl\`emes de notation seront r\'esolus par les masses (version abr\'eg\'ee)}

De toutes les formes d'art soutenues et rendues possibles par le
Logiciel Libre et Open Source, la programmation live ou \quote{livecoding}
est apparue \^etre celle qui incarne le plus directement les principes
cl\'es de la production de logiciels libres et open source de
cr\'eation et d'exp\'erience m\^eme de l'{\oe}uvre. Dans le
{\em livecoding}, l'{\oe}uvre est exprim\'ee par du code logiciel
\'ecrit et r\'e\'ecrit en direct durant son ex\'ecution. Beaucoup
d'artistes du {\em livecoding} r\'edigent leurs propres outils
logiciels pour permettre ce genre de programmation. Le
{\em feedback.pl} d'Alex McLean fut un des premiers outils de ce
type. Il s'agit d'un simple script Perl qui lit et ex\'ecute en continu
un extrait de son propre code affich\'e par un \'editeur de texte. Ce
code d\'efinit divers algorithmes qui g\'en\`erent de la musique.
Durant la performance, il est r\'e\'ecrit par l'artiste, changeant la
structure musicale et improvisant vraiment avec le code. Une projection
du bureau de l'artiste rend visible le processus, mettant ainsi en
valeur la fa\c{c}on dont le code et les modifications qui lui sont
apport\'ees font partie int\'egrante du travail et de son exp\'erience
par les spectateurs. La relation formelle et mat\'erielle entre le code
et la musique sont par cons\'equent discernables, m\^eme si de nombreux
spectateurs ne sont pas familiers des langages de programmations
eux{}-m\^emes. C'est un peu comme quand on assiste \`a une performance
sur un instrument acoustique telle qu'une guitare ou une clarinette.
Bien qu'on puisse ne pas comprendre comment jouer de ces instruments
soi{}-m\^eme, on peut relier les gestes des musiciens aux sons que nous
entendons et ainsi acqu\'erir une notion de la relation entre le son et
sa production mat\'erielle. Ceci contraste nettement avec les formes
ant\'erieures de performance musicale \'electronique, comme celles de
Jean Michel Jarre et Todd Machover, dans lesquelles les appareils
interfaces sont souvent pr\'esent\'es sur sc\`ene simulant ou en
r\'ef\'erence \`a des instruments acoustiques. Le {\em livecoding}
se dispense de ce type de \quote{f\'etiches} et expose sans honte la
mat\'erialit\'e nue de sa production. La pr\'esentation inhabituelle du
code comme un mat\'eriau brut m\`ene toutefois \`a quelque chose de
fort diff\'erent de la performance de la guitare ou de la clarinette,
mais ressemble davantage \`a la r\'ev\'elation de la machinerie
sc\'enographique dans une pi\`ece de Brecht. Un avantage est ainsi
cr\'e\'e en exposant quelque chose qui est normalement dissimul\'e.

Si le {\em livecoding} s'est initialement d\'evelopp\'e comme une
forme de musique, il n'est pas limit\'e \`a cela. Fluxus de David
Griffith et PacketForth{\em } de{\em } Tom Schouten sont des
outils qui permettent de cr\'eer des travaux visuels, le premier
\'etant bas\'e sur un moteur graphique 3D et le second \'etant un
syst\`eme de traitement de la vid\'eo. Certains outils informatiques
existants, tels que SuperCollider,{\em } Chuck et Pure Data ont
aussi \'et\'e utilis\'es pour du{\em  livecoding}. En fait,
n'importe quel langage ou outil de programmation qui peut ex\'ecuter un
code au vol peut potentiellement \^etre utilis\'e pour du
{\em livecoding}. Le concept a aussi \'et\'e \'etendu \`a d'autres
formes de travaux: le Social Versioning System (SVS) permet \`a des
jeux de simulation multi{}-joueurs d'\^etre cr\'e\'es et cod\'es en
direct, le nouveau code \'etant distribu\'e aux joueurs simultan\'ement
au d\'eroulement du jeu. Life Coding de Ap est une performance \`a
grande \'echelle qui combine la programmation logicielle, la m\'ethode
de {\em circuit bending} en cr\'eation musicale et des
pr\'esentations parl\'ees de type conf\'erence.

\SubSubTitle{Esth\'etique du livecoding}

Deux facettes{}-cl\'es du {\em livecoding} incarnent les principes du
Logiciel Libre et Open Source. D'une part, la fa\c{c}on dont il fait de
la continuelle r\'e\'ecriture de code elle{}-m\^eme un mode primaire de
production artistique, d'autre part, sa pr\'esentation de
l'\quote{{\oe}uvre} elle{}-m\^eme comme une partie de code \'evolutif ouvert
plut\^ot que comme un artefact statique distinct. A l'oppos\'e de la
plupart de l'art non{}-digital et de l'art des nouveaux m\'edias qui
est pr\'esent\'e exclusivement comme une marchandise \`a consommer, le
{\em livecoding} rend son propre contenu et sa pratique de
production accessible aux autres. Le {\em livecoding} met en valeur
le principe de production du Logiciel Libre et Open Source bas\'e sur
le code comme une forme de production qui est elle{}-m\^eme \quote{live} et
vivante, permettant la production par d'autres \`a leurs propres fins.

\PlaceImage{ottoroessler.jpg}{Otto Roessler at ap 'life coding' event, Piksel, 2007}
\PlaceImage{pattern-cascade.jpg}{David Griffths, fluxus, screengrab}

Ce \quote{permettre la production par d'autres} se prolonge souvent
au{}-del\`a de la performance, non seulement par l'usage d'une
distribution de type Logiciel Libre, mais aussi dans le recours
conscient aux ateliers comme moyen de pr\'esentation des {\oe}uvres et
d'enseignement des savoirs utilis\'es dans leur cr\'eation. Cet aspect
p\'edagogique se prolonge dans la pro\'eminence donn\'ee aux rencontres
techniques et aux ateliers de d\'eveloppement au sein de festivals
organis\'es par des artistes tels que Piksel et MAKEART, ou dans des
groupes tels que Dorkbot et OpenLab, ainsi que dans la cr\'eation de
plateformes et projets de diss\'emination tels que pure:dyne ou FLOSS
Manuals. De nombreuses performances et projets de {\em livecoding}
sont eux{}-m\^emes souvent par nature des \quote{ateliers} ad hoc et donc un
prolongement de l'\'ethique de partage et de diffusion publique du
{\em livecoding}. Les participants des \'ev\`enements ap qui sont
organis\'es sur de longues p\'eriodes de 12h et plus, apprennent et
adaptent les outils de la performance tandis qu'elle se d\'eroule. Sur
une plus petite \'echelle, le OpenLab de Londres accueille les
performances \quote{drumming cercle} auxquelles quiconque peut se joindre
avec ses propres algorithmes et codes, afin de simultan\'ement
construire et d\'evelopper une {\oe}uvre rythmique collective et des
performances qui commencent sur une partie du code qui est r\'e\'ecrite
par des artistes successifs. Plut\^ot que d'\^etre marginale ou
h\'et\'erog\`ene \`a l'\quote{art}, l'id\'ee d'atelier a \'et\'e absorb\'ee
comme un aspect int\'egral de l'esth\'etique du {\em livecoding}.

Le {\em livecoding} n'est pas la seule, ni m\^eme la forme dominante
de pratique, emprunt\'ee par tous ceux qui sont engag\'es dans les arts
li\'es au Logiciel Libre et Open Source. Tous les praticiens
impliqu\'es dans ces projets ont en commun d'\^etre attach\'es \`a la
notion \'elargie de \quote{code live} en tant que mode de production et
d'avoir une pr\'ef\'erence commune pour une esth\'etique de type
atelier. C'est aussi dans ces pratiques plus \quote{p\'edagogiques} que la
production artistique au sein du Logiciel Libre et Open Source
rencontre les autres aspects du monde du Logiciel Libre, et plus
particuli\`erement les pratiques politiques et sociales engag\'ees qui
\'emergent des hacklabs et hackmeets.

\SubSubTitle{Hacklabs, Hackmeets et Centres Sociaux}

Les hacklabs sont des lieux autog\'er\'es qui offrent un acc\`es libre
\`a des ordinateurs et \`a internet. Ils utilisent en g\'en\'eral des
machines r\'ecup\'er\'ees ou recycl\'ees tournant sur GNU/Linux et,
parall\`element au fait de procurer un acc\`es internet, la plupart des
hacklabs organisent des ateliers sur un \'eventail de sujets, de
l'usage de base d'un ordinateur et l'installation de Linux, \`a la
programmation, \`a l'\'electronique et \`a la radiodiffusion libre (ou
pirate). Les premiers hacklabs se sont d\'evelopp\'es en Europe,
\'emergeant souvent de la tradition des centres sociaux squatt\'es et
des laboratoires de m\'edia communautaires. En Italie, ils ont \'et\'e
li\'es avec les centres sociaux autonomes et en Espagne, en Allemagne
et aux Pays{}-Bas, avec le mouvement anarchiste des squats. Les
hackmeets sont des rassemblements temporaires de hackers et
d'activistes, dans lesquels on \'echange outils et savoirs et dans
lesquels on d\'eveloppe des projets. C'est en Italie notamment, dans
les ann\'ees 1990, que se sont tenus les premiers hackmeets. Il existe
des liens directs entre nombre d'entre eux et les artistes travaillant
avec le Logiciel Libre et Open Source. Le projet dyne:bolic (duquel est
sorti pure:dyne) s'est partiellement d\'evelopp\'e dans les hackmeets
italiens et les hacklabs hollandais. Le hacklab RampArts de Londres a
offert un point de rencontre au groupe local OpenLab, et \`a Barcelone,
des espaces tels que Hackitectura et Riereta ont soutenu plusieurs
projets artistiques et politiques \`a base de Logiciel Libre et Open
Source. Tous les artistes travaillant avec le Logiciel Libre et le
{\em livecoding} ne partagent pas n\'ecessairement la politique de
la sc\`ene hacklab, et tous les participants aux hacklabs ne voient pas
n\'ecessairement leurs activit\'es en tant qu'art, et certains sont,
parfois \`a raison, sceptiques quant au caract\`ere artistique de ce
qu'ils font. Toutefois, les hacklabs ont \'et\'e absolument
fondamentaux pour le d\'eveloppement du Logiciel Libre et Open Source
r\'ecent, plus particuli\`erement en Europe et en Am\'erique du Sud, et
ont apport\'e une orientation politique et \'ethique claire en
contraste avec les perspectives politiques et sociales quelque peu
confuses et souvent contradictoires articul\'ees dans les autres
communaut\'es et contextes du monde Logiciel Libre et Open Source plus
large.

Si le {\em livecoding} est une des manifestations artistiques du
Logiciel Libre et Open Source des plus embl\'ematiques, les hacklabs
sont devenus une de ses formes sociales les plus embl\'ematiques. Bien
que tous deux ne suivent pas des trajectoires identiques, elles se
chevauchent n\'eanmoins et se compl\`etent l'une l'autre de mani\`ere
significative. Le principe partag\'e de \quote{permettre la production par
d'autres} y joue un r\^ole central. Il s'agit d'une question de
distribution, pas simplement au niveau du produit, tel un logiciel qui
peut \^etre facilement distribu\'e par exemple, mais au niveau de la
pratique. La pratique elle{}-m\^eme est distributive en soi, parce
qu'elle int\`egre la distribution du savoir sur le comment produire
dans le produit m\^eme. Bien que cela ouvre des possibilit\'es de
production collaborative, il faudrait voir cela comme distinct d'une
collaboration en tant que telle. Alors qu'une pratique collaborative
regroupe la production de plusieurs personnes en un objectif unique,
dirigeant de la sorte la perspective de leur travail, une pratique
distributive permet le cadrage du travail sous leur propre conduite. Ce
qui se r\'ealise gr\^ace \`a la sortie de leur production en notation,
en code qui ne cr\'ee pas seulement un produit, mais acquiert une vie
active au{}-del\`a de son impl\'ementation initiale.

\SubSubTitle{La Production de Notation}

La notation n'est pas propre au logiciel. L'\'emergence du
{\em livecoding} en tant qu'activit\'e initialement musicale
refl\`ete l'engagement envers une production de notation qui a
caract\'eris\'e maintes traditions musicales diff\'erentes. Appliquer
le code informatique \`a la construction de son est, en un sens,
simplement un \'episode suppl\'ementaire du processus. Le
{\em livecoding} fonctionne sur une relation particuli\`ere entre la
notation et la contingence. La sp\'ecificit\'e du code est ouverte \`a
l'ind\'eterminisme de l'improvisation. A cet \'egard, le
{\em livecoding} ne s'ajoute pas simplement \`a l'\'evolution de la
production de notation musicale mais fait \'egalement \'echo \`a une
p\'eriode particuli\`ere durant laquelle une relation similaire entre
la notation et la contingence \'etait \`a l'avant{}-plan. C'\'etait une
p\'eriode o\`u l'improvisation du jazz exp\'erimental d\'evelopp\'ee
par des gens tels que John Coltrane, Ornette Coleman et Sun Ra,
croisait les syst\`emes de composition \quote{ouverte} de l'avant{}-garde,
qui furent d\'evelopp\'es par John Cage, Karlheinz Stockhausen, Earle
Brown et d'autres. Tout comme le Logiciel Libre et Open Source fait se
rencontrer deux \'ethiques de production de logiciel apparent\'ees,
soit le \quote{Logiciel Libre} et le principe de l' \quote{Open Source}, mais
cependant divergentes, nous pourrions d\'ecrire cette musique comme
Free Open Form Performance (Performance de Forme Libre et Ouverte,
abr\'eviation anglaise: FOFP). Le \quote{free jazz} \'etait le terme choisi
par Coleman et d'autres musiciens de jazz qui rejetaient l'usage du
terme \quote{improvisation}, jugeant qu'il \'etait souvent appliqu\'e \`a de
la musique noire par un public blanc afin de souligner une musicalit\'e
intuitive inn\'ee, refusant ainsi de reconna\^itre l'h\'eritage de
savoir{}-faire et la tradition formelle auquel font appel les musiciens
noirs. \quote{Ouvert} vient de {\em L'{\oe}uvre ouverte} de Umberto Eco,
un essai \'ecrit en 1959 qui fut parmi les premiers \`a couvrir et \`a
analyser les exp\'eriences d'{\oe}uvres al\'eatoires, ind\'etermin\'ees
et partiellement compos\'ees qui \'emergeaient dans l'avant{}-garde en
musique classique. A la fin des ann\'ees 1960, ces deux courants de
d\'eveloppement se sont rejoints, avec des compositeurs de jazz tels
que Coleman et Anthony Braxton travaillant consciemment avec
l'instrumentation et les formes structurales de l'avant{}-garde de la
musique classique, et des groupes tels que le Scratch Orchestra
adoptant la structure collective de formations tels que l'Art Ensemble
of Chicago. Les exp\'eriences sur la notation ont \'et\'e importantes
pour nombre de ces groupes et compositeurs, mais l'exploration de la
production notationnelle fut, pour le Scratch Orchestra, un des
objectifs fondateurs.

\PlaceFramedImage{scratch1.jpg}{Scratch Orchestra score}
\PlaceFramedImage{scratch2.jpg}{Scratch Orchestra score}
\PlaceFramedImage{scratch3.jpg}{Scratch Orchestra score}

\SubSubTitle{Histoire du Scratch}

Le Scratch Orchestra est n\'e d'une s\'erie de cours publics de musique
exp\'erimentale que Cornelius Cardew et d'autres compositeurs
organisaient \`a Londres \`a la fin des ann\'ees 1960. Elles
commenc\`erent \`a l'Anti{}-Universit\'e de la rue de Rivington et se
poursuivirent au Morley College, un centre d'\'education pour
travailleurs mis en place au 19\high{e} si\`ecle. Ce fut
l\`a que les membres fondateurs du Scratch Orchestra se
rencontr\`erent: Cardew, Michael Parson, Howard Skempton et les
personnes qui suivaient leurs cours. La fondation de l'Orchestre fut
officiellement annonc\'ee en juin 1969 lors de la publication dans le
{\em Musical Times} de \quote{Un Scratch Orchestra: un brouillon de
constitution} \'ecrit par Cardew. La constitution d\'efinit l'Orchestre
comme suit:

\QuoteStyle{({\dots}) Un grand nombre d'enthousiastes mettant leurs ressources en
commun (pas principalement des ressources mat\'erielles) et se
rassemblant pour des actions (faire de la musique, de la performance,
de l'\'edification).}

Quiconque pouvait s'y affilier, ind\'ependamment de sa capacit\'e
musicale. Nombre d'artistes visuels, comme Stefan Szczelkun, l'ont
rejoint et apport\'e avec eux l'int\'er\^et et l'exp\'erience des
happenings artistiques et des travaux d'intervention urbaine. Par ce
biais et par des concerts plus conventionnels, l'Orchestre avait pour
objectif de \quote{fonctionner dans la sph\`ere publique} en pr\'esentant des
{\oe}uvres cr\'e\'ees par le groupe. La constitution souligne les formes
d'activit\'es vari\'ees que l'Orchestre poursuivrait en cr\'eant ces
{\oe}uvres. Une de ses activit\'es les plus importantes fut l'\'ecriture
de \quote{Scratch Music}. Chaque membre de l'Orchestre avait un ordinateur
portable, ou \quote{Scratchbook}, sur lequel il \'ecrivait de petites {\oe}uvres
qui pouvaient \^etre combin\'ees en de plus grandes pi\`eces pour
orchestre. La constitution insiste pour que ces pi\`eces de Scratch
Music soient un processus actif d'exp\'erimentation avec diff\'erentes
formes de notation: \quotation{verbale, graphique, musicale, collage, etc}. En
1972, un processus clairement d\'efini pour le d\'eveloppement du
Scratch Orchestra \'emerge. Chaque morceau \'etait jou\'e par son
auteur, puis chaque partition \'etait \'echang\'ee et jou\'ee par
d'autres membres de l'Orchestre, offrant un genre de \quote{critique par ses
pairs} des compositions. Les \quote{Scratchers} avaient pour consigne de ne
pas \'ecrire plus d'un nouveau morceau par jour, mais \'etaient
encourag\'es de garder une \quote{production r\'eguli\`ere}, afin qu'il y ait
une boucle de feedback serr\'ee entre l'\'ecriture et la performance.

D\`es ses origines, le Scratch Orchestra prit la d\'ecision consciente
que toutes leurs notations puissent \^etre librement distribu\'ees, en
d\'eclarant que les travaux de Scratch Music \'etaient exempts de droit
d'auteur. Sur l'une de leurs premi\`eres collections
de partitions, publi\'ees en 1969, appel\'ee {\em Nature Study
Notes: Improvisation Rites}, la notice de copyright conventionnelle est
remplac\'ee par la suivante:

\QuoteStyle{Il n'y a pas de droits r\'eserv\'es dans ce livre de rites. Ils peuvent
\^etre reproduits et jou\'es librement. Les contributions que quiconque
souhaiterait envoyer pour une seconde s\'erie, doivent \^etre
adress\'ees \`a l'\'editeur: C.Cardew, 112 Elm Grove Road, Londres,
SW13.}

Bien que l'abandon de son droit d'auteur ne constituait
rien de nouveau {--} les situationnistes et le chanteur folk Woody
Guthrie avaient plac\'e des notices anti{}-copyright dans leurs travaux
{--} il est notable que le Scratch Orchestra ait aussi encourag\'e
d'autres \`a modifier et adapter leurs compositions, en d\'eclarant
qu'elles pouvaient \^etre incorpor\'ees dans la version suivante.

\SubSubTitle{Interruptions de Sons}

Les {\oe}uvres de {\em Nature Study Notes} sont tous des morceaux
d'instru\-ctions textuelles. Toutefois peu d'entre elles d\'ecrivent des
moyens de produire des sons, mais se concentrent plut\^ot sur diverses
interactions sociales qui construisent et jouent avec les relations de
pouvoir entre les artistes. Certains ressemblent \`a des jeux
d'\'equipe:

\QuoteStyle{Former un cercle debout. Nommer un meneur, il se place dans le cercle,
les yeux band\'es. Le restant du groupe tourne lentement autour de
lui/elle. ({\dots}) Lorsque le meneur est touch\'e, il c\`ede son
r\^ole en criant \quote{Porridge}.}

D'autres sont comme des automatons g\'en\'eratifs:

\QuoteStyle{Chaque personne qui rentre dans l'espace de performance re\c{c}oit un
nombre suivant son arriv\'ee. Quiconque peut donner un ordre (\`a
ob\'eir imp\'erativement) \`a un nombre plus \'elev\'e, et doit ob\'eir
aux ordres que lui donne un nombre moins \'elev\'e. Le N{\textdegree}1
re\c{c}oit ses ordres du num\'ero le plus \'elev\'e du moment (le
dernier joueur \`a entrer); le num\'ero le plus \'elev\'e ne peut
donner d'ordres qu'au N{\textdegree}1.}

De nombreuses compositions de {\em Nature Study Notes} mettent en
place des \quote{syst\`emes d'exploitation} \`a petite \'echelle, des
structures organisationnelles simples qui permettent de produire
d'autres travaux en leur sein. La notion de performance en tant que
syst\`eme d'exploitation a \'et\'e utilis\'ee par ap dans leur projet
{\em Life Coding}. Adaptant les m\'ecanismes des syst\`emes
informatiques, l'interaction des ex\'ecutants est dict\'ee par des
signaux d' \quote{interruption} connect\'es \`a des actions d\'efinies par
des tableaux de consultation (lookup tables). Dans les ordinateurs
conventionnels, le m\'ecanisme d'interruption permet aux signaux
d'appareils p\'eriph\'eriques tels que souris, claviers ou cartes de
r\'eseaux de rentrer dans le syst\`eme d'exploitation. D\`es la
r\'eception d'un signal interrompu, l'ordinateur s\'electionne une
action r\'eponse en confrontant un code d'identification pour chaque
signal \`a un tableau \`a consulter de routines programm\'ees
appel\'ees \quote{agents interrupteurs}. De la sorte, presser une cl\'e du
clavier ou bouger la souris peut changer le cours des \'ev\`enements
qui prennent place au moment m\^eme. L'interruption cr\'ee un vecteur
entre l'op\'eration interne du processeur central (CPU), le domaine des
op\'erations notationnelles et les contingences du monde ext\'erieur.
Comme Edsger Dijkstra, un des inventeurs du syst\`eme d'interruption,
l'a not\'e:

\QuoteStyle{Ce fut une belle invention, mais aussi une bo\^ite de Pandore. Les
moments exacts d'interruption \'etant impr\'evisibles et hors de
contr\^ole, le m\'ecanisme d'interruption a transform\'e l'ordinateur
en une machine non{}-d\'eterministe dot\'ee d'un comportement
non{}-reproductible. Pourrions{}-nous contr\^oler une telle b\^ete?}

\PlaceImage{sunra.jpg}{Sun Ra, film still from Space is the Place, 1974}

L'interruption rompt le d\'eroulement lin\'eaire ferm\'e de la machine
de Turing, permettant aux programmes d'\^etre arr\^et\'es, alt\'er\'es
ou red\'emarr\'es. Ceci a permis le d\'eveloppement de langages qui
pouvaient \^etre ex\'ecut\'es comme des formulations individuelles, pas
\`a pas, suscitant des commandes Shell (la commande textuelle
utilis\'ee sur les terminaux UNIX) et la~read{}-evaluate{}-print{}-loop
(parfois \quote{read{}-eval{}-print{}-loop} ou REPL) qui constitue la base
des langages de programmation interactifs tels que Lisp. L'interruption
et le~read{}-eval{}-print{}-loop sont au c{\oe}ur de tout programme de
{\em livecoding} et de tous les syst\`emes d'exploitation
d\'eriv\'es d'UNIX. Dans sa note sur la premi\`ere version de Linux,
Linus Torvalds \'ecrivit: \quote{les interruptions ne sont pas cach\'ees.}
C'est ici que la contingence et la notation se rencontrent, mais c'est
aussi ici que la possibilit\'e d'erreurs appara\^it. Toutefois,
plut\^ot que de pi\'etiner doucement de peur d'un accident, l'erreur
amen\'ee par un signal interrompu est pour certains une opportunit\'e
positive et productive. Ceci n'est pas limit\'e \`a des interruptions
d'ordinateurs. Sun Ra interrompait d\'elib\'er\'ement ses
r\'ep\'etitions et pi\'egeait ses musiciens. Les erreurs ainsi
produites n'\'etaient toutefois pas des fautes mais plut\^ot des formes
d'\'evolution:

\QuoteStyle{Il n'y a pas de fautes. Si quelqu'un joue faux ou si \c{c}a sonne mal,
les autres feront la m\^eme chose. Et alors cela para\^itra juste.}

Le syst\`eme d'exploitation du Arkestra de Ra incorporait des \quote{bruits}
de ce genre et se restructurait simultan\'ement. Ce \quote{bruit} n'est pas
simplement celui d'un son non musical, mais aussi, dans le sens que
Jacques Attali donne de la th\'eorie de l'information et des
syst\`emes, tout contenu qui n'est pas reconnu par un syst\`eme
existant et est d\`es lors oppos\'e \`a l' \quote{information}, contenu qui a
de la valeur ou du sens dans un syst\`eme donn\'e. Attali d\'ecrit
l'\'evolution des styles musicaux comme un syst\`eme existant de
musique devenant expos\'e au \quote{bruit} qui l'avait initialement
d\'erang\'e, et qui le restructure par incorporation et donne naissance
\`a un nouveau syst\`eme. Lors du voyage de l'Arkestra, des syst\`emes
se sont effondr\'es puis sont n\'es \`a nouveau quotidiennement.

\SubSubTitle{Compositions Scolaires}

Ce pouvoir sur les syst\`emes n'\'etait pas limit\'e au d\'emiurge ou
ma\^itre du jazz intergalactique. A la m\^eme \'epoque que celle o\`u
le Scratch Orchestra r\'einventait la musique \`a partir de ses
fondations, un groupe d'enfants \`a la Muzzey Junior High School aux
Etats{}-Unis exp\'erimentait leur propre syst\`eme de notation
improvis\'e. Ces enfants n'\'ecrivaient cependant pas de la musique,
mais apprenaient en autodidactes \`a programmer des ordinateurs. Ils
faisaient partie du premier LOGO Lab, un projet de Seymour Papert, un
chercheur du Laboratoire d'Intelligence Artificielle du MIT. LOGO
\'etait un simple langage de programmation qui dirigeait une entit\'e
appel\'ee \quote{tortue}. La tortue pouvait \^etre soit un personnage virtuel
sur \'ecran soit un petit robot qui suivait des instructions pour se
d\'eplacer sur le terrain (sur l'\'ecran ou l'espace au sol) et qui
pouvait dessiner une tra\^in\'ee dans son sillage. Les \'etudiants du
LOGO Lab d\'evelopp\`erent leurs propres programmes sur lesquels les
tortues ex\'ecutaient des dessins ou des exercices spatiaux. Dans la
mesure o\`u LOGO exprimait une s\'erie d'actions potentielles
desquelles \'emerge un dessin, il ressemble \`a la notation du Scratch
Orchestra, qui n'a pas souvent produit directement du son mais plut\^ot
des actions desquelles pouvait na\^itre du son. Comme l'\'ecrivait
Cardew dans ses notes \`a {\em Treatise}: \quotation{La notation est un moyen
pour faire bouger les gens.}

\PlaceFramedImage{cardew.jpg}{Cornelius Cardew, Treatise, 1963-67}

Tout comme le Scratch Orchestra, le LOGO Lab s'est d\'evelopp\'e \`a
partir d'un int\'er\^et p\'edagogique conscient pour le d\'eveloppement
de recherches pratiques de formes collectives et autog\'er\'ees. Elles
furent r\'ealis\'ees dans des activit\'es \quote{improvis\'ees}
semi{}-structur\'ees et utilisaient des syst\`emes de notation
auto{}-d\'evelopp\'es comme moyen de construction, de communication et
de r\'eflexion sur ces activit\'es. 

Comme il appara\^it clairement dans la constitution, le Scratch
Orchestra \'etait une exploration consciente de ce que pouvait \^etre
la notation et quel lien elle entretenait avec les tentatives
d'\'etablir une autre compr\'ehension de ce que peut \^etre la pratique
de la musique elle{}-m\^eme. Ceci fut d\'evelopp\'e au{}-del\`a du
contexte p\'edagogique des classes du Morley College, et c'est
peut{}-\^etre dans un geste d'autod\'erision, que les {\em Nature
Study Notes} du Scratch Orchestra et les plus anciennes partitions
{\em School Compositions} de Cardew prirent d\'elib\'er\'ement la
forme de livres d'exercices.

Papert \'etait convaincu que la programmation \'etait un savoir qui
devait \^etre accessible \`a tout un chacun, pas en tant que
\quote{technologie} {--} soit en tant que m\'ecanisme pour une production
manufactur\'ee d\'etach\'ee du travail humain {--} mais comme un moyen
d'exploration conceptuelle. Il existe des parall\`eles politiques entre
les deux projets. L'approche de l'informatique de Papert a \'et\'e
influenc\'ee par son implication ant\'erieure dans les mouvements
politiques radicaux de gauche. Dans les ann\'ees 1950, il fit partie du
groupe qui publiait la {\em Socialist Review} \`a Londres. Le
concept du LOGO Lab combinait des id\'ees des \'etudes psychologiques
de Jean Piaget et de Lev Vygotsky sur le d\'eveloppement des enfants
suivant les principes de non{}-scolarit\'e d'Ivan Illich. L'approche
invoqu\'ee est celle \quotation{d'un enfant qui programme l'ordinateur plut\^ot
qu'un ordinateur qui sert \`a programmer un enfant.} Papert d\'eclare
aussi que la conception d'un langage de programmation pourrait
refl\'eter une position politique et \'ethique particuli\`ere. Il
critique BASIC, un autre langage con\c{c}u \`a l'origine pour enseigner
la programmation, comme d\'emontrant \quotation{comment un syst\`eme social
conservateur s'approprie et tente de neutraliser un instrument
potentiellement r\'evolutionnaire.} Bien que le Scratch Orchestra ne se
soit pas d\'evelopp\'e \`a partir d'un programme politique d\'efini, il
a n\'eanmoins agi comme un contexte pour le d\'eveloppement d'une
pratique politis\'ee des arts, instruite \`a la fois par des tendances
marxistes et anarchistes. C'est par le Scratch Orchestra que Cardew a
acquis une conscience politique profonde, appliquant une perspective
explicitement mao\"iste \`a sa propre pratique, et qui le mena \`a son
implication dans la fondation du Parti Communiste R\'evolutionnaire de
Grande{}-Bretagne (Marxiste{}-L\'eniniste). Se faisant l'\'echo des
critiques de Papert au sujet de BASIC, Cardew a \'egalement critiqu\'e
le conservatisme institutionnel de nombreuses notations de musique,
exigeant plut\^ot que \quotation{tous les probl\`emes de notation soient
r\'esolus par les masses.} Pour Papert et Cardew, la p\'edagogie
\'etait une voie \`a deux sens. Le labo et l'orchestre ont bris\'e les
distinctions entre le ma\^itre et l'\'el\`eve et plac\'e
l'apprentissage dans le contexte de la production autog\'er\'ee. Ainsi,
il s'agissait de formes de pratique distributive.

\SubSubTitle{Apprentissage de la Contingence} 

Un \'el\'ement de la contingence fut essentiel \`a cette forme de
p\'edagogie radicale. Aux yeux de Papert, une des forces de la
programmation comme outil d'apprentissage, \'etait qu'elle encourageait
l'attitude \`a l'erreur. Rencontrer l'erreur, sous forme de bugs,
\'etait un aspect in\'evitable et n\'ecessaire de la programmation,
surtout en ce qui concerne la pratique singuli\`ere de programmation
d\'evelopp\'ee aux AI Labs du MIT, appel\'ee \quote{hacking}. Papert
soulignait que dans l'\'education conventionnelle, les erreurs avaient
une connotation purement n\'egative. Lorsqu'un{}-e \'etudiant{}-e
commet une erreur, il{}-elle est discr\'edit\'e{}-e, perd des points ou
est puni{}-e, ce qui inculque une peur de l'erreur et m\`ene \`a ne pas
vouloir s'\'ecarter des limites conventionnelles et prendre des
risques. Pour le{}-la hacker, par contre, ce qui importe n'est pas
qu'une erreur soit commise ou pas mais bien comment lui r\'epondre de
fa\c{c}on cr\'eative. Tout comme pour l'Arkestra, inclure l'erreur est
une possibilit\'e productrice. La prise en compte de l'erreur est
\'evoqu\'ee dans des documents tels que HAKMEM. Diminutif de \quote{hack
memo}, il s'agissait d'une s\'erie de bribes de codes et d'id\'ees de
programmation distribu\'ees parmi les hackers des AI Labs, m\'emos
auxquels ont contribu\'e notamment Richard Stallman, James Gosling et
Marvin Minsky. Il y est fait mention \`a de nombreuses reprises des
possibilit\'es d\'ecouvertes \`a la suite de bugs et d'incoh\'erences
au sein des ordinateurs du PDP sur lesquels travaillait le AI Lab.
D'autres contributions proposent des fa\c{c}ons de jouer sur un nouvel
algorithme particulier et encouragent les gens \`a le chambouler, d'une
mani\`ere qui ne peut \^etre d\'ecrite que comme une forme de jeu de
code esth\'etique. On peut voir HAKMEM comme l'\'equivalent des AI Labs
aux Scratchbooks \'echang\'es entre les membres du Scratch Orchestra.
Au sein des LOGO Labs, du code \'etait \'ecrit et \'echang\'e entre
\'etudiants de fa\c{c}on similaire. Plut\^ot que de pr\'eparer des
programmes \`a l'avance, les \'el\`eves \quote{improvisaient} avec leur code
en r\'epondant \`a la performance de la tortue et modifiaient leurs
programmes en fonction. L'apprentissage de LOGO passait donc par une
boucle de feedback semblable de code{}-performance que des livecoders
tels que Alex McLean identifient comme \'etant la base de leur pratique
et qui est construite sur le principe de read{}-eval{}-print{}-loop.

Les langages informatiques et de programmation proposent des
environnements hautement contraignants qui limitent l'\'eventail
d'interpr\'etations possibles d'une notation particuli\`ere.
L'interpr\'etation d'une notation par un humain peut \^etre beaucoup
moins contraignante. Pour Cardew, il s'agit d'une pr\'eoccupation
majeure du d\'eveloppement de nouvelles formes de notations, car
c'\'etait \`a la fois un danger et une opportunit\'e. Une opportunit\'e
car les notations ne doivent pas seulement encoder des mod\`eles
existants ou des syst\`emes d\'efinis de sons, mais peuvent aussi
\^etre des propositions et des provocations pour en \'ecrire de
nouveaux. Un danger car le musicien professionnel qui sera confront\'e
\`a un syst\`eme de notations inhabituel, pourrait se reposer sur ses
habitudes et pr\'edispositions personnelles, plut\^ot que d'y
r\'epondre directement. L'ex\'ecution risque de se r\'esumer \`a une
r\'egurgitation de vieux clich\'es et de formules \`a l'instar du
musicien de jazz amateur d\'ecrit par Adorno, qui est incapable de
s'\'eloigner des mod\`eles existants auxquels il s'est adapt\'e et
soumis. Lors de la performance, le musicien professionnel arrive avec
un syst\`eme pr\'ed\'efini de production sonore dans le cadre duquel il
interpr\`ete la nouvelle notation. On peut r\'eagir \`a ce qui \'etait
novateur dans la nouvelle notation comme \`a une \quote{erreur} ou \`a un
bruit interne au syst\`eme et donc l'\'eviter. Les nouvelles notations
n\'ecessitent des artistes ayant une attitude similaire \`a celle du
hacker ou d'un \'etudiant du LOGO lab, quelqu'un qui
peut r\'eagir de mani\`ere cr\'eative face \`a l'inconnu et \`a
l'inattendu. L'artiste ne peut donc pas r\'ep\'eter une telle musique
mais plut\^ot s'\quote{entra\^iner} \`a la fa\c{c}on d'un art martial, en
d\'eveloppant des fa\c{c}ons d'agir sur la contingence. Ceci s'est
\'egalement d\'evelopp\'e au moyen d'une boucle de feedback de
performance{}-code qui a constitu\'e la base de la pratique de la
Scratch Music. 

C'est par de telles boucles de feedback que les notations incorporent
l'exp\'erience du contingent dans une pratique future. Ce qui
constituait l'\quote{erreur} inattendue pass\'ee devient la pr\'eparation
pour des possibilit\'es inconnues futures. En l'int\'egrant, une
notation enregistre le d\'eveloppement historique d'une pratique, en
captant diff\'erentes versions de \quote{comment faire} et en permettant la
comparaison, l'analyse et la synth\`ese. Les LOGO Labs et le Scratch
Orchestra s'engageaient consciemment dans ce processus d'enregistrement
de versions successives, y m\^elant le savoir progressif, les
intentions et les standards de la communaut\'e des praticiens, qui
agissait comme une forme de contr\^ole des versions, distinguant les
pratiques les plus courantes des pratiques plus contradictoires ou
tangentes. 

\SubSubTitle{La Musique Noire en Notation}

Comment on en arrive \`a d\'efinir une notation et comment celle{}-ci
est distribu\'ee sont des questions essentiellement politiques. Cette
distribution s'\'etend au{}-del\`a de la publication des partitions ou
du code logiciel, sous la forme appliqu\'ee par l'usage des
m\'ecanismes de copyleft par le Scratch Orchestra. Ainsi que le
rappelle Ornette Coleman, la visibilit\'e m\^eme des notations au sein
du processus de production, comment elles sont r\'ev\'el\'ees et
dissimul\'ees, d\'ependent et expriment des relations de pouvoir
particuli\`eres: 

\QuoteStyle{Une fois, j'ai entendu Eubie Blake dire que quand il jouait dans des
groupes noirs pour des audiences blanches, \`a l'\'epoque o\`u la
s\'egr\'egation \'etait forte, les musiciens devaient arriver sur
sc\`ene sans aucune partition \'ecrite. Les musiciens regardaient les
partitions dans les coulisses, les laissaient l\`a et partaient la
jouer. Ils disaient qu'ils \'etaient plus vendables s'ils
pr\'etendaient savoir ce qu'ils faisaient. L'audience blanche se
sentait plus en s\'ecurit\'e.}

\PlaceImage{bag0.jpg}{Black Artists' Group, St Louis.}
\PlaceImage{bag1.jpg}{Black Artists' Group}

Le d\'esaveu de la notation d\'ecrit dans cet exemple est un refus de
l'auto{}-l\'egitimisation du musicien noir. Si l'usage d'une notation
peut aider \`a documenter le d\'eveloppement d'une pratique, son
histoire et auto{}-analyse, alors le refus de notation est le refus de
cette histoire, et donc, le refus de la base d'une l\'egitimation de
l'artiste. C'est dans cette optique que Coleman met une distance entre
sa propre pratique et l'id\'ee d'improvisation, car cette forme de
\quote{virtuosit\'e} est devenue la base d'un d\'eni de l\'egitimation. Le
\quote{free jazz} que lui et d'autres musiciens noirs ont mis en avant dans
les ann\'ees 60, n'\'etait pas juste libre dans le sens d'une cassure
de la structure musicale conventionnelle, mais aussi libre car il
rompait avec la condition \quotation{d'improvisateur dans une situation requise}.
Ceci a men\'e au d\'eveloppement de nouveaux espaces de performances,
beaucoup d'entre eux \'etant situ\'es directement au sein des
communaut\'es noires, et \`a l'articulation consciente de la pratique
comme une forme de recherche. Lester Bowie de l'Art Ensemble of Chicago
a choisi de rev\^etir un tablier de laborantin sur sc\`ene pour
annoncer la performance elle{}-m\^eme comme un lieu d'exp\'erimentation
radicale. Sun Ra encourageait son Arkestra en d\'eclarant: \quotation{Vous
n'\^etes pas des musiciens, vous \^etes des scientifiques de la
sonorit\'e.} Ra a pouss\'e ce concept \ encore plus loin avec la
cr\'eation de Ihnfinity Inc en 1967, une soci\'et\'e de recherche qui
\'etait cens\'ee \quotation{poss\'eder et op\'erer toutes sortes de laboratoires
de recherche, de studios, d'\'equipement \'electronique, d'appareils
\'electrochimiques communicationnels de notre propre design et
cr\'eativit\'e{\dots}} \`A St. Louis le Black Artists' Group a mis en
place un centre d'apprentissage afin de cr\'eer un forum de discussion
pour la communaut\'e locale qui, \`a c\^ot\'e des performances, des
r\'ep\'etitions et des ateliers, organisait aussi des r\'eunions et
d\'ebats quotidiens sur des questions d'int\'er\^et local. Selon
Anthony Braxton, la relation de la notation \`a la l\'egitimation est
devenue la base des recherches qui forment d\'esormais le centre de son
travail, le d\'eveloppement de la \quote{Musique Noire en Notation}. Ce
concept va au{}-del\`a de la simple description de sons sur une page et
se confronte au prolongement du r\^ole du son \`a un niveau socialement
structurant: \quotation{la notation peut \^etre per\c{c}ue comme un facteur
d'\'etablissement de la plateforme de la r\'ealit\'e de la musique.}

\PlaceImage{lesterbowie.jpg}{Lester Bowie of the Art Ensemble of Chicago}

Tandis qu'\`a la surface cela peut para\^itre refl\'eter la base
p\'edagogique de projets tels que le Scratch Orchestra et les LOGO
Labs, ceux{}-ci se sont d\'evelopp\'es \`a partir d'une trajectoire
compl\`etement diff\'erente. M\^eme si, d'une part, les p\'edagogies de
Cardew et Papert visaient \`a briser les structures sociales existantes
qui d\'eterminaient l'acquisition de musique et les aptitudes \`a la
programmation, la p\'edagogie constituait aussi la base \`a partir de
laquelle ils r\'eint\'egraient leurs travaux au sein du cadre
institutionnel existant. De cette fa\c{c}on leur pratique \'etait
institutionnellement l\'egitim\'ee. En particulier, la p\'edagogie
l\'egitimait leur statut \quote{non{}-commercial}. D'une mani\`ere identique,
la d\'ependance des Logiciels Libres sur l'acad\'emique sugg\`ere un
conflit d'int\'er\^ets potentiel au sein des ateliers g\'er\'es par des
artistes, ou du moins souligne les tensions sous{}-jacentes au travail
auto{}-valorisant qui est cens\'e \quote{payer le loyer}. Pour les musiciens
noirs des ann\'ees 1960 aux Etats{}-Unis, pour lesquels m\^eme un
acc\`es de base \`a l'\'education \'etait un probl\`eme, de telles
voies n'\'etaient pas accessibles. S'approprier des tabliers \quote{blancs}
de laborantins et une culture de recherche n'\'etait pas le meilleur
moyen pour obtenir la reconnaissance institutionnelle, mais
questionnait plut\^ot leur usage m\^eme en tant que m\'ecanismes de
l\'egitimation. Pour finir le Scratch Orchestra est devenu conscient de
sa d\'ependance \`a de telles formes externes de l\'egitimation et de
\quote{la situation obligatoire} dans laquelle elle op\'erait.

\SubSubTitle{Instrumentalisation du Collectif}

En 1972 des tensions ont commenc\'e \`a \'emerger au sein du Scracth
Orchestra. Certains ont ressenti que le groupe fonctionnait d'une
fa\c{c}on qui \'etait en contradiction avec ses objectifs, et un
\quote{dossier des m\'econtents} fut \'etabli pour que les gens puissent y
adresser leurs dol\'eances. En r\'eaction, Cardew, Keith Rowe et John
Tilbury cr\'e\`erent un groupe id\'eologique du Scratch Orchestra qui
appliquait une pratique de l'auto{}-critique mao\"iste parmi les
membres de l'Orchestra. M\^eme si un processus d'auto{}-critique au
sein de l'Orchestra a pu \^etre b\'en\'efique, cette approche
avant{}-gardiste ne fit qu'exacerber la situation. Beaucoup pens\`erent
qu'il s'agissait d'une imposition de la part d'une \'elite
auto{}-promue exer\c{c}ant son autorit\'e sur l'ensemble du Scratch
Orchestra, et que le rejet de certaines initiatives des autres membres
de la part du groupe id\'eologique, ne reconnaissait pas leur propre
base politique. Plut\^ot que de retrouver un but clair, l'Orchestra
s'est d\'ecompos\'e. Comme l'a dit par la suite un de ses membres,
Eddie Prevost, la contradiction fondamentale que rencontrait
l'Orchestra \'etait sans doute sa d\'ependance \`a sa propre
constitution, \`a l'objectif paradoxal de \quote{l\'egif\'erer pour la
non{}-conformit\'e}. Un autre membre, Michael Chant, fit observer que
la constitution elle{}-m\^eme fut une \quote{partition}. L'Orchestra \'etait
le produit de cette partition, une partition qui portait le nom d'un
unique auteur: Cornelius Cardew. De ce point de vue, la cr\'eation du
groupe id\'eologique du Scratch pourrait \^etre vu comme une tentative
de r\'ecup\'erer la paternit\'e de la \quote{composition} de Cardew, faisant
\'echo \`a la pr\'eoccupation de ses premiers \'ecrits selon lesquels
\quotation{la partition doit gouverner la musique}. Voici peut{}-\^etre un
exemple classique d'une avant{}-garde id\'eologique qui s'empare et
instrumentalise le collectif \`a ses propres fins, ou de la renaissance
de l'auteur dans un groupe qui tente de d\'epasser de telles notions
d'auteur unique. En refusant de succomber \`a l'acquisition de telles
id\'eologies et de la notion d'auteur, une restructuration n\'ecessaire
de la \quote{composition} de l'Orchestra prenait place. La qualit\'e
essentiellement distributive de l'Orchestra a investi des formes
d'auto{}-actualisation qui ont rendu le besoin d'un groupe unique
coh\'erent superflu. De nombreux membres se sont plus tard engag\'es
dans des activit\'es qui prolongeaient la {\em praxis} radicale
d\'evelopp\'ee au sein de l'Orchestra. La rupture, par cons\'equent, ne
repr\'esentait pas l'\'echec de ses membres, mais bien l'\'eclatement
de la limite entre la structure formelle de la partition/constitution
et les gens qui \'etaient la \quote{substance} de l'Orchestra. Comme le
disait Adorno pour d\'ecrire une erreur de notation dans unes des
compositions en s\'erie de Schoenberg, cela repr\'esentait:

\QuoteStyle{({\dots}) la perc\'ee de la substance devant \^etre structur\'ee, le
point o\`u elle rencontre le processus structurant et sans lequel cette
derni\`ere n'eut pu \^etre l\'egitim\'ee.}

\SubSubTitle{L\'egif\'erer pour la Non{}-conformit\'e}

Il y a des parall\`eles \`a \'etablir avec la mani\`ere dont le Logiciel
Libre s'appuie sur le copyleft et la GPL qui peut aussi \^etre vue
comme une mode de \quote{l\'egislation pour la non{}-conformit\'e}. La GPL
peut certes \quote{retourner} les restrictions normales cr\'e\'ees par le
droit d'auteur traditionnel, mais cela d\'epend
n\'eanmoins de leur cadre l\'egal de base et donc d'une notion
l\'egalis\'ee de libert\'e r\'ealis\'ee par l'entremise de la
propri\'et\'e exclusive. De l\`a vient l'attraction du copyleft pour
des libertariens de droite tels qu'Eric Raymond. En effet, on peut
avancer que le copyleft, dans sa r\'ealisation actuelle, plut\^ot que
d'incarner une forme de \quote{production en commun} illustre en r\'ealit\'e
quelque chose de plus proche de la \quote{transaction juste} de Robert
Nozick. Le probl\`eme avec le copyleft est sa forme actuelle et les
notions de \quote{remix} et de culture l\'egalis\'ee de l' \quote{appropriation}
qui s'y sont d\'evelopp\'ees, est qu'il pr\'esente simplement une
alternative {\em au sein} de la production propri\'etaire et
acquisitive (capitalisme) plut\^ot qu'une alternative \`a celle{}-ci.
Cela appara\^it dans la promotion active de la \quote{libert\'e}
jeffersonienne parmi les avocats de l'open source et des Creative
Commons, tels que Raymond et Lawrence Lessig. Mettre l'emphase sur le
copyleft comme une fin en soi et sur la GPL comme le document
d\'efinissant le logiciel libre, est donc potentiellement contraire aux
objectifs du Logiciel Libre. Un commentaire de Stallman corrobore cela:

\QuoteStyle{Le logiciel libre est une question de libert\'e. De notre point de vue,
savoir quel est pr\'ecis\'ement le m\'ecanisme l\'egal utilis\'e pour
d\'enier toute libert\'e aux utilisateurs du logiciel est juste un
d\'etail d'impl\'ementation. Que ce soit fait avec le droit
d'auteur, avec les contrats, ou de tout autre
mani\`ere, il est faux de refuser au public les libert\'es
n\'ecessaires pour former une communaut\'e et coop\'erer. C'est
pourquoi il est inexact de comprendre le Mouvement du Logiciel Libre
comme \'etant sp\'ecifiquement une question d'opposition au droit
d'auteur sur le logiciel. C'est \`a la fois plus et
moins que cela.}

Il est significatif que cette remarque \'etait une r\'eponse \`a la
promotion du copyleft par Robert T. Long comme \'etant compatible avec
les valeurs d'un march\'e libre libertarien. C'est peut{}-\^etre mieux
d\`es lors d'envisager la GPL et le copyleft comme des tactiques
conf\'erant un certain levier dans les circonstances actuelles. La
prolif\'eration des licenses \quote{libres} dans les derni\`eres ann\'ees
pourrait \^etre plus le signe de l'am\'enagement de pratiques
r\'esistantes \`a un ordre de l\'egitimation qu'ils feraient mieux
d'\'eviter, dans la mesure o\`u, dans le droit actuel, il n'existe
aucun sch\'ema magique de license qui mettra fin \`a la production
propri\'etaire.

\SubSubTitle{Production Distributive}

Les conflits au sein du Scratch Orchestra et les conflits entre Logiciel
Libre et Logiciel Open Source illustrent les distinctions, parmi les
formes de production, entre celles qui sont collectives et
distributives, et celles qui sont collaboratives et acquisitives. Une
pratique distributive permet l'arrangement du travail par d'autres sous
leur propre direction, tandis qu'une pratique acquisitive accumule le
travail des autres sans \'egard pour leur auto disposition. Elle expose
\'egalement le conflit qui peut \'emerger quand une pratique qui s'est
d\'evelopp\'ee au sein d'une communaut\'e auto{}-constitu\'ee devient
l'objet de formes externes de constitution et de l\'egitimation. Par
cons\'equent, toute collaboration n'est pas par essence distributive.
La nature des relations de pouvoir en son sein, ainsi que la
disposition et la l\'egitimation de la production qu'elles permettent,
peut \^etre soumise \`a des forces contradictoires.

L'importance de groupes tels que le Scratch Orchestra \`a la fin des
ann\'ees 60, jusqu'\`a l'\'emergence, presque quarante ans plus tard,
du livecoding, peut \^etre li\'ee aux changements de formes
g\'en\'erales de production qui ont pris place durant cette p\'eriode.
A une \'epoque o\`u l'\quote{\'economie de l'information} \'etait encore
\'emergente, et les outils et cadres conceptuels qui l'ont soutenue
encore embryonnaires, des projets tels que le Scratch Orchestra et les
LOGO Labs furent des tentatives de cr\'eer une trajectoire
\'emancipatrice avec les ressources et le savoir disponible.
Aujourd'hui nous sommes \`a une \'epoque o\`u l'\quote{\'economie de
l'information} s'est consolid\'ee et ses modes distinctifs de
production sont plus \'etablis et envahissants. C'est UNIX, pr\'etend
Martin Hardie, avec son syst\`eme de fichiers mis en r\'eseau et distribu\'e, qui a cr\'e\'e l'inscription de la notation de base pour ces modes de
production. La production de notation elle{}-m\^eme est devenu un
\'el\'ement cl\'e de la production et de la consommation contemporaine,
les masses y \'etant impliqu\'ees d'une mani\`ere que Cardew n'aurait
jamais pu pr\'evoir ni souhaiter. Chaque aspect de nos vies est not\'e
\`a un degr\'e inconnu jusqu'alors et nous sommes sans cesse mis au
d\'efi par de nouvelles compositions et scripts que nous devons
ex\'ecuter pour accomplir m\^eme la t\^ache la plus m\'ediocre. C'est
par une telle notation que le travail immat\'eriel est valoris\'e et
aussi g\'er\'e, et que nous sommes amen\'es \`a collaborer avec les
processus m\^emes de production qu'elle inscrit. En effet, une telle
collaboration est devenue le paradigme dominant \`a la fois du
contr\^ole manag\'erial et de la consommation quotidienne, tel
qu'illustr\'e dans la prolif\'eration de produits et services hautement
\quote{personnalis\'es}, le loisir de r\'ealit\'e et les r\'eseaux sociaux du
Web 2.0. Cette forme de collaboration, cependant, est construite par
des m\'ecanismes acquisitifs plut\^ot que distributifs. Par ce biais,
l'usine en tant qu'une unit\'e singuli\`ere et coh\'erente de
production a laiss\'e la place \`a des syst\`emes de r\'eseaux
amorphes. Dans une certaine mesure ces d\'eveloppements ont \'et\'e
accompagn\'es par l'\'evolution de groupes compos\'es d'un nombre de
membres relativement stable tels le Scratch Orchestra ou Art Ensemble
of Chicago, vers des groupes et individus connect\'es de mani\`ere
beaucoup plus diffuse qui sont caract\'eristiques de la sc\`ene
artistique li\'ee au Logiciel Libre et Open Source. De la m\^eme
mani\`ere, des pratiques qui auparavant pouvaient \^etre limit\'ees \`a
un seul groupe, telles que la composition de musique scratch, sont
devenues de plus en plus diss\'emin\'ees et r\'epandues, des
r\'epertoires de code en ligne rempla\c{c}ant la circulation de livre
de scratch et la pratique artistique \'etant valoris\'ee dans une plus
grande mesure qu'avant, et parfois comme un but en soi. Ceci traduit
les intersections et les conflits entre les pratiques dominantes et
r\'esistantes qui caract\'erisent la nature dialectique de la
production en g\'en\'eral. Si le livecoding est embl\'ematique d'une
nouvelle trajectoire \'emancipatrice \'emergeant au sein de cette
dialectique, alors il sera utile de r\'eexaminer les probl\`emes de
notations et la politique de la production de notation tels
qu'exp\'eriment\'es et travaill\'es par ceux qui ont dans le pass\'e
apport\'e le code sur le devant de la sc\`ene.}
