\PlaceImage{rabadan2.jpg}{Screening Modern Times at V/J10}
\PlaceImage{rabadan0.jpg}{}
\PlaceImage{rabadan1.jpg}{}
\PlaceImage{rabadan3.jpg}{}

\AuthorStyle{In\`es Rabadan}

\licenseStyle{Creative Commons Attribution{}-NonCommercial{}-ShareAlike}

\Eng{\Title{Does the repetition of a gesture irrevocably\crlf lead to madness?}

\SubTitle{A personal introduction to Modern Times\crlf (Charles Chaplin, 1936)}

One of the most memorable moments of Modern Times, is the one where
the tramp goes mad after having spent the whole day screwing bolts on the
assembly line. He is free: neither husband, nor worker, nor
follower of some kind of movement, nor even politically engaged. His
gestures are burlesque responses to the adversity in his life, or just
plain \quote{exuberant}. But through the interaction with the machine,
however, he completely goes off the rails and ends up in prison. 

In\`es Rabadan made two short films in which a female protagonist is
confined by the fast{}-paced work of the assembly line. Tragically and
mercilessly, the machine changes the woman and reduces her to a
mechanical gesture {--} a gesture in which she sometimes takes pride,
precisely in order not to lose her sanity. Or else, she really goes mad, ruined by the machine, eventually managing to free herself.}

\Ned{\Title{Leidt een repeterende beweging\crlf noodzakelijkerwijs tot waanzin?}

\SubTitle{Persoonlijke introductie tot Modern Times\crlf (Charles Chaplin, 1936)}

Een van de meest gedenkwaardige momenten uit Modern Times is de sc\`ene
waarin de zwerver gek wordt, nadat hij de hele dag aan een lopende band
bouten heeft staan aandraaien. Gewoonlijk is hij vrij; hij is geen
echtgenoot, geen arbeider, geen aanhanger van een beweging, zelfs niet
politiek betrokken. Zijn gebaren zijn steeds opnieuw burleske
antwoorden op de tegenspoed die hem te beurt valt, of gewoonweg
uitgelaten. Door de interactie met de machines ontspoort hij echter
volkomen, en eindigt in de gevangenis. 

In\`es Rabadan maakte twee kortfilms waarin een vrouwelijke protagonist
vastgekluisterd zit aan jachtig lopende bandwerk. Droefgeestig of
onverbiddelijk, altereert de machine de persoon, en reduceert haar tot
het mechanische gebaar {--} een gebaar waarover ze soms enige trots
voelt, precies om niet gek te worden. Het kan echter ook dat ze
werkelijk gek wordt, dat de machine haar in de war brengt, en ze er
daardoor uiteindelijk in slaagt haar eraan te onttrekken.}

\Fra{\Title{La r\'ep\'etition d'un geste rend{}-elle
forc\'ement fou?}

\SubTitle{Introduction personnelle \`a Modern Times\crlf (Charles Chaplin, 1936)}

Un des moments les plus m\'emorables des Temps Modernes,
c'est celui o\`u Charlot, d'avoir
viss\'e des boulons toute la journ\'ee sur une cha\^ine, devient fou. 

Il est libre, d'ordinaire. Ni mari\'e, ni travailleur,
ni pratiquant de quoi que ce soit, ni m\^eme engag\'e politiquement,
ses gestes sont toujours nouveaux, r\'eponses burlesques \`a
l'adversit\'e, ou alors ils sont \quote{fol\^atres} (le pied
jet\'e de c\^ot\'e). Mais l'interaction \ avec la
machine le fait d\'erailler puis jeter en prison.

Dans deux courts m\'etrages, In\`es Rabadan a film\'e un personnage
(f\'eminin) riv\'e au flux du travail \`a la cha\^ine. \ Morne ou
implacable, la machine alt\`ere la personne, la r\'eduit \`a ce geste
{--} dont elle a parfois la fiert\'e, pour ne pas devenir folle,
justement. Ou alors, elle devient vraiment folle, la d\'etraque, et
r\'eussit \`a s'en lib\'erer.}
