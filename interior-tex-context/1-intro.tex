{\tfa\setupinterlinespace \Eng{How does the information we seize in search engines circulate, what happens to our data entered in social networking sites, health records, news sites, forums and chat services we use? Who is interested? How does the \quote{market} of the electronic profile function? These questions constitute the framework of the E{}-traces project.

For this, we started to work on Yoogle!, an online game. This game, still in an early phase of development, will allow users to play with the parameters of the Web 2.0  e\-co\-no\-my and to exchange roles between the different actors of this economy. We presented a first demo of this game, accompanied by a public discussion with lawyers, artists and developers. The discussion and lecture were meant to analyse more deeply the mechanism of the economy behind its friendly interface, the speculation on profiling, the exploitation of free labor, but also to develop further the scenario of the game.

}

\Ned{Hoe circuleert informatie die we vinden via zoekmachines, wat gebeurt er met de data die we invoeren in sociale netwerksites, medische databanken, nieuwssites, forums en chatplatformen? Hoe werkt de \quote{markt} voor electronische profielen?

Michel Cleempoel en Nicolas Malev\'{e} werken aan Yoo\-gle!; een online spel waarin gebruikers kunnen spelen met de parameters van de Web 2.0 economie en met de rolverdeling tussen verschillende actoren die daarbij horen. Tijdens het festival lieten ze een eerste demo van dit spel zien. Ook organiseerden ze een publieke discussie met advocaten, kunstenaars en ontwikkelaars. De discussies en lezingen waren bedoeld om de mechanismes van de economie achter de vriendelijke interface, de speculatie met profielen, de exploitatie van gratis arbeid te analyseren, maar ook om het scenario van het Yoogle! spel verder te brengen.

}

\Fra{Comment circulent les informations que nous saisissons dans les moteurs de recherche, les sites de rencontre, de sant\'e, de news, les forums et les chats? Qui s'y int\'eresse? Comment fonctionne le march\'e du profil \'electronique? Ces questions forment le cadre du projet E{}-traces.

Pour ce faire, nous avons commenc\'e à travailler \`a Yoogle!, un jeu en ligne. Ce jeu, encore dans une phase premi\`ere de d\'eveloppement, permet de prendre tour{}-\`a{}-tour les r\^oles des diff\'erents acteurs de ce \quote{march\'e} et de participer aux man{\oe}uvres des uns et des autres. Une premi\`ere d\'{e}mo du jeu  accompagn\'ee de pr\'esentations de projets aux int\'er\^ets similaires, de discussions avec des artistes, juristes, d\'eveloppeur/euse/s et d'une conf\'e\-rence. Ces activit\'es ont pour but d'analyser les m\'ecanismes \'economiques \`a l'{\oe}uvre derri\`ere les interfaces riantes du Web 2.0, la sp\'eculation sur les profils, l'exploitation du travail gratuit, mais aussi de d\'evelopper plus avant le scenario du jeu. Enfin, cette rencontre est aussi con\c{c}ue comme un appel aux diff\'erentes personnes et organisations int\'eress\'ees de se constituer en groupe de veille technologique sur les questions d'identit\'es et de surveillance en ligne.

}

}