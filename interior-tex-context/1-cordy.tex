\PlaceImage{cor.jpg}{DOPPELGÄNGER III, n.demello/MéTAmorphoZ} 

\AuthorStyle{M\'eTAmorphoZ}

\licenseStyle{Creative Commons Attribution{}-NonCommercial{}-ShareAlike}

\Fra{\Title{Doppelg\"anger}

M\'eTAmorphoZ est un collectif pluridisciplinaire cr\'e\'e en 2001 autour d'un noyau fixe constitu\'e de Val\'erie Cordy, metteur en sc\`ene, et Natalia de Mello, artiste plasticienne. Au gr\'e des projets et sur base d'affinit\'es sensibles, le collectif s'est enrichi de collaborations r\'ecurrentes avec des concepteurs issus de diff\'erents horizons, tels Derek Sein (musicien et DJ), Marc Doutrepont (ing\'enieur du son), Sandra Naz\'e (chanteuse lyrique), Ali Contu (photographe), Laurent d'Ursel (artiste performer), Laurence Drevard (artiste multim\'edia), Manu Flety (ing\'enieur en \'electronique/Ircam), Alexandre Cordy (programmation), Nicolas d'Alessandro (ing\'enieur Numediarts), Alain Cofino Gomez (auteur dramatique), H\'el\`ene Heinrichs et Doroth\'ee Schoonooghe (com\'ediennes).

Les projets de M\'eTAmorphoZ traitent des implications interpersonnelles et des enjeux collectifs des technologies ordinaires de notre vie quotidienne qui m\^elent inextricablement l'humain et la machine (les jeux vid\'eos et les relations de l'humain et du virtuel dans {\em Zone temporaire} (2001{}-2003); les cam\'eras de surveillance de la soci\'et\'e de contr\^ole et les codes barres avec la tra\c{c}abilit\'e \'electronique dans {\em Doppelg\"anger I}, {\em II} et {\em III} (2002{}-2007); la construction de l'humain et les relations de d\'ependance affective avec {\em Ami} (2002{}-2007); le t\'el\'ephone mobile dans la communication dans {\em j'tapLDkej'pe} (2004); le zapping visuel et auditif et les al\'eas du choix et de la libert\'e dans {\em M\'etamorphoses} (2005); le chat sur Internet dans {\em Wired Dreams} (2006), etc.).

Avec le projet, le collectif s'int\'eresse \`a la th\'ematique du double \'electronique dans la soci\'et\'e de contr\^ole et de surveillance.

\quotation{Notre identit\'e \'electronique, embl\`eme de cette nouvelle soci\'et\'e de contr\^ole, redouble d\'esormais notre identit\'e organique et sociale. Mais l'obligation l\'egale de se voir assigner une identit\'e unique, stable et infalsifiable n'est{}-elle pas, en d\'efinitive, un danger pour notre libert\'e fondamentale de revendiquer des identit\'es qui sont forc\'ement et irr\'em\'ediablement multiples pour chacun d'entre nous?}}


\Ned{\Title{Doppelg\"{a}nger}
Het collectief M\'eTAmorphoZ (Val\'erie Cordy, Natalia de Mello) werd opgericht in september 2001 als een multidisciplinaire vereniging voor de creatie van installaties, voorstellingen en transdisciplinaire performances waarbij artistiek experiment samengaat met digitale praktijken.

Met het project Doppelganger, focust het collectief M\'eTAmorphoZ op de thematiek van de electronische verdubbeling (duplicaat, tweeling) in een maatschappij gedomineerd door controle en toezicht. 

\quotation{Onze elektronische identiteit, symbool van deze nieuwe controle-maatschappij, dupliceert onze organische en sociale identiteit. Maar is de legale verplichting van een unieke, stabiele en onvervalsbare identiteit uiteindelijk geen gevaar voor de fundamentele vrijheid die elk van ons heeft om veelvoudige identiteiten voor zichzelf te kunnen opeisen?}}


\Eng{\Title{Doppelg\"{a}nger}
Born in September 2001, represented here by Val\'erie Cordy et Natalia De Mello, the M\'eTAmorphoZ collective is a multidisciplinary association that create installations, spectacles and transdisciplinary performances that mix artistic experiments and digital practices.

With the project Doppelganger, the collective M\'eTAmorphoZ focuses on the thematic of the electronic double(duplicate, twin) in a society of control and surveillance. 

\quotation{Our electronic identity, symbol of this new society of control, duplicates our organic and social identity. But this legal obligation to be assigned a unique, stable and unforgeable identity isn’t, in the end, a danger for our fundamental freedom to claim identitites which are irreducibly multiple for each of us?}}
