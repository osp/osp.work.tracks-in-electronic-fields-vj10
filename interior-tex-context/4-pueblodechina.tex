\setupcaptions[width=.9\textwidth,style=\tfx\setupinterlinespace,number=yes,numberconversion=Characters]
\setupindenting[no]

\resetnumber[figure]

\AuthorStyle{elpueblodechina a.k.a.

Alejandra Maria Perez Nunez}

\licenseStyle{??}

\Eng{\Title{El Curanto}

Curanto is a traditional method of cooking in the ground by the people
of Chiloe, in the south of Chile. This technique is practiced
throughout the world under different names. What follows is a summary
of the \SHOUT{elements} and steps enunciated and executed during el curanto,
which was performed in the centre of Brussels during V/J10.
\godown[2em]

\startcolumns[n=2,balance=yes]

\Title{Recipe}
\SubTitle{Free Libre Open Source\crlf Curanto in the center\crlf of Bruxelles }

\Curanto{For making a curanto you need to take the following steps and arrange
the following \SHOUT{elements}:

\CurantoFigure{elpueblo1.jpg}{a slow cooking \SHOUT{oven}} 

\godown[2em]

\startitemize[3,unpacked]

\item{\SHOUT{Oven}, a hole in the ground filled with fire resistant \SHOUT{stones}.
}

\stopitemize

This image is repeated in many different cultures. Might be an ancient way of
cooking. What does this underground cooking imply? Most of all, it
takes a lot of \SHOUT{time}.

\startitemize[3,unpacked]

\item{Find a way to get a good deal at the market to get fresh \SHOUT{mussels} for x
people. It helps to have a \SHOUT{charismatic woman} do it for you.}

\stopitemize

\CurantoFigure{elpueblo2.jpg}{a \SHOUT{terrain vague} in the centre of Brussels and a \SHOUT{neighbour}
willing to let you in.}

\CurantoFigure{elpueblo3.jpg}{\SHOUT{A hole} in the ground 1.5 m deep, 1 m diameter. (It makes me
think of a hole in my head).}

A hole in the ground reminds me of the unknown. \SHOUT{Food} cooked inside the
ground relates to ideas, creativity and \SHOUT{gift}. It helps to have
\SHOUT{Guillaume} or a strong and positive \SHOUT{man} to help you dig the hole. A
second \SHOUT{person} would be of great help, especially if, while digging, he
would talk about taxonomies of immaterial labour.

\startitemize[3,unpacked]

\item{\SHOUT{A bright woman friend} to find out about \SHOUT{Belgian porphyry} and tell you
about the mining {\em carri\`ere} in Quenast (Hainaut).} 

\item{A \SHOUT{camera woman} to hand you a \SHOUT{marble stone} to put inside the \SHOUT{oven}.}

\item{\SHOUT{Wendy} or some other \SHOUT{multitasking woman} who is extremely \SHOUT{patient} and
\SHOUT{Humouristic} and who helps you to focus and takes pictures.}

\item{\SHOUT{Femke} and \SHOUT{Peter} or some \SHOUT{Excentric couple} that \SHOUT{Trusts} the carrier of the
performance, will tell their \SHOUT{story} about \SHOUT{traveling mussels}.} 
\stopitemize

Mussels eaten in the centre of Brussels are grown in Ireland and immersed in
Dutch seawater and are then officially called Dutch. After 2 days in
Dutch water, they are ready to be exported to Brussels and become
Belgian mussels that are in fact Dutch{}-Irish.

\CurantoFigure{elpueblo4.jpg}{Original curanto \SHOUT{stones} are round fire resistant stones. I
couldn't find them in Brussels.}

The only round and granite stones were very expensive design ones. In Chile you just dig a
hole anywhere and find them. The only fire resistant rock in Brussels
was the \SHOUT{street} itself.

\startitemize[3,unpacked]

\item{Square shaped rocks collected randomly throughout the city by means of appropriation.}

\stopitemize

Streets are made of a type of granite rock, might be
Belgian porphyry. Note that there is a message on one of the stones we
picked up in the centre. It reads 'watch your
head'.

\CurantoFigure{elpueblo5.jpg} {A good \SHOUT{Bucket} to scoop the rain out of your newly dug \SHOUT{hole}}

\CurantoFigure{elpueblo6.jpg} {A tent to protect your \SHOUT{fire} from random \SHOUT{rain}}

\CurantoFigure{elpueblo7.jpg} {\SHOUT{Laia} or some psychonaut, hierophant friend.}

Should be someone who is able to transmit confidence to the execution of el curanto and
who will keep you company while you are appropriating stones in
Brussels.


\startitemize[3,unpacked]

\item{A good \SHOUT{bouillon} made of cheap white wine and concentrated bio
vegetables and spices is one of the secrets.}

\stopitemize

\CurantoFigure{elpueblo8.jpg}{You need to find \SHOUT{Moam} or some Palestinian fellow to help you
keep the fire burning}

\CurantoFigure{elpueblo9.jpg}{\SHOUT{Girl} that will randomly come to the place with her \SHOUT{mother} and
speak in Spanish to the carrier of the performance.}

She will play the flute, give the \SHOUT{oven} some orders to cook well and sing improvised
\SHOUT{songs}. She and some other children will play around by digging holes
and making their own \SHOUT{curanto}.

\CurantoFigure{elpueblo10.jpg}{A big \SHOUT{fire} to heat up the wet cold ground of Brussels}

\CurantoFigure{elpueblo11.jpg} {\SHOUT{Red hot coal}}

\CurantoFigure{elpueblo12.jpg} {Using some cabbage leaves to cover the \SHOUT{red hot coal} to place
the \SHOUT{food} on top of} 

\CurantoFigure{elpueblo13.jpg} {\SHOUT{A sack cloth} to cover the food and to retain \SHOUT{steam} for
cooking.}

\CurantoFigure{elpueblo14.jpg} {\SHOUT{Didier} or some \SHOUT{Panic cook man} who is happy to \SHOUT{share} his expert
knowledge and willing to join in the performance.}

\CurantoFigure{elpueblo15.jpg}{\SHOUT{Onions}, \SHOUT{gestures} and \SHOUT{speculations}.}

While reading \SHOUT{Valis}, the carrier of the performance will become reverend \SHOUT{Timothy Archer} and read
about \SHOUT{time} (something that has mainly been forgotten is Palestine).

\CurantoFigure{elpueblo16.jpg}{el curanto is to be made together with \SHOUT{people} and for \SHOUT{everyone}.}

\column

\startitemize[3,unpacked]

\item{\SHOUT{Wood} found in a dismantled house. It helps to find a ride to transport
it.}

\item{\SHOUT{Hole}}

\item{\SHOUT{Mussels}}

\item{\SHOUT{Spices}, rosemary and bay leaf.}

\item{\SHOUT{Michael} or some \SHOUT{dedicated} friend that will assist with the execution of
the performance and keep the pictures of it afterwards for months.}

\stopitemize

\CurantoFigure{elpueblo17.jpg}{You can eat from the shell by using your hands or a little
\SHOUT{wooden spoon}.}

If you want to eat later, take the mussels out of their
shell, add \SHOUT{olive oil}, make a spread and keep it cold in a jar. Find
\SHOUT{queer} couples to savour it with \SHOUT{bread} while talking about \SHOUT{sex}.

\column

\startitemize[3,unpacked]

\item{\SHOUT{Fire}}

\item{\SHOUT{Red hot coal}}

\item{\SHOUT{Food}}

\item{\SHOUT{Noise} from the cooking \SHOUT{mussels}. It helps to use
'hot' \SHOUT{Piezzo microphones}.
}
\stopitemize

Here \SHOUT{time} turns into space. \quotation{Time can be overcome}, Mircea Eliade
wrote. That's what it's all about.
\blank
The great mystery of Eleusis, of the Orphics, of the early Christians,
of Sarapis, of the Greco {}-Roman mystery religions, of Hermes
Trismegistos, of the Renaissance Hermetic alchemists, of the Rose Cross
Brotherhood, of Apollonius of Tyana, of Simon Magus, of Asklepios, of
Paracelsus, of Bruno, consists of the abolition of time. The techniques
are there. Dante discusses them in the Comedy. It has to do with the
loss of amnesia; when forgetfulness is lost, true memory spreads out
backward and forward, into the past and into the future, and also,
oddly, into alternate universes; it is orthogonal as well as linear.
\footnote{Philip K. Dick Valis (1972)} 

}

\stopcolumns

}

%RESTORE DEFAULTS
\setupcaptions[style=\tfxx\setupinterlinespace,number=no,location=bottom,width=1.45cm,align=middle]
\setupindenting[yes, 1em, next]
