\setupthinrules[interlinespace=small]

\def\VideoRemark#1{{\tf\setupinterlinespace #1}\godown[0.25em]\thinrule\godown[0.5em]}
\def\VideoTitle#1{\blank{\bia\setupinterlinespace #1}\blank[small]}
\def\VideoSubTitle#1{\blank{\bfa\setupinterlinespace #1}\blank[small]}

\PlaceImage{video0.jpg}{}
\PlaceImage{video1.jpg}{}
\PlaceImage{video2.jpg}{}

\def\VideoLib#1{\noindenting\blank{\startcolumns[n=2,tolerance=verytolerant,balance=yes]{\tfx\setupinterlinespace #1}\stopcolumns}}

\Eng{\Title{Mutual Motions Video Library}
\SubTitle{To be browsed, a vision to be displaced}

Wearing the video library, performer Isabelle Bats presents a selection
of films related to the themes of V/J10. As a living memory, the discs
and media players in the video library are embedded in a dress designed
by artists collective De Geuzen. Isabelle embodies an accessible
interface between you (the viewer), and the videos. This human
interface allows for a mutual relationship: viewing the films
influences the experience of other parts of the program, and the
situation and context in which you watch the films play a role in
experiencing and interpreting the videos. A physical exchange between
existing imagery, real{}-time interpretation, experiences and context,
emerges as a result.

The V/J10 video library collects excerpts of performance and dance video
art, and (documentary) film, which reflect upon our complex
body{--}technique relations. Searching for the indicating, probing,
disturbing or subverting gesture(s) in the endless feedback loop
between technology, tools, data and bodies, we collected historical as
well as contemporary material for this temporary archive.
\godown[2em]

\VideoTitle{Modern Times or the Assembly Line}
{\em Reflects the body in work environments, which are structured by
technology, ranging from the pre{}-industrial manual work with analogue
tools, to the assembly line, to postmodern surveillance configurations.}

\VideoLib{\VideoSubTitle{24 Portraits}

\VideoRemark{Excerpt from a series of documentary portraits by Alain Cavalier, FR, 1988{}-1991.}

{\em 24 Portraits} is a series of short documentaries paying tribute
to women's manual work. The intriguing and sensitive portraits of 24
women working in different trades reveal the intimacy of bodies and
their working tools.

\VideoSubTitle{Humain, trop humain} 

\VideoRemark{Quotes from a documentary by Louis Malle, FR, 1972.}

A documentary filmed at the Citroen car factory in Rennes and at the
1972 Paris auto show, documenting the monotonous daily routines of
working the assembly lines, the close interaction between bodies and
machines.

\VideoSubTitle{Performing the Border}

\VideoRemark{Video essay by Ursula Biemann, CH, 1999, 45 min. }

\quotation{{\em Performing the Border} is a video essay set in the
Mexican{}-U.S. border town Ciudad Juarez, where the U.S. industries
assemble their electronic and digital equipment, located right across
El Paso, Texas. The video discusses the sexualization of the border
region through labour division, prostitution, the expression of female
desires in the entertainment industry, and sexual violence in the
public sphere. The border is presented as a metaphor for
marginalization and the artificial maintenance of subjective boundaries
at a moment when the distinctions between body and machine, between
reproduction and production, between female and male, have become more
fluid than ever.} (Ursula Biemann)
\blank
\Url{http://www.geobodies.org}

\VideoSubTitle{Maquilapolis (city of factories)}

\VideoRemark{A film by Vicky Funari and Sergio De La Torre, Mexico/U.S.A., 2006, 68
min.}

Carmen works the graveyard shift in one of Tijuana's
{\em maquiladoras}, the multinationally{}-owned factories that came
to Mexico for its cheap labour. After making television components all
night, Carmen comes home to a shack she built out of recycled garage
doors, in a neighbourhood with no sewage lines or electricity. She
suffers from kidney damage and lead poisoning from her years of
exposure to toxic chemicals. She earns six dollars a day. But Carmen is
not a victim. She is a dynamic young woman, busy making a life for
herself and her children.

As Carmen and a million other {\em maquiladora} workers produce
televisions, electrical cables, toys, clothes, batteries and IV tubes,
they weave the very fabric of life for consumer nations. They also
confront labour violations, environmental devastation and urban chaos
{--} life on the frontier of the global economy. In
{\em Maquilapolis} Carmen and her colleague Lourdes reach beyond the
daily struggle for survival to organize for change: Carmen takes a
major television manufacturer to task for violating her labour rights,
Lourdes pressures the government to clean up a toxic waste dump left
behind by a departing factory. 

As they work for change, the world changes too: a global economic crisis
and the availability of cheaper labour in China begin to pull the
factories away from Tijuana, leaving Carmen, Lourdes and their
colleagues with an uncertain future.

A co{}-production of the Independent Television Service (ITVS), project
of Creative Capital.
\blank
\Url{http://www.maquilapolis.com}
}

\VideoTitle{Practices of everyday life}
{\em Everyday life as the place of a performative encounter between bodies
and tools, from the U.S.A. of the 70s to contemporary South
Africa.}

\godown[2em]

\VideoLib{
\VideoSubTitle{Saute ma ville}

\VideoRemark{Chantal Akerman, B, 1968, 13 min.}

A girl returns home happily. She locks herself up in her kitchen and
messes up the domestic world. In her first film, Chantal Akerman
explores a scattered form of being, where the relationship with the
controlled human world literally explodes. Abolition of oneself,
explosion of oneself.

\VideoSubTitle{Semiotics of the Kitchen}

\VideoRemark{Video by Martha Rosler, U.S.A., 1975, 05:30 min.}

{\em Semiotics of the Kitchen} adopts the form of a parodic cooking
demonstration in which, Rosler states, \quotation{An anti{}-Julia Child replaces
the domesticated \quote{meaning} of tools with a lexicon of rage and
frustration.} In this performance{}-based work, a static camera is
focused on a woman in a kitchen. On a counter before her are a variety
of utensils, each of which she picks up, names and proceeds to
demonstrate, but with gestures that depart from the normal uses of the
tool. In an ironic grammatology of sound and gesture, the woman and her
implements enter and transgress the familiar system of everyday kitchen
meanings {--} the securely understood signs of domestic industry and
food production erupt into anger and violence. In this alphabet of
kitchen implements, Rosler states that, \quotation{When the woman speaks, she
names her own oppression.}
\blank
\quotation{I was concerned with something like the notion of \quote{language speaking
the subject}, and with the transformation of the woman herself into a
sign in a system of signs that represent a system of food production, a
system of harnessed subjectivity.} (Martha Rosler)

\VideoSubTitle{Choreography}

\VideoRemark{Video installation preview by Anke Sch\"afer, NL/South Africa, 13:07
min (loop), 2007.}

{\em Choreography} reflects on the notion \quote{Armed Response} as an
inner state of mind. The split screen projection shows the movements of
two women commuting to their work. On the one side, the German{}-South
African Edda Holl, who lives in the rich Northern suburbs of
Johannesburg. Her search for a safe journey is characterized by
electronic security systems, remote controls, panic buttons, her
constant cautiousness, the reassuring glances in the tinted car
windows. On the other side, you see the African{}-South African Gloria
Fumba, who lives in Soweto and whose security techniques are very
basic: clutching her handbag to her body, the way she cues for the bus,
avoiding to go home alone when it's dark. A classical continuity
editing, as seen fiction film, suggests at first a narrative storyline,
but is soon interrupted by moments of pause. These pauses represent the
desires of both women to break with the safety mechanism that motivates
their daily movements. 
\blank
\Url{http://www.livemovie.org}

\VideoSubTitle{Television}

\VideoRemark{Ximena Cuevas, Mexico, 1999, 2 min.}

\quotation{The vacuum cleaner becomes the device of the feminist \quote{liberation},
or the monster that devours us.} (Insite 2000 program, San Diego Museum
of Art)\par
}

\VideoTitle{Perform the script, write the score}
{\em Considers dance and performance as knowledge systems where movement and
data interact. With excerpts of performance documents, interviews and
(dance) films. But also the script, the code, as system of perversion,
as an explorative space for the circulation of bodies.}

\VideoLib{
\VideoSubTitle{William Forsythe's works}

Choreography can be understood as writing moving bodies into space, a
complex act of inscription, which is situated on the
borderline between creating and remembering, future and past. Movement
is prescribed and is passing at the same time. It can be inscribed into
the visceral body memory through constant repetition, but it is also
always undone: 

As Laurie Anderson says: 
\blank
\quotation{You're walking. And you
don't always realize it, but you're
always falling. With each step you fall forward slightly. And then
catch yourself from falling. Over and over, you're
falling. And then catching your self from falling.} (Quoted after
Gabriele Brandstetter, {\em ReMembering the Body})
\blank
William Forsythe, for instance, considers classical ballet as a
historical form of a knowledge system loaded with ideologies about
society, the self, the body, rather than a fixed set of rules, which
simply can be implemented. An arabesque is a platonic ideal for him, a
prescription, but it can't be danced: \quotation{There is no arabesque, there is
only everyone's arabesque.} His choreography is concerned with
remembering and forgetting: referencing classical ballet, creating a
geometrical alphabet, which expands the classical form, and searching
for the moment of forgetfulness, where new movement can arise. Over the
years, he and his company developed an understanding of dance as a
complex system of processing information with some analogies to
computer programming. 

\VideoSubTitle{Chance favours the prepared mind}

\VideoRemark{Educational dance film, produced by Vlaams Theaterinstituut, Ministerie
van Onderwijs dienst Media and Informatie, dir. Anne Quirynen, 1990, 25
min.}

{\em Chance favours the prepared mind} features discussions and
demonstrations by William Forsythe and four Frankfurt Ballet Dancers
about their understanding of movement and their working methods: \quotation{Dance
is like writing or drawing, some sort of inscription.} (William
Forsythe)

\VideoSubTitle{The way of the weed}

\VideoRemark{Experimental dance film featuring William Forsythe, Thomas McManus and
dancers of the Frankfurt Ballet, An{}-Marie Lambrechts, Peter Missotten
and Anne Quirynen, soundtrack: Peter Vermeersch, 1997, 83 min.}

In this experimental dance film, investigator Thomas is dropped in a
desert in 7079, not only to investigate the growth movements of the
plant life there, but also the life's work of the obscure scientist
William F. (William Forsythe), who has achieved numerous insights and
discoveries on the growth and movement of plants. This knowledge is
stored in the enormous data bank of an underground laboratory. It is
Thomas's task to hack into his computer and check the professor's
secret discoveries. His research leads him into the catacombs of a
complex building, where he finds people stored in cupboards in a
comatose state. They are loaded with professor F.'s knowledge
of vegetation. He puts the \quote{people{}-plants} into a large transparent
pool of water and notices that in the water the \quote{samples} come to life
again{\dots} A complex reflection on (body) memory, (digital) archives
and movement as repetition and interference.

\VideoSubTitle{Rehearsal Last Supper}

\VideoRemark{Video installation preview by Anke Sch\"afer, NL/South Africa, 16:40
min. (loop), 2007.}

The work {\em Rehearsal Last Supper} combines a kind of \quote{Three
Stooges} physical, slapstick{}-style comedy, but with far more serious
subject matters such as abuse, gender violence, and the general
breakdown of family relationships. It's a South
African and mixed couple re{}-enactment of a similar scene that Bruce
Nauman realized in the 70s with a white, middle{}-aged man and woman. 

The experience, the \quote{Gestalt} of the experienced violence, the
frustration and the unwillingly or even forced internalization are felt
to the core of the voice and the body. Humour can help to express the
suppressed and to use your pain as power.

Actors: Nat Ramabulana, Tarryn Lee, Megan Reeks, Raymond Ngomane (from
Wits University Drama department), Kekeletso Matlabe, Lebogang Inno,
Thabang Kwebu, Paul Noko (from Market Theatre Laboratory).
\blank
\Url{http://www.livemovie.org}

\VideoSubTitle{Nest Of Tens}

\VideoRemark{Miranda July, U.S.A., 1999, 27 min.}

{\em Nest Of Tens} is comprised of four alternating stories, which
reveal mundane yet personal methods of control. These systems are
derived from intuitive sources. Children and a retarded adult operate
control panels made out of paper, lists, monsters, and their own
bodies. 
\blank
\quotation{A young boy, home alone, performing a bizarre ritual with a baby; an
uneasy, aborted sexual flirtation between a teenage babysitter and an
older man; an airport lounge encounter between a businesswoman (played
by July) and a young girl. Linked by a lecturer enumerating phobias in
a quasi{}-academic seminar, these three perverse, unnerving scenarios
involving children and adults provide authentic glimpses into the
queasy strangeness that lies behind the everyday.} (New York Video
Festival, 2000)
\blank

\VideoSubTitle{In the field of players}

\VideoRemark{Jeanne Van Heeswijk \& Marten Winters, 2004, NL\crlf
Duration: 25.01.2004 {--} 31.01.2004\crlf
Location: TENT.Rotterdam\crlf
Participants: 106 through casting, 260 visitors of TENT.}

Together with artist Marten Winters, Van
Heeswijk developed a \quote{game:set}. In cooperation with graphic designer
Roger Teeuwen, they marked out a set of lines and fields on the ground.
Just like in a sporting venue, these lines had no meaning until used by
the players. The relationship between the players was revealed by the
rules of the game. 

Designer Arienne Boelens created special game cards that were handed out
during the festival by the performance artists Bliss. Both Bliss and
the cards turned up all over the festival, showing up at every hot spot
or special event. Through these game cards people were invited to
fulfil the various roles of the game {--} like \quote{Round Miss} (the girl
who walks around the ring holding up a numbered card at the start of
each round at boxing matches), \quote{40-plus male in (high) cultural position},
\quote{Teen girl with star ambitions}, \quote{Vital 65-plus}. But even \quote{Whisperer}, and
\quote{Audience} were specific roles.

\VideoSubTitle{Writing Desire}

\VideoRemark{Video essay by Ursula Biemann, CH, 2000, 25 min.}
\blank
{\em Writing Desire} is a video essay on the new dream screen of the
Internet, and its impact on the global circulation of
women's bodies from the \quote{Third World} to the {\quote{First
World}.} Although underage Philippine \quote{pen pals} and post{}-Soviet
mail{}-order brides have been part of the transnational exchange of sex
in the post{}-colonial and post{}-Cold War marketplace of desire before
the digital age, the Internet has accelerated these transactions. The
video provides the viewers with a thoughtful meditation on the obvious
political, economic and gender inequalities of these exchanges by
simulating the gaze of the Internet shopper looking for the imagined
docile, traditional, pre{}-feminist, but Web{}-savvy mate.
\blank
\Url{http://www.geobodies.org}

}
}

\Fra{\Title{Mouvements Mutuels Vid\'eoth\`eque}
\SubTitle{Des vid\'eos \`a feuilleter, une vid\'eoth\`eque\crlf et un regard \`a d\'eplacer.}
La com\'edienne Isabelle Bats, qui porte la vid\'eoth\`eque, invite \`a
regarder une s\'election de films en lien avec les
questions de V/J10. Telle une m\'emoire vivante, les disques et les
lecteurs DVDs de la vid\'eoth\`eque font partie de sa robe, con\c{c}ue
par le collectif d'artistes De Geuzen. Isabelle
incarne une interface accessible entre vous (celui{}-celle qui regarde)
et la vid\'eo. Cette interface humaine permet une relation mutuelle.
Regarder les films influence l'exp\'erience des autres
parties du programme, la situation et le contexte dans lequel vous
regardez les films jouent un r\^ole dans
l'exp\'erience et l'interpr\'etation
des vid\'eos. Un \'echange physique existe entre les images existantes,
l'interpr\'etation en temps r\'eels, les
exp\'eriences, et le contexte.

La vid\'eoth\`eque temporaire propose une collection
d'extraits de performances, de documents sur la danse,
d'art vid\'eo et de documentaires. Documents
r\'ef\'erents, documents r\'ef\'erences.

Cette s\'election est une r\'eflexion sur les relations complexes
qu'entretiennent les corps et les machines. Nous avons
cherch\'e les gestes signifiants, exp\'erimentaux, perturbateurs ou
subversifs dans la boucle de r\'etroaction entre les technologies, les
outils, les donn\'ees et les corps. Nous vous pr\'esentons le
mat\'eriau historique et contemporain d'une archive
temporaire.

\godown[1em]

\VideoTitle{Les Temps Modernes ou la Cha\^ine de Production}
{\em Une r\'eflexion sur le corps dans les environnements de travail, depuis
le travail manuel pr\'eindustriel, avec des outils analogues,
jusqu'au travail \`a la cha\^ine et aux dispositifs de
surveillance postmodernes.} 

\VideoLib{
\VideoSubTitle{24 portraits}
\VideoRemark{Extrait d'une s\'erie de portraits documentaires\crlf
Alain Cavalier, FR, 1988{}-1991.}

{\em 24 portraits} est une s\'erie de courts documentaires qui
rendent hommage au travail manuel des femmes. Ces portraits doux et
intrigants de 24 femmes qui pratiquent des m\'etiers diff\'erents,
r\'ev\`elent l'intimit\'e des corps et des outils de
travail.

\VideoSubTitle{Humain, trop humain}
\VideoRemark{Extraits d'un documentaire de Louis Malle, FR, 1972.}

Un documentaire, tourn\'e \`a l'usine automobile
Citro\"en \`a Rennes et au salon de l'auto de Paris en
1972, qui montre les routines monotones du travail \`a la cha\^ine et
la forte interaction des corps et des machines.

\VideoSubTitle{Performing the Border}
\VideoRemark{Un essai vid\'eo d'Ursula Biemann, CH, 1999, 45 min.}
\blank
\quotation{{\em Performing the Border} est un essai vid\'eo situ\'e \`a la
fronti\`ere entre les \'Etats{}-Unis et le Mexique, \`a Ciudad Juarez,
juste \`a c\^ot\'e d'El Paso, Texas, o\`u les
industries am\'ericaines d\'elocalisent l'assemblage
des appareils \'electroniques et digitaux. La vid\'eo probl\'ematise la
sexualisation de la r\'egion frontali\`ere en montrant les liens entre
la division du travail, la prostitution, l'expression
du d\'esir f\'eminin dans l'industrie du
divertissement et la violence sexuelle qui est exerc\'ee dans la
sph\`ere publique. La fronti\`ere devient une m\'etaphore pour la
marginalisation et la construction artificielle, et renouvel\'ee, des
fronti\`eres subjectives au moment o\`u les distinctions entre corps et
machine, entre reproduction et production, entre f\'eminin et masculin
sont devenues plus fluides que jamais.} (Ursula Biemann)
\blank
\Url{http://www.geobodies.org}

\VideoSubTitle{Maquilapolis (city of factories)}
\VideoRemark{Un film de Vicky Funari et Sergio De La Torre, Mexique/U.S.A., 2006, 68 min.}

Carmen travaille la nuit dans une des {\em maquiladoras} de
Tijuana, ces usines multinationales qui sont venues \`a Mexico,
attir\'ees par sa main{}-d'{\oe}uvre bon march\'ee.
Apr\`es avoir fait des composants de t\'el\'evisions toute la nuit,
Carmen revient chez elle, dans un abri fait de portes de garage
recycl\'ees, dans un quartier sans \'egout ni \'electricit\'e. Elle
souffre d'un probl\`eme aux reins et
d'intoxication au plomb apr\`es des ann\'ees
d'exposition \`a des substances toxiques. Elle gagne
six dollars par jour. Mais Carmen n'est pas une
victime. Elle est une jeune femme dynamique, occup\'ee \`a se frayer un
chemin dans la vie, pour elle et pour ses enfants.

Comme Carmen, un million de travailleurs des {\em maquilladoras}
produisent des t\'el\'evisions, des c\^ables \'electriques, des jouets,
des v\^etements, des batteries et des tubes n\'eons, ils tissent
l'\'etoffe vivante de nos pays consum\'eristes. Ils
affrontent aussi les violations du droit du travail, un environnement
d\'evast\'e et le chaos urbain {--} c'est la vie \`a
la fronti\`ere de l'\'economie globale. Dans
{\em Maquilapolis}, Carmen et sa coll\`egue Lourdes se hissent
au{}-del\`a de la lutte pour la survie et se livrent \`a un combat pour
le changement: Carmen s'oppose \`a un important
fabricant de t\'el\'evisions parce qu'il viole ses
droits de travailleuse. Lourdes fait pression sur le gouvernement pour
assainir un d\'ep\^ot toxique laiss\'e derri\`ere elle par une
compagnie sur le d\'epart.

Pendant qu'elles luttent pour le changement, le monde
change lui aussi: une crise \'economique globale et la disponibilit\'e
d'une main{}-d'{\oe}uvre encore moins
ch\`ere en Chine commencent \`a \'eloigner les entreprises de Tijuana,
abandonnant Carmen, Lourdes et leurs coll\`egues \`a un avenir
incertain.

Une coproduction de l'Independent Television Service
(ITVS). Un projet de Creative Capital.
\blank
\Url{http://www.maquilapolis.com}\par
}

\VideoTitle{L'invention du quotidien} 
{\em Le quotidien comme lieu de performance et m\'ecanismes, de l'Am\'erique des ann\'ees 70 \`a
l'Afrique du Sud contemporaine.}

\VideoLib{
\VideoSubTitle{Saute ma ville}
\VideoRemark{Chantal Akerman, B, 1968, 11 min.}

Une jeune fille rentre joyeuse chez elle. Elle
s'enferme dans sa cuisine et d\'etraque le monde
m\'enager. Chantal Akerman, avec son premier film, explore une forme
d'\^etre{}-l\`a, \'eparse, o\`u le rapport au monde humain, r\'egul\'e, explos\'e litt\'eralement. Abolition de soi, explosion de
soi.

\VideoSubTitle{Semiotics of the Kitchen}
\VideoRemark{Vid\'eo de Martha Rosler, U.S.A., 1975, 05:30 min.}

Parodie d'une \'emission culinaire, {\em Semiotics
of The Kitchen} remplace l'usage domestique des
ustensiles par un vocabulaire de rage et de frustration. Film\'ee en
plan fixe dans une cuisine et v\^etue d'un tablier,
une femme, Martha Rosler, a devant elle une s\'erie
d'ustensiles qu'elle saisit
l'un apr\`es l'autre, les nomme et
mime leur utilisation de fa\c{c}on didactique au d\'ebut, puis avec des
mouvements qui sortent du quotidien. Dans une grammaire sonore et
gestuelle ironique, la femme et ses instruments transgresse le
syst\`eme quotidien des signifiants culinaires, les symboles bien
ancr\'es de l'industrie domestique et de la production
alimentaire explosent en col\`ere et violence. Dans cet alphabet
d'instruments culinaires, Martha Rosler affirme \quotation{Lorsque la femme parle,
elle nomme sa propre oppression}.

\VideoSubTitle{Choreography}
\VideoRemark{Vid\'eo de Anke Sch\"afer, NL/Afrique du Sud, 13:07 min (loop), 2007.}

Une r\'eflexion sur la notion de \quote{r\'eponse arm\'ee} comme \'etat
d'esprit. Le mouvement de deux femmes en route vers
leur travail, peut \^etre suivi simultan\'ement, d'un
c\^ot\'e Edda Holl, Sud{}-Africaine d'origine
allemande et de l'autre Gloria Fumba, Sud{}-Africaine
d'origine africaine. Dans le cas
d'Edda, qui vit dans la banlieue riche au nord de
Johannesbourg, la poursuite d'un voyage s\^ur est
repr\'esent\'ee par le syst\`eme \'electronique de s\'ecurit\'e, les
commandes \`a distance, le \quote{panic buttons}, son attention
permanente, son regard scrutateur dans le r\'etroviseur de la voiture.
De l'autre c\^ot\'e, Gloria, vivant \`a Soweto, dont
les techniques de s\'ecurit\'e sont tr\`es limit\'ees et
s'expriment seulement par la mani\`ere dont elle porte
son sac, tout pr\`es d'elle, dont elle regarde les
autres dans les minibus, dont elle essaie de ne jamais rester seule et
\'evite de rentrer \`a la maison lorsqu'il fait noir.
Un montage classique continu comme nous le connaissons des films de
fictions mais qui bient\^ot est interrompu par des moments de pause.
Ces pauses accueillent le d\'esir des deux femmes de rompre avec les
m\'ecanismes de s\'ecurit\'e de leurs mouvements quotidiens.
\blank
\Url{http://www.livemovie.org}

\VideoSubTitle{Television}
\VideoRemark{Ximena Cuevas, Mexico, 1999, 2 min. }

\quotation{L'aspirateur, symbole de la \quote{lib\'eration} de la
femme ou monstre pr\^et \`a nous d\'evorer.} (Insite 2000 program, San Diego Museum of Art)\par
}

\VideoTitle{Performe le script, \'ecris la partition}
{\em Envisager la danse et la performance comme des syst\`emes de connaissance
dans lesquels le mouvement et les donn\'ees interagissent, d'apr\`es des
extraits de performance, des interviews et des films (de danse). Mais
aussi du script, du code comme syst\`eme de perversion ou comme espace
d'exploration du quotidien.}

\VideoLib{
\VideoSubTitle{Du travail de\crlf
William Forsythe}
La chor\'egraphie peut \^etre comprise comme
l'\'ecriture de corps en mouvement dans
l'espace, un acte complexe
d'inscription qui est situ\'e \`a la fronti\`ere entre
cr\'eer et se rappeler le futur et le pass\'e. Le mouvement est
pre\-scrit et passe en m\^eme temps. Il peut \^etre inscrit dans la
m\'emoire visc\'erale du corps par la r\'ep\'etition constante, mais il
est aussi toujours d\'efait.

Comme le dit Laurie Anderson: 
\blank
\quotation{Vous marchez. Et vous ne le r\'ealisez
pas toujours, mais vous tombez. A chaque pas vous tombez l\'eg\`erement
en avant. Et puis vous vous rattrapez et vous vous emp\^echez de
tomber. Encore et encore vous tombez, et vous vous emp\^echez de
tomber.} (D'apr\`es Gabriele Brandstetter,
{\em ReMembering the Body})
\blank
William Forsythe, par exemple, consid\`ere le ballet classique comme une
forme de syst\`eme de connaissance historique charg\'e
d'id\'eologies sur la soci\'et\'e, le soi, le corps,
plut\^ot que comme un syst\`eme fixe de r\`egles qui peut \^etre
simplement impl\'ement\'e. Une arabesque est pour lui un id\'eal
platonique, une prescription, mais cela ne peut \^etre dans\'e: \quotation{Il n'y
a pas d'arabesque, il n'y a que
l'arabesque de chacun.} Sa chor\'egraphie est
concern\'ee par le souvenir et l'oubli: se
r\'ef\'erant au ballet classique, cr\'eant un alphabet g\'eom\'etrique
qui \'etire la forme classique, et poursuivant le moment
d'oubli, lorsqu'un nouveau mouvement peut surgir.
Pendant des ann\'ees, sa compagnie et lui ont d\'evelopp\'e une
compr\'ehension de la danse comme un syst\`eme complexe de processus de
l'information qui peut avoir des analogies avec la
programmation informatique.

\VideoSubTitle{Chance favors the prepared mind}
\VideoRemark{Film \'educatif de danse, produit par le Vlaams Theaterinstituut,
Ministerie van Ouderwijs dienst Media en Informatie, r\'eal. Anne
Quirynen, 1990, 25 min.}
 Conversations et d\'emonstrations par William Forsythe et par quatre
danseurs du Ballet de Francfort sur la compr\'ehension du mouvement et
sur leurs m\'ethodes de travail. \quotation{La danse est comme
l'\'ecrit ou le dessin, une sorte
d'inscription.} (William Forsythe)

\VideoSubTitle{The way of the weed}
\VideoRemark{Film exp\'erimental de danse, avec William Forsythe, Thomas McManus et
les danseurs du Ballet de Francfort, r\'eal. An{}-Marie Lambrechts,
Peter Missotten et Anne Quirynen, bande son Peter Vermeersch, 1997, 83
min.}

Dans ce film de danse exp\'erimental, le chercheur, Thomas, est
l\^ach\'e dans un d\'esert en 7079, pas seulement pour y \'etudier les
mouvements de croissance de la vie v\'eg\'etale, mais aussi les travaux
de toute une vie de l'obscur scientifique William F.
(William Forsythe), qui a collect\'e un champ
d'\'etudes et de d\'ecouvertes de la croissance et du
mouvement des plantes. Cette connaissance est stock\'ee dans une
\'enorme base de donn\'ees d'un laboratoire
souterrain. La t\^ache de Thomas est donc de
s'introduire dans l'ordinateur et de
jeter un {\oe}il sur les d\'ecouvertes secr\`etes du professeur. Sa
recherche l'am\`ene aux catacombes d'une construction
complexe, o\`u il trouve des personnes dans le coma, rang\'ees dans des
armoires. Elles sont charg\'ees des connaissances du professeur F. sur
la v\'eg\'etation. Thomas place alors les \quote{personnes{}-plantes} dans
une piscine transparente o\`u elles reprennent vie...

Une r\'eflexion complexe sur la m\'emoire corporelle, les archives
(digitales) et le mouvement comme r\'ep\'etition et interf\'erence.

\VideoSubTitle{Rehearsal Last Supper}
\VideoRemark{Anke Sch\"afer, NL/Afrique du Sud, 16:40 min (loop), 2007.}

Ce travail combine une sorte de \quote{Trois Faire-valoirs} physiques, la grosse farce, la comédie de style, mais avec des sujets beaucoup plus sérieux comme l'abus, la violence de genre et l'\'echec dans les relations familiales.
Un couple
Sud{}-Africain mixte re{}-joue une sc\`ene similaire \`a celle mise en
sc\`ene par Bruce Nauman dans les ann\'ees septante avec un homme et
une femme blanches d'\^age moyen. La voix et le corps
enregistrent dans chaque cellule l'exp\'erience, la
\quote{Gestalt} de la violence v\'ecue, la frustration et
l'int\'eriorisation volontaire ou non.
L'humour peut aider \`a exprimer
l'oppression et \`a utiliser la douleur comme pouvoir.

Acteurs{}-rices: Nat Ramabulana, Tarryn Lee, Megan Reeks, Raymond
Ngomane (Wits University Drama department), Kekeletso Matlabe, Lebogang
Inno, Thabang Kwebu, Paul Noko (Market Theatre Laboratory) 
\blank
\Url{http://www.livemovie.org}
 \VideoSubTitle{Nest Of Tens}
\VideoRemark{Miranda July, U.S.A., 27 min, 1999.}

Quatre histoires altern\'ees qui r\'ev\`elent des mod\`eles ordinaires
et personnels de contr\^ole. Ces syst\`emes sont tir\'es de sources
intuitives. Des enfants et un adulte attard\'es produisent des tableaux
de commandes faits de papier, de liste,
d'\'echantillons et de leur propre corps.
\blank
\quotation{Un jeune gar\c{c}on seul \`a la maison pratique un rituel \'etrange
avec un b\'eb\'e; une relation sexuelle d\'erangeante avort\'ee entre
une baby{}-sitter et un homme plus \^ag\'e; une rencontre dans une
salle d'attente d'a\'eroport entre une
femme d'affaire (jou\'ee par
l'auteur) et une jeune fille. Reli\'es entre eux par
un conf\'erencier \'enum\'erant des phobies dans un s\'eminaire quasi
acad\'emique, ces trois sc\'enarios pervers et d\'econcertants
impliquant des enfants et des adultes fournissent des aper\c{c}us
authentiques sur l'\'etranget\'e d\'erangeante tapie
derri\`ere le quotidien.} (New York vid\'eo Festival, 2000)
\blank
\VideoSubTitle{In the field of players}
\VideoRemark{Jeanne Van Heeswijk \& Martin Winters, 2004, NL.
25.01.2004 {}- 31.01.2004. TENT. Rotterdam. 
Participant{}-e{}-s: 106 sur casting, 260 visiteurs de TENT.}

Jeanne Van Heeswijk en collaboration avec
l'artiste Martin Winters, d\'evelopp\`erent un
\quote{game:set} (\quote{sc\`ene de jeu}). En coop\'eration avec le graphiste Roger
Teeuwen, ils marqu\`erent une s\'erie de lignes et de champs sur le
sol. Tout comme sur un terrain de sport, ces lignes
n'ont aucune signification jusqu'\`a
ce qu'elles soient utilis\'ees par les
joueurs{}-euses. La relation entre les joueurs{}-euses est
r\'ev\'el\'ee par les r\`egles du jeu.

La graphiste Arienne Boelens a cr\'e\'e un jeu de cartes sp\'ecial qui a
\'et\'e distribu\'e pendant le festival par les artistes de performance
Bliss. Et les cartes et les artistes tournaient dans tout le festival
et pouvaient \^etre trouv\'e dans tous les lieux essentiels. A travers ce jeu
de cartes, les gens \'etaient invit\'es \`a remplir les diff\'erents
r\^oles dans le jeu {--} tels que \quote{Round Miss} (la fille qui marche
autour du ring montrant une carte num\'erot\'ee au d\'ebut de chaque
round d'un match de boxe), \quote{Homme 40-plus dans une position culturelle importante},
\quote{Adolescente avec de l'ambition},
\quote{Vital 65-plus}. Mais il y avait aussi le \quote{Souffleur}, et m\^eme le
\quote{Public} avait un r\^ole sp\'ecifique.

\VideoSubTitle{Writing Desire}
\VideoRemark{Essai vid\'eo par Ursula Biemann, 2000, 25 min. }
\blank
\quotation{Sur le nouvel \'ecran du r\^eve que repr\'esente Internet et son impact sur la circulation globale du corps des femmes du \quote{tiers{}-monde} au {\quote{premier\\monde}.} Bien avant l'\`ere digitale, les
trop jeunes correspondantes Philippines et les \'epouses de vente par correspondance
post{}-Sovi\'etique ont fait partie de l'\'echange
transnational du sexe sur le march\'e du d\'esir post{}-colonial et
post{}-Guerre Froide. A pr\'esent, Internet a acc\'el\'er\'e ces
transactions. Cette vid\'eo fournit aux spectateurs{}-trices une
m\'editation sur les \'evidentes in\'egalit\'es politiques,
\'economiques, de genre de ces \'echanges, en simulant le regard du
consommateur sur Internet cherchant la compagne fantasm\'ee, docile,
traditionnelle, pr\'e{}-f\'eministe, mais connaisseuse du Web.} (Ursula
Biemann)
\blank
\Url{http://www.geobodies.org}
}
}

\Ned{
\Title{Wederkerige Bewegingen Videotheek}
\SubTitle{Videos om door te bladeren, om je blik van positie te laten veranderen}

Performer Isabelle Bats, de draagster van de videotheek, presenteert u
een selectie van films die gerelateerd zijn aan de thema's die in V/J10
aan bod komen. De videotheek is een levend geheugen; de schijven en
spelers zijn als \quote{embedded} media ondergebracht in een jurk
ontworpen door kunstenaarscollectief De Geuzen. Isabelle belichaamt een
toegankelijke interface tussen u (het publiek) en de
video's. De mens als interface staat een wederzijdse
relatie toe: zoals het bekijken van de films uw ervaring van andere
programmaonderdelen zal be\"invloeden, verandert de situatie in La
Bellone waaronder u de films bekijkt op haar beurt de interpretatie van
de film{}- en videowerken. Zo ontstaat een fysieke wisselwerking tussen
bestaande inhoud, context, ervaring en levende interpretatie.

Het tijdelijke V/J10{}-videoarchief bevat een keuze uit performances,
dansregistraties, videokunst en (documentaire) film, en reflecteert op
de complexe relatie tussen lichamen en machines. We verzamelden
historisch en hedendaags materiaal, waarin gebaren en bewegingen
richting geven, testen, verstoren en subverteren, en zich in de
eindeloze feedbackverbinding tussen technologie, gereedschap, data en
lichamen, bevinden.

\godown[1em]

% \VideoTitle{Moderne Tijden of de Lopende Band}
% {\em Bekijkt het lichaam in de context van technologische werkomgevingen,
% gaande van pre\"industrieel handwerk en analoge gereedschappen, tot de
% fabriek, de lopende band en hedendaagse, postmoderne, door surveillance
% gecontroleerde werkconfiguraties.}
% 
% \VideoLib{
% \VideoSubTitle{24 Portraits} 
% \VideoRemark{Een keuze uit een serie korte documentaire portretten van Alain
% Cavalier, 1988{}-1991.}
% 
% {\em 24 Portraits} brengt een ode aan het handwerk van vrouwen. De
% intrigerende en sensitieve portretten van 24 vrouwen die in
% verschillende beroepen werkzaam zijn, tonen de intimiteit die tussen
% lichamen en hun werktuigen kan bestaan.
% 
% \VideoSubTitle{Humain, trop humain} 
% \VideoRemark{Een fragment uit een documentaire door Louis Malle, 1975.}
% 
% Een documentaire gefilmd in de Citro\"en{}-autofabriek in Rennes en
% tijdens de Parijse autotentoonstelling van 1972. De film toont de
% monotone dagelijkse routines van het werken aan de lopende band, de
% hechte interactie tussen lichamen en machines.
% 
% \VideoSubTitle{Performing the Border}
% \VideoRemark{Video{}-essay van Ursula Biemann, CH, 1999, 45 min. }
% \blank
% \quotation{{\em Performing the Border} is een video{}-essay met Ciudad Juarez,
% de grensstad tussen Mexico en de V.S., als setting. Het is een plaats
% recht tegenover het Texaanse El Paso gelegen, waar de
% Noord{}-Amerikaanse industrie haar elektronisch en digitaal materiaal
% assembleert. De video bediscussieert de seksualisering van dit
% grensgebied door arbeidsverdeling, prostitutie, de expressie van
% vrouwelijke verlangens in de entertainmentindustrie, en seksueel geweld
% in de publieke sfeer. De grens wordt aldus gepresenteerd als een
% metafoor voor marginalisatie en het artificieel in stand houden van
% subjectieve grenzen, op het moment dat de distincties tussen lichaam en
% machine, tussen reproductie en productie, tussen mannelijk en
% vrouwelijk, onderhand vager zijn dan ooit.} (Ursula Biemann)
% \blank
% \Url{http://www.geobodies.org}
% 
% \VideoSubTitle{Maquilapolis}
% \VideoRemark{Een film van Vicky Funari en Sergio De La Torre, Mexico/VS, 2006, 68
% min.}
% 
% De globalisering houdt ook huis in Tijuana, waar de Amerikaanse
% bedrijven naartoe gelokt worden door de lagere loonkosten. In dit
% grensstadje tussen Mexico en de Verenigde Staten, waar de
% \quote{American dream} van talrijke Latino's een stille dood sterft,
% mobiliseren de werkneemsters zich om hun collega's te informeren over
% hun rechten en om deze rechten ook af te dwingen.
% \blank
% \Url{http://www.maquilapolis.com}\par
% }

\VideoTitle{De praktijk van het dagelijks leven}
{\em Brengt een aantal performancedocumenten bij elkaar, van de jaren
zeventig tot nu, waarin dagelijkse ontmoetingen tussen lichamen en
gereedschappen centraal staan.}

\page

\VideoLib{
\VideoSubTitle{Saute ma ville}
\VideoRemark{Chantal Akerman, B, 1968, 11 min.}

Een jong meisje komt vrolijk thuis. Ze sluit zich op in haar keuken en
brengt de huishoudelijke wereld in de war. In haar eerste film verkent
Chantal Akerman een verwarde manier van zijn, oftewel een bestaan
waarin de relatie met de reguliere menselijke omgeving letterlijk is
ontploft. De opheffing van het zelf als een explosie van het zelf.

\VideoSubTitle{Semiotics of the Kitchen}
\VideoRemark{Video door Martha Rosler, VS, 1975, 05:30 min.}

{\em Semiotics of the Kitchen} neemt de vorm aan van een
parodi\"erende kookdemonstratie die, \quotation{De huiselijke betekenis van
gereedschap vervangt door een lexicon van woede en frustratie}
(Rosler). In dit performatieve werk focust een statische camera op een
vrouw in een keuken. Op het werkblad voor haar ligt een verscheidenheid
aan keukengereedschappen die ze \'e\'en voor \'e\'en oppakt, benoemt en
demonstreert op een wijze die zich verwijdert van het normale gebruik
tijdens het koken. In een ironische \quote{grammaturgie} van geluid en
beweging, omarmt en verandert de vrouw de veilige dagdagelijkse
semiotiek van de keuken. De overbekende uiting van een industrie van
huiselijkheid en voedselproductie ontaardt nu in agressie en geweld. In
relatie tot dit semiotische keukenalfabet zegt Rosler: \quotation{Als de vrouw
spreekt, benoemt ze haar eigen onderdrukking.}
\blank
\quotation{Ik was begaan met de notie van hoe taal het subject \quote{uitspreekt} en
articuleert (\quote{language speaking the subject}), en met de
transformatie van de vrouw zelf in een teken in een systeem van tekens,
dat een systeem van voedselproductie representeert, een systeem van een
in een keurslijf geduwde subjectiviteit ter controle en gebruik.} (Martha Rosler)

\VideoSubTitle{Television}
\VideoRemark{Ximena Cuevas, Mexico, 1999, 2 min.}

\quotation{De stofzuiger transformeert in een hulpstuk ter feministische
\quote{bevrijding}, of in het monster dat ons verslindt.} (Insite 2000 program, San Diego Museum of Art)

\VideoSubTitle{Choreography}
\VideoRemark{Preview van een video{}-installatie door Anke Sch\"afer,
NL/Zuid{}-Afrika, 13:07 min (loop), 2007.}

{\em Choreography} reflecteert op het begrip {\em {\quote{Armed Response}}} als een innerlijke staat van zijn. In de
splitscreenprojectie kunnen de bewegingen gevolgd worden van twee
vrouwen die naar hun werk gaan: aan de ene kant de
Duits{}-Zuid{}-Afrikaanse Edda Holl, aan de andere kant de
Afrikaans{}-Zuid{}-Afrikaanse Gloria Fumba.

Edda woont in de rijke noordelijke buitenwijken van Johannesburg. Haar
zoektocht naar een veilige route wordt gekenmerkt door elektronische
veiligheidssystemen, afstandbedieningen, \quote{panic buttons}, haar
voortdurende waakzaamheid, de verzekerende blikken in spiegelende
autoramen. Aan de andere kant zien we Gloria die in Soweto woont, en
wiens veiligheidstechnieken basaal zijn en voornamelijk bestaan uit:
het dicht tegen haar lichaam aandrukken van haar handtas, de manier
waarop ze zich in de rij voor de bus positioneert, het vermijden van
alleen naar huis te gaan wanneer het donker is. Een klassieke, op
continu\"iteit gebaseerde montage zoals we die kennen van de
fictiefilm, suggereert in eerste instantie een narratieve verhaallijn,
maar wordt al snel onderbroken door pauzemomenten. Deze pauzes
vertolken de verlangens van beide vrouwen om te breken met het
veiligheidsmechanisme dat hun dagelijkse bewegingen motiveert.
\blank
\Url{http://www.livemovie.org}\par
}

\VideoTitle{Voer het script uit, schrijf het programma}
{\em Benadert dans en performance als kennissystemen waarin beweging en data
op elkaar reageren. Met registraties van performances, interviews en
dansfilms. Maar ook over het script, de code, als een systeem van
perversies, als een exploratieve ruimte voor het dagdagelijkse en de
circulatie van lichamen.}

\VideoLib{
\VideoSubTitle{Werken van William Forsythe}
Choreografie kan worden begrepen als het ruimtelijk schrijven van/met
bewegende lichamen, een complexe daad van inscriptie die zich situeert
op de grens tussen creatie en herinnering, toekomst en verleden.
Beweging wordt voorgeschreven en speelt zich tegelijkertijd ook af. Ze
wordt in het lichaam gegrift door constante repetitie, en ze wordt
altijd gelijktijdig opgeslagen en ongedaan gemaakt. Zoals Laurie
Anderson zegt: 
\blank
\quotation{Je bent aan het wandelen. En je realiseert het je niet
altijd, maar je bent eigenlijk voortdurend aan het vallen. Met elke
stap val je lichtelijk naar voor. En dan voorkom je dat je valt.
Opnieuw en opnieuw, ben je aan het vallen. En dan aan het voorkomen dat
je valt.} (citaat: Gabriele Brandstetter, {\em ReMembering the
Body})
\blank
William Forsythe bijvoorbeeld ziet het klassieke ballet eerder als een
historische vorm van kennissysteem beladen met ideologie\"en over de
maatschappij, het zelf en het lichaam, dan als een vaststaande set
toepasbare regels. Voor hem is een arabesk een platonisch ideaal, een
voorschrift dat niet gedanst kan worden: \quotation{Er bestaat geen arabesk, er
bestaat enkel ieders eigen arabesk.} 

Zijn choreografie onderzoekt herinneren en vergeten: met referenties aan
klassiek ballet, met het cre\"eren van een geometrisch alfabet als
uitbreiding op de klassieke vormen, maar ook een choreografie die zoekt
naar momenten van vergetelheid, waarop nieuwe bewegingen kunnen
ontstaan. In de loop van de jaren ontwikkelde hij met zijn gezelschap
een opvatting van dans als een complex systeem van informatieverwerking
met analogie\"en naar computerprogrammering.

\page

\VideoSubTitle{Chance favors the prepared mind}
\VideoRemark{Een educatieve dansfilm geproduceerd door het Vlaams Theaterinstituut en
het Ministerie van Ouderwijs, dienst Media en Informatie, regie: Anne
Quirynen, 1990, 25 min.}

{\em Chance favours the prepared mind} toont discussies en
demonstraties door William Forsythe en vier dansers van het Frankfurter
Ballet waarin ze hun begrip van beweging en hun werkmethoden
toelichten: \quotation{Dans is zoals schrijven of tekenen, een soort van
inscriptie.} (William Forsythe)

\VideoSubTitle{The way of the weed}
\VideoRemark{Experimentele dansfilm met o.a. William Forsythe, Thomas McManus en de
dansers van het Frankfurter Ballet, realisatie: An{}-Marie Lambrechts,
Peter Missotten en Anne Quirynen, soundtrack: Peter Vermeersch, 1997,
83 min.}

In deze experimentele dansfilm wordt de onderzoeker Thomas in 7079 in
een woestijn gedropt. Hij bestudeert er de groeiende bewegingen van de
plaatselijke flora, en ook de obscure wetenschapper William F. (William
Forsythe). Deze laatste sprokkelde inzichten en ontdekkingen bijeen op
het gebied van de groei en beweging van planten. Die kennis ligt
opgeslagen in de enorme databank van een ondergronds laboratorium. Het
is Thomas' taak om zijn computer te hacken en de geheime ontdekkingen
van de wetenschapper te doorzoeken. 

Zijn onderzoek leidt hem naar een catacombe van een ingewikkeld gebouw,
waar hij comateuze mensen opgeborgen in ladekasten aantreft. Ze zijn
geladen met de kennis van vegetatie van de professor. Hij stopt deze
\quote{mens{}-planten} in een grote transparante waterbak, waardoor de
\quote{samples} weer tot leven komen. Een complexe reflectie op lichamelijk
geheugen, digitale archieven en beweging als repetitie en
interferentie.

\VideoSubTitle{Rehearsal Last Supper}
\VideoRemark{Preview van een video{}-installatie door Anke Sch\"afer,
NL/Zuid{}-Afrika, 16:40 min (loop).}

Het werk {\em Rehearsal Last Supper} combineert een fysieke komedie
in de stijl van een slapstick, met serieuze onderwerpen als misbruik,
gendergeweld en een algehele ondermijning van familierelaties. Het werk
betreft een {\em re{}-enactment} door Zuid{}-Afrikaanse gemengde
koppels van een zelfde sc\`ene die Bruce Nauman in de jaren '70
uitvoerde met een blanke man en vrouw van middelbare leeftijd.

De ervaring, de \quote{Gestalt} van het ondergane geweld, van de frustratie en
ongewilde of zelfs oppressieve internalisatie, worden tot in elke vezel
van het lichaam en de stem ingeschreven. Humor kan helpen om het
onderdrukte te uiten, en om je pijn in je voordeel te gebruiken.

Acteurs: Nat Ramabulana, Tarryn Lee, Megan Reeks, Raymond Ngomane (Wits
University Drama department), Kekeletso Matlabe, Lebogang Inno, Thabang
Kwebu, Paul Noko (Market Theatre Laboratory) 
\blank
\Url{http://www.livemovie.org}

\VideoSubTitle{Nest Of Tens}
\VideoRemark{Miranda July, VS, 1999, 27 min. }

{\em Nest of Tens} is samengesteld uit vier afwisselende verhalen die
alledaagse, maar persoonlijke controlemethodes tonen. Deze systemen
hebben een intu\"itieve oorsprong. Kinderen en een geestelijk
gehandicapte volwassene bedienen papieren controlepanelen, lijsten,
monsters en hun eigen lichamen.

\quotation{Een jongeman, alleen thuis, die een bizar ritueel met een baby
uitvoert; een ongemakkelijke, onderbroken flirt tussen een puberende
babysitter en een oudere man; een ontmoeting in de wachtruimte van een
luchthaven tussen een zakenvrouw (gespeeld door July) en een jong
meisje. Dit alles wordt aan elkaar gerijgd door een spreker die tijdens
een quasi{}-academisch seminarie fobie\"en opsomt. Deze drie perverse,
perturberende scenario's, met zowel kinderen als volwassenen als
protagonisten, werpen een authentieke glimp op de beklemmende
vreemdheid die achter het dagdagelijkse verscholen ligt.} (New York
Video Festival, 2000)

\VideoSubTitle{In the field of players}
\VideoRemark{Jeanne Van Heeswijk \& Martin Winters, 2004, NL. Duur: 25.01.2004 {}-
31.01.2004. Plaats: TENT. Rotterdam. Deelnemers: 106 door casting,
260 bezoekers van TENT.}

Martin Winters en Jeanne van Heeswijk ontwikkelden samen een spel dat in
lijnen en velden op een speelvloer werd uitgetekend. De relaties tussen
de spelers werden onthuld door de regels van het spel. Er werden
spelkaarten gemaakt die verspreid werden door de
performancekunstenaarsgroep Bliss. Door middel van de kaarten werden
mensen uitgenodigd om de verschillende rollen in het spel te komen
helpen invullen. Voorbeelden van rollen zijn: \quote{Ronde Miss} (het meisje
dat rond de ring loopt en dat een genummerde kaart ophoudt bij de start
van de rondes van een boxwedstrijd), \quote{40 plus man in (hoge) culturele
positie}, \quote{Tienermeisje met sterallures}, \quote{Vitale 65 plusser}. Ook het aanwezige
publiek werd door de Bliss{}-kunstenaressen een rol toebedeeld. 

I.o.v. International FilmFestival Rotterdam, I.s.m.: Arienne Boelens,
Roger Teeuwen, Bliss.

\VideoSubTitle{Writing Desire}
\VideoRemark{Video{}-essay door Ursula Biemann, CH, 2000, 25 min.}

{\em Writing Desire} is een video{}-essay over hoe het internet de
globale circulatie van vrouwenlichamen van de \quote{derde} naar de \quote{eerste
wereld} be\"invloedt. Hoewel minderjarige Filipijnse \quote{penvriendinnen}
en post{}-Sovjet \quote{mailorderbruiden} ook voor de aanvang van het
digitale tijdperk al deel uitmaakten van de transnationale uitwisseling
van seks op de postkoloniale en post{}-Koude Oorlog marktplaats voor
verlangen, heeft het internet deze transacties versneld. Deze video
stelt de kijker een overdachte meditatie voor over de evidente
politieke, economische en genderongelijkheden van deze uitwisselingen,
door de blik van de \quote{internetshopper} te simuleren, naarstig op zoek
naar gedwee\"e traditionele prefeministische, maar webbewuste,
partners. 
\blank
\Url{http://www.geobodies.org}
}
}
