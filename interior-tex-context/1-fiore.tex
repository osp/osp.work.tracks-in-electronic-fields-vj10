\PlaceFramedImage{cookiesensus.jpg}{Cookies found on Washingtonpost.com} 
\PlaceFramedImage{tacoda.jpg}{Cookies sent by Tacodo.net} 

\AuthorStyle{Andrea Fiore}

\licenseStyle{Creative Commons Attribution{}-NonCommercial{}-ShareAlike}

\Eng{\Title{Cookiecensus}

Although still largely perceived as a private activity, web surfing
leaves persistent trails. While users browse and interact through the
web, sites watch them read, write, chat and buy. Even on the basis of a
few basic web publishing experiences one can conclude that most web
servers record \quote{by default} their entire clickstream in persistent
\quote{log} files.

\quote{Web cookies} are sort of digital labels sent by websites to web
browsers in order to assign them a unique identity and automatically
recognize their users over several visits. Today, this technology,
which was introduced with the first version of the Netscape browser in
1994, constitutes the de facto standard upon which a wide range of
interactive functionalities are built that were not conceived by the
early web protocol design. Think, for example, of user accounts and
authentications, personalized content and layouts, e{}-commerce and
shopping charts.

While it has undeniably contributed to the development and the social
spread of the new medium, web cookie technology is still to be
considered as problematic. Especially the so{}-called \quote{third party
cookies} issue -- a technological loophole enabling marketeers and
advertisement firms to invisibly track users over large networks of
syndicated websites -- has been the object of a serious controversy,
involving a varied set of actors and stakeholders.

Cookiecensus is a software prototype. A wannabe info tool for studying
electronic surveillance in one of its natively digital environments.
Its core functionality consists of mapping and analyzing third party's
cookies distribution patterns within a given web, in order to identify
its trackers and its network of syndicated sites. A further feature of
the tool is the possibility to inspect the content of a web page in
relation to its third party cookie sources.

It is an attempt to deconstruct the perceived unity and consistency of
web pages by making their underlying content assemblage and their
related attention flows visible.}
