\PlaceImage{maleve01.JPG}{De algemene context van toezicht en controle vormt het onderwerp
van acties, discussies en workshops}
\PlaceImage{maleve02.JPG}{Le contexte g\'en\'eral de surveillance g\'en\'eralis\'ee fait l'objet d'interventions, conf\'erences, ateliers}

\AuthorStyle{Nicolas Malev\'e, Michel Cleempoel}

\licenseStyle{Free Art License}

\Ned{\Title{E{}-traces in Context}

\Url{http://etraces.constantvzw.org}\blank

Het project e{}-traces kaart de ontwikkelingen rond Web 2.0 aan in de
context van de uitbouw van de controlemaatschappij. E{}-traces
ontwikkelt zich rond 3 assen:

\startitemize
\item Een site met online verzamelde informatie (artikels, links,
dossiers{\dots})
\item De ontwikkeling van Yoogle!, een spelproject dat nog in een eerste
ontwikkelingsfase is, waarbij spelers om beurten de rollen van de
verschillende actoren op de markt van de persoonlijke gegevens kunnen
aannemen, en aan alle man{\oe}uvres kunnen deelnemen.
\item De algemene context van toezicht en controle vormt het onderwerp
van acties, discussies en workshops over onderwerpen als: hoe kun je
software zo gebruiken (e{}-mail, webbrowsers) dat je anonimiteit
online gewaarborgd blijft.
\stopitemize

\SubSubTitle{Hoe Web 2.0 defini\"eren?}
De meeste onder ons zien Web 2.0 als een verzameling tools die ons in
staat stellen berichten te verzenden, webpagina's te schrijven, foto's,
video's en muziek met elkaar te delen, online te kopen etc. Deze
diensten kennen de gebruiker een centrale positie toe. Dit dankzij
eenvoudige, gebruiksvriendelijke en uitnodigende interfaces, evenals
door het vrijwaren van een grote vrijheid inzake de configuratie, die
de gebruiker het gevoel geven \quote{thuis} te zijn in zijn of haar
persoonlijke webruimten. De bedrijven die deze diensten ontwikkelen
profileren zich als intermediairs tussen de gebruikers en hun online
ervaring.

Volgens de uitgever Tim O'Reilly, een van de emblematische figuren die aan de wieg stond van het concept \quote{Web 2.0}, kennen deze nieuwe applicaties een reeks gemeenzame technische
karakteristieken en volgen ze een zelfde commerci\"ele logica. Onder de
grote principes die hij excerpeert, vinden we onder andere het
volgende:

\SubSubTitle{Het web als platform}
Om gebruik te kunnen maken van Web 2.0 heb je enkel een webbrowser
nodig. De gehele applicatie draait op de server: de gegevens van de
gebruiker zijn er gestockeerd en een webinterface geeft toegang tot
alle functionaliteiten. In tegenstelling tot de informaticawereld van
de jaren '80, die software ontwikkelde die op de machine van de
gebruiker diende ge\"installeerd te worden, ontwikkelt de wereld van
Web 2.0 software die van op afstand toegankelijk is, en zelfs
beschikbaar wanneer de gebruiker op vakantie is of wanneer hij/zij van
zijn/haar persoonlijke machine overschakelt naar diegene die hij/zij op
kantoor gebruikt. Deze situatie, die erg praktisch is voor de
gebruikers en gebruiksters, is het ook voor de bedrijven die deze
diensten aanleveren: de installatie van een update kan zich voor alle
geconnecteerde gebruikers onmiddellijk voltrekken. Een verbetering van
het algoritme van een zoekmachine is voor alle gebruikers/gebruiksters
meteen beschikbaar. Het is bijgevolg voor de producent niet nodig om
zoveel versies te cre\"eren als er systemen bestaan, noch om de
gebruikers en gebruiksters te overhalen om een nieuwe versie op hun
machine te installeren. 

\SubSubTitle{Het web en heel het web}
Web 2.0 bedrijven ontwikkelen applicaties die zich op het web in zijn
geheel richten en op al haar gebruikers en gebruiksters. Het gaat dus
niet gewoon om pakweg tweehonderd belangrijke cli\"enten, maar om
miljoenen en miljoenen gebruikers. Elke persoon die een blog heeft, hoe
klein en gespecialiseerd ook, heeft het recht om publicitaire
boodschappen te plaatsen.

In de verveelvoudigde wereld van Web 2.0 bestaan er geen onbelangrijke
actoren, want eens bij elkaar opgeteld, telt iedereen mee. Uit de
wereld van Web 2.0 haalt men nooit slechts klein profijt.

\page

\SubSubTitle{Profiteren van collectieve intelligentie} 
Het web lijkt het evenbeeld van een groot brein. Hoe talrijker de
synaptische verbinden zijn, hoe groter ook de intelligentie is. Hoe
talrijker de verbindingen en connecties, hoe beter de informatie
circuleert. Web 2.0 biedt aan de internauten eenvoudige middelen om te
classificeren, zaken naar elkaar te laten refereren, te filteren, samen
te werken. Deze tools functioneren op een \quote{horizontale} wijze, wat
wil zeggen dat ze de kleinst mogelijke hi\"erarchie impliceren, en de
structuur extraheren die zich volgens de keuzes en de acties van de
internauten ontwikkelt. De tagclouds vervangen de taxonomie\"en, elke
blogger kan er een andere becommentari\"eren. Dit laat zeer
interessante realisaties toe zoals de creatie van online
encyclopedie\"en, fundamentele vergaarbakken van kennis, onophoudelijk
herlezen en gecorrigeerd, verbeterd en gecompleteerd door duizenden
mensen, maar ook diverse publicaties, kritische journalistiek,
discussies waarin talrijke participanten betrokken zijn, evenals de
creatie van virtuele gemeenschappen.

Maar dit is natuurlijk ook een uitgelezen hulpmiddel voor de virale
marketing, waarbij de gebruikers en gebruiksters merken en producten
langzamerhand en van niche tot niche verspreiden. Men is nooit zo
ontvankelijk voor een product als binnen een kader dat ons vertrouwen
wekt, als in de ruimte van iemand waarmee we de keuzes, smaak,
interesses en opinies delen.

De controle behouden over gegevensbronnen die moeilijk te recre\"eren
zijn en die al naar gelang hun gebruik verrijkt worden. 

Aan wie behoren nu deze gegevens, die aan de basis liggen van deze
collectieve intelligentie, toe? 

Aan de internauten die ze cre\"eren of aan de bedrijven die ze op hun
servers accumuleren? Deze gegevens zijn welteverstaan de re\"ele waarde
van de bedrijven van Web 2.0, enerzijds door hun overvloedigheid, maar
bovenal omdat ze moeilijk ex nihilo te cre\"eren zijn. Ze zijn
inderdaad begerenswaardig voor zij die er op belust zijn de gedragingen
van de consumenten, de opinies van een segment van de populatie etc.,
te analyseren. 

Kortom, het bezitten van een aanzienlijke basis aan gebruikersgegevens
bevordert voor de bedrijven van Web 2.0 een exponenti\"ele groei: hoe
meer een zoekmachine een groot deel van het web zal indexeren, hoe meer
mensen haar zullen gebruiken en in haar resultaten zullen willen
figureren. Hoe meer een site voor online verkoop aanbevelingen van haar
gebruikers verzamelt, hoe meer deze ook producten zullen vinden die
nauw aansluiten bij hun voorkeuren.

Ook al vrezen we Big Brother, dit alles speelt zich af alsof we
vrijwillig de meest intieme informatie aan een resem \quote{little brothers}
toevertrouwen, van wie het schijnbare doel is ons het leven
gemakkelijker te maken, maar die in feite tersluiks onze persoonlijke
gegevens verzamelen en opslaan.

De grote actoren van Web 2.0 (Google, Yahoo!, Microsoft...) breiden
zich uit en rivaliseren door middel van een aankooppolitiek van
websites voor het delen van bestanden, portaalsites, e{}-commerces,
sociale netwerken... Op deze wijze kunnen ze de correlatie
diagnosticeren tussen de gegevens die door de gebruiker ingevoerd
werden op sites met verschillende identiteiten. De bezoeker gebruikt de
diensten, neemt deel aan community websites, brengt in alle vertrouwen
gegevens in, zonder dat hun naderhand gebruik duidelijk gedefinieerd
wordt. De bepalingen van de dienst laten het aan zijn eigenaar nochtans
toe om ze in extreem verschillende contexten buiten de controle van de
gebruiker, te exploiteren.

\SubSubTitle{Persoonlijke gegevens: de nieuwe olie?}
Het geheel van deze gegevens laat de gespecialiseerde firma's in
gedragsmarketing toe om onze profielen als consument, burger en
werknemer op te stellen, en ze te verhandelen aan ge\"interesseerde
cli\"enten.

De commerci\"ele firma's interesseren zich in de gegevens die te maken
hebben met de consumptie en de koopkracht. De werkgevers interesseren
zich in de gegevens die te maken hebben met gezondheid, betrouwbaarheid
en opinies. De firma's van de sociale zekerheid, van verzekeringen, en
de staat, interesseren zich in gegevens aangaande gezondheid, gedrag en
stabiliteit. 

In onze consumptieve context heeft men het met de studie van deze
profielen gemunt op de portefeuille van de internaut, maar in meer
autoritaire regimes draagt ze reeds bij aan de toenemende uitbouw van
een beklemmende sociale controle.
\godown[2em]
\midaligned\framed[offset=1em,frame=off,background=color,backgroundcolor=lightgray,corner=round,width=.6\textwidth,align=middle]{{\ss\bf Download Yoogle! add{}-on}
\blank
{\ss\tfx\setupinterlinespace Download de Firefox add{}-on om uw zoekopdrachten in Google te
registreren en uw gegevens beschikbaar te maken voor de ontwikkeling
van Yoogle!
\blank
\Url{http://dev.yoogle.be}}}\par}
\godown[2em]
\Fra{\Title{E{}-traces en contexte}

\Url{http://etraces.constantvzw.org}\blank

Le projet e{}-traces aborde le Web 2.0 dans le contexte de
l'instauration progressive d'une
soci\'et\'e de la surveillance. Il s'articule autour
de 3 axes:

\startitemize
\item Un site d'informations r\'ecolt\'ees en ligne
(articles, liens, dossiers...).
\item Un projet de jeu Yoogle!, encore dans une phase premi\`ere de
d\'eveloppement, qui permettra de prendre tour{}-\`a{}-tour les r\^oles
des diff\'erents acteurs du march\'e des donn\'ees personnelles et de
participer aux man{\oe}uvres des uns et des autres.
\item Le contexte général de surveillance généralisée qui fait l'objet d'interventions, conférences, ateliers sur des sujets en relation comme, par exemple, l'initiation aux outils (navigateurs, messageries...)
pr\'eservant l'anonymat sur Internet.
\stopitemize

\page

\SubSubTitle{Comment d\'efinir le Web 2.0?}
Pour la plupart d'entre nous, le Web 2.0 est con\c{c}u
comme un ensemble d'outils qui nous permettent
d'envoyer des messages, \'ecrire des pages web,
partager des photos, videos, de la musique, d'acheter
en ligne etc. Ces services mettent l'utilisateur au
centre de la toile. Gr\^{a}ce \`a des interfaces simples, conviviales et
engageantes. Et en laissant une grande libert\'e de configuration qui
donne le sentiment \`a l'utilisateur
d'\^etre chez lui dans ses espaces personnels. Ces
compagnies se profilent comme les interm\'ediaires entre les
utilisateurs et leur exp\'erience en ligne.

Selon l'\'editeur Tim O'Reilly, une
des figures embl\'ematiques \`a l'origine du concept
du Web 2.0, ces nouvelles applications suivent une s\'erie de
caract\'eristiques techniques et une logique commerciale communes.
Parmi les grands principes qu'il d\'egage, on
retrouve:

\SubSubTitle{Le web comme plateforme}
Pour profiter des services du Web 2.0, pas besoin
d'autre chose que d'un navigateur
web. Toute l'application tourne sur le serveur: les
donn\'ees de l'utilisateur y sont stock\'ees et une
interface web donne acc\`es \`a toutes les fonctionnalit\'es.
Contrairement au monde de l'informatique des ann\'ees
'80 qui produisait des logiciels \`a installer sur la
machine de l'utilisateur, le monde du Web 2.0 produit
des logiciels accessibles \`a distance et disponibles m\^eme lorsque
l'utilisateur/trice est en voyage ou
lorsqu'il/elle passe de sa machine personnelle \`a
celle qu'il/elle utilise au bureau. Cette situation,
pratique pour les utilisateur/trice/s, l'est aussi
pour les compagnies qui fournissent ces services: leur mise \`a jour
est imm\'ediate pour tous les usagers connect\'es. Une am\'elioration
de l'algorithme d'un moteur de
recherche est disponible pour tous les utilisateur/trice/s. Pas besoin
pour le producteur d'en faire autant de versions
qu'il existe de syst\`emes et de convaincre les
utilisateur/trice/s d'effectuer une mise \`a jour sur
leur machine.

\page

\SubSubTitle{Le web et tout le web}
Les compagnies du Web 2.0 d\'eveloppent des applications qui visent le
web tout entier et tous ses utilisateur/trice/s. Pas seulement quelques
deux cents clients importants, mais des millions et des millions. Toute
personne qui a un blog, aussi petit et sp\'ecialis\'e
qu'il soit, a droit \`a placer des messages
publicitaires. Dans le monde d\'emultipli\'e du Web 2.0, il
n'y a pas d'acteurs n\'egligeables,
car additionn\'es les uns aux autres, ils finissent toujours par
compter. Il n'y a pas de petits profits dans le monde
du Web 2.0.

\SubSubTitle{Profiter de l'intelligence collective}
Le web est \`a l'image d'un grand
cerveau. Plus nombreuses sont les connections synaptiques, plus grande
est l'intelligence. Plus il y a de liens et de
connections, au mieux l'information circule. Le web
2.0 offre aux internautes des moyens simples de classifier, de
r\'ef\'erer les uns aux autres, de filtrer, de collaborer. Ces outils
fonctionnent de mani\`ere \quote{horizontale},
c'est{}-\`a{}-dire qu'ils impliquent
le moins de hi\'erarchie possible, et extraient la structure qui se
d\'egage des choix et des actions des internautes. Les nuages de
mots{}-cl\'e remplacent les taxinomies, chaque blogger peut en
commenter un autre. Ceci permet des r\'ealisations tr\`es
int\'eressantes comme la cr\'eation d'encyclop\'edies
en ligne, des bases de connaissances, sans cesse relues et corrig\'ees,
amend\'ees et compl\'et\'ees par des milliers de gens, mais aussi des
publications diverses, du journalisme critique, des discussions qui
impliquent de tr\`es nombreux acteurs ainsi que la cr\'eation de
communaut\'es virtuelles.

Mais c'est bien s\^ur aussi un formidable outil pour le
marketing viral par lequel les utilisateur/trice/s diffusent des
marques et des produits de proche en proche, de niche en niche. Un
produit n'est jamais aussi bien re\c{c}u que dans un
cadre o\`u nous sommes en confiance, dans l'espace de
quelqu'un dont nous partageons les choix, les go\^uts,
les int\'er\^ets ou les opinions.

Garder le contr\^ole sur des sources de donn\'ees difficiles \`a
recr\'eer, et qui s'enrichissent au fur et \`a mesure
de leur utilisation.

Mais \`a qui appartiennent ces donn\'ees qui sont \`a la base de cette
intelligence collective?

Aux internautes qui les cr\'eent ou aux compagnies qui les accumulent
sur leurs serveurs? Ces donn\'ees sont bien entendu la valeur r\'eelle
des compagnies du Web 2.0, tant elles sont riches, mais surtout
difficiles \`a recr\'eer ex nihilo. Elles attirent, en effet, tous ceux
qui sont avides d'analyser les comportements des
consommateurs, les opinions d'un segment de la
population, etc.

Enfin, pour les compagnies du Web 2.0, poss\'eder une base importante de
donn\'ees d'utilisateurs favorise une croissance exponentielle: plus un
moteur de recherche va indexer une large partie du web et plus de
personnes vont l'utiliser et vouloir figurer dans ses
r\'esultats. Plus un site de vente en ligne collecte des
recommandations de ses usagers et plus ceux{}-ci vont trouver des
produits proches de leurs go\^uts.

Tout se passe comme si, alors que nous craignons Big Brother, nous
confions volontairement des informations les plus intimes \`a une
s\'erie de little brothers, dont le but apparent est de nous faciliter
la vie mais qui, discr\`etement, collectent et enregistrent nos
donn\'ees personnelles.

Les grands acteurs du Web 2.0 (Google, Yahoo!, Microsoft...)
s'\'etendent et rivalisent au moyen
d'une politique d'achat de sites de
partage, de portails, d'e{}-commerces, de r\'eseaux
sociaux... Ainsi, ils peuvent corr\'eler les donn\'ees introduites par
l'usager dans des sites aux identit\'es diff\'erentes.
Le visiteur utilise les services, participe aux sites communautaires,
entre des informations en toute confiance, sans que leur utilisation ne
soient clairement d\'efinie. Les termes du service permettent pourtant
\`a son propri\'etaire de les exploiter dans des contextes
extr\^emement diff\'erents non maitris\'es par
l'utilisateur.

\SubSubTitle{Les donn\'ees personnelles, un nouveau p\'etrole?}
L'ensemble de ces donn\'ees permet aux firmes
sp\'ecialis\'ees dans la publicit\'e comportementale
d'\'etablir nos profils de consommateur, citoyen et
travailleur et d'en faire commerce aupr\`es de clients
int\'eress\'es.

Les firmes commerciales s'int\'eresseront aux donn\'ees
concernant la consommation et le pouvoir d'achat. Les
employeurs s'int\'eresseront aux donn\'ees concernant
la sant\'e, la stabilit\'e et les opinions. Les firmes de couverture
sociale, d'assurances et l'Etat
s'int\'eresseront aux donn\'ees concernant la sant\'e,
le comportement et la stabilit\'e.

Dans notre contexte consum\'eriste, l'\'etude de ces
profils visera \`a ce que l'internaute ouvre son
portefeuille mais, d\'ej\`a, dans des r\'egimes plus autoritaires, elle
participe \`a la mise en place progressive d'un
contr\^ole social accentu\'e.
\godown[2em]
\midaligned\framed[offset=1em,frame=off,background=color,backgroundcolor=lightgray,corner=round,width=.6\textwidth,align=middle]{{\ss\bf T\'el\'echargez Yoogle! add{}-on}
\blank
{\ss\tfx\setupinterlinespace T\'el\'echargez l'add{}-on Firefox pour
enregistrer votre historique de recherche dans Google et introduire vos
donn\'ees dans notre jeu Yoogle!
\blank
\Url{http://dev.yoogle.be}}}\par}
