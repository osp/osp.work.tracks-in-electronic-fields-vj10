\PlaceImage{lazaro01.jpg}{Conf\'erence Christophe Lazaro, V/J10}

\AuthorStyle{Christophe Lazaro}

\licenseStyle{Creative Commons Attribution{}-NonCommercial{}-ShareAlike}

\Fra{\Title{La vie priv\'{e}e et les droits d'auteur\crlf li\'{e}s aux profils}
\SubTitle{Transcription d'un expos\'{e} oral}

Merci. Effectivement, ce n'est pas une t\^ache tr\`es ais\'ee de vous parler de communaut\'es parce qu'aujourd'hui, on a principalement parl\'e d'objets techniques: de Web 2.0, de moteurs de recherche, de {\em tracking systems}. Quand on entend les gens parler durant toute la journ\'ee de ces objets techniques, il est assez fascinant de constater qu'on les appr\'ehende de plus en plus comme des individus. Il y a quelques ann\'ees, un philosophe fran\c{c}ais du nom de Simondon a d'ailleurs fait un remarquable essai sur ce qu'il appelle \quotation{le mode d'existence des objets techniques}. Il a montr\'e qu'\`a un certain stade de sophistication, ces objets techniques t\'emoignent d'une forme d'existence qui leur est propre et s'apparentent \`a des individus. Aujourd'hui, au moment de la pr\'esentation du projet Yoogle!, on a parl\'e de {\em \quote{little brothers}}, de \quote{petits fr\`eres}. Donc, on qualifie ces objets techniques d'individus, on les consid\`ere comme des acteurs.

\`A cet \'egard, il faut souligner plusieurs choses. D'abord on voit bien qu'il n'y a plus de l\'egitimit\'e \`a s\'eparer, d'un c\^ot\'e, la sph\`ere des objets techniques et, de l'autre c\^ot\'e, le monde des hommes ou, peut{}-on dire, la sph\`ere du politique. La technique et la politique sont deux sph\`eres qui sont compl\`etement encastr\'ees et qui doivent \^etre appr\'ehend\'ees de mani\`ere commune; il faudrait m\^eme, plus largement, faire r\'ef\'erence \`a des r\'eseaux socio{}-, technico{}-, politico{}-, \'economiques au sein desquels interagissent des actants humains et non{}-humains. En outre, ce qu'on a bien vu \`a travers les diff\'erents expos\'es d'aujourd'hui, c'est que la technique n'est pas neutre; derri\`ere le design technologique, derri\`ere la conception des objets techniques, il y a des id\'ees, il y a des valeurs et, pour ces raisons, les objets techniques rev\^etent un caract\`ere \'eminemment normatif, voire performatif. Le cas des {\em search engines} abord\'e aujourd'hui est exemplaire \`a cet \'egard: on a bien vu que la mani\`ere dont le {\em ranking} \'etait constitu\'e avait une influence sur nos choix, sur nos capacit\'es de recherche, et qu'il y avait moyen de cr\'eer des alternatives \`a ce que Google propose, par exemple. De la m\^eme mani\`ere, on a aussi pu constater que le syst\`eme de {\em cookies}, d'apparence anodine, permet de tracer l'internaute, permet de l'identifier. Ces exemples doivent nous amener \`a consid\'erer {--} et on en a tr\`es rapidement parl\'e apr\`es l'expos\'e. d'Andrea Fiore {--}, le code (informatique) comme une norme. Lessig, un juriste{}-philosophe am\'ericain de r\'eputation internationale, a explicitement parl\'e de code {\em \quote{as law}}. Donc v\'eritablement, on peut faire le parall\`ele entre la loi, entre la norme et le code informatique. Pourquoi? Parce que cela influence nos choix, \c{c}a identifie nos trajectoires, et \c{c}a peut {--} le cas \'ech\'eant {--} r\'eduire notre marge de man{\oe}uvre, r\'eduire notre libert\'e.

On a beaucoup parl\'e d'individus techniques, comme je le disais. D'une personnalisation de l'objet technique, on a m\^eme abouti \`a son anthropomorphisation totale, puisqu'un des intervenants nous a parl\'e du Syndicat des Robots. A cette occasion, on a m\^eme envisag\'e la possibilit\'e que ces individus se mutualisent et forment une communaut\'e. Il faut ici souligner un \'etrange paradoxe: lors de la discussion, d'un c\^ot\'e, on a reconnu que les robots n'avaient pas conscience d'eux{}-m\^emes, mais de l'autre, on a envisag\'e la possibilit\'e qu'ils puissent former une communaut\'e.

Ceci me permet de passer \`a ce que Nicolas m'a demand\'e de vous exposer, \`a savoir de vous parler de la notion de \quote{communaut\'e}. Il y a beaucoup de questions qui se posent au sujet de cette notion en sociologie, en anthropologie, en philosophie, pr\'ecis\'ement depuis l'\'emergence de ce qu'on appelle \quote{les nouvelles technologies}. C'est comme si on avait d\^u {\em remettre} ces notions fondamentales des sciences humaines sur le m\'etier, pour essayer de comprendre les nouvelles formes de socialit\'e et de solidarit\'e qui se d\'eveloppent sur Internet. Alors, que faut{}-il entendre par cette notion de \quote{communaut\'e}? Vous verrez que je n'ai absolument aucune r\'eponse pr\'ecise \`a cette question. Les robots n'ont pas conscience d'eux{}-m\^emes, est{}-ce que d\`es lors ils peuvent avoir conscience de leur appartenance \`a une communaut\'e? Est{}-ce que pour appartenir, la communaut\'e suppose cette r\'eflexivit\'e, suppose cette conscience \quote{d'appartenir \`a}? Ce sont des questions tr\`es vastes. Bri\`evement il n'est pas inutile de revenir ici sur l'\'etymologie du mot \quote{communaut\'e}. L'\'etymologie nous informe sur le fait que le terme \quote{communaut\'e} provient du mot latin {\em communis}, lequel d\'erive lui{}-m\^eme de {\em cum} (avec, ensemble) et {\em munus} (charge, dette). \`A ce titre, il renverrait donc originellement \`a une relation sociale caract\'eris\'ee par des {\em obligations mutuelles}, autrement dit un r\'eseau d\'efini par des r\`egles du type \quote{donner{}-recevoir{}-rendre}. La notion de \quote{communaut\'e} renvoie donc \`a des relations sociales bas\'ees sur le triptyque du don tel qu'on le d\'ecrit en anthropologie. Parall\`element \`a cette notion de don et de \quote{dette}, le terme de \quote{communaut\'e} renvoie aussi \`a l'id\'ee de \quote{communion}, de {\em communis}, au sens de partager, de mettre en commun.

La notion de communaut\'e a \'et\'e interrog\'ee par les sciences sociales de mani\`ere tr\`es approfondie \`a la fin du XIX\`eme si\`ecle. On peut \`a cet \'egard faire r\'ef\'erence au travail du sociologue T\"onnies qui a fait une distinction entre ce qu'il appelle la {\em \quote{Gemeinschaft}} et la {\em \quote{Gesellschaft}.} La {\em Gemeinschaft} renverrait \`a une \quote{volont\'e organique}, au caract\`ere organique de la communaut\'e. Qu'est{}-ce que \c{c}a veut dire pratiquement? La {\em Gemeinschaft} se pr\'esenterait comme une forme d'organisation sociale o\`u les liens entre les individus sont tr\`es serr\'es, o\`u il existe une tr\`es grande interd\'ependance entre les individus. Il s'agit ici typiquement du mod\`ele du voisinage, du mod\`ele parental, ou encore du mod\`ele de l'amiti\'e. Et \`a cette forme d'organisation du vivre ensemble, il oppose la {\em Gesellschaft}, qui correspond \`a la \quote{soci\'et\'e}: cette notion renverrait plut\^ot \`a une forme d'organisation sociale o\`u les liens entre les individus sont plus l\^aches et davantage rationalis\'es. Pour T\"onnies, ces deux notions sont totalement exclusives l'une de l'autre. Mais on verra qu'en r\'ealit\'e, il faudrait plut\^ot les replacer dans une sorte de continuum empirique et les consid\'erer comme deux id\'eals types, deux p\^oles entre lesquels il existe diff\'erentes modalit\'es de communaut\'es ou de soci\'et\'es.

Si la cat\'egorisation propos\'ee par T\"onnies peut servir de base \`a la r\'eflexion sur la notion de communaut\'e, il faut cependant souligner le fait que les technologies de l'information ont forc\'e les sociologues et les anthropologues \`a affiner cette notion, \`a la nuancer. Aujourd'hui, la pluralit\'e de termes utilis\'es pour t\'emoigner des diff\'erentes formes de socialit\'e et de solidarit\'e sur Internet t\'emoigne de la difficult\'e de cerner cette notion. Ainsi, les qualificatifs abondent pour caract\'eriser ces \'etranges entit\'es se d\'eveloppant dans l'environnement num\'erique; elles sont dites \quote{virtuelles}, \quote{distantes}, \quote{m\'edi\'ees par ordinateurs}, \quote{en r\'eseau} ou encore \quote{interactives}. Certains sociologues ont essay\'e de proposer des nouvelles cat\'egories, privil\'egiant notamment une approche constructiviste. Il ne s'agit plus alors de r\'efl\'echir \`a ce que pourrait \^etre une communaut\'e au sens naturel ou au sens substantiel du mot, mais de montrer le caract\`ere construit de cette notion et de prendre distance par rapport \`a toute conception essentialiste du collectif. Par exemple, Benedict Anderson a forg\'e la notion de \quote{communaut\'e imagin\'ee} en r\'efl\'echissant \`a ce que repr\'esentait la nation pour les gens. Dans son travail, il montre que la notion de la communaut\'e finalement se rapporte \`a quelque chose qui serait de l'ordre de la fiction, de l'imaginaire collectif. La nation correspondrait \`a une \quote{communaut\'e politique imagin\'ee}, r\'eunissant des gens qui ne se connaissent pas et qui ne se croiseront jamais mais qui \'eprouvent un fort sentiment d'appartenance \`a une communaut\'e.

Les sociologues contemporains ont propos\'e une deuxi\`eme notion qui vaut la peine d'\^etre mentionn\'ee. C'est la notion de \quote{public}, qui a \'et\'e emprunt\'ee au sociologue Gabriel Tarde, qui lui a \'ecrit au d\'ebut du XX\high{\`eme} si\`ecle. Pour Tarde, le \quote{public} d\'esigne une collectivit\'e purement spirituelle entre des individus physiquement s\'epar\'es et dont la coh\'esion est toute mentale. Cette notion de \quote{public} a eu une valeur heuristique principalement dans le domaine des pratiques m\'ediatiques; elle vise notamment \`a d\'ecrire les publics de t\'el\'evision. Mais, elle est tr\`es int\'eressante en ce qui concerne les pr\'etendues communaut\'es \quote{virtuelles}, parce qu'elle implique la possibilit\'e pour un individu d'appartenir \`a plusieurs entit\'es \`a la fois, donc d'appartenir \`a plusieurs communaut\'es virtuelles. On s'\'eloigne ici fortement de la notion de communaut\'e telle que on la concevait au XIX\high{\`eme} si\`ecle, parce qu'\`a l'\'epoque la communaut\'e faisait r\'ef\'erence \`a quelque chose de totalement exclusif. On appartenait \`a une et une seule communaut\'e \`a la fois et on ne pouvait pas concevoir qu'un individu appartienne simultan\'ement \`a plusieurs entit\'es. Or ici, on constate, quand on observe par exemple les communaut\'es libres sur Internet, notamment celles qui rassemblent des fans des logiciels libres, qu'il y a une possibilit\'e pour un m\^eme individu d'adh\'erer par exemple \`a la communaut\'e Debian, qui d\'eveloppe le syst\`eme d'exploitation Debian, mais en m\^eme temps d'appartenir \`a la sous{}-communaut\'e Ubuntu, qui est une \quote{{\em children distro}}, c'est{}-\`a{}-dire une distribution fille de Debian.

Parall\`element aux notions de \quote{communaut\'e imagin\'ee}, et de \quote{public}, on peut encore mobiliser une troisi\`eme notion pour essayer de rendre compte des formes de socialit\'es originales qui fleurissent sur Internet. Les sociologues et les anthropologues contemporains utilisent tr\`es souvent la notion du \quote{r\'eseau social}. La notion de \quote{r\'eseau social} bien entendu n'est pas neuve, mais avec l'\'emergence nouvelles technologies, il est int\'eressant de constater qu'on a essay\'e de faire {\em coller} la notion de \quote{r\'eseau social} \`a la sp\'ecificit\'e m\^eme des r\'eseaux num\'eriques, \`a savoir leur caract\`ere {\em horizontal}. Alors, \`a nouveau, il faut ici \^etre tr\`es prudent, parce qu'on constate, lorsqu'on analyse par exemple la communaut\'e Debian ou les communaut\'es logiciels libres en g\'en\'eral, que derri\`ere l'horizontalit\'e li\'ee \`a la mani\`ere de communiquer, d'utiliser l'outil technique, le lien social est aussi d\'etermin\'e par une certaine {\em verticalit\'e}. Cette verticalit\'e se manifeste par exemple \`a travers le fait de gagner en r\'eputation, d'obtenir de la reconnaissance, de s'\'elever sur l'\'echelle de l'expertise technique. Donc, la notion du \quote{r\'eseau social} est importante, mais on ne doit pas oublier que derri\`ere cette fameuse horizontalit\'e qui semble inh\'erente aux architectures num\'eriques et aux technologies de la communication, il y a des processus subtils de hi\'erarchie, de verticalit\'e qui op\`erent.

Face \`a cette complexit\'e, que font les sociologues ou les anthropologues? Et bien, ils essayent d'{\em unifier} et quand ils d\'ecrivent une communaut\'e sur Internet, ils mobilisent diverses notions, tentent de faire un mix entre les notions de \quote{public} et de \quote{r\'eseau social}, par exemple; mais ce qu'on constate de plus en plus, c'est qu'ils font r\'ef\'erence \`a des communaut\'es {\em abstraites}, des communaut\'es repr\'esentationnelles, des communaut\'es imagin\'ees. Ce qu'il faut sans doute retenir de tout cela, c'est que finalement la notion de \quote{communaut\'e} est une notion {\em dynamique} et cette caract\'eristique appelle vraiment, selon moi, un positionnement tr\`es clair par rapport \`a toute vision fig\'ee et nostalgique de la \quote{communaut\'e} telle qu'on la d\'ecrivait au XIX\high{\`eme} si\`ecle. Ce qu'on remarque aujourd'hui, c'est que les partisans des nouvelles technologies, qui c\'el\`ebrent l'apparition d'une intelligence collective, qui ressuscitent la noosph\`ere de Teilhard de Chardin et des id\'ees de la sorte, \`a l'instar de ceux qui critiquent les nouvelles technologies et qui les diabolisent, succombent souvent, malgr\'e leurs divergences, \`a une m\^eme vision pass\'eiste et totalement nostalgique de la communaut\'e. Or, il faut bien tenir compte du fait que si on doit r\'efl\'echir \`a la notion de \quote{communaut\'e} telle qu'elle est aujourd'hui v\'ecue sur les r\'eseaux num\'eriques, il faut actualiser cette notion et ne pas rester emprisonn\'e par ces visions nostalgiques, voire tribales, de la communaut\'e.

Alors, que peut encore ajouter? Il faut admettre que tous les agr\'egats sociaux qui se d\'eveloppent en ligne ne se valent pas en termes de pertinence et de sophistication sociale et politique. En effet, on ne peut \'evidemment pas mettre sur un m\^eme pied, une \quote{communaut\'e} Facebook, une \quote{communaut\'e} YouTube et, par exemple, une \quote{communaut\'e} Debian {--} je parle beaucoup de Debian parce que j'ai travaill\'e pendant un an sur cette communaut\'e{}-l\`a. Ces entit\'es ne sont pas du tout, au niveau de l'organisation politique et sociale, organis\'ees de la m\^eme mani\`ere. La communaut\'e Debian t\'emoigne d'un grand formalisme institutionnel: il y a un contrat social, il y a des textes fondateurs, il y a un fonctionnement hautement d\'emocratique, il y a une \'ethique qui sont \`a la base du \quote{vivre ensemble} de la communaut\'e. Pour conclure, je dirais donc qu'il faut \'evidemment faire preuve de nuance et, de mani\`ere g\'en\'erale, qu'il faut \'eviter de f\'etichiser la notion de \quote{communaut\'e virtuelle}, lorsqu'on tente de comprendre les nouvelles formes de socialit\'e et de solidarit\'e qui se d\'eveloppent sur Internet.{\em } Sur cette base, il s'agit notamment de d\'epasser l'opposition trop souvent \'evoqu\'ee entre \quote{r\'eel} et \quote{virtuel}. Le terme \quote{virtuel} est assur\'ement probl\'ematique. Ce qui compte, ce n'est pas seulement l'objet technique de communication qui est utilis\'e, mais c'est surtout le type de relation qu'il g\'en\`ere; c'est \c{c}a qui est int\'eressant. Et dans notre soci\'et\'e contemporaine, on peut d\'ej\`a constater qu'{\em hors{}-ligne} une part grandissante des interactions et des \'echanges symboliques est d\'ej\`a m\'ediatis\'ee par toute une s\'erie de processus technologiques.}