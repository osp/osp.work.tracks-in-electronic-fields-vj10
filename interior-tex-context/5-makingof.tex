\blank
{\em This publication was produced with a set of digital tools that are
rarely used outside the world of scientific publishing: \TeX, \LaTeX\ and \ConTeXt. As early as the summer of 2008, when most contributions and
translations to {\em Tracks in electronic fields} were reaching
their final stage, we started discussing at OSP\footnote{Open Source Publishing \Url{http://ospublish.constantvzw.org}} how we could design and
produce a book in a way that responded to the theme of the festival
itself. OSP is a design collective working
with Free Software, and our relation to the software we design with, is
particular on purpose. At the core of our design practice is the
ongoing investigation of the intimate connection between form,
content and technology. What follows, is a report of an experiment that
stretched out over a little more than a year.}\par

For the production of previous books, OSP used Scribus, an Open Source
Desktop Publishing tool which resembles its proprietary variants
PageMaker, InDesign or QuarkXpress. In this type of software, each
single page is virtually present as a
\quote{canvas} that has the same
proportions as a physical page and each of these
\quote{pages} can be individually altered
through adding or manipulating the virtual objects on it. Templates or
\quote{master pages} allow the automatic
placement of repeated elements such as page numbers and text blocks,
but like in a paper-based design workflow, each single page can be
treated as an autonomous unit that can be moved, duplicated and when
necessary removed. Scribus would have certainly been fit for this job,
though the rapidly developing project is currently in a stage that the
production of books with more than 40 pages can become tedious. Users
are advised to split up such documents into multiple sections which
means that in able to keep continuity between pages, design decisions
are best made beforehand. As a result, the design workflow is rendered
less flexible than you would expect from state-of-the-art creative
software. In previous projects, Scribus' rigid
workflow challenged us to relocate our creative energy to another
territory: that of computation. We experimented with its powerful
Python scripting API to create 500 unique books. In another project,
we transformed a text block over a sequence of pages with the help of a
fairy-tale script. But for {\em Tracks in electronic fields} we
dreamed of something else.


Pierre Huyghebaert takes on the responsibility for the design of the
book. He had been using various generations of lay-out software since
the early 90's, and gathered an extensive body of
knowledge about their potential and limitations. More than once he
brought up the desire to try out a legendary typesetting system called
\TeX\, a sublime typographic engine that allegedly implemented the work
of grandmaster Jan Tshichold\footnote{In {\em Die neue Typographie}
(1928), Jan Tschichold formulated the classic canon of modernist
bookdesign.} with mathematical precision.


\TeX\ is a computer language designed by Donald Knuth in the
1970's, specifically for typesetting mathematical and
other scientific material. Powerful algorithms automatize widow and
orphan control and can handle intelligent image placement. It is
renowned for being extremely stable, for running on many different
kinds of computers and for being virtually bug free. In the academic
tradition of free knowledge exchange, Knuth decided to make \TeX\
available \quote{for no monetary fee} and
modifications of or experimentations with the source code are
encouraged. In typical self referential style, the near perfection of
its software design is expressed in a version number which is
converging to ${\pi}$\footnote{The value of ${\Pi}$
(3.141592653589793...) is the ratio of any circle's
circumference to its diameter and it's decimal
representation never repeats. The current version number of \TeX\ is
3.141592}.

For OSP, \TeX\ represents the potential of doing design differently.
Through shifting our software habits, we try to change our way of
working too. But Scribus, like the kinds of proprietary softwares it is
modeled on, has a \quote{productionalist}
view of design built into it\footnote{\quotation{A DTP
program is the equivalent of a final assembly in an industrial
process} Christoph~Sch\"afer, Gregory~Pittman et
al. {\em The Official Scribus Manual.}Fles Books, 2009}, which is
undeniably seeping through in the way we use it. An exotic Free
Software tool like \TeX, rooted firmly in an academic context rather
than in commercial design, might help us to re-imagine the familiar
skill of putting type on a page. By making this kind of
\quote{domain shift} \footnote{
See: Richard
Sennett. {\em The Craftsman}. Allen Lane (Penguin Press), 2008} we
hope to discover another experience of making, and find a more
constructive relation between software, content and form. So when Pierre
suggests that this V/J10 publication is possibly the right occasion to
try, we respond with enthusiasm.

By the end of 2008, Pierre starts carving out a path in the dense forest
of manuals, advice, tips-and-tricks with the help of Ivan Monroy Lopez.
Ivan is trained as mathematician and more or less familiar with the
exotic culture of \TeX. They decide to use the popular macro-package
\LaTeX\footnote{\LaTeX\ is a high-level markup language that was first
developed by Leslie Lamport in 1985. Lamport is a computer scientist
also known for his work on distributed systems and multi-treading
algorithms.} to interface with \TeX\ and find out about the tong-in-cheek
concept of \quote{badness} (depending on the
tension put on hyphenated paragraphs, compiling a .tex document
produces \quote{badness} for each block on a
scale from 0 to 10.000), and encounter a long history of wonderful but
often incoherent layers of development that envelope the mysterious
lasagna beauty of \TeX's typographic algorithms.

Laying-out a publication in \LaTeX\ is an entirely different experience
than working with a canvas-based software. First of all, design
decisions are executed through the application of markup which vaguely
reminds of working with CSS or HTML. The actual design is only complete
after \quote{compiling} the document, and
this is where \TeX\ magic happens. The software passes several times over
a marked up .tex file, incrementally deciding where to hyphenate a
word, place a paragraph or image. In principle, the concept of a page
only applies after compilation is complete. Design work therefore
radically shifts from the act of absolute placement to co-managing a
flow. All elements remain relatively placed until the last {\em 
tour} has passed, and while error messages, warnings and hyphenation
decisions scroll by on the command line, the sensation of elasticity is
almost tangible. And indeed, when within the acceptable
\quote{stretch} of the program placement of
a paragraph is exceeded, words literally break out of the grid (see
page 34 example).

When I join Pierre to continue the work in January 2009, the book is
still far from finished. By now, we can produce those typical
academic-style documents with ease, but we still have not managed to
use our own fonts\footnote{\quotation{Installing fonts
in \LaTeX\ has the name of being a very hard task to accomplish. But it
is nothing more than following instructions. However, the problem is
that, first, the proper instructions have to be found and, second, the
instructions then have to be read and understood}.
\Url{http://www.ntg.nl/maps/29/13.pdf}}. Flipping back and forth in the many
manuals and handbooks that exist, we enjoy discovering a new culture.
Though we occasionally cringe at the paternalist humour that seems to
have infected every corner of the \TeX\ community and which is clearly
inspired by witticisms of the founding father, Donald Knuth himself, we
experience how the lightweight, flexible document structure of \TeX\
allows for a less hierarchical and non-linear workflow, making it
easier to collaborate on a project. It is an exhilarating experience to
produce a lay-out in dialogue with a tool and the design process
takes on an almost rhythmical quality, iterative and incremental. It
also starts to dawn on us, that {\em souplesse} comes with a price.


\quotation{Users only need to learn a few
easy-to-understand commands that specify the logical structure of a
document} promises {\em The Not So Short
Introduction to \LaTeX}. \quotation{They almost never need to
tinker with the actual layout of the document}. It
explains why using \LaTeX\ stops being easy-to-understand once you
attempt to expand its strict model of
\quote{book},
\quote{article} or
\quote{thesis}: the
\quote{users} that \LaTeX\ addresses are not designers
and editors like us. At this point, we doubt whether to give up or push
through, and decide to set ourselves a limit of a week in which we
should be able to to tick off a minimal amount of items from a list of
essential design elements. Custom page size and headers, working with
URL's... they each require a separate
\quote{package} that may or may not be
compatible with another one. At the end of the week, just when we start
to regain confidence in the usability of \LaTeX\ for our purpose, our
document breaks beyond repair when we try to use custom paper size
with custom headers at the same time.

In February, more than 6 months into the process, we briefly consider
switching to OpenOffice instead (which we had never tried for such a
large publication) or go back to Scribus (which means for Pierre,
learning a new tool). Then we remember \ConTeXt, a relatively young
\quote{macro package} that uses the \TeX\ engine
as well. \quotation{While \LaTeX\ insulates the writer
from typographical details, \ConTeXt\ takes a complementary approach by
providing structured interfaces for handling typography, including
extensive support for colors, backgrounds, hyperlinks, presentations,
figure-text integration, and conditional
compilation}\footnote{Interview with Hans Hagen
\Url{http://www.tug.org/interviews/interview-files/hans-hagen.html}}. This is
what we have been looking for.

\ConTeXt\ was developed in the 1990's by a Dutch company
specialised in \quote{Advanced Document
Engineering}. They needed to produce complex
educational materials and workplace manuals and came up with their own
interface to \TeX. \quotation{The development was purely
driven by demand and configurability, and this meant that we
could optimize most workflows that involved text
editing}.\footnote{Interview with Hans Hagen
\Url{http://www.tug.org/interviews/interview-files/hans-hagen.html}}

However frustrating it is to re-learn yet another type of markup (even
if both are based on the same \TeX\ language, most of the \LaTeX\ commands
do not work in \ConTeXt\ and vice versa), many of the things that we
could only achieve by means of \quote{hack} in
\LaTeX, are built in and readily available in \ConTeXt. With the help of
the very active \ConTeXt\ mailinglist we find a way to finally use our
own fonts and while plenty of questions, bugs and dark areas remain, it
feels we are close to producing the kind of multilingual, multi-format,
multi-layered publication we imagine {\em Tracks in Electr(on)ic
Fields} to be.

However, Pierre and I are working on different versions of Ubuntu,
respectively on a Mac and on a PC and we soon discover that our
installations of \ConTeXt\ produce different results. We
can't find a solution in the nerve-wrackingly
incomplete, fragmented though extensive documentation of \ConTeXt\ and by
June 2009, we still have not managed to print the book. As time passes,
we find it increasingly difficult to allocate concentrated time for
learning and it is a humbling experience that acquiring some sort of
fluency seems to pull us in all directions. The stretched out nature of
the process also feeds our insecurity: Maybe we should have tried this
package also? Have we read that manual correctly? Have we read the
right manual? Did we understand those instructions really? If we were
computer scientists ourselves, would we know what to do? Paradoxically,
the more we invest into this process, mentally and physically, the
harder it is to let go. Are we refusing to see the limits of this tool,
or even scarier, our own limitations? Can we accept that the experience
we'd hoped for, is a lot more banal than the sublime
results we secretly expected? A fellow Constant member suggests in
desperation: \quotation{You can't just
make a book, can you?}


In July, Pierre decides to pay for a consult with the developers of
\ConTeXt\ themselves, and once and for all solve some of the issues we
continue to struggle with. We drive up expectantly to the headquarters
of Pragma in Hasselt (NL) and discuss our problems, seated in the
recently redecorated rooms of a former bank building. Hans Hagen
himself reinstalls markIV (the latest in \ConTeXt) on the machine of
Pierre, while his colleague Ton Otten tours me through samples of the
colorful publications produced by Pragma. In the afternoon, Hans
gathers up some code examples that could help us place thumbnail images
and before we know it we are on our way South again. Our visit confirms
the impression we had from the awkwardly written manuals and peculiar
syntax, that \ConTeXt\ is in essence a one man mission. It is hard to
imagine that a tool written to solve particular problems of a certain
document engineer, will ever grow into the kind of tool that we desire
too as well.

In August, as I type up this report, the book is more or less ready to
go to print. Although it looks
\quote{handsome} according to some, due to
unexpected bugs and time restraints, we have had to let go of some of
the features we hoped to implement. Looking at it now, just before
going to print, it has certainly not turned out to be the kind of
eye-opening typographic experience we dreamt of and sadly, we will
never know whether that is due to our own limited understanding of \TeX,
\LaTeX\ and \ConTeXt, to the inherent limits of those tools themselves, or
to the crude decision to finally force through a lay-out in two weeks.
Probably a mix of all of the above, it is first of all a relief that
the publication finally exists. Looking back at the process, I am
reminded of the wise words of Joseph Weizenbaum, who observed that
\quotation{Only rarely, if indeed ever, are a tool and
an altogether original job it is to do, invented
together}\footnote{Joseph Weizenbaum.
{\em Computer power and human reason: from judgment to calculation}.
MIT, 1976}.

While this book nearly crumbled under the weight of the
projections it had to carry, I often thought that outside academic
publishing, the power of \TeX\ is much like a Fata Morgana. Mesmerizing
and always out of reach, \TeX\ continues to represent a promise of an
alternative technological landscape that keeps our dream of changing
software habits alive.
\blank
\blank
Femke Snelting (OSP), August 2009

