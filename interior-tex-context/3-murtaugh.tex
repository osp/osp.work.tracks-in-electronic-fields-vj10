\PlaceFramedImage{murtaugh0.jpg}{Start broadcasting yourself!}
\PlaceFramedImage{murtaugh1.jpg}{Join the largest worldwide video-sharing community!}

\AuthorStyle{Michael Murtaugh}

\licenseStyle{Free Art License}

\Eng{\Title{Active Archives}
\SubTitle{or: What's wrong with the YouTube documentary?}

As someone who has shot video and programmed web{}-based interfaces to
video over the past decade, it has been exciting to see how
distributing video via the Internet has become increasingly
popularized, thanks in large part to video sharing sites like YouTube.
At the same time, I continue to design and write software in search of
new forms of collaborative and
\quote{evolving} documentaries; and for
myself, and others around me, I feel disinterest, even aversion, to
posting videos on YouTube. This essay has two threads: (1) I revisit an
earlier essay describing the \quote{Evolving
Documentary} model to get at the roots of my
enthusiasm for working with video online, and (2) I examine why I find
YouTube problematic, and more a reflection of television than the
possibilities that the web offers.

In 1996, I co{}-authored an essay with Glorianna Davenport, then my
teacher and director of the Interactive Cinema group at the MIT Media
Lab, called {\em Automatist storyteller systems and the
shifting sands of story}. \footnote{\Url{http://www.research.ibm.com/journal/sj/363/davenport.html}} In it, we
described a model for supporting \quote{Evolving
Documentaries}, or an \quotation{approach to
documentary storytelling that celebrates electronic narrative as a
process in which the author(s), a networked presentation system, and
the audience actively collaborate in the co{}-construction of
meaning.} In this paper, Glorianna included a section
entitled \quote{What's wrong with the
Television Documentary?} The main points of this
argument were as follows:

\page

\SubSubTitle{1.}

\QuoteStyle{[... T]elevision consumes the viewer. Sitting
passively in front of a TV screen, you may appreciate an hour{}-long
documentary; you may even find the story of interest; however, your
ability to learn from the program is less than what it might be if you
were actively engaged with it, able to control its shape and probe its
contents.}

Here, it is crucial to understand what is meant by the word\\
\quote{active}. In a naive comparison
between the activities of watching television and surfing the web, one
might say that the latter is inherently more active in the sense that
the process is \quote{driven} by the
choices of the user; in the early days of the web it became popular to
refer to this split as \quote{lean back vs. lean
forward} media. Of course, if one means to talk about
cognitive activity, this is clearly misleading as aimlessly surfing the
net can be achieved at near comatose levels of brain function (as any
late night surfer can attest to) and watching a particularly sharp
television program can be incredibly engaging, even life changing.
Glorianna would often describe her frustration with traditional
documentary by observing the vast difference between her own sense of
engagement with a story gained through the process of shooting and
editing, versus the experience of an audience member from simply
viewing the end result. Thus \quote{active}
here relates to the act of authoring and the construction of meaning.
Rather than talking about leaning forward or backward, a more useful
split might be between reading and writing. Rather than being a
question of bad versus good access, the issue becomes about two
interconnected cognitive processes, both hopefully
\quote{active} and involving thought. An
ideal platform for online documentary would be one that facilitates a
fluid movement between moments of reflection (reading) and of
construction (writing).

\page

\SubSubTitle{2.}

\QuoteStyle{Television severely limits the ways in which an author
can \quote{grow} a story. A story must be
composed into a fixed, unchanging form before the audience can see and
react to it: there is no obvious way to connect viewers to the process
of story construction. Similarly, the medium offers no intrinsic,
immediately available way to interconnect the larger community of
viewers who wish to engage in debate about a particular
story.}

Part of the promise of crossing video with computation is the potential
to combine the computers' ability to construct models
and run simulations with the random access possibilities of digitized
media. Instead of editing a story down into a fixed form or
\quote{final cut}, one can program a
\quote{storytelling system} that can act as
an \quote{editor in software}. Thus the
system can maintain a dynamic representation of the context of a
particular telling, on which to base (or support a viewer in making)
editing decisions \quote{on the fly}. The
\quote{Evolving Documentary} was intended
to support complex stories that would develop over time, and which
could best be told from a variety of points of view.

\SubSubTitle{3.}

\QuoteStyle{Like published books and movies, television is
designed for unidirectional, one{}-to{}-many transmission to a mass
audience, without variation or personalization of presentation. The
remote{}-control unit and the VCR (videocassette recorder) {}-
currently the only devices that allow the viewer any degree of
independent control over the play{}-out of television {}- are
considered anathema by commercial broadcasters. Grazing,
time{}-shifting, and \quote{commercial zapping} run contrary to the desire of the industry for a demographically
correct audience that passively absorbs the programming {}- and the
intrusive commercial messages {}- that the broadcasters
offer.}

Adding a decentralized means of distribution and feedback such as the
Internet provides the final piece of the puzzle in creating a
compelling new medium for the evolving documentary. No longer would
footage have to be excluded for reasons of reaching a
\quote{broad} or average audience. An ideal
storytelling system would be one that could connect an individual
viewer to whatever material was most personally relevant. The Internet
is a unique \quote{mass media} in its
potential support for enabling access to non{}-mainstream, individually
relevant and personal subject matter.

\SubSubTitle{What's wrong with the YouTube documentary?}

YouTube has massively popularized the sharing and consumption of video
online. That said, most of the core concerns made in the arguments
related to television, are still relevant to YouTube when considered as
a platform for online collaborative documentary.

\SubSubTitle{Clips are primarily \quote{view{}-only}}
Already in it's name, \quote{YouTube} consciously invokes the
television set, thus inviting visitors to \quote{lean
back} and watch. The YouTube interface functions
primarily as a showcase of static monolithic elements. Clips are
presented as fixed and finished, to be commented upon, rated, and
possibly bookmarked, but no more. The clip is
\quote{atomic} in the sense that
it's not possible to make selections within a clip, to
export images or sound, or even to link to a particular starting point.
Without special plugins, the site doesn't even allow
downloading of the clip. While users are encouraged \quote{to
embed} YouTube content in other
websites (by cutting and pasting special HTML codes that refer back to
the YouTube site), the resulting video plays using the YouTube player,
complete with \quote{related} links back
into the service. It is in fact a violation of the YouTube terms of use
to attempt to display videos from the service in any other way.

\page

\SubSubTitle{The format of the clip is fixed and uniform for all kinds
of content}

Technically, YouTube places some rather arbitrary limits on the format
of clips: all clips must contain an image and a sound track and may not
be longer than 10 minutes in length. Furthermore all clips are treated
equally, there is no notion of a
\quote{lecture}, versus a
\quote{slideshow}, versus a
\quote{music video}, together with a sense
that these different kinds of material might need to be handled
differently. Each clip is compressed in a uniform way, meaning at the
moment into a flash format video file of fixed data rate and screen
size.

\SubSubTitle{Clips have no history}

Despite these limitations, users of YouTube have found workarounds to,
for instance, download clips to then rework them into derived clips.
Although the derived works are often placed back again on YouTube, the
system itself has no means representing this kind of relationship.
(There is a mechanism for posting video responses to other clips, but
this kind of general purpose solution seems not to be understood or
used to track this kind of \quote{derived}
relationship.) The system is unable to model or otherwise make
available the \quote{history} of a
particular piece of media. Contrast this with a system like Wikipedia,
where the full history of an article, with a record of what was
changed, by whom, when, and even
\quote{meta{}-level} discussions about the
changes (including possible disagreement) is explicitly facilitated.

\SubSubTitle{Weak or \quote{flat} narrative structure}

YouTube's primary model for narrative is a broad (and
somewhat obscure) sense of
\quote{relatedness} (based on
user{}-defined tags) modulated by popularity. As with many
\quote{social networking} and media sharing
sites, YouTube relies on \quote{positive
feedback} popularity mechanisms, such as view counts,
\quote{star} ratings and favorites, to
create ranked lists of clips. Entry points like
\quote{Videos being watched right now},
\quote{Most Viewed}, \quote{Top
Favorites}, only close the loop of featuring
what's already popular to begin with. In addition,
YouTube's commercial model of enabling special paid
levels of membership leads to ambiguous selection criteria, complicated
by language as in the \quote{Promoted
Videos} and \quote{Featured
Videos} of YouTube's front page
(promoting what?, featured by whom?).

The \quote{editing logic} threading the user
through the various clips is flat, in that a clip is shown the same way
regardless of what has been viewed before it. Thus YouTube makes no
visible use of a particular viewing history (though the fact that this
information is stored has been brought to the attention of the public
via the ongoing Viacom lawsuit,
\Url{http://news.bbc.co.uk/2/hi/technology/7506948.stm}). In this way
it's difficult to get a sense of being in a particular
\quote{story arc} or thread when moving
from clip to clip in YouTube as in a sense each click and each clip
restarts the narrative experience.

\SubSubTitle{No licenses for sharing / reuse}

The lack of a download feature in YouTube could be said to protect the
interests of those who wish to assert a claim of copyright. However,
YouTube ignores and thus obscures the question of license altogether.
One can find for instance the early films of Hitchcock, now part of the
public domain, in 10 minute chunks on YouTube; despite this status (not
indicated on the site), these clips are, like all YouTube clips,
unavailable for any kind of manipulation. This approach, and the
limitations it places on the use of YouTube material, highlights the
fact that YouTube is primarily focused on getting users to consume
YouTube material, framed in YouTube's media player, on
YouTube's terms.

\SubSubTitle{Traditional models for (software) authorship}

While YouTube is built using open source software (Python and ffmpeg for
instance), the source code of the system itself is closed, leaving
little room for negotiation about how the software of the site itself
operates. This is a pity on a variety of levels. Free and open source
software is inextricably bound to the web not only in terms of
providing many of the underlying software (like the Apache web server),
but also in the reverse, as the possibilities for collaborative
development that the web provides has catalyzed the process of open
source development. Software designed to support collaborative work on
code, like Subversion and other CVS's (concurrent
versioning systems), and platforms for tracking and discussing software
(like TRAC), provide much richer models of use and relationship to work
than those which YouTube offer for video production.

\SubSubTitle{Broadcasting over coherence}
From it's slogan (\quote{Broadcast
yourself}), to the language the service uses around
joining and uploading videos (see images), YouTube falls very much into
a traditional model of commercial broadcast television. In this model
sharing means getting others to watch your clips, with the more
eyeballs the better.

The desire for broadness and the building of a
\quote{worldwide} community united only by
a desire to \quote{broadcast one's
self} means creating coherence is not a top priority.
YouTube comments, for instance, seem to suffer from this lack of
coherence and context. Given no particular focus, comments seem doomed
to be similarly ungrounded and broad. Indeed, comments in YouTube often
seem to take on more the character of public toilets than of public
broadcasting, replete with the kind of sexism, racism, and homophobia
that more or less anonymous \quote{blank
wall} access seems to encourage.

\SubSubTitle{A problematic space for \quote{sharing}}

The combination of all these aspects make YouTube for many a problematic
space for \quote{sharing} {}- particularly
when the material is of a personal or particular nature. While on the
one hand appearing to pose an alternative platform to television,
YouTube unfortunately transposes many of that form's
limitations and conventions onto the web.

Looking to the future, what still remains challenging, is figuring out
how to fuse all those aspects that make the Internet so compelling as a
medium and enable them in the realm of online video: the
net's decentralized nature, the possibilities for
participatory/collaboration production, the ability to draw on diverse
sources of knowledge (from \quote{amateur}
and home{}-based, to \quote{expert}). How
can the successful examples of collaborative text{}-based projects like
Wikipedia inspire new forms of collaborative video online; and in a way
that escapes the \quote{heaviness} and
inertia of traditional forms of film/video. This fusion can and needs
to take place on a variety of levels, from the concept of what a
documentary is and can be, to the production tools and content
management systems media makers use, to a legal status of media that
reflects an understanding that culture is something which is shared,
down to the technical details of the formats and codecs carrying the
media in a way that facilitates sharing, instead of complicating it.}
