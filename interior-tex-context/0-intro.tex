{\tfa\setupinterlinespace \Ned{Sporen in het elektr(on)ische veld {\em documenteert de gelijknamige 10\high{e} editie van Verbindingen/Jonctions, het multidisciplinaire festival dat Constant, vereniging voor kunst en media, elke twee jaar organiseert. Het is een ontmoetingsmoment voor mensen die vanuit een artistiek, activistisch en/of theoretisch perspectief ge\"interesseerd zijn in experimentele reflecties op technologische cultuur.

Niet voor het eerst, maar tijdens de 10\high{e} editie explicieter dan ooit, lag de vraag naar de wisselwerking tussen lichaam en technologie op tafel. Hoe na te denken over de werkelijke effecten van surveillance, de alomaanwezigheid van camera's en maatschappelijke veiligheidsvoorschriften die individuen processen alsof het niets anders dan analyseerbare data zijn? Wat is de status van \quote{identiteit} wanneer deze tegelijkertijd ongrijpbaar en onveranderbaar blijkt te zijn? Hoe worden wij geconditioneerd door de technologie die we gebruiken? Wat is de relatie tussen toewijding en beloning? Flexibiliteit van werken en gezond leven? Welke sporen laat die technologie na in ons denken, gedrag, onze routineuze bewegingen? En welke residuen laten we z\'elf na op elektr(on)ische velden middels onze aanwezigheid op en in fora, sociale platformen, databases, logfiles?

Het dubbelkarakter van het begrip \quote{notatie} vormde een belangrijke bron van inspiratie. Systemen die choreografen, componisten \'en computerprogrammeurs gebruiken om idee\"en en waarnemingen vast te leggen, kunnen vervolgens worden opgevat als aanwijzing om een acteur, software, uitvoerend artiest of \ machine in beweging te zetten. Van ponskaart tot notenbalk, van programmeertaal tot Laban notatie, we interesseerden ons voor standaarden en protocollen die nodig zijn om zo'n document \quote{uitvoerbaar} te maken. Het was de aanleiding om het festival onder te brengen in de documentatiecentrum, bibliotheek en werkplaats voor theater en dans, \quote{maison du spectacle} La Bellone. Dit minutieus gerenoveerde 17\high{e} eeuwse paleisje in het hart van Brussel bood gastvrijheid aan een pluriforme groep denkers, dansers, kunstenaars, programmeurs, interface designers en andere specialisten en vormde het perfecte decor voor dit intense programma.

Doorheen het festival liepen een aantal thema's, niet bedoeld als afbakening en inperking van deelgebieden, maar eerder als \quote{spindraden} die de projecten met elkaar verbonden:} 

E-traces {\em (p. 35) onderwierp de huidige realiteit van het Web 2.0 aan een aantal kritische overwegingen. Hoe heroveren we grip op de overvloedige data-correlatie die we in ruil voor de diensten van mega-bedrijven als Google en Yahoo produceren? Hoe begrijpen we \quote{service} als we aangestaard worden door de corporate Januskop: enerzijds bestaand uit een vriendelijke interface, anderzijds uit machiavellistische gebruikerslicenties? } 

Om ons heen resoneren ongeziene golven {\em (p. 77) nam de spookachtige aanwezigheid van technologie als vertrekpunt en {\em Read Feel Feed Real} (p. 101) luisterde naar ongehoorde geluiden en keek achter de gordijnen in Doe-Het-Zelf, wandel- en stadsinterventies. Door radiogolven te ontleden en te gebruiken in artistieke installaties, door elektro-magnetische velden hoorbaar te maken, werden onverklaarbare fenomenen hanteerbaar gemaakt.

Terwijl machines leren over lichamen, leren lichamen over machines en de bewegingen die er uit ontstaan zijn niet gemakkelijk te herleiden tot oorzaak en gevolg.} Wederkerige bewegingen {\em (p. 139) begon in de keuken, de uitgelezen plek om mens-machine configuraties te heroverwegen, zonder ze los te zien van het dagelijkse leven en de rollen die daarin een rol spelen. Zou een ander idee over \quote{gebruiker} ook onze benadering van \quote{gebruik} kunnen veranderen?

Aan het einde van het avontuur zei Sadie Plant in haar \quote{gesitueerde rapport} over} Sporen in het elektr(on)ische veld {\em (p. 275): 'Het is uiteindelijk zeer moeilijk om onderscheid te maken tussen de gebruiker en de ontwikkelaar, of tussen de deskundige en de amateur. Het experiment, het onderzoek, de ontwikkeling gebeurt voortdurend in de keuken, in de slaapkamer, op de bus, met behulp van je mobiele telefoon of je computer. (...) Dit gevoel van zich herhalende bewegingen, die plaatsvinden in vele disciplines en langs verschillende lijnen, raken echt de diepe onbewuste geschiedenis van de menselijke activiteit. En misschien is dat wel waar de meest interessante ontwikkelingen gebeuren, zij het in een onbezongen, onzichtbare, vaak bijna verborgen manier. Het is dit soort van diepe collectiviteit, dit diepe gevoel van micro-samenwerking, die hier werd aangeboord.'
\blank
Constant, oktober 2009}

}

\page

\Eng{Traces in electr(on)ic fields {\em documents the 10\high{th} edition of Verbindingen/Jonctions with the same name, a bi-annual multidisciplinary festival organised by Constant, association for arts and media. It is a meeting point for a diverse public that from an artistic, activist and / or theoretical perspective is interested in experimental reflections on technological culture.

Not for the first time, but during this edition more explicit than ever, we put the question of the interaction between body and technology on the table. How to think about the actual effects of surveillance, the ubiquitous presence of cameras and public safety procedures that can only regard \ individuals as an amalgamate of analysable data? What is the status of \quote{identity} when it appears both elusive and unchangeable? How are we conditioned by the technology we use? What is the relationship between commitment and reward? Flexibility of work and healthy life? Which traces does technology leave in our thinking, behavior, our routine movements? And what residue do we leave behind ourselves on electr(on)ic fields through our presence in forums, social platforms, databases, log files?

The dual nature of the term \quote{notation} formed an important source of inspiration. Systems that choreographers, composers and computer programmers use to record ideas and observations, can then be interpreted as instruction, as a command which puts an actor, software, performing artist or machine in to motion. From punch card to musical scale, from programming language to Laban notation, we were interested in the standards and protocols needed to make such documents work. It was the reason to organise the festival inside the documentation, library and workshop for theater and dance, \quote{maison du spectacle} La Bellone. Located in the heart of Brussels, La Bellone offered hospitality to a diverse group of thinkers, dancers, artists, programmers, interface designers and others and its meticulously renovated 17\high{th} century fa\c{c}ade formed the perfect backdrop for this intense program.

Throughout the festival we worked with a number of themes, not meant to isolate areas of thinking, but rather as \quote{spider threads} interlinking various projects: }

E-traces {\em (p. 35) subjected the current reality of Web 2.0 to a number of critical considerations. How do we regain control of the abundant data correlation that mega-companies such as Google and Yahoo produce, in exchange for our usage of their services? How do we understand \quote{service} when we are confronted with their corporate Janus face: one a friendly interface, the other Machiavellian user licenses?}

Around us, magnetic fields resonate unseen waves {\em (p. 77) took the ghostly presence of technology as a starting point and} Read Feel Feed Real {\em (p. 101) listened to unheard sounds and looked behind the curtains in Do-It-Yourself, walks and urban interventions. Through the analysis of radio waves and their use in artistic installations, by making electro-magnetic fields heard, we made unexplained phenomena tangible. 

As machines learn about bodies, bodies learn about machines and the movements that emerge as a result, are not readily reduced to cause and effect.} Mutual movements {\em (p. 139) started in the kitchen, the perfect place to reconsider human-machine configurations, without having to separate these from everyday life and the patterns that are ingrained in it. Would a different idea of \quote{user} also change our approach to \quote{use}? 

At the end of the adventure Sadie Plant remarked in her \quote{situated report} on }Tracks in electr(on)ic fields {\em (p. 275): \quotation{It is ultimately very difficult to distinguish between the user and the developer, or the expert and the amateur. The experiment, the research, the development is always happening in the kitchen, in the bedroom, on the bus, using your mobile or using your computer. (...) this sense of repetitive activity, which is done in many trades and many lines, and that really is the deep unconscious history of human activity. And arguably that's where the most interesting developments happen, albeit in a very unsung, unseen, often almost hidden way. It is this kind of deep collectivity, this profound sense of micro-collaboration, which has often been tapped into.}
\blank
Constant, October 2009}

}

\godown[3em]

\Fra{Empreintes dans les champs \'electr(on)iques {\em documente la dixi\`eme \'edition de Verbindingen/Jonctions, le festival multidisciplinaire bi-annuel organis\'e par l'association Constant, pour les Arts et Media. Ce festival est un lieu de rencontre pour un public vari\'e s'int\'eressant \`a une r\'eflexion exp\'erimentale sur la culture technologique men\'ee par des artistes, des activistes et des th\'eoricien/ne/s.

Cette fois plus encore que dans les \'editions pr\'ec\'edentes, nous avons plac\'e la question de l'interaction entre le corps et la technologie au centre du d\'ebat. Comment penser les effets de la surveillance, de la toute pr\'esence des cam\'eras et des mesures de s\'ecurti\'e dans l'espace public, qui ne per\c{c}oivent les individus que comme des amalgames de donn\'ees \`a analyser? Quel est le statut de l'identit\'e lorsqu'elle appara\^it \`a la fois insaisissable et fig\'ee? Comment sommes-nous conditionn\'es par les technologies que nous utilisons? Quelle est la relation entre l'engagement et la r\'ecompense? La flexibilit\'e du travail et une vie saine? Quelles traces laisse la technologie dans notre pens\'ee, notre comportement et nos d\'eplacements routiniers? Et quels r\'esidus laissons-nous derri\`ere nous dans les champs \'electr(on)iques par notre participation aux forums, plateformes sociales, bases de donn\'ees et fichiers de log?

La double nature du terme \quote{notation} forme une part importante de notre inspiration. Les syst\`emes que les chor\'egraphes, composit/eur/rice/s et les programmeu/r/se/s utilisent pour enregistrer leurs id\'ees et observations peuvent \^etre interpr\'et\'es comme une s\'erie d'instructions, \ de commandes qui peuvent mettre un/e act/eur/rice, un logiciel, un/e performeu/r/se ou une machine en mouvement. De la carte perfor\'ee \`a la partition musicale, du langage de programmation \`a la notation Laban, nous nous sommes int\'eress\'es aux standards et aux protocoles n\'ecessaires qui \ rendent ces documents op\'erationnels. Ce fut la raison pour laquelle ce festival a \'et\'e organis\'e dans la maison du spectacle, La Bellone, centre de documentation, biblioth\`eque et atelier pour le th\'e\^atre et la danse. Situ\'ee au centre de Bruxelles, La Bellone a accueilli un groupe de penseur/se/s, danseu/r/se/s, artistes et programmeu/r/se/s, designers d'interface et autres enthousiastes et sa fa\c{c}ade du 17\high{\`eme}, m\'eticuleusement r\'enov\'ee, s'est trouv\'ee \^etre le parfait d\'ecor pour ce programme intense.

Tout au long du festival, nous avons travaill\'e sur une s\'erie de th\`emes qui comme les fils d'une toile d'araign\'ee ont tiss\'e des liens entre les diff\'erents projets:}

E-traces {\em (p.35) a soumis la r\'ealit\'e actuelle du Web 2.0 \`a un certain nombre de consid\'erations critiques. Comment reprendre le contr\^ole sur les donn\'ees corr\'el\'ees que des mega-compagnies comme Google et Yahoo produisent \`a propos de nous, en \'echange de leurs services? Que doit-on entendre par \quote{service} lorsque nous sommes confront\'es au visage de Janus de ces soci\'et\'es: d'un c\^ot\'e, une interface souriante et facile \`a utiliser, de l'autre une licence d'utilisation machiav\'elique?}

Autour de nous, les champs magn\'etiques font r\'esonner des ondes invisibles {\em (p.77) pris comme point de d\'epart la pr\'esence fant\^omatique de la technologie et} Read Feel Feed Real {\em (p.101) nous fit \'ecouter des sons inou\"is et regarder l'envers du d\'ecor le temps de ballades Do-It-Yourself et d'interventions urbaines. Gr\^ace \`a l'analyse des ondes radio et leur utilisation dans des installations artistiques, nous avons pu rendre audibles les champs \'electro-magn\'etiques. Nous avons rendus tangibles des ph\'enom\`enes inexpliqu\'es.

Comme les machines apprennent des corps, les corps apprennent des machines et les mouvements qui \'emergent en retour ne peuvent se r\'eduire \`a une relation de cause \`a effet.} Mouvements Mutuels {\em (p.139) ont commenc\'e dans la cuisine, le meilleur endroit pour reconsid\'erer les configurations humain/machine, sans avoir \`a les s\'eparer de la vie quotidienne et des motifs qui y sont ancr\'es. Est-ce qu'une id\'ee diff\'erente de l'utilisat/eur/rice pourrait changer notre approche de l'usage? 

A la fin de l'aventure, Sadie Plant notait dans son \quote{Situated Report} sur} Empreintes dans les champs \'electr(on)iques {\em (p.275): \quotation{Il est tr\`es difficile de distinguer l'utilisat/eur/trice du/de la d\'eveloppeu/r/se, l'expert/e de l'amat/eur/trice. L'exp\'erimentation, la recherche, le d\'eveloppement ont toujours lieu dans la cuisine, la chambre \`a coucher, le bus, en utilisant votre t\'el\'ephone portable ou votre ordinateur. (...) cette sorte d'activit\'e r\'ep\'etitive, qui se fait de tant de mani\`eres diff\'erentes, et qui est r\'eellement l'histoire inconsciente profonde de l'activit\'e humaine. Et on peut soutenir que c'est l\`a que les d\'eveloppements les plus int\'eressants ont lieu, quoique d'une mani\`ere peu c\'el\'ebr\'ee, peu remarqu\'ee voire cach\'ee. C'est cette sorte de profonde collectivit\'e, cette forme riche de micro-collaboration, qui a souvent \'et\'e exploit\'ee.}
\blank

Constant, octobre 2009}

}

}