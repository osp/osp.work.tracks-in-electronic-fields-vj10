\PlaceImage{dusollier1.jpg}{Val\'erie Laure Benabou en S\'everine Dusollier discutent}
\PlaceImage{dusollier2.jpg}{}

\AuthorStyle{Val\'erie Laure Benabou, S\'everine Dusollier}

\licenseStyle{Creative Commons Attribution{}-NonCommercial{}-ShareAlike}

\Fra{\Title{Du droit d'auteur sur les mouvements,\crlf de l'interpr\'etation du droit d'auteur}

\SubTitle{Transcription}

\Interview{S\'everine Dusollier (SD)}Merci. Vous l'avez remarqu\'e sans doute: on
s'est mises face \`a face, non pas pour un match ou un combat d'\'echec
ou je ne sais quoi{\dots} L'id\'ee c'\'etait plut\^ot que, quand on a
commenc\'e \`a discuter du sujet {--} est{}-ce que le droit d'auteur
prot\`ege les mouvements? {--} en fait on n'est pas du tout arriv\'ees
\`a une conclusion tr\`es fixe, mais on se renvoyait tout le temps des
questions, on \'etait plut\^ot dans un mouvement de dialectique entre
est{}-ce qu'on est d'accord, est{}-ce qu'on n'est pas d'accord avec ces
propositions; comment est{}-ce qu'on est d'accord? Et donc on a
d\'ecid\'e de plut\^ot vous le pr\'esenter dans ce sens{}-l\`a,
c'est{}-\`a{}-dire de suivre une s\'erie de questions qui vont tenter
de structurer nos propres r\'eflexions par rapport \`a la question
principale: est{}-ce que le droit d'auteur peut prot\'eger les
mouvements et de quelle mani\`ere? On va essayer de mener plut\^ot une
interrogation crois\'ee. On n'a pas voulu non plus nous mettre dans des
positions tout \`a fait extr\^emes o\`u l'une serait pour et l'autre
serait contre, parce que nos propres positions sont mixtes, je crois,
et absolument pas certaines ni stables. On va ainsi se passer la parole
de mani\`ere \`a parfois prendre la position contraire sans forc\'ement
qu'on soit tout le temps dans la m\^eme posture; et la prise de parole
va aussi se faire de mani\`ere ludique, {--} en tout cas c'est ce qu'on
va tenter {--} sur base de certains mouvements, pour indiquer qu'on va
parfois retourner, renverser le discours. Voir \Url{http://www.doly.com/ici.d\&a.htm}\par

Venons{}-en au sujet: du droit d'auteur sur les mouvements{\dots} La
premi\`ere question qu'on voulait se poser est la suivante: de quels
mouvements parlons{}-nous, quels sont les mouvements qui pourraient
\^etre susceptibles d'\^etre prot\'eg\'es par le droit d'auteur?\par

\Interview{Val\'erie{}-Laure Benabou (VLB)}On est parti d'une id\'ee qu'il y avait
plusieurs \ cat\'egories de mouvements, dont certains pouvaient \^etre
prot\'eg\'es par le droit d'auteur, et d'autres qui pouvaient
\'eventuellement \^etre prot\'eg\'es, ou plus exactement \quote{r\'eserv\'es}
{--} c'est{}-\`a{}-dire faire l'objet d'une propri\'et\'e {--} par
d'autres instruments du droit, comme les brevets, comme le contrat,
comme \'eventuellement d'autres syst\`emes juridiques. Et on a essay\'e
d'avoir une cl\'e de r\'epartition entre ce qui pourrait relever du
droit d'auteur et ce qui rel\`everait des autres droits de la
propri\'et\'e intellectuelle. Et il y a une premi\`ere distinction qui
est apparue, c'est une distinction entre {\em l'utile} et {\em le
beau}. Il y a des mouvements qui ont une valeur parce qu'ils sont
utiles, et il y a des mouvements qui ont une valeur parce qu'ils sont
beaux. Et les mouvements utiles ne vont pas forc\'ement donner prise au
m\^eme type de syst\`eme de r\'eservation que les mouvements qui sont
beaux. C'est un peu arch\'etypal, c'est un petit peu manich\'een comme
distinction, mais malgr\'e tout c'est une ligne de force qu'on peut
\'eventuellement envisager.\par

Dans les mouvements utiles, parce que les mouvements ont des finalit\'es
diff\'erentes, on va pouvoir avoir des mouvements m\'ecanis\'es, par
exemple le mouvement d'un objet, le mouvement d'un objet dans une
cha\^ine de production. Les mouvements des objets peuvent
\'eventuellement donner lieu \`a une certaine valeur, et cette valeur
pourrait, le cas \'ech\'eant, \^etre r\'eserv\'ee par des brevets. On
pourrait imaginer une invention dans laquelle ce qui est l'invention
c'est le mouvement, c'est un certain mouvement, dans une machine par
exemple, mouvement qui produit un effet technique particulier. Donc
l\`a il y a une possibilit\'e de r\'eservation du mouvement d'un objet
comme un effet induit et comme le r\'esultat d'une invention.\par

On a aussi la possibilit\'e d'envisager des protections ou des
r\'eservations, des syst\`emes de r\'eservation de certains mouvements
comme des mouvements manuels \`a travers des savoir{}-faire. Le
savoir{}-faire d'un artisan, d'un boulanger, d'une brodeuse, d'un
couturier, qui va exprimer, qui va exercer un certain nombre de
mouvements et qui va par la r\'ep\'etition de ses mouvements, conduire
\`a une expertise et un savoir{}-faire. Ce savoir{}-faire peut \^etre
prot\'eg\'e par le secret bien \'evidemment. C'est{}-\`a{}-dire que
tant que le mouvement est tu, tant qu'il est conserv\'e dans une petite
communaut\'e des individus, il va avoir une valeur \'economique
particuli\`ere, et cette valeur \'economique peut \^etre transmise par
des contrats, des contrats de savoir{}-faire. Et l\`a on est sur une
captation du mouvement, en ce qu'il est porteur d'une valeur
\'economique dans les r\'esultats qu'il produit: le bon pain, le bel
ouvrage, qui va \^etre le r\'esultat du mouvement. Mais, le mouvement
n'est pas envisag\'e {\em en tant que tel}, comme une {\em valeur
esth\'etique}, alors qu'il y a d'autres syst\`emes pour envisager la
valeur esth\'etique du mouvement.\par

\Interview{SD}On a travaill\'e plut\^ot sur le mouvement qui n'est pas
consid\'er\'e dans ses effets pratiques, mais qui est consid\'er\'e en
tant que tel, par rapport \`a sa pr\'etention artistique ou \`a sa
pr\'etention esth\'etique. Donc ce qu'on pourrait appeler {\em \quote{le
beau mouvement}}. Et en m\^eme temps, se concentrer sur le beau
mouvement est un peu trompeur en droit d'auteur, puisque le droit
d'auteur n'a pas pr\'etention de prot\'eger uniquement le beau, il peut
prot\'eger le laid aussi, le droit d'auteur \'etant indiff\'erent aux
consid\'erations de valeur esth\'etique ou de valeur artistique. Mais
quand on parle du \quote{beau} en droit d'auteur, ce qu'on essaie en fait de
prot\'eger c'est l'esth\'etisme, c'est la forme qui peut se
d\'evelopper dans un m\'edia litt\'eraire, artistique, visuel,... {--}
enfin, toutes les formes de m\'edia qui peuvent \^etre possibles. Et
c'est \'egalement trompeur \`a mon avis de faire cette distinction qui
a l'air tr\`es claire entre le beau et l'utile, parce qu'on va se
retrouver avec toute une s\'erie de mouvements qui vont se retrouver
\`a la {\em lisi\`ere} entre le beau et l'utile, tel le mouvement
sportif. Le mouvement sportif est sans doute utile, si on arrive \`a
mettre la balle dans le but, mais il peut \^etre aussi beau quand on
parle par exemple d'un mouvement d'une patineuse, entre prouesse
technique et valeur esth\'etique. Donc ces mouvements{}-l\`a, o\`u se
trouvent{}-ils dans notre division? Pensons \'egalement au mouvement
virtuel {--} et \quote{virtuel} est sans doute un mot qui est mal choisi
{--} le mouvement en tous cas qui se produit dans un environnement
virtualis\'e, digitalis\'e, dans ce cas l\`a il est \'evident que c'est
un mouvement qui prend une certaine forme. Mais, est{}-ce qu'on est
vraiment dans une forme qui veut \^etre une forme belle ou qui veut
surtout \^etre une traduction du personnage, de la personne dans un
univers virtuel? Donc l\`a aussi on va se retrouver avec des cas de
mouvements un peu limites, on ne va pas forc\'ement pouvoir les glisser
du c\^ot\'e de l'utile ou du c\^ot\'e du beau.\par

Mais peut{}-\^etre avant de rentrer vraiment dans le vif du sujet, il
serait utile de rappeler tr\`es vite ce qu'est le droit d'auteur et
comment il fonctionne, peut{}-\^etre pour ceux d'entre vous qui n'ont
aucune id\'ee ou qu'une id\'ee assez vague de ce qui est le droit
d'auteur. Le droit d'auteur c'est un droit qui prot\`ege des
cr\'eations dans le domaine litt\'eraire et artistique. C'est un droit
qui se distingue donc du droit qui touche plus \`a des productions
industrielles {--} comme le droit de brevets, le droit des marques
{--} pour viser vraiment la protection des formes artistiques.
L'int\'er\^et de ce qu'on voudrait voir avec vous c'est \`a partir de
quel moment le mouvement {\em entre} dans le droit d'auteur, sans
forc\'ement prendre parti sur l'opportunit\'e d'un droit d'auteur ou
l'opportunit\'e de la r\'eservation du r\'egime du droit d'auteur. Une
fois que le mouvement est entr\'e dans le droit d'auteur, le droit
d'auteur peut d\'eployer tout son arsenal de mesures protectrices, \`a
premi\`ere vue en tout cas plut\^ot d\'efensives. Lorsque vous avez un
droit d'auteur sur quelque chose {--} par exemple sur un mouvement
{--} vous allez pouvoir emp\^echer d'autres personnes de reproduire ce
mouvement, de refaire la m\^eme chose. Donc, il y a un r\^ole de
protection assez d\'efensif. Mais le droit d'auteur accorde avant tout
une esp\`ece de contr\^ole de la personne sur ce qu'il ou elle a
cr\'e\'e. Et donc forc\'ement, \c{c}a peut \^etre pour
{\em interdire}, mais \c{c}a peut \^etre aussi pour
{\em autoriser}, pour permettre \`a d'autres de reproduire, de
communiquer, c'est ce qu'on voit par exemple dans les logiciels libres,
dans les cr\'eations libres, o\`u le droit d'auteur est utilis\'e \`a
rebours non pas pour interdire, mais pour autoriser. En voulant poser
la question de la protection du mouvement par le droit d'auteur,
l'id\'ee n'est pas de pr\'etendre \`a un monopole sur les mouvements,
qui serait une interdiction pure de reproduction des mouvements, mais
plut\^ot de voir dans quelle mesure le droit d'auteur peut venir
s'accrocher au mouvement. En cons\'equence de cette protection par le
droit d'auteur, l'auteur, le titulaire de ce droit, a
le choix entre diff\'erentes strat\'egies d'exploitation de son
mouvement, entre interdiction et autorisation. L'autre int\'er\^et
\'egalement du droit d'auteur, c'est que le droit d'auteur accorde un
droit moral qui permet \`a l'auteur de se voir reconna\^itre la
paternit\'e de son {\oe}uvre (expression sexiste s'il en est), enfin
que l'{\oe}uvre en tout cas lui soit attribu\'ee, qu'il ou elle soit
reconnue comme cr\'eateur, comme cr\'eatrice, du mouvement dans ce
cas{}-ci, et \'egalement qu'il ou elle puisse
prot\'eger l'int\'egrit\'e de son {\oe}uvre. Le droit moral prot\`ege
plus le lien entre l'auteur et l'{\oe}uvre, le mouvement. Voil\`a en
tout cas quel est notre point de d\'epart, la question qu'on va poser:
est{}-ce que le mouvement entre dans la sph\`ere du droit d'auteur et
\`a partir de quel moment? Et pour \c{c}a je pense que la premi\`ere
question qu'on doit se poser est la suivante: {\em le mouvement
peut{}-il \^etre consid\'er\'e comme une {\oe}uvre?}\par

\Interview{VLB}Mais bien s\^ur, \'evidemment! Quelle question! Le mouvement est
bien \'evidemment prot\'eg\'e et r\'eserv\'e, c'est une {\oe}uvre!
Pourquoi est{}-ce une {\oe}uvre? Parce qu'il n'y a pas de d\'efinition
des {\oe}uvres, alors c'est facile de dire, \quotation{Pourquoi le mouvement ne
serait pas une {\oe}uvre, puisque je ne sais pas ce que c'est, une
{\oe}uvre}. L'{\oe}uvre, c'est une cr\'eation intellectuelle. Bien
s\^ur \quote{intellectuel}, qu'est ce que \c{c}a veut dire \quote{intellectuel}?
\c{C}a veut dire que quelqu'un a mis son sens, son c{\oe}ur, son \^ame,
son esprit au service d'un r\'esultat, il a exsud\'e une {\oe}uvre, il
a transpir\'e une {\oe}uvre, et l'{\oe}uvre est le r\'esultat de cette
personne, c'est le \quote{petit b\'eb\'e} de l'auteur. Donc, pourquoi
est{}-ce que le mouvement ne pourrait pas \^etre prot\'eg\'e par le
droit d'auteur? D\`es lors que l'auteur met l'empreinte de sa
personnalit\'e, d\`es lors qu'effectivement on peut voir dans le
mouvement l'auteur, que le mouvement est la traduction dans l'espace de
ce que veut l'auteur, de ce qu'a cr\'e\'e l'auteur, bien s\^ur qu'il
peut \^etre une {\oe}uvre, prot\'eg\'ee par la propri\'et\'e
intellectuelle, par le droit d'auteur. Le droit d'auteur est une sorte
d'aspirateur \`a prot\'eger, \`a r\'eserver {--} il n'a pas de seuil
tr\`es lourd, tr\`es \'elev\'e. La seule chose qu'on demande pour
\^etre prot\'eg\'e dans le droit d'auteur, ce n'est pas de d\'eposer,
ce n'est pas d'enregistrer, c'est de {\em cr\'eer}, de cr\'eer une
{\oe}uvre qui est {\em originale}. \quote{Originale}, qu'est{}-ce que
\c{c}a veut dire? \c{C}a veut dire que c'est une {\oe}uvre qui est
l'expression de la personnalit\'e de son cr\'eateur. La seule chose
qu'on demande, c'est qu'on soit dans le monde de la forme, qu'on ne
reste pas au niveau de l'id\'ee. Mais, le mouvement peut{}-il \^etre
r\'eduit \`a une simple id\'ee? Non, le mouvement est {\em toujours}
forme, le mouvement n'est jamais une simple id\'ee, il est toujours la
traduction dans un espace, dans un temps d'une id\'ee, mais il n'est
jamais l'id\'ee elle{}-m\^eme. La question ne se pose pas, \'evidemment
que le mouvement est prot\'eg\'e par le droit d'auteur.\par

Le mouvement est prot\'eg\'e par le droit d'auteur \`a la condition
d'\^etre propre, c'est{}-\`a{}-dire, d'incarner la d\'emarche de
cr\'eation de l'auteur. S'il n'y a pas cette conscience de l'auteur de
cr\'eer un mouvement, il faut peut{}-\^etre se poser la question de
savoir s'il y a une {\oe}uvre. Il faut peut{}-\^etre penser que seuls
les mouvements qui ont \'et\'e voulus comme {\oe}uvres pourraient
acc\'eder \`a cette cat\'egorie. Mais, est{}-ce qu'il suffit de vouloir
cr\'eer quelque chose, pour que le r\'esultat de cet acte de volont\'e, soit une {\oe}uvre?\par

\Interview{SD}Mais c'est justement l\`a qu'est la faille du raisonnement, parce
que le mouvement, dis{}-tu, est \quote{une expression}, qu'il n'est jamais
une simple id\'ee. Mais encore faut{}-il que cette expression soit
vraiment le r\'esultat d'une cr\'eation, qu'il y donc
{\em conscience} de la cr\'eation, conscience de l'{\oe}uvre en
train de se faire. Hors de nombreux mouvements, \`a mon sens, ne sont
pas une {\oe}uvre, ce sont des mouvements inconscients, des mouvements
qui ne sont que la traduction de gestes tout \`a fait n\'ecessaires, ou
de gestes tout \`a fait banals. Si mon nez me gratte, et je me gratte
le nez, je n'ai peut{}-\^etre m\^eme pas conscience du mouvement. En
quoi est{}-ce que je pourrais avoir un droit d'auteur sur ce mouvement,
dont je n'ai absolument aucune conscience? Si je traverse la pi\`ece en
prenant un pas particulier, c'est peut{}-\^etre mon propre pas, c'est
vrai qu'il m'est propre, il participe de {\em ma} d\'emarche, de ma
mani\`ere de marcher, mais est{}-ce que j'ai en m\^eme temps conscience
que je suis en train de cr\'eer une {\oe}uvre, et ai{}-je la
{\em volont\'e} de cr\'eer une {\oe}uvre? Donc, il me semble que le
mouvement ne peut pas \^etre per\c{c}u en tant qu'{\oe}uvre, surtout
les ex\'ecutions d'une action naturelle, il n'est pas forc\'ement
construit, il se construit peut{}-\^etre mais sans aucune volont\'e
d'\^etre aussi construit. Et si on reprend l'id\'ee qu'en tout cas, le
mouvement ne pourrait \^etre prot\'eg\'e que s'il est original, cela
laisserait beaucoup de mouvements hors de la protection, tels les
mouvements qui sont d'une banalit\'e affligeante: le fait de marcher,
le fait de s'asseoir, de se lever, de b\^ailler, de se frotter le nez,
de se frotter les yeux {--} ce sont des mouvements tout \`a fait
banals, qui donc ne pourront pas pr\'etendre \`a la protection du droit
d'auteur.\par

D\`es lors peut{}-\^etre pourrait{}-on imaginer de prot\'eger le
mouvement, mais tout sera alors une question de {\em degr\'e}.
\c{C}a signifierait qu'on essaierait non pas de prot\'eger un mouvement
isol\'e, mais de prot\'eger un mouvement dans une succession de
mouvements, un mouvement dans une expression plus large, puisque le
mouvement lui{}-m\^eme finalement ne serait qu'un \'el\'ement d'une
syntaxe corporelle, mais il serait compl\`etement arbitraire. Ce qu'on
essaierait de prot\'eger c'est plut\^ot la succession du mouvement, qui
\`a partir de cette syntaxe, du langage du corps, permettrait de
cr\'eer une {\oe}uvre. Et on va le voir tout \`a
l'heure sans doute, que le fait de cr\'eer, de mettre des mouvements
ensemble, par exemple dans une chor\'egraphie, peut parvenir \`a une
protection, mais qui alors viendrait investir la succession des
mouvements, l'expression qui ressort de diff\'erents mouvements mis
ensemble et pens\'es ensemble. La question essentielle sans doute en
droit d'auteur, n'est pas de dire que \quotation{Le mouvement ne pourra pas
\^etre prot\'eg\'e}, mais plut\^ot de pouvoir d\'eterminer {\em \`a
partir de quand} un mouvement devient une {\oe}uvre prot\'egeable par
le droit d'auteur.\par

\Interview{VLB}Mais, le mouvement ne peut pas \^etre prot\'eg\'e par le droit
d'auteur, parce que le mouvement c'est un {\em langage}, c'est un
{\em m\'edium}. Ce n'est pas \quote{une {\oe}uvre}, c'est un instrument
pour exprimer, mais ce n'est pas une expression, le r\'esultat de cette
expression. Ce n'est qu'une parcelle d'une {\oe}uvre, ce n'est qu'un
fragment d'une {\oe}uvre, le mouvement. Pourquoi? Parce que le
mouvement, il est indispensable \`a tous. Imaginez deux minutes,
S\'everine dit, \quotation{Je me gratte le nez} {--} c'est un mouvement commun.
Je le fais deux fois, je le fais trois fois, je le fais quatre fois,
cinq fois, six fois{\dots} \c{C}a devient une chor\'egraphie? \c{C}a
devient une {\oe}uvre prot\'eg\'ee? \c{C}a devient quelque chose sur
laquelle je peux dire \`a autrui: \quotation{Tu n'auras pas le droit de
reproduire ce mouvement sans mon autorisation?} Ce n'est pas s\'erieux.
On ne pourrait plus avoir le droit de se gratter le nez sans demander
l'autorisation du chor\'egraphe qui aurait fait un gratt\'e de nez de
vingt{}-cinq minutes de long?\par

On a dans la jurisprudence fran\c{c}aise, pas sur la chor\'egraphie,
mais sur les {\oe}uvres litt\'eraires, eu un d\'ebat int\'eressant~sur
la question suivante: \`a partir de quand est{}-ce qu'il y a une
{\oe}uvre litt\'eraire? Et notamment, il y a quelqu'un qui avait fait
un livre \`a partir du vocabulaire cajun, de la langue cajine de la
Louisiane, et quelqu'un d'autres avait repris des mots de ce livre, des
mots de la langue cajine, et l'auteur du premier livre l'avait
poursuivi en contrefa\c{c}on, en disant, \quotation{Mais vous avez repris les
mots.} Et l'autre de r\'epondre \quotation{Mais les mots de la langue cajine
existent dans la langue cajine.} Donc ce n'est pas parce que j'ai
\'ecrit un livre avec des mots qui existent dans une langue, que je
peux m'approprier ces mots l\`a. Ce n'est pas parce que je fais une
chor\'egraphie compos\'ee de plusieurs mouvements, que j'ai un droit
sur l'ensemble des mouvements, ou sur chacun de ces mouvements {--} ce
serait une violation totale de notre mobilit\'e, de notre droit \`a la
mobilit\'e, de notre libert\'e de mouvement. Et est{}-ce qu'une
succession, une simple succession de mouvements peut suffire \`a faire
na\^itre l'{\oe}uvre? Ce n'est pas s\'erieux; si je me gratte le nez et
puis je me gratte l'oreille, est{}-ce que je vais avoir un droit de
propri\'et\'e intellectuelle qui va pouvoir emp\^echer \`a autrui de se
gratter le nez et puis l'oreille ou de se gratter l'oreille et puis le
nez, puisque la contrefa\c{c}on va s'appr\'ecier en fonction de
ressemblances? Ce n'est pas s\'erieux. D'ailleurs, la cour d'appel de
Paris dans un arr\^et de 1967 avait dit: \quotation{Un pas de danse est par sa
nature et sa destination, soustrait \`a l'appropriation priv\'ee.} On
ne peut pas prot\'eger un seul pas de danse, on ne peut pas prot\'eger
{\em deux} pas de danses {--} imaginez que tout d'un coup, je dise:
\quotation{Tiens, je prot\`ege un pas de bourr\'ee.} Est{}-ce que quelqu'un
pourrait maintenant dire tous ceux qui voudront faire un pas de bourr\'ee et un assembl\'e apr\`es, ou
devront me demander l'autorisation? Ce n'est pas s\'erieux; ce n'est
pas s\'erieux!\par

\Interview{SD}Je trouve m\^eme \c{c}a tellement pas s\'erieux qu'alors l'id\'ee
qu'on est en train de d\'evelopper, c'est que le mouvement lui{}-m\^eme
ne pourra jamais \^etre prot\'eg\'e, que c'est juste la fixation du
mouvement qui pourrait \^etre prot\'eg\'ee. Donc, le fait qu'un moment,
le mouvement donne lieu \`a une autre {\oe}uvre, \`a une {\oe}uvre
fix\'ee, que ce soit par notation {--} dont on a d\'ej\`a parl\'e
durant ce week{}-end {--} \`a partir du moment o\`u l'{\oe}uvre est
fix\'ee dans une forme, \`a ce moment{}-l\`a c'est cette forme
elle{}-m\^eme qui va \^etre prot\'eg\'ee, mais pas le mouvement qui
constitue les diff\'erents \'el\'ements de cette forme. Qu'en
penses{}-tu?\par

\Interview{VLB}Je pense que si on parle du mouvement not\'e, du mouvement
fix\'e, alors ce n'est plus le mouvement qu'on prot\`ege: c'est une
{\em traduction} du mouvement. C'est une traduction dans un autre
langage d'un mouvement. C'est un signe. Mais c'est un signe
litt\'eraire, comme une note de musique, qui n'est pas la musique, qui
n'est pas r\'eduite \`a la musique. Parce que si je ne sais pas lire la
notation, je ne sais pas voir l'{\oe}uvre, je ne la comprends pas et
elle ne m'est plus perceptible, or la d\'efinition de l'{\oe}uvre
c'est qu'elle me soit perceptible au
sens. Si elle ne m'est pas intelligible, parce que
la {\em forme} dans laquelle elle aurait \'et\'e transcrite m'est inaccessible, je n'ai pas en face de moi une
{\oe}uvre, ou alors je n'ai pas une chor\'egraphie, j'ai une {\oe}uvre
litt\'eraire, que je peux lire ou que je peux ne pas comprendre, mais
simplement contempler dans sa forme \'ecrite.\par 

En m\^eme temps, ce n'est pas une condition du droit d'auteur, la
fixation. Jamais on a exig\'e en droit d'auteur de fixer pour
prot\'eger. Pourquoi fixer pour prot\'eger? L'acte de cr\'eation, il
existe dans son imm\'ediatet\'e, la protection nette du simple fait de
la cr\'eation. \'Evidemment j'aurais apr\`es un probl\`eme \`a prouver
que je suis bien l'auteur, parce que je ne peux pas prouver ce qui est
fugace, je ne peux pas prouver ce qui s'\'evapore, ce qui est immanent.
Mais ma protection existe du simple fait que je cr\'ee, je n'ai pas
besoin de fixer. Je n'ai pas besoin de noter pour avoir une protection
{--} ce serait r\'eduire l'improvisation improt\'egeable du simple fait
qu'elle \'emerge; elle ne peut pas demander pr\'ealablement une
fixation, ou a posteriori, une fixation. Il faut qu'il y ait de la
libert\'e de mouvement dans la cr\'eation.\par

\Interview{SD}Mais pourtant la loi fran\c{c}aise est ambigu\"e sur ce point,
parce que la loi fran\c{c}aise dit qu'elle ne prot\`egera les
chor\'egraphies que si elles {\em font} l'objet d'une fixation.
C'est donc bien la loi fran\c{c}aise qui exige ces conditions de
fixation pour les {\oe}uvres chor\'egraphiques. Mais, en m\^eme temps
cela pose une question quand m\^eme un petit peu emb\^etante:
c'est qu'il n'y a pas de raison de faire une
discrimination entre les diff\'erents types de cr\'eation en droit
d'auteur. Toutes les cr\'eations doivent \^etre prot\'eg\'ees de la
m\^eme mani\`ere. Or, il n'y a que pour les chor\'egraphies en effet
que la loi fran\c{c}aise en tous cas, {--} la loi belge est plus
accueillante {--} {\em demande} que l'{\oe}uvre soit fix\'ee. Sans
doute peut{}-on y voir surtout une question de preuve, de pouvoir un
moment au moins d\'elimiter les fronti\`eres de l'{\oe}uvre
chor\'egraphique en en ayant gard\'e une trace. Mais il va de soi que
le droit d'auteur prot\`ege les {\oe}uvres litt\'eraires et
artistiques, sans vouloir distinguer entre les cat\'egories
d'{\oe}uvres ou les formes, les m\'edia dans lesquels se d\'eveloppent
les {\oe}uvres. Donc il n'y a pas de raison finalement d'\^etre moins
accueillant pour les mouvements que pour les autres {\oe}uvres. On
prot\`ege toute une s\'erie de choses en droit d'auteur, on prot\`ege
m\^eme des logiciels {--} vous vous rendez compte {--} des logiciels
sont des {\oe}uvres soi{}-disant artistiques ou litt\'eraires. Donc il
va de soi que les mouvements pourraient tr\`es bien aussi pr\'etendre
\`a cette protection par le droit d'auteur. C'est ce qu'on appelle en
droit d'auteur \quote{la th\'eorie de l'unit\'e de l'art}, selon laquelle il
n'est pas question pour les juges appliquant le droit d'auteur, pour
les l\'egislateurs, de faire une distinction entre diff\'erentes
cr\'eations selon le type d'art auquel elles se rattachent. Ce qui
permet d'ailleurs des choses un petit peu absurdes, puisqu'on a pu
prot\'eger par le droit d'auteur {--} en France de nouveau {--} un
panier \`a salade, en disant que ce panier \`a salade \'etait une forme
d'art comme une autre. On voit bien que dans ce cas{}-l\`a \'evidemment
la th\'eorie de l'unit\'e de l'art et l'absence de discrimination
permettent de mettre dans le champ du droit d'auteur
un peu tout et n'importe quoi. Mais des questions plus pr\'ecises se
posent alors actuellement en droit d'auteur. Une des grandes questions
qui se pose par exemple c'est: \quotation{Peut{}-on prot\'eger un parfum par le
droit d'auteur?} Le parfum est une cr\'eation, il s'adresse au sens,
\`a l'odorat, un sens qui n'a pas encore \'et\'e tr\`es fort
sollicit\'e par des {\oe}uvres prot\'eg\'ees par le droit d'auteur.
Donc, si on envisage de prot\'eger le parfum, pourquoi
n'envisagerait{}-on pas de prot\'eger les mouvements?\par

\Interview{VLB}Parce que, ce serait, comme je l'ai dit toute \`a l'heure, un
obstacle \`a notre mobilit\'e. Ce n'est pas parce que le droit d'auteur
a une vocation expansionniste, qu'il veut tout grignoter, tout faire
entrer dans son escarcelle, qu'il faut accepter cette id\'ee pour les
mouvements. Pourquoi? Parce qu'accepter l'emprise, la
g\'en\'eralisation du droit d'auteur sur les mouvements, aurait une
cons\'equence terrible, qui serait de brider notre propre capacit\'e
\`a se mouvoir. Et donc qu'il y a ici quelque chose d'imp\'erieux, que
les mouvements restent dans le domaine public, que les mouvements
soient communs, partag\'es par tous. Peut{}-\^etre qu'on pourrait
envisager que certains mouvements particuliers, extr\^emement
extraordinaires, qui ne refl\`etent pas un mouvement commun. Mais
jusqu'o\`u est{}-ce que les mouvements deviennent communs ou deviennent
extraordinaires? Ce serait tr\`es compliqu\'e de mettre une barri\`ere,
un jugement de valeur. Or le droit d'auteur ne met pas de jugement de
valeur; \`a partir de quel moment doit{}-on consid\'erer qu'un
mouvement est beau, qu'il est utile, qu'il est commun, qu'il est
extraordinaire? Tout \c{c}a, ce n'est pas un juge qui doit d\'ecider si
on est ou non dans le droit d'auteur. Il ne faut pas laisser l'espace
au droit d'auteur pour se d\'evelopper partout, tout le temps. Il faut
au contraire pr\'eserver ce statut particulier qu'est le mouvement en
droit d'auteur et cette timidit\'e qu'a le droit d'auteur \`a se porter
sur le mouvement, en raison de l'imp\'erieuse n\'ecessit\'e que nous
avons de partager les mouvements les uns avec les autres, de pouvoir
les reproduire. Si je fais une caresse sur la joue de mon enfant,
est{}-ce que je peux deux secondes imaginer que cette possibilit\'e me
soit refus\'ee parce qu'un chor\'egraphe, aussi talentueux soit{}-il,
aura repris l'id\'ee ou l'image de la caresse sur la joue de mon
enfant? Ce serait dramatique.\par

Et, puis quoi? Parce que je reconstruis un mouvement? Parce que j'aurais
d\'ecid\'e de la mani\`ere dont il doit se d\'erouler, j'aurais
n\'ecessairement la possibilit\'e d'interdire \`a autrui de recommencer
ce mouvement{}-l\`a? Si par exemple, je con\c{c}ois le mouvement d'une
proth\`ese, si par exemple je peux impulser \`a une proth\`ese un
certain mouvement, est{}-ce que \c{c}a veut dire pour autant que toute
personne humaine qui sait recopier, reconstruire ce mouvement se
verrait interdit~de le faire, parce que j'ai d\'evelopp\'e le processus
qui a conduit \`a ce mouvement? M\^eme si ce mouvement est construit de
fa\c{c}on totalement arbitraire par mon imaginaire? Jusqu'o\`u est{}-ce
que je peux interdire \`a autrui de reconstituer un mouvement?
Impossible, \'epouvantable!\par

La question est la suivante: une fois que j'ai accept\'e que le droit
d'auteur s'appuie sur le mouvement, {--} obligatoire, \'evidemment le
mouvement est prot\'eg\'e par le droit d'auteur {--} je ne suis pas
celle qui peut dire le contraire{\dots} Donc, \`a partir du moment o\`u
je consid\`ere que le mouvement est prot\'eg\'e, {\em qui} va \^etre
prot\'eg\'e? {\em Qui} est celui qui va b\'en\'eficier de la
protection, de la r\'eservation? Qui est celui qui pourra interdire,
{--}comme le disait S\'everine tout \`a l'heure {--} ou autoriser
\'egalement, autrui \`a refaire, recopier, reproduire, reconstituer ce
mouvement? Le chor\'egraphe? Le danseur? Celui qui aura donn\'e
l'id\'ee de la chor\'egraphie? Le sc\'enographe?\par

Il y a une d\'ecision int\'eressante, une vieille d\'ecision que
S\'everine a \'et\'e rechercher, a exhum\'ee. C'est une d\'ecision de
la cour d'appel de Paris du 8 juin 1960, sur le ballet, sur le fameux
ballet du {\em Jeune Homme et La Mort}. Vous savez de qui est ce
ballet? De qui est ce ballet, {\em Le Jeune Homme et La Mort~}?\par

\Interview{Public}Roland Petit.\par

\Interview{VLB}Faux! Jean Cocteau! C'est ce qui a \'et\'e d\'ecid\'e dans la
d\'ecision. Jean Cocteau est l'auteur du ballet. Pourquoi Jean Cocteau
est l'auteur de ballet? Parce que Jean Cocteau est celui qui a dit \`a
Roland Petit: \quotation{Voil\`a ce que je veux faire!} C'est celui qui a dit \`a
l'auteur du d\'ecor: \quotation{Voil\`a comment je veux que le d\'ecor soit
fait!} C'est celui qui a dit au costumier: \quotation{Voil\`a comment je veux que
soit le costume!} Et Roland Petit n'est qu'un {\em ex\'ecutant} ici,
il n'est qu'un petit artiste de seconde zone; la personne qui est
prot\'eg\'ee, c'est Jean Cocteau, parce que c'est Jean Cocteau qui a
donn\'e l'impulsion. C'est choquant parce que j'ai toujours cru que
c'\'etait Roland Petit qui avait \'et\'e l'auteur du {\em Jeune
Homme et La Mort}, parce que le chor\'egraphe, c'est quand m\^eme celui
qui traduit, dans le mouvement, l'id\'ee. Et moi j'ai dit tout \`a
l'heure, que l'id\'ee n'\'etait pas prot\'eg\'ee, que la simple id\'ee
ne pouvait pas \^etre prot\'eg\'e, que c'est le mouvement,
c'est{}-\`a{}-dire la {\em traduction} dans l'espace, qui pouvait
\^etre prot\'eg\'ee par le droit d'auteur.\par

Il y a une autre d\'ecision int\'eressante en France dans un autre
domaine, qui est le domaine de la sculpture. Renoir \`a la fin de sa
vie ne pouvait plus sculpter, il n'avait plus suffisamment de sens pour
pouvoir sculpter et donc il travaillait avec son assistant Guino, et il
donnait \`a Guino des indications, il lui disait comment sculpter. Et
Guino sculptait selon les indications de Renoir, et on s'est pos\'e la
question de savoir qui des deux \'etait prot\'eg\'e, qui des deux
\'etait l'auteur. La r\'eponse de la cour de cassation \'etait une
r\'eponse un peu simple: \quotation{Les deux.} Les deux: Renoir parce que c'est
lui qui avait con\c{c}u intellectuellement l'{\oe}uvre, Guino parce que
c'est lui qui avait con\c{c}u manuellement l'{\oe}uvre. C'est comme si
tout d'un coup dans un individu, il y avait la dissociation entre
l'esprit et la main, et que les deux parties du corps devaient recevoir
une protection. Eh bien, est{}-ce qu'on ne devra pas r\'efl\'echir de
la m\^eme mani\`ere: Roland Petit bien s\^ur \'egalement l'auteur,
Cocteau aussi l'auteur, parce que Cocteau a donn\'e l'impulsion, sans
laquelle Roland Petit n'aurait pas pu faire le ballet, et inversement,
le ballet n'aurait pas pu \^etre fait si Roland Petit n'avait pas
\'et\'e l\`a, parce que Cocteau ne sait pas chor\'egraphier.\par

\Interview{SD}Et ne pourrait{}-on pas alors y rajouter, apr\`es la main et
l'esprit, le {\em corps}~{--} et parler aussi de danseurs, de
danseuses, de com\'ediens, de com\'ediennes, qui inscrivent dans leurs
corps les mouvements qu'on leur demande de faire? On n'est plus
vraiment en droit d'auteur, on est dans la protection des droits
voisins {--} on appelle \c{c}a les \quote{droits voisins}, parce qu'ils sont
\`a c\^ot\'e du droit d'auteur. Ce sont des droits qui notamment
investissent l'artiste, l'interpr\`ete, d'un droit sur sa prestation,
sur sa performance. Dans le cas du mouvement, il est clair que le
mouvement, le sens du mouvement, est d\'ecid\'e par le chor\'egraphe,
mais le danseur lui{}-m\^eme ou la danseuse {\em ex\'ecutent} ce
mouvement, {\em r\'ealisent} ce mouvement. Donc o\`u se trouve la
cr\'eation? Dans celui qui {\em d\'ecide} quel mouvement doit \^etre
fait, ou dans celui qui {\em r\'ealise} ce mouvement? Et qu'en
est{}-il si le danseur ou le com\'edien improvisent~un mouvement? Dans
ce cas{}-l\`a, le danseur et le com\'edien ont{}-ils un droit d'auteur
sur le mouvement qu'ils viennent de faire, ou juste un droit
d'interpr\`ete sur un mouvement? Mais, un interpr\`ete d'un mouvement
inexistant auparavant?\par

Or en g\'en\'eral les artistes{}-interpr\`etes n'ont des droits que
lorsqu'ils interpr\`etent une {\oe}uvre existante, et qui est d\'ej\`a
prot\'eg\'ee par le droit d'auteur. Donc l\`a on voit bien je pense,
qu'en prot\'egeant le mouvement, on ne sait plus tr\`es bien \`a qui il
appartient, \ qui l'a cr\'e\'e, quels sont les droits qui vont
peut{}-\^etre venir se cumuler sur les mouvements? Et \`a ce
moment{}-l\`a on arrive plus facilement \`a glisser en dehors de la
simple cr\'eation. Il va de soi que lorsqu'on parle de mouvement dans
une {\oe}uvre chor\'egraphique, on peut se dire que le chor\'egraphe a
un droit d'auteur sur ses mouvements, et les artistes auront un droit
d'interpr\'etation sur le mouvement, et on ne va peut{}-\^etre pas
distinguer les mouvements improvis\'es, ou les mouvements command\'es
par le ou la chor\'egraphe. Mais, lorsqu'il s'agit d'interpr\`etes de
mouvements tout \`a fait improvis\'es, ou qui demandent une certaine
prouesse technique {--} je pense aux sportives, ou je pense m\^eme
simplement \`a des mannequins qui d\'efilent pour des collections {--}
la protection par un droit voisin devient moins \'evidente. La
jurisprudence belge par exemple a dit qu'un mannequin qui d\'efilait
dans une collection, ne pouvait pas recevoir un droit voisin sur son
interpr\'etation, car il n'y avait pas d'{\oe}uvre prot\'eg\'ee par le
droit d'auteur. Elle ne faisait que marcher. Donc l\`a on est de
nouveau dans un cas limite, o\`u le mouvement lui{}-m\^eme ne va pas
attirer sur lui certains droits.\par

On est arriv\'e \`a vous montrer toute une s\'erie de contradictions,
et, sur base de ces contradictions, comment r\'eagissent les cours et
tribunaux. On peut maintenant arriver \`a des cas d'applications un peu
plus concrets pour montrer quelles difficult\'es se posent.\par

Le premier cas concernait une chor\'egraphie. C'est un cas assez
c\'el\`ebre en Belgique, mais qu'on cite \'egalement
\quote{outre{}-Qui\'evrain}, comme on dit. La chor\'egraphie est en
r\'ealit\'e de temps en temps prot\'eg\'ee par la jurisprudence. On
admet que les chor\'egraphies peuvent \^etre l'objet de droit d'auteur.
Ici il s'agissait d'un cas relatif \`a un ballet assez c\'el\`ebre de
Fr\'ed\'eric Flamand, qui s'appelait {\em La chute d'Icare}. Je n'ai
malheureusement pas d'images en mouvement de ce ballet, je n'ai qu'une
image fixe qui montre la sc\`ene la plus connue de ce ballet: un
danseur nu traverse la sc\`ene avec des t\'el\'evisions accroch\'ees
\`a ses pieds, et il est habill\'e en ange avec des ailes de plumes. Il
traverse l'ensemble de la sc\`ene d'un pas tr\`es lent avec ses
t\'el\'evisions. Quelques ann\'ees plus tard Fr\'ed\'eric Flamand
s'aper\c{c}oit que Maurice B\'ejart, dans un de ses ballets, reprend ce
type de mouvement. Je vais vous \'epargner cette sc\`ene du ballet de
B\'ejart, qui n'est pas un exemple le plus flagrant de son talent. Donc
il s'agissait d'un autre ballet, dans lequel un danseur, habill\'e de
mani\`ere similaire avec \'egalement des t\'el\'evisions aux pieds,
traverse la sc\`ene en reproduisant assez fid\`element le mouvement du
danseur qu'il l'avait pr\'ec\'ed\'e dans le ballet de Fr\'ed\'eric
Flamand. Le premier chor\'egraphe, Fr\'ed\'eric Flamand, poursuit en
justice Maurice B\'ejart, en disant, \quotation{Il y a contrefa\c{c}on de mon
droit d'auteur, il y a violation de mon droit d'auteur sur ce
mouvement.} Et ce qui est int\'eressant c'est que le juge en fait dans
sa d\'ecision va regarder ce qui peut \^etre prot\'eg\'e par le droit
d'auteur dans cette sc\`ene. Le juge explique que la sc\`ene est
form\'ee par la combinaison des \'el\'ements suivants: un homme nu ou
quasi nu, muni d'ailes et chauss\'e de
t\'el\'evisions, qui traverse lentement la sc\`ene de part en part, de
droite \`a gauche, et fait un arr\^et au milieu de la sc\`ene {--}
cette premi\`ere image. Que cette sc\`ene comprend donc le mouvement,
l'encha\^inement des mouvements {--} donc les deux \'el\'ements sur
lesquels on s'est un peu disput\'e pour savoir s'ils pouvaient
pr\'etendre \`a la protection {--} plus le costume, les accessoires
utilis\'es, le positionnement {--} l\`a on est aussi tr\`es proche du
mouvement {--} la mise en \'evidence du personnage, la puissance
\'evocatrice, sa signification symbolique. Avec les derniers
\'el\'ements, \quote{puissance \'evocatrice}, \quote{signification symbolique}, on
est plus dans du concept, dans l'abstrait, pas vraiment dans le concret
d'une {\oe}uvre. Et le juge poursuit: \quotation{Que cette sc\`ene d\'egage donc
une grande force \'evocatrice symbolique, repr\'esentative, qui n'a pas
\'echapp\'e au critique d'art, dans la mesure o\`u cette sc\`ene est
devenue la sc\`ene phare de {\em La chute d'Icare}. Attendu donc que
cette combinaison d'\'el\'ements forme un tout qui ne peut \^etre
divis\'e entre ces diff\'erents \'el\'ements pour tenter de d\'emontrer
que l'{\oe}uvre ne serait pas prot\'egeable, parce que chacun de ses
\'el\'ements pris individuellement ne le serait pas.} Donc le juge va
consid\'erer que {\em l'ensemble} est prot\'eg\'e par le droit
d'auteur, mais qu'on ne va pas pouvoir diviser les diff\'erents
\'el\'ements de cette sc\`ene, pour essayer de prot\'eger chaque
\'el\'ement par le droit d'auteur. Donc le mouvement lui{}-m\^eme, le
fait que le danseur traverse lentement la sc\`ene, ne serait pas
suffisant en lui{}-m\^eme pour \^etre prot\'eg\'e par le droit
d'auteur. C'est uniquement la mani\`ere dont ce mouvement est mis en
sc\`ene avec des costumes, avec certains accessoires {--} qui rend le
mouvement, si vous voulez, \quote{habill\'e} par le droit d'auteur, par une
certaine cr\'eation, et qui peut alors pr\'etendre \`a la protection
par le droit d'auteur.\par

\Interview{VLB}La question s'est pos\'ee aussi de savoir si les mises en
sc\`enes pouvaient \^etre prot\'eg\'ees. Pourquoi? Parce que mettre en
sc\`ene, c'est partir d'une {\oe}uvre litt\'eraire statique, pour en
faire une {\oe}uvre en mouvement, une {\oe}uvre interpr\'et\'ee, une
{\oe}uvre sur sc\`ene. Et, la question s'est pos\'ee pendant tr\`es
longtemps de savoir si les metteurs en sc\`ene pouvaient \^etre
consid\'er\'es comme des auteurs. D'ailleurs, les soci\'et\'es de
gestions collectives, Sacem en t\^ete, pendant
longtemps ont fait obstacle \`a avoir dans leurs rangs des metteurs en
sc\`ene. Cette question est d\'epass\'ee, je crois, et la jurisprudence
admet maintenant assez r\'eguli\`erement le fait que des metteurs en
sc\`ene puissent \^etre consid\'er\'es comme des auteurs. Qu'est ce que
fait un metteur en sc\`ene? Il va donner un certain nombre de
directions sur la mani\`ere dont l'{\oe}uvre dramatique va \^etre
traduite dans l'espace, il va choisir le d\'ecor, il va choisir
l'emplacement des objets. Il va d\'ecider des entr\'ees et des sorties
des personnages, de certains mouvements de ses personnages, de leur
comportement bien \'evidemment. Mais \`a l'int\'erieur de ces
comportements, il peut aller tr\`es loin et leur impulser certains
types de mouvements et on sait tr\`es bien que aujourd'hui d'ailleurs
la distinction entre mise en sc\`ene et chor\'egraphie est tr\`es
t\'enue. Quand on voit des gens comme Robert Wilson par exemple, qui
font \`a la fois du th\'e\^atre, de la danse{\dots} On n'a plus de
fronti\`eres extr\^emement pr\'ecises entre le spectacle th\'e\^atral,
le spectacle chor\'egraphique. Donc si on consid\`ere qu'une {\oe}uvre
chor\'egraphique peut \^etre prot\'eg\'ee, pourquoi ne pas prot\'eger
quelque chose comme une mise en sc\`ene, qui m\^eme si les mouvements
utilis\'es ne sont plus courants, ne sont plus du domaine du commun,
peuvent \^etre bien marqu\'es dans un espace, dans un temps
particulier. Il y a chez le metteur en sc\`ene \`a la fois la
protection d'une certaine musicalit\'e de la parole, et aussi d'une
mise en mouvement de la pi\`ece. La seule chose qu'on puisse dire,
c'est peut{}-\^etre que le metteur en sc\`ene ne sera pas lui{}-m\^eme
l'auteur, notamment s'il est trop prisonnier des indications de
l'auteur. L\`a encore on a ce probl\`eme de rapport de
d\'ependance: plus l'auteur va mettre des pr\'ecisions dans sa pi\`ece,
va donner des indications de mise en sc\`ene {--} \quote{il tousse}, \quote{il
s'assoit}, \quote{il se mouche}, etc. {--} moins le metteur en sc\`ene aura
de marge de man{\oe}uvre, et moins il sera cr\'eatif, et donc moins il
pourra \^etre prot\'eg\'e par le droit d'auteur. Mais la question de la
mise en sc\`ene est un petit peu derri\`ere nous et les jurisprudences
consid\`erent de fa\c{c}on traditionnelle que ces types de cr\'eation
sont des {\oe}uvres prot\'eg\'ees par le droit d'auteur.\par

\Interview{SD}La loi dit \'egalement que les num\'eros de cirques peuvent \^etre
prot\'eg\'es par un droit voisin. Pas la prouesse technique mais
simplement la mani\`ere dont on va exprimer ce num\'ero de cirque dans
une certaine forme. On a m\^eme prot\'eg\'e dans certaines d\'ecisions
des tours de magie. Il y a quelques d\'ecisions qui vont prot\'eger
ainsi des spectacles de magie, ou en tout cas certains \'el\'ements de
ces spectacles de magie. Et dans ce cas{}-l\`a en fait, ce qui va
\^etre prot\'eg\'e, c'est la forme que prend l'ex\'ecution du tour de
magie. Comme dans cette d\'ecision de la cour d'appel de Paris de 2003,
il s'agissait d'un tour de magie qui datait d'il y a 20, 25 ans, et qui
faisait en sorte qu'un piano s'envolait dans les airs. Un pianiste
s'asseyait au piano, et puis le piano s'enlevait dans les airs. Et
cette sc\`ene se caract\'erisait par les mouvements suivants, dit le
jugement: \quotation{Le piano s'\'el\`eve lentement dans les airs selon une
trajectoire en {\em looping} apparemment irr\'egulier, le pied avant
du piano se d\'ecolle en premier du sol, l'avant du piano se soul\`eve.
Puis, il balance d'un c\^ot\'e de l'autre jusqu'\`a la position
verticale, marque un stop dans son mouvement \'evolutif, avant de le
poursuivre, jusqu'\`a se retrouver totalement \`a l'envers, le moment
o\`u le pianiste qui a gard\'e les jambes serr\'ees tout au long de
l'illusion se retrouve dos au sol, et que des {\em loopings} d'avant
en arri\`ere sont effectu\'es, {--} un seul ou plusieurs, selon le
{\em timing} du spectacle {--} \`a la suite desquels le piano et
le musicien atterrissent, retour au sol, le pied avant se pose en
premier, que ces \'el\'ements caract\'eristiques sont bien ceux fix\'es
sur le vid\'eogramme.} On avait enregistr\'e ce tour de
prestidigitation, et la question se posait de savoir si quelqu'un
d'autre qui avait refait le m\^eme tour en imprimant au piano le m\^eme
mouvement dans son truc de magie \'etait en contrefa\c{c}on avec la
premi\`ere {\oe}uvre. Et bien l\`a aussi le juge va faire une
distinction entre la prouesse m\^eme de magie, le tour de
prestidigitation, et le num\'ero de magie, qui est l'expression de ce
tour, la mise en sc\`ene, le tour mis en sc\`ene. Mais la d\'ecision
pose quelques difficult\'es: \`a partir de quand on est dans le tour de
magie {--} le piano qui vole {--} ou \`a partir de quand est{}-on dans
un mouvement qui va exprimer d'une mani\`ere diff\'erente tel ou tel
tour de magie? Il y a sans doute plusieurs mani\`eres de faire
s'envoler le piano. Donc ce sont des d\'ecisions qui peuvent \^etre
tr\`es difficiles \`a appr\'ecier.\par

Autres mises en sc\`enes \'egalement, c'est lorsque les artistes se
mettent en sc\`ene eux{}-m\^emes; dans l'art contemporain, toute la
protection des performances par exemple par le droit d'auteur. L\`a je
ne pense pas qu'on ait de d\'ecision de jurisprudence, l'art
contemporain fait rarement l'objet de d\'ecisions devant les cours et
tribunaux. Mais, l\`a aussi on peut se poser la question: est{}-ce
qu'il y a vraiment \quote{{\oe}uvre} lorsqu'un artiste en fait ne fait que se
mettre en sc\`ene, et de faire une performance? Sans doute la fixation,
le fait d'avoir enregistr\'e la performance est une {\oe}uvre, mais la
performance elle{}-m\^eme, certains mouvements, certaines postures qui
sont prises par l'artiste pour exprimer certains concepts: est{}-ce
vraiment prot\'egeable par le droit d'auteur? Bien souvent les
performances seront surtout des concepts, et pas vraiment des
expressions, donc on va \^etre en dehors du droit d'auteur. Mais donc
peut{}-on vraiment admettre que le droit d'auteur passe ainsi \`a
c\^ot\'e d'un pan entier de l'art contemporain? Simplement parce que
l'art contemporain a choisi une autre voie que le droit d'auteur, et a
choisi de travailler plus sur l'abstrait, sur le concept?\par

\Interview{VLB}Autre champ dans lequel on peut s'interroger sur l'opportunit\'e
ou l'existence d'une protection, c'est ce qu'on appelle \quote{les
expressions du folklore}, les mouvements traditionnels, les mouvements
sacr\'es: les pri\`eres, les danses traditionnelles{\dots} Est{}-ce
qu'il y a ici un terrain pour le droit d'auteur? Alors, il y a
plusieurs possibilit\'es: soit on consid\`ere que \c{c}a ne peut pas
\^etre appropri\'e par le droit d'auteur, parce que c'est une
expression qui est vieille d\'ej\`a, qui vient de temps ancestraux et
qui est transmise de g\'en\'eration en g\'en\'eration, dans une
certaine fid\'elit\'e. Justement, par rapport au folklore c'est
pr\'ecis\'ement cette tradition, cette fid\'elit\'e de transmission qui
est assur\'ee \`a travers le folklore, et que donc les gens qui
expriment ce folklore \`a un moment donn\'e, n'en sont que les
d\'epositaires, et n'en sont pas les cr\'eateurs. Donc \'evidemment,
\`a cette enseigne, ils ne peuvent pas y avoir de droits d'auteur sur
le folklore, parce que le folklore {\em d\'epasse} la cr\'eation
individuelle de ceux qui interpr\`etent \`a un moment donn\'e. Il
existe avant, il existera apr\`es, il y est permanent, il est immanent.\par

En m\^eme temps, il y a eu un exemple de jurisprudence int\'eressante
sur {\em la lambada}. Je ne sais pas si vous souvenez, un moment, un
\'et\'e, tout le monde s'est mis \`a danser \quote{{\em Chorando se
foi}{\dots}}. Eh bien, \c{c}a c'\'etait une chanson qui \'etait de
l'ordre du folklore br\'esilien depuis tr\`es longtemps et qui est
devenue un hit tout d'un coup, parce que la r\'eorchestration, la
nouvelle rythmique qui \'etait impuls\'ee \`a ce morceau de musique, et
d'ailleurs il y a une danse qui va avec {\em la lambada}
\'evidemment {--} vous voulez qu'on danse {\em la lambada}? Non? Je
peux hein, si vous voulez{\dots} Ce morceau de musique traditionnel
tout d'un coup est devenue une {\oe}uvre avec une forte attraction
\'economique, avec un march\'e. S'il y a un march\'e qui arrive, parce
que quelqu'un a su donner tout d'un coup une certaine impulsion \`a
quelque chose qui est folklorique, et lui a donn\'e une certaine
nouveaut\'e, une certaine attractivit\'e, est{}-ce qu'il ne pourrait
pas profiter de ce march\'e et en fait essayer d'avoir un droit
d'auteur dessus? On voit bien ici qu'il y a des tensions diff\'erentes:
une tension entre le march\'e, le profit, et puis la n\'ecessit\'e
\'egalement de partager, de garder une expression commune, et de la
garder en dehors du droit d'auteur. Mais d'un autre c\^ot\'e, le
probl\`eme de rejeter le folklore en dehors du droit d'auteur, c'est
qu'on va arriver \`a cette cons\'equence particuli\`ere, que seules les
expressions occidentales qui correspondent \`a la d\'efinition que l'on
a dans le droit d'auteur ou le copyright, vont donner lieu \`a un
march\'e, une r\'emun\'eration, alors que les expressions de types
folkloriques qui viennent d'Afrique, d'Asie etcetera, tout d'un coup
elles seront compl\`etement \'ecart\'ees de toute possibilit\'e de
profit d'exploitation, parce qu'elles sont du domaine du folklore. Donc
il y a ici un discours qui n'est pas seulement un discours de droit
d'auteur, au regard des principes du droit d'auteur, mais un discours
de discrimination, selon le r\'egime culturel et de possibilit\'e
d'exploitation \'economique, selon le r\'egime culturel. Je crois que
S\'everine est d'accord avec moi sur ce point.\par

\Interview{SD}Oui. Alors jusqu'ici {--} on va conclure l\`a{}-dessus {--}
jusqu'ici on a parl\'e surtout du mouvement des corps, mais on pourrait
aussi parler du mouvement des {\em objets}. Je vous ai d\'ej\`a
parl\'e du piano qui vole, mais on peut \'evidemment partir de
certaines {\oe}uvres d'art. Les mobiles de Calder par exemple: est{}-ce
qu'ils sont prot\'eg\'es par le droit d'auteur en tant qu'objet? Ou,
est{}-ce qu'ils sont prot\'eg\'es en tant qu'objet {\em en
mouvement}? Est{}-ce que c'est le mouvement du mobile qui peut faire
l'objet d'une appropriation ou seul l'objet? D'autres mouvements
d'objets aussi en art pourraient poser questions. Les fameux films
r\'ealis\'es par Fischli et Weiss qui sont sur des successions de
mouvements. Je suis d\'esol\'ee, je vous am\`ene sur YouTube, vous
allez avoir affaire \`a la publicit\'e de YouTube [projection du
{\em Mouvement perp\'etuel} de Fischli et Weiss]. Ces auteurs font
en fait des {\oe}uvres o\`u ils ne font que reprendre des mouvements,
quand c'est surtout cette succession des mouvements d'objets qui cr\'ee
l'{\oe}uvre, qu'est{}-ce qu'on va prot\'eger?\par

Est{}-ce que c'est le film qu'on va prot\'eger, donc la fixation de tous
ces mouvements d'objets, ou est{}-ce que c'est vraiment les mouvements
d'objets eux{}-m\^emes qui sont prot\'eg\'es? Et l\`a avec ces
mouvements d'objets, si on admet la protection par le droit d'auteur,
on peut alors arriver en effet \`a reposer la question de d\'epart,
mais par l'intervention du droit des brevets, qu'en
est{}-il du mouvement des machines ou des robots?\par

La semaine pass\'ee on a parl\'e du Syndicat des Robots, parce que le
Syndicat des Robots va pouvoir demander une protection par le droit
d'auteur du mouvement que les robots produisent. Si on admet que les
robots ont droit \`a un syndicat, sans doute leurs cr\'eations ont
droit \`a la protection par le droit d'auteur, n'est{}-ce pas? Sinon
\c{c}a serait vraiment compl\`etement ridicule. Donc est{}-ce qu'alors
on n'arrive pas en effet \`a ce que cette histoire de protection du
mouvement ne soit le pr\'etexte \`a un glissement tout \`a fait
irrecevable du droit d'auteur, ou un mouvement compl\`etement excessif
du droit d'auteur? Voil\`a un petit peu la question sur laquelle je
m'arr\^eterais, mais je laisse la parole \`a Val\'erie{\dots}

\Interview{VLB}Je n'ai rien \`a ajouter, juste qu'il faut se m\'efier
de~l'\'emouvant mouvement mouvant. Mais \c{c}a n'est pas tr\`es
traduisible, je comprends bien.\par

\Interview{SD}Redis{}-le.\par

\Interview{VLB}Alors je disais: il faut se m\'efier de l'\'emouvant mouvement
mouvant.\par}
