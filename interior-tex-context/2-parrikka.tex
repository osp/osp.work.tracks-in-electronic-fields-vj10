\PlaceImage{parikka02.JPG}{Jussa Parrikka at V/J10}

\AuthorStyle{Jussi Parikka}

\licenseStyle{Creative Commons Attribution{}-NonCommercial{}-ShareAlike}

\Eng{\Title{Insects, Affects and Imagining New Sensoriums}
\SubTitle{A Media Archaeological Rewiring\crlf from Geniuses to Animals}

An insect media artist or a media archaeologist imagining a
potential weird medium might end up with something that sounds quite
mundane to us humans. For the insect probe head, the question of what
it feels like to perceive with two eyes and ears and move with two legs
would be a novel one, instead of the multiple legs and compound eyes
that it has to use to manoeuvre through space. The uncanny formations
often used in science fiction to describe something radically inhuman
(like the killing machine insects of {\em Alien} movies) differ from
the human being in their anatomy, behaviour and morals. The human brain
might be a much more efficient problem solver and the human hands are
quite handy tool making metatools, and the human body could be seen as
an original form of any model of technics, as Ernst Kapp already
suggested by the end of the 19\high{th} century. But still,
such realisations do not take away the fascination that emerges from
the question {\em of what would it be like to move, perceive and
think differently}; what does a becoming{}-animal entail.

I am of course taking my cue here from the philosopher Manuel DeLanda
who in his 1991 book {\em War in the Age of Intelligent Machines},
asked what would the history of warfare look like from the viewpoint of
a future robot historian? An exercise perhaps in creative imagination,
DeLanda's question also served other ends relating to physics of
self{}-organization. My point is not to discuss DeLanda, or the history
of war machines, but I want to pick an idea from this kind of an
approach, an idea that could be integrated into media archaeological
considerations, concerning actual or imaginary media. As already said,
imagining alternative worlds is not the endpoint of this exercise in
\quote{insect media}, but a way to dip into an alternative understanding of
media and technology, where such general categories as \quote{humans} and
\quote{machines} are merely the endpoints of intensive flows, capacities,
tendencies and functions. Such a stance takes much of its force from
Gilles Deleuze's philosophical ontology of abstract
materialism, which focuses primarily on a Spinozian ontology of
intensities, capacities and functions. In this sense, the human being
is not a distinct being in the world with secondary qualities, but a
\quotation{capacity to signify, exchange, and communicate}, as Claire Colebrook
has pointed out in her article \quote{The Sense of Space} ({\em Postmodern
Culture}). This opens up a new agenda not focused on \quote{beings} and their
tools, but on capacities and tendencies that construct and create
beings in a move which emphasizes Deleuze's interest in pre{}-Kantian
worlds of baroque. In addition, this move includes a multiplication of
subjectivities and objects of the world, a certain autonomy of the
material world beyond the privileged observer. Like everybody who has
done gardening knows: there is a world teeming with life outside the
human sphere, with every bush and tree being a whole society in itself.

To put it shortly, still following Colebrook's recent writing on the
concept of affect, what Deleuze found in the baroque worlds of
windowless monads was a capacity of perception that does not stem from
a universalising idea of {\em perception in general}. Man or any
general condition of perception is not the primary privileged position
of perception but perceptions and creations of space and temporality
are multiplied in the numerous monadic worlds, a distributed perception
of a kind that according to Deleuze later found resonance in the
philosophy of A.N.Whitehead. For Whitehead, the perceiving subject is
more akin to a \quote{superject}, a second order construction from the sum of
its perceptions. It is the world perceived that makes up superjects and
based on the variations of perceptions also {\em alternative}
worlds. Baroque worlds, argues Deleuze in his book {\em Le Pli} from
1988, are characterised by the primacy of variation and perspectivism
which is a much more radical notion than a relativist idea of different
subjects having different perspectives on the world. Instead, \quotation{the
subject will be what comes to the point of view}, and where \quotation{the point
of view is not what varies with the subject, at least in the first
instance; it is, to the contrary, the condition in which an eventual
subject apprehends a variation (metamorphosis){\dots}}.

Now why this focus on philosophy, this short excursion that merely
sketches some themes around variation and imagination? What I am after
is an idea of how to smuggle certain ideas of variation, modulation and
perception into considerations of media culture, media archaeology and
potentially also imaginary media, where imaginary media become less a
matter of a Lacanian mirror phase looking for utopian communication
offering unity, but a deterritorialising way of understanding the
distributed ontology of the world and media technologies. Variation and
imagination become something else than the imaginations of a point of
view {--} quite the contrary, the imagination and variation
{\em give rise} to points of view, which opens up a whole new agenda
of a past paradoxically not determined, and even further, future as
open to variation. This would mean taking into account perceptions
unheard of, unfelt, unthought{}-of, but still real in their intensive
potentiality, a becoming{}-other of the sensorium so to speak. Hence,
imagination becomes not a human characteristic but an epistemological
tool that interfaces analytics of media theory and history with the
world of animals and novel affects.

Imaginary media and variations at the heart of media cultural modes of
seeing and hearing have been discussed in various recent books. The
most obvious one is {\em The Book of Imaginary Media}, edited by
Eric Kluitenberg. According to the introduction, all media consist of a
real and an imagined part, a functional coupling of material
characteristics and discursive dreams which fabricate the crucial
features of modern communication tied intimately with utopian ideals.
Imaginary media {--} or actual media imagined beyond its real
capacities {--} have been dreamed to compensate insufficient
communication, a realisation that Kluitenberg elaborates with the
argument that \quotation{central to the archaeology of imaginary media in the end
are not the machines, but the human aspirations that more often than
not are left unresolved by the machines{\dots}}. Powers of
imagination are then based in the human beings doing the imagining, in
the human powers able to transcend the actual and factual ways of
perception and to grasp the unseen, unheard and unthought of media
creations. Variation remains connected to the principle of the central
point where variation is perceived.

Talking of the primacy of variation, we are easily reminded of Siegfried
Zielinski's application of the idea of \quote{variantology} as an
\quote{anarchaeology of media}, a task dedicated to the primacy of variation
resisting the homogeneous drive of commercialised media spheres.
Excavating dreams of past geniuses, from Empedocles to Athanius
Kircher's cosmic machines and communication networks to Ernst Florens
Friedrich Chladni's visualisation of sound, Zielinski has been
underlining the creative potential in an exercise of imagining media.
In this context, he defines in threefold the term \quote{imaginary media} in
his chapter in the {\em Book of Imaginary Media}:

\startitemize
\item {\em Untimely media/apparatus/machines}: \quotation{Media devised and
designed either much too late or much too early{\dots}}
\item {\em Conceptual media/apparatus/machines}: \quotation{Artefacts that were
only ever sketched as models{\dots} but never actually built.}
\item {\em Impossible media/apparatus/machines}: \quotation{Imaginary media in
the true sense, by which I mean hermetic and hermeneutic
machines{\dots} they cannot actually be built, and whose implied
meanings nonetheless have an impact on the factual world of media.}
\stopitemize

A bit reminiscent of the baroque idea, variation is primary, claims
Zielinski. Whereas the capitalist orientated consumer media culture is
working towards a {\em psychopathia medialis} of homogenized media
technological environments, variantology is committed to promoting
heterogeneity, finding dynamic moments of media archaeological past,
and excavating radical experiments that push the limits of what can be
seen, heard and thought. Variantology is then implicitly suggested as a
mode of ontogenesis, of bringing forth, of modulation and change {--}
an active mode of creation instead of distanced contemplation.

Indeed, the aim of promoting diversity is a much welcomed one, but I
would like to propose a slight adjustment to this task, something that
I engage under the banner of \quote{insect media}. Whereas Zielinski and much
of the existing media archaeological research still starts off from the
human world of male inventor{}-geniuses, I propose a slightly more
distributed look at the media archaeology of affects, capacities, modes
of perception and movement, which are primarily not attached to a
specific substance (animal, technology), but since the 19\high{th} century at
least, refer to a certain passage, vector from animals to technology
and vice versa. Here, a mode of baroque thought, a thought tuned in
terms of variations becomes unravelled with the help of animality that
is not to be seen as a metaphor, but as a metamorphosis, as \quote{teachings}
in weird perceptions, novel ways of moving, new ways of sensing,
opening up to the world of sensations and contracting them. Instead of
looking for variations through inventions of people, we can turn to the
\quote{storehouses of invention} of for example insects that from the
19\high{th} century on were introduced as an alien form of
media in themselves. Next I will elaborate how we can use these tiny
animals as philosophical and media archaeological tools to address
media and technology as intensities that signal weird sensory
experiences.

\SubSubTitle{Novel Sensoriums}

During the latter half of the 19\high{th} century, insects
were seen as uncanny but powerful forms of media in themselves, capable
of weird sensory and kinaesthetic experiences. Examples range from
popular newspaper discourse to scientific measurements and such early
best{}-sellers as {\em An Introduction to Entomology; or,
Elements of the Natural History of Insects: Comprising an Account of
Noxious and Useful Insects, of Their Metamorphoses, Hybernation,
Instinct} (1815{--}\-1826) by William Kirby and William
Spence.

Since the 19\high{th} century, insects and animal affects are not only
found in biology but also in art, technology and popular culture. In
this sense, the 19\high{th} century interest in insects produces a
valuable perspective on the intertwining of biology (entomology),
technology and art, where the basics of perception are radically
detached from human{}-centred models towards the animal kingdom. In
addition, this science{}-technology{}-art trio presents a challenge to
rethink the forces which form what we habitually refer to as \quote{media} as
modes of perception. By expanding our notions of \quote{media} from the
technological apparatuses to the more comprehensive assemblages that
connect biological, technological, social and aesthetic issues, we are
also able to bring forth novel contexts for contemporary analysis and
design of media systems. In a way, then, the concept of the \quote{insect}
functions here as a displacing and a deterritorialising force that
seeks a questioning of where and in what kind of conditions we approach
media technologies. This is perhaps an approach that moves beyond a
focus on technology {\em per se}, but still does
not remain blind to the material forces of the world. It presents an
alternative to the \quote{substance{}-approaches} that start from a stability
or a ground like \quote{technology} or \quote{humans}. It is my claim that
Deleuzian biophilosophy, that has taken elements from Spinozian
ontology, von Uexk\"ull's ethology, Whitehead's ideas as well as
Simondon's notions on individuation, is able to approach the world as
media in itself: a contracting of forces and analysing them in terms of
their affects, movements, speeds and slownesses. These affects are
primary defining capacities of an entity, instead of a substance or a
class it belongs to, as Deleuze explains in his short book
{\em Spinoza: Practical Philosophy}. From this
perspective we can adopt a novel media archaeological rewiring that
looks at media history not as one of inventors, geniuses and solid
technologies, but as a field of affects, interactions and modes of
sensation and perception.

Examples from the 19\high{th} century popular discourse are illustrative. In
1897, {\em New York Times} addressed spiders as \quote{builders, engineers
and weavers}, and also as \quote{the original inventors of a system of
telegraphy}. Spiders' webs offer themselves as ingenious communication
systems which do not merely signal according to a binary setting
(something has hit the web/has not hit the web) but transmits
information regarding the \quotation{general character and weight of any object
touching it ({\dots})} Or take for example the book {\em Beaut\'es
et merveilles de la nature et des arts} by Eli\c{c}agaray from the
18\high{th} century which lists both technological and
animal wonders, for example bees and ants, electricity and
architectural constructions as marvels of artifice and nature. 

Similar accounts abound since the mid 19\high{th} century. Insects sense, move,
build, communicate and even create art in various ways that raised
wonder and awe for example in U.S. popular culture. Apt example of the
19\high{th} century insect mania is the {\em New York Times} story (May 29,
1880) about the \quote{cricket mania} of a certain young lady who collected
and trained crickets as musical instruments:

{\QuoteStyle{200 crickets in a
wirework{}-house, filled with ferns and shells, which she called a
\quote{fernery}. The constant rubbing of the wings of these insects,
producing the sounds so familiar to thousands everywhere seemed to be
the finest music to her ears. She admitted at once that she had a mania
for capturing crickets.}}

Besides entertainment, and in a much earlier framework, the classic of
modern entomology, the aforementioned {\em An Introduction to
Entomology} by Kirby and Spence already implicitly presented throughout
its four volume best seller the idea of a primitive technics of nature
{--} insect technics that were immanent to their surroundings. 

Kirby and Spence's take probably attracted the attention it did because
of the catchy language but also what could be called its ethological
touch. Insects were approached as living and interacting entities that
are intimately coupled with their environment. Insects intertwine with
human lives (\quotation{Direct and indirect injuries caused by insects, injuries
to our living vegetable property but also direct and indirect benefits
derived from insects}), but also engage in ingenious building projects,
stratagems, sexual behaviour and other expressive modes of motion,
perception and sensation. Instead of pertaining to a taxonomic account
of the interrelations between insect species, their forms, growth or
for example structural anatomy, {\em An Introduction to Entomology}
(vol. 1) is traversed by a curiosity cabinet kind of touch on the
ethnographics of insects. Here, insects are for example war machines,
like the horse{}-fly ({\em Tabanus L}.): \quotation{Wonderful and various are
the weapons that enable them to enforce their demand. What would you
think of any large animal that should come to attack you with a
tremendous apparatus of knives and lancets issuing from its mouth?}. 

From Kirby and Spence to later entomologists and other writers, insects'
powers of building continuously attracted the early entomological gaze.
Buildings of nature were described as more fabulous than the pyramids
of Egypt or the aqueducts of Rome. Suddenly, in this weird parallel
world, such minuscule and admittedly small{}-brained entities like
termites were pictured as alike to the ancient monarchies and empires
of Western civilization. The Victorian appreciation of ancient
civilization could also incorporate animal kingdoms and their buildings
of monarchic measurements. Perhaps the parallel was not to be taken
literally, but in any case it expressed a curious interest towards
microcosmical worlds. A recurring trope was that of \quote{insect geometrics}
which seemed with accuracy, paralleled only in mathematics, to follow
and fold nature's resources into micro versions of emerging urban
culture. To quote Kirby and Spence's {\em An Introduction to
Entomology}, vol. 2:

\QuoteStyle{No thinking man ever witnesses the complexness and yet regularity and
efficiency of a great establishment, such as the Bank of England or the
Post Office without marvelling that even human reason can put together,
with so little friction and such slight deviations from correctness,
machines whose wheels are composed not of wood and iron, but of fickle
mortals of a thousand different inclinations, powers, and capacities.
But if such establishments be surprising even with reason for their
prime mover, how much more so is a hive of bees whose proceedings are
guided by their instincts alone!}

Whereas the imperialist powers of Europe headed for overseas conquests,
the mentality of exposition and mapping new terrains turned also
towards other fields than the geographical. The Seeing Eye {--} a key
figure of hierarchical modern power {--} could also be a non{}-human
eye, as with the fly which according to Steven Connor can be seen as
the recurring mode of \quotation{radically alien mode of entomological vision} with its huge eyes consisting of 4000 sensors. Hence, it is
fitting how in 1898 the idea of \quotation{photographing through a fly's eye} was
suggested as a mode of experimental vision {--} able also to catch
queen Victoria with \quotation{the most infinitesimal lens known to science},
that of a dragon fly. 

Jean{}-Jacques Lecercle explains how the Victorian enthusiasm for
entomology and insect worlds is related to a general discourse of
natural history that as a genre labelled the century. Through the
themes of \quote{exploration} and \quote{taxonomy} Lecercle claims how
{\em Alice in Wonderland} can be read as a key novel of the era in
its evaluation and classification of various life worlds beyond the
human. Like Alice in the 1865 novel, new landscapes and exotic species
are offered as an armchair exploration of worlds not merely extensive
but also opened up by intensive gaze into microcosms. Uncanny
phenomenal worlds are what tie together the entomological quest,
Darwinian inspired biological accounts of curious species and Alice's
adventures into imaginative worlds of twisting logic. In taxonomic
terms, the entomologist is surrounded by a new cult of private and
public archiving. New modes of visualizing and representing insect life
produce a new phase of taxonomy becoming a public craze instead of
merely a scientific tool. Again the wonder worlds of Alice or Edward
Lear, the Victorian nonsense poet, are the ideal point of reference for
19\high{th} century natural historian and entomologist, as Lecercle 
writes:

\QuoteStyle{And it is part of a craze for discovering and classifying new species.
Its advantage over natural history is that it can invent those species
(like the Snap{}-dragon{}-fly) in the imaginative sense, whereas
natural history can invent them only in the archaeological sense, that
is discover what already exists. Nonsense is the entomologist's dream
come true, or the Linnaean classification gone mad, because gone
creative ({\dots})}

For Alice, the feeling of not being herself and \quotation{being so many different
sizes in a day is very confusing}, which of course is something
incomprehensible to the Caterpillar she encounters. It is not queer for
the Caterpillar whose mode of being is defined by the metamorphosis and
the various perception/action{}-modulations it brings about. It is only
the suddenness of the becoming{}-insect of Alice that dizzies her. A
couple of years later, in {\em The Population of an Old{}-Pear Tree,
or Stories of insect life} (1870) an everyday meadow is disclosed as a
vivacious microcosm in itself. The harmonious scene, \quotation{like a great
amphitheatre}, is filled with life that easily escapes the (human)
eye. Like Alice, the protagonist wandering in the meadow is \quotation{lulled and
benumbed by dreamy sensations} which however transport him suddenly
into new perceptions and bodily affects. What is revealed to our boy
hero in this educational novel fashioned in the style of travel
literature (connecting it thus to the colonialist contexts of its age)
is a world teeming with sounds, movements, sensations and insect beings
(huge spiders, cruel mole{}-crickets, energetic bees) that are beyond
the human form (despite the constant tension of such narratives as
educational and moralising tales that anthropomorphize affective
qualities into human characteristics). True to entomological
classification, a big part is reserved for the structural{}-anatomical
differences of the insect life but also the affect{}-life of how
insects relate to their surroundings is under scrutiny. 

As precursors of ethology, such natural historical quests (whether
archaeological, entomological or imaginative) were expressing an
appreciation of phenomenal worlds differing from that of the human with
its two hands, two eyes and two feet. In a way, this entailed a kind of
an extended Kantianism interested not only in the conditions of
possibility of experiences, but the emergence of alternative potentials
on the immanent level of life that functions through a technics of
nature. Curiously the inspiration with new phenomenal worlds was
connected to the emergence of new technologies of movements, sensation
and communication (all challenging the Kantian apperception of Man as
the historically constant basis of knowledge and perception). Nature
was gradually becoming the \quotation{new storehouse of invention} ({\em New
York Times}, August 4, 1901) that was to entice inventors into
perfecting their developments. What I argue is that this theme can also
be read as an expression of a shift in understanding technology {--} a
shift that marked the rise of modern discourse concerning media
technologies since the end of the 19\high{th} century and that has usually
been attributed to an anthropological and ethnological turn in
understanding technology. I also address this theme in another text of
mine, \quote{Insect Technics}. For several writers such as Ernst Kapp who
became one of the predecessors of later theories of media as
\quote{extensions of man}, it was the human body that served as a storage
house of potential media. However, at the same time, another
undercurrent proposed to think of technologies, inventions and
solutions to problems posed by life as stemming from a much more
different class of bodies, namely insects. 

So beyond Kant, we move onto a baroque world, not as a period of art,
but as a mode of folding and enveloping new ways of perception and
movement. The early years and decades of technical media were
characterized by the new imaginary of communication, from work by
inventors such as Nikola Tesla to various modes of e.g. spiritualism
analyzed recently in her art works by Zoe Beloff. However, one can
radicalize the viewpoint even further and take an animal turn and not
look for alien but for animal and insect ways of sensing the world.
Naturally, this is exactly what is being proposed in a variety of media
art pieces and exhibitions. Insects have made their appearance for
example in Toshio Iwai's {\em Music Insects} (1990), Sarah Peebles'
{\em electroacoustic Insect Grooves} as an example of imaginary
soundscapes, David Dunn's acoustic ecology pieces with insect sounds,
{\em the Sci{}-Art: Bio{}-Robotic Choreography} project (2001, with
Stelarc as one of the participators), and Laura Beloff's
{\em Spinne} (2002), a networked spider installation that works
according to the web spider/ant/crawler technology.

Here we are dealing not just with representing the insect, but engaging
with the animal affects, indistinguishable from those of the
technological, as in Stelarc's work where the experimentation with new
bodily realities is a form of becoming{}-insect of the technological
human body. Imagining by doing is a way to engage directly with affects
of becoming{}-animal of media where the work of sound and body artists
doubles the media archaeological analysis of historical strata. In
other words, one should not reside on the level of intriguing
representations of imagined ways of communication, or imagined
apparatuses that never existed, but realize the overabundance of real
sensations, perceptions to contract, to fold, the neomaterialist view
towards imagined media.

\page

\SubSubTitle{Literature}

Ernest van Bruyssel, {\em The population of an old pear{}-tree; or,
Stories of insect life}. (New York: Macmillan and co., 1870).

Lewis Carroll, {\em Alice's Adventures in Wonderland and Through the
Looking Glass}. Edited with an Introduction and Notes by Roger Lancelyn
Green. (Oxford: Oxford University Press, 1998).

Claire Colebrook, \quote{The Sense of Space. On the Specificity of Affect in
Deleuze and Guattari.} In: {\em Postmodern Culture}, vol. 15, issue
1, 2004.

Steven Connor, {\em Fly}. (London: Reaktion Books, 2006).

Manuel DeLanda, {\em War in the Age of Intelligent Machines}. (New
York: Zone Books, 1991).

Gilles Deleuze, {\em Spinoza: Practical Philosophy}. Transl. Robert
Hurley. (San Francisco: City Lights, 1988).

Gilles Deleuze, {\em The Fold}. Transl. Tom Conley. (Minneapolis:
University of Minnesota Press, 1993).

Ernst Kapp, {\em Grundlinien einer Philosophie der Technik: Zur
Entstehungsgeschichte der Kultur aus neuen Gesichtspunkten}.
(Braunschweig: Druck und Verlag von George Westermann, 1877).

William Kirby \& William Spence, {\em An Introduction to Entomology,
or Elements of the Natural History of Insects}. Volumes 1 and 2.
Unabridged Faximile of the 1843 edition. (London: Elibron, 2005).

Eric Kluitenberg (ed.), {\em Book of Imaginary Media. Excavating the
Dream of the Ultimate Communication Medium}. (Rotterdam: NAi
publishers, 2006).

Jean{}-Jacques Lecercle, {\em Philosophy of Nonsense: The Intuitions
of Victorian Nonsense Literature}. (London: Routledge, 1994).

Jussi Parikka, \quote{Insect Technics: Intensities of Animal
Bodies.} In: {\em (Un)Easy Alliance {}- Thinking the
Environment with Deleuze/Guattari}, edited by Bernd
Herzogenrath. (Newcastle: Cambridge Scholars Press, Forthcoming
2008).

Siegfried Zielinski, \quote{Modelling Media for Ignatius Loyola. A Case Study
on Athanius Kircher's World of Apparatus between the Imaginary and the
Real.} In: {\em Book of Imaginary Media}, edited by Kluitenberg.
(Rotterdam: NAi, 2006).}
