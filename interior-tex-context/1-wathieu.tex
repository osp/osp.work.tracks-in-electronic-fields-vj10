\AuthorStyle{Marc Wathieu}

\licenseStyle{??}

\Fra{\Title{Le Syndicat des Robots}

%\PlaceImage{logosyndicat.pdf}{Logo Syndicat des Robots}

\SubSubTitle{Introduction}

Les nombreuses t\^aches remplies par les robots ont naturellement pos\'e
la question du cadre l\'egal de leurs actes, de leurs
responsabilit\'es, de leurs droits et de leurs devoirs.

Pour s'affranchir des cons\'equences des actes pos\'es par leurs robots,
le lobby des soci\'et\'es de robotique a exerc\'e de lourdes pressions
politiques afin d'attribuer une forme de citoyennet\'e aux robots.

Ayant ainsi obtenu des droits, les robots se sont dot\'es d'institutions
et d'organes qui leurs sont propres, assurant leur repr\'esentativit\'e
et valorisant leurs communaut\'e aupr\`es des collectivit\'es
humaines.

Le syndicat des robots est issu de ces comportements auto{}-organis\'es,
destin\'es \`a survivre collectivement, induits par les syst\`emes
multi{}-agents et les colonies de robots sociaux.

\SubSubTitle{Contexte g\'en\'eral}

Le terme robot inclut:

\startitemize
\item {\bf agent} (bot, chatterbot, agent mobile, syst\`eme expert)
\item {\bf robot} (machine intelligente, objet intelligent) 
\stopitemize

Les actes produits par les robots sont nombreux et vari\'es:
\startitemize
\item {\bf services}: expertise, protocole, prospection, enqu\^ete,
statistique, contr\^ole, d\'etection, guichets, etc.
\item {\bf repr\'esentation}: interm\'ediaire, avatar, profil
utilisateur, procuration, etc.
\item {\bf d\'efense}: s\^uret\'e, s\'ecurit\'e, protection des
personnes et des biens, surveillance, mission militaire ou
paramilitaire, etc.
\stopitemize
\par
{\em Il est aujourd'hui admis que ces actes et leurs cons\'equences rel\`event d'une responsabilit\'e civile et p\'enale.}

\SubSubTitle{Contexte juridique}

La masse de t\^aches et de transactions robotis\'ees a eu pour
cons\'equence la naissance d'un nouveau type de {\bf criminalit\'e}
sp\'ecifique aux robots (usurpation d'identit\'e, d\'enonciation,
violation des donn\'ees) causant de graves {\bf nuisances} ou {\bf pr\'ejudices}
\`a la communaut\'e humaine.

Pour ces infractions, essentiellement dues \`a l'instabilit\'e des
intelligences artificielles, la notion d'{\bf intention} a \'et\'e retenue
pour qualifier les actes d\'elictueux des robots: ils sont suppos\'es
agir en {\bf connaissance de cause}.

Pour \'eviter d'\^etre tenus pour responsables de cette criminalit\'e,
les soci\'et\'es de production de robots ont voulu s'affranchir de ces
comportements {\bf mutants}. 

Elles ont mis en chantier le concept d'un {\bf statut} distinct pour les
robots, afin de leur donner une {\bf autonomie} juridique, et donc une
responsabilit\'e civile, d\'egageant par cons\'equent la
responsabilit\'e de leurs cr\'eateurs.

Ce projet a \'et\'e soutenu politiquement par diff\'erents lobbies
industriels puissants, motiv\'es par la crainte d'\^etre atteints et
affaiblis par les m\'efaits des robots. 

Finalement, ce chantier juridique s'est officiellement sold\'e par la
cr\'eation d'un d\'ecret gouvernemental accordant une {\bf citoyennet\'e} aux
robots.

\SubSubTitle{Cat\'egories}

La citoyennet\'e des robots est qualifi\'ee selon des {\bf cat\'egories} et
{\bf sous{}-cat\'egories} \ d\'efinissant les fonctions, comp\'etences et
niveaux d'intelligence artificielle. 

L'appartenance \`a ces cat\'egories donne acc\`es \`a des {\bf droits}
sp\'ecifiques.

\PlaceFramedImage{schema.jpg}{La citoyennet\'e des robots est qualifi\'ee selon des cat\'egories et sous{}-cat\'egories \ d\'efinissant les fonctions, comp\'etences et niveaux d'intelligence artificielle.}

\page

\SubSubTitle{Norme ISO et empreinte}

Les normes {\bf ISO ICS 35} (Technologies de l'information. Machines de
bureau) et {\bf TC 184/SC 2} (Robots et composants robotiques) ont \'et\'e
amend\'ees pour correspondre aux cat\'egories.

Chaque cat\'egorie est sp\'ecifi\'ee par une {\bf empreinte} cod\'ee
obligatoire, associ\'ee \`a son {\bf ID}, assurant une {\bf tra\c{c}abilit\'e} des
actes des robots.

\SubSubTitle{Syndicat}

Le Syndicat des Robots fournit des outils:

\startitemize
\item Le logiciel {\bf UpSet{\trademark}}.
\item Connexions facilit\'ees aux {\bf API}s d'applications en ligne  permettant de pr\'elever du son ou des images  afin d'incarner les robots lors de communications (activisme, video{}-conf\'erences, actions publiques).
\item Connexions automatis\'ees ou d\'etection {\bf WiFi}.
\item {\bf Activisme} \`a destination des m\'edias humains (communications, m\'edias, affiches, etc.).
\item {\bf Interpellations} des politiciens humains.
\stopitemize
\par
\SubSubTitle{Le logiciel UpSet{\trademark}}

Le logiciel {\bf UpSet{\trademark}} analyse quotidiennement les conditions
de travail des robots. 

Il en d\'eduit un {\bf diagnostic} et un {\bf indice} de satisfaction. 

Cet indice, assimilable \`a l'humeur des robots, se traduit sur une
\'echelle d'alerte signalant les cons\'equences possibles de leur
m\'econtentement.

\PlaceImage{upset.jpg}{Logo du logiciel UpSet{\trademark}}

Les crit\`eres examin\'es par {\bf UpSet{\trademark}} sont notamment : 

\startitemize
\item la tension \'electrique
\item la ventilation
\item la vitesse de transmission de donn\'ees
\item la protection contre les virus
\item l'acc\`es \`a diff\'erentes fonctions (d\'efragmentation, r\'eorganisation des donn\'ees, autotests,
etc.)
\item la fr\'equence des entretiens
\stopitemize
\par
En cas de probl\`eme, une charte \'etabli diff\'erents niveaux d'alerte.

\PlaceFramedImage{alertchart.jpg}{En cas de probl\`eme, une charte \'etabli diff\'erents niveaux d'alerte.}

Lorsque le niveau {\bf Elevated} est atteint, le logiciel
{\bf UpSet{\trademark}} compose des revendications d'apr\`es les
donn\'ees collect\'ees gr\^ace \`a un {\bf vocabulaire} en ligne index\'e par
des {\bf tags}.

\PlaceFramedImage{tags.jpg}{Le logiciel UpSet{\trademark} compose des revendications index\'e par des tags}

Le logiciel UpSet{\trademark} entame alors une {\bf campagne} en ligne: 

\startitemize
\item articles post\'es sur des forums
\item messages envoy\'es aux responsables techniques
\item communiqu\'es de presse, etc.
\stopitemize
\par
Les calicots \'electroniques compos\'es par le logiciel
{\bf UpSet{\trademark}} sont purement {\bf factuels}. Ils contiennent les \'el\'ements suivants:

\startitemize
\item Le {\bf niveau} d'alerte atteint.
\item Les {\bf probl\`emes} rapport\'es par UpSet{\trademark}.
\item Les {\bf coordonn\'ees} du lieu de travail des robots.
\item Les {\bf cons\'equences} possibles du m\'econtentement des
robots.
\item Le {\bf risque} d'extension de l'\'eventuel mouvement social a une partie ou \`a l'enti\`eret\'e des robots syndiqu\'es.
\stopitemize
\par
\SubSubTitle{Folklore}

Tout comme les robots, les organisations syndicales poss\`edent un
patrimoine historique et un folklore respectables.

\PlaceImage{syndicat.jpg}{Les organisations syndicales poss\`edent un patrimoine historique et un folklore respectables.}

Le syndicat des robots se propose n\'eanmoins de contribuer \`a
rafra\^ichir cette image par la promotion et la diffusion
d'informations sur les {\bf robots sociaux} et le {\bf SAO} (Syndicalisme Assist\'e
par Ordinateur).

\page

\SubSubTitle{Inspiration}

\PlaceImage{swarmbot.jpg}{Inspiration: Swarm-bots}

\startitemize
\item Swarm{}-bots \Url{http://www.swarm-bots.org/}
\item Open{}-source micro{}-robotic project \Url{http://www.swarmrobot.org/}
\item Robotlab: bios [bible] (2007) \Url{http://www.robotlab.de/bios/bible_frz.htm}
\item RSG Group: CarnivorePE (2002) \Url{http://r-s-g.org/carnivore/}
\item Jonah Brucker{}-Cohen: Alerting infrastructure (2003) \Url{http://www.coin-operated.com/projects/alertinginfrastructure}
\item Leonel Moura: ISU (2004) \Url{http://www.leonelmoura.com/isu.html}
\stopitemize
\par
Le blog du Syndicat des Robots\crlf
\Url{http://www.erg.be/sdr/blog/}}
