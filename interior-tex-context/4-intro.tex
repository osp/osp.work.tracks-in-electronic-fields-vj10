{\tfa\setupinterlinespace \Eng{Whether we operate a computer with the help of a command line interface,
or by using buttons, switches and clicks{\dots} the exact location of
interaction often serves as conduit for mutual knowledge {}- machines
learn about bodies and bodies learn about machines. Dialogues happen at
different levels and in various forms: code, hardware, interface,
language, gestures, circuits.

Those conversations are sometimes gentle in tone {}- ubiquitous requests
almost go unnoticed {}- and other times they take us by surprise
because of their authoritative and demanding nature: \quotation{Put That
There}. How can we think about such feed back loops in productive
ways? How are interactions translated into software, and how does
software result in interaction? Could the practice of using and
producing free software help us find a middle ground between
technophobia and technofetishism? Can we imagine ourselves and our
realities differently, when we try to re{}-design interfaces in a
collaborative environment? Would a different idea about \quote{user} change
our approach to \quote{use} as well?}

\Ned{Of we nu een computer met behulp van een commandline{}-interface
bedienen, of door op knoppen te klikken en schakelaars te
gebruiken{\dots} de plek waar de interactie tussen mens en machine
plaatsvindt, vormt een doorgeefluik voor wederzijdse kennis {}-
machines leren over lichamen, en lichamen leren over machines. De
dialoog speelt zich af op verschillende niveaus en in diverse vormen:
code, hardware, interface, taal, gebaren. De uitwisseling is meestal
subtiel van toon {--} soms gaat ze zelfs bijna ongemerkt voorbij, en
soms zijn we verrast door haar gebiedende wijs: \quotation{Zet Dat Daar!}.

Hoe kunnen we constructief over die {\em feed{}-back loop} denken? Hoe wordt
interactie vertaald in software, en hoe resulteert software in
interactie? \quotation{Computers zijn niet meer slechts hulpmiddelen (als zij
dat ooit al waren), maar complexe systemen die meer en meer de
voorwaarden, ideologie\"en, veronderstellingen en praktijken helpen
vormen die bepalen wat wij werkelijkheid noemen} schrijft Katherine
Hayles, en we zijn het met haar eens dat we het ons niet kunnen
veroorloven slechts eindgebruikers te zijn van de technologie\"en
waarmee we zo intensief werken. Zou de praktijk van gebruik en
productie van vrije software ons een tussenweg kunnen bieden tussen
technophobia en technofetishisme? Kunnen wij onszelf en onze omgeving
misschien anders voorstellen, wanneer we proberen interfaces collectief
te herontwerpen? Zou een ander idee over \quote{gebruiker} ook onze
benadering van \quote{gebruik} veranderen?

Wederkerige Bewegingen hanteerde het bereiden van voedsel als
referentiekader; de keuken als plaats om te beginnen mens{}-machine
configuraties te heroverwegen zonder ze los te zien van het dagelijkse
leven en de genderpatronen die daarin een rol spelen. Plannen, code,
instructies, notaties en manuals vertegenwoordigen elk een fase in de
cyclus van bewegingingen die voorwerpen en onderwerpen in elkaar
veroorzaken; van verbeelding, tot ingre\-diënten en geheugen {}-
recepten in een kookbook voor een gedeelde ruimte.}

\Fra{Que nous exploitions un ordinateur avec l'aide d'une interface de ligne de commande, ou en utilisant des boutons, des switches ou des clicks..., l'emplacement exact d'interaction sert souvent \`a produire une connaissance commun{}- les machines apprennent des corps et les corps apprennent des machines. Les dialogues se produisent \`a diff\'erents niveaux et sous diff\'erentes formes: le code, le hardware, l'interface, la langue, les gestes, les circuits.

Ces conversations sont parfois courtoises {}- et leur omnipr\'esence
passe inaper\c{c}ue {--} et d'autres fois, elles nous surprennent par
leur nature autoritaire: \quotation{Mets cela ici}. Comment pouvons{}-nous approcher ces boucles de
feedback de mani\`ere productive? Comment les interactions sont{}-elles traduites dans le logiciel et comment le logiciel aboutit-il à l'interaction? Est{}-ce que la pratique du logiciel libre peut nous
permettre de trouver un moyen terme entre la technophobie et le
technof\'etichisme? Pouvons{}-nous imaginer diff\'eremment nous{}-m\^emes et nos
r\'ealit\'es, lorsque nous tentons de re{}-cr\'eer les
interfaces d'un environnement de travail collaboratif? Est{}-ce qu'une
id\'ee diff\'erente de l'usage change notre approche de l'ergonomie?

}

}
