\PlaceImage{praticable0.jpg}{Workshop for participants with different body practices at V/J10}
\PlaceImage{praticable1.jpg}{The body as a locus of knowledge production was made tangible}
\PlaceImage{praticable2.jpg}{}
\PlaceImage{praticable3.jpg}{}

\AuthorStyle{Alice Chauchat, Fr\'ed\'eric Gies}

\licenseStyle{Attribution{}-Noncommercial{}-No Derivative Work}

\Eng{\Title{Praticable}

Praticable is a collaborative research project between several artists
(currently: Alice Chauchat, Fr\'ed\'eric de Carlo, Fr\'ed\'eric Gies,
Isabelle Schad and Odile Seitz).

Praticable proposes itself as a horizontal work structure, which brings
research, creation, transmission and production structure into relation
with each other. This structure is the basis for the creation of a
variety of performances by either one or several of the project's
participants. In one way or another, these performances start from the
exploration of body practices, leading to a questioning of its
representation. More concretely, Praticable takes the form of
collective periods of research and shared physical practices, both of
which are the basis for various creations. These periods of research
can either be independent of the different creation projects or
integrated within them.

During Jonctions/Verbindingen 10, Alice Chauchat and Fr\'ed\'eric Gies
gave a workshop for participants dealing with different \quote{body
practices}. On the basis of Body{}-Mind Centering (BMC) techniques, the
body as a locus of knowledge production was made tangible. The notation
of the Dance performance with which Fr\'ed\'eric Gies concluded the day
is reproduced in this book and published under an open license.}

\Ned{\Title{Praticable}

Praticable is een onderzoeks{}- en samenwerkingsproject van
verschillende kunstenaars (momenteel: Alice Chauchat, Fr\'ed\'eric de
Carlo, Fr\'ed\'eric Gies, Isabelle Schad en Odile Seitz). Praticable
tekent zich af als een horizontale werkstructuur waarbinnen onderzoek,
creatie, overdracht en productiestructuur, met elkaar in relatie worden
gebracht. Deze structuur vormt de basis voor de creatie van
verschillende voorstellingen, gedragen door een of meerdere
participanten aan het project. Deze voorstellingen vertrekken steeds op
een of andere manier van een exploratie van praktijken van het lichaam
om zo tot een bevraging van haar representatie te komen.

Concreet neemt Praticable de vorm aan van gedeelde tijd voor onderzoek
alsook van gemeenschappelijke fysieke praktijk, die beide als basis
dienen voor verschillende creaties. Deze periodes van onderzoek kunnen
zich hetzij onafhankelijk ontwikkelen van de verschillende
creatieprojecten, hetzij zich er in integreren.

Tijdens Jonctions/Verbindingen 10 gaven Alice Chauchat en Fr\'ed\'eric
Gies een workshop voor deelnemers met ieder hun eigen
\quote{praktijk van het lichaam}. Aan de
hand van Body{}-Mind Centering (BMC) technieken werd het lichaam als
locus van kennisproductie, ervaarbaar gemaakt. De notatie van de
voorstelling Dance waarmee Fr\'ed\'eric Gies de dag afsloot, is in dit
boek opgenomen en gepubliceerd onder een open licentie.}

\Fra{\Title{Praticable}

Praticable est un projet de recherche et de collaboration entre
plusieurs artistes (\`a ce jour: Alice Chauchat, Fr\'ed\'eric de
Carlo, Fr\'ed\'eric Gies, Isabelle Schad et Odile Seitz). Praticable
est une structure horizontale de travail qui met en relation recherche,
cr\'eation, transmission et structure de production. Cette structure
est la base pour la cr\'eation de plusieurs pi\`eces, sign\'ees par un
ou plusieurs participants au projet. Ces pi\`eces s'attachent, d'une
mani\`ere ou d'une autre, \`a partir de l'exploration de pratiques du
corps pour aller vers la repr\'esentation. Concr\`etement, Praticable
prend la forme de temps de recherche et de pratique physique en commun,
servant de terreau pour les diff\'erentes cr\'eations. Ces temps de
recherches peuvent se d\'erouler de mani\`ere ind\'ependante par
rapports aux diff\'erents projets de cr\'eations, aussi bien que
s'int\'egrer \`a ceux{}-ci.

Pendant Verbindingen/Jonctions 10, Alice Chauchat et Fr\'ed\'eric Gies
ont donn\'e un atelier qui a permis d'exp\'erimenter
diff\'erentes \quote{pratiques du corps}. Sur la base des techniques de
Body{}-Mind Centering (BMC), les participant{}-e{}-s ont pu d\'ecouvrir
les potentialit\'es de leur corps comme un lieu de production de
savoir. La notation de la performance Dance avec laquelle
\ Fr\'ed\'eric Gies a conclu la journ\'ee est reproduite dans ce livre
et publi\'ee sous une license ouverte.}

