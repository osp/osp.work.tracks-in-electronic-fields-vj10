\PlaceImage{kleiner01.JPG}{Lezing Dmytri Kleiner tijdens V/J10}

\AuthorStyle{Dmytri Kleiner, Brian Wyrick}
\licenseStyle{Dmytri Kleiner \& Brian Wyrick, 2007. Anti{}-Copyright. Use as desired in whole or in part. Independent or collective commercial use encouraged. Attribution optional.\par}
\Remark{Text first published in English in Mute: http://www.metamute.org/InfoEnclosure-2.0. For translations in Polish and Portuguese, see \Url{http://www.telekommunisten.net}\par}

\Ned{\Title{InfoEnclosure 2.0}

{\em De hype die zich rond het vermogen van Web 2.0 om inhoudproductie te democratiseren voordoet, verdonkeremaant het feit dat het eigenaarschap en de middelen tot} sharing centraal {\em stelt. Dmytri Kleiner \& Brian Wyrick ontsluieren Web 2.0 als het paradijs van de}  venture capitalist{\em, waar investeerders de waarde geproduceerd door onbezoldigde gebruikers in eigen zak steken, zich verrijken aan de technische innovaties van de vrijesoftwarebeweging, en het decentraliserend potentieel van peer{}-\-to{}-\-peer{}-\-productie de kop indrukken.}

Op Wikipedia lezen we dat \quotation{Web 2.0, een uitspraak gedaan door O'Reilly
Media in 2004, refereert naar een veronderstelde tweede generatie van
internetgebaseerde diensten {--} zoals sociale netwerksites, wiki's,
communicatietools, en folksonomie\"en {--}, die online samenwerking en
sharing tussen gebruikers benadrukken.}

Het gebruik van het woord \quote{verondersteld} is hier noemenswaardig. Als
het waarschijnlijk meest omvangrijke werk uit de geschiedenis ontstaan
uit collaboratief auteurschap, en als een van de huidige favorieten van
de internetgemeenschap, zal Wikipedia het wel weten.

In tegenstelling tot de meeste leden van de Web 2.0{}-generatie, wordt
Wikipedia gecontroleerd door een non{}-profitstichting, verwerft ze
enkel inkomsten door donatie, en geeft ze haar inhoud vrij onder de
copyleft GNU Vrije Documentatie Licentie. Het is veelzeggend dat
Wikipedia haar relaas vervolgt met: \quotation{[Web 2.0] is een populair (maar
onvolkomen gedefinieerd en vaak bekritiseerd) modewoord onder bepaalde
technische gemeenschappen en marketinggemeenschappen.}

De vrijesoftwaregemeenschap heeft steeds {--} als ze al niet volstrekt
afwijzend was {--} een neiging tot achterdocht gekend tegenover de term
\quote{Web 2.0}. Tim Berners{}-Lee doet de
term als volgt af: \quotation{Web 2.0 is klaarblijkelijk niets meer dan hol
jargon, en niemand weet dan ook wat het precies betekent.} Hij vervolgt
met de opmerking dat \quotation{het louter betekent de standaarden te gebruiken,
geproduceerd door al die mensen die aan Web 1.0 werkten.}

In realiteit is er noch sprake van \quote{Web 1.0} noch van \quote{Web 2.0}, er is
enkel een continue ontwikkeling van online toepassingen die niet
glashelder van elkaar te onderscheiden zijn.

In een poging te defini\"eren wat Web 2.0 is, kunnen we gerust stellen
dat het grotendeel van de significante ontwikkelingen gericht was op
het in staat stellen van de gemeenschap om op een dergelijke manier
inhoud te cre\"eren, aan te passen, en te delen, die voordien
voorbehouden was aan gecentraliseerde organisaties, die dure
softwarepakketten aankochten, die staf betaalden om zich van de
technische aspecten van de site te kwijten en om inhoud te cre\"eren
die over het algemeen slechts op de site van de organisatie werd
gepubliceerd.

Een Web 2.0{}-bedrijf verandert de productiewijze van internetinhoud
fundamenteel. Webtoepassingen en webdiensten zijn goedkoper en
gemakkelijker implementeerbaar geworden, en door de eindgebruikers
toegang tot deze toepassingen te verschaffen, kan een bedrijf op
effectieve wijze de creatie en de organisatie van haar inhoud aan de
eindgebruikers zelf uitbesteden. In plaats van het traditionele model
van een inhoudprovider die zijn eigen inhoud publiceert, en de
eindgebruiker die dit louter consumeert, laat het nieuwe model toe dat
de site van het bedrijf ageert als het gecentraliseerd portaal tussen
gebruikers, die zowel cre\"eren als consumeren.

Toegang tot deze toepassingen stelt de gebruikers bij machte inhoud te
cre\"eren en te publiceren waarvoor ze zich voordien desktopsoftware
hadden moeten aanschaffen, en waarvoor ze over een ruimer stel
technologische vaardigheden dienden te beschikken. Bijvoorbeeld, twee
van de meest elementaire middelen om tekstgebaseerde inhoudproductie
voort te brengen zijn blogs en wiki's, die de gebruikers in staat
stellen om onmiddellijk vanuit hun browser inhoud te cre\"eren en te
publiceren, zonder enige vereiste kennis van markuptaal,
bestandenoverdracht of syndicatieprotocollen, en zonder zich enige
software te moeten aanschaffen.

Het gebruik van de webapplicatie om desktopsoftware te vervangen, is
zelfs nog significanter voor de gebruiker als het op inhoud aankomt die
niet louter tekstueel is. Niet enkel kunnen webpagina's in de browser
gecre\"eerd en ge\"edit worden zonder de aanschaf van
HTML{}-editingsoftware, foto's kunnen ook online door de browser
ge\"upload en gemanipuleerd worden, zonder nood aan kostelijke
desktop{}-beeldmanipulatietoepassingen. Een video opgenomen met een
{\em gewone} camcorder, kan ondergebracht worden op een
videohosting{}-site, kan ge\"upload worden, ge\"encodeerd, ingebed in
een HTML{}-pagina, gepubliceerd, getagd, en gesyndiceerd op het web, en
dit allemaal via de browser van de gebruiker.

In Paul Grahams artikel over Web 2.0 splitst deze de verschillende
rollen van de gemeenschap/gebruiker op in meer specifieke rollen,
namelijk die van de Professional, de Amateur, en de Gebruiker (meer
bepaald, de eindgebruiker). Volgens Graham werden de rollen van de
Professional en de Gebruiker in Web 1.0 goed begrepen, maar kreeg de
Amateur geen welomlijnde plaats toebedeeld. Zoals Graham het beschrijft
in \quote{{\em What Business Can Learn From Open Source}}, houdt de Amateur er
gewoonweg van om te werken, zonder bekommernis om compensatie voor, of
eigenaarschap van dat werk; in de ontwikkeling draagt de Amateur bij aan de opensourcesoftware,
terwijl de Professioneel betaald wordt voor werk dat haar/zijn eigendom
is.

Grahams karakterisering van de \quote{Amateur} reminisceert aan {\em If I
Ran The Circus} van Dr. Suess, waarin de jonge Morris McGurk de staf
van zijn imaginaire Circus McGurkus als volgt beschrijft:

\QuoteStyle{Mijn werkers {\em houden} van werk. Ze zeggen, \quote{Werk ons! Alstublieft werk ons!}. We zullen werken en werken, en zovele verrassingen voortbrengen. Nauwelijks de helft zou je kunnen zien, al had je veertig paar ogen!}

En hoewel \quote{Web 2.0} dan wel niets mag betekenen voor Tim Berners{}-Lee,
die recente innovaties als niets meer dan de voortgezette ontwikkeling
van het web ziet, en hoewel {\em venture capitalists} net zoals
Morris McGurk dagdromen van onvermoeibare werkers die eindeloze inhoud
produceren en hier geen looncheque voor vragen, toch klinkt het
allemaal wonderbaarlijk. En inderdaad, van YouTube tot Flickr tot
Wikipedia: al bezat je veertig paar ogen, dan nog zou je ogen te kort
komen om nog maar de helft te kunnen bekijken.

Tim{}-Berners Lee heeft gelijk. Vanuit een technisch standpunt of een
gebruikersstandpunt valt er niets in Web 2.0 te bespeuren dat zijn
wortels niet in Web 1.0 heeft, of dat er geen natuurlijke ontwikkeling
uit is. De technologie die onder de Web 2.0{}-vaandel ondergebracht
wordt, was reeds voordien mogelijk en in sommige gevallen onmiddellijk
voorhanden, maar de hype waarin dit gebruik baadt, heeft de groei van
Web 2.0{}-internetsites onomstotelijk be\"invloed. 

Het internet (dat in feite meer omvat dan het web) ging altijd om
{\em sharing} tussen gebruikers. Meer bepaald Usenet, een
gedistribueerd berichtensysteem, is reeds operatief sinds 1979! Dus
reeds lang voor er zelfs maar sprake was van Web 1.0, hostte Usenet
discussies en \quote{amateurjournalistiek}, en liet het {\em sharing} van
foto's en bestanden toe. Net zoals internet is het een gedistribueerd
systeem dat van niemand het bezit is, dat door niemand gecontroleerd
wordt. Het is deze kwaliteit, dit ontbreken van centraal eigenaarschap
en centrale controle, dat diensten zoals Usenet onderscheidt van Web
2.0.

Als Web 2.0 in enigermate iets zou betekenen, dan ligt haar betekenis in
de rationale van venturekapitaal. Web 2.0 betekent de terugkeer van
investering in internet{}-startups. Na de dotcom{}-flop (de werkelijke
doodsteek van Web 1.0) hadden zij die naar investeringsdollars dongen
een nieuwe rationale nodig voor het investeren in online ondernemingen.
{\em \quote{Build it and they will come}}, de dominante attitude van de
dotcom{}-boom uit de jaren '90, die hand in hand ging met het waanbeeld
van de \quote{nieuwe economie}, was niet langer aantrekkelijk nadat zoveel
online bedrijven het onderspit dolven. Infrastructuur uitbouwen en
werkelijke kapitalisatie financieren was niet langer wat investeerders
voor ogen hadden. Beslag leggen op waarde die door anderen geproduceerd
wordt, bleek echter een meer begerenswaardig voorstel.

Web 2.0 is \quote{Internet Investeringsboom 2.0}. Web 2.0 is een
ondernemingsmodel, het houdt de private inbezitneming van door de
community gecre\"eerde waarde in. Niemand ontkent dat de technologie
van sites zoals bijvoorbeeld YouTube triviaal is. Dit is overduidelijk
als men naar het grote aantal identieke diensten zoals DailyMotion
kijkt. De werkelijke waarde van YouTube wordt niet gecre\"eerd door de
ontwikkelaars van de site, maar eerder door de mensen die video's op de
site uploaden. Toen YouTube echter gekocht werd voor Google{}-aandelen
die meer dan een biljoen dollar waard zijn, hoeveel van deze aandelen
werden dan verworven door diegenen die al de video's gemaakt hebben?
Zero. Zilch. Nada. Een {\em great deal} als je een eigenaar van een
Web 2.0{}-bedrijf bent.

De waarde geproduceerd door gebruikers van Web 2.0{}-diensten zoals
YouTube, wordt in bezit genomen door kapitalistische investeerders. In
sommige gevallen eindigt de feitelijke inhoud die ze aanleveren als het
eigendom van de eigenaars van de site. Door de private toe{}-eigening
van door de gemeenschap gecre\"eerde waarde, wordt de belofte van het
delen van technologie en van vrije samenwerking verbroken.

In tegenstelling tot Web 1.0, waarbij investeerders vaak dure
kapitaalacquisitie, softwareontwikkeling en inhoudscreatie
financierden, dient een Web 2.0{}-investeerder voornamelijk
financi\"ele middelen in het voeden van de hype te pompen, in marketing
en {\em buzz}. De infrastructuur is ruim verkrijgbaar aan lage
prijs, de inhoud is gratis, en de kost van de software {--} ook al dan
toch \'e\'en zaak die niet volkomen gratis verworven kan worden {--},
is tevens verwaarloosbaar. Eigenlijk ben je door het aanbieden van wat
bandbreedte en diskruimte in staat een succesvolle internetsite op touw
te zetten, zolang je jezelf maar effectief weet aan te prijzen.

Het succes van een Web 2.0{}-bedrijf drijft hoofdzakelijk op haar
relatie tot de gemeenschap, meer bepaald op het vermogen van het
bedrijf om de \quote{collectieve intelligentie te benutten}, zoals O'Reilly
het formuleert. Web 1.0{}-bedrijven waren te monolithisch en
unilateraal in hun benadering van inhoud. Succesverhalen over de
transitie van Web 1.0 naar Web 2.0 waren gestoeld op het vermogen van
een bedrijf om monolithisch te blijven in haar merkinhoud, of beter
nog, in haar onverdeeld eigenaarschap van die inhoud, terwijl ze
tegelijkertijd de methode voor deze inhoudscreatie voor de gemeenschap
ontsloot. Yahoo! cre\"eerde een portaal naar community{}-inhoud,
terwijl ze ook de gecentraliseerde locatie voor het vinden van die
inhoud bleef. Ebay stelt de community in staat haar goederen te
verhandelen, terwijl ze de marktplaats voor deze goederen bezit.
Amazon, dat dezelfde producten als vele andere sites verkoopt, verwierf
succes door de community de mogelijkheid te verlenen om te participeren
in de {\em \quote{flow}} die hun producten omgeeft.

Omdat de kapitalisten die in Web 2.0{}-startups investeren haast nooit
vroege kapitalisatie financieren, is hun gedrag ook markant meer
parasitair. Ze zijn vaak de \quote{laatkomers} in het economische spel:
wanneer waardecreatie reeds een goed momentum kent, verschijnen ze op
het toneel om het eigenaarschap naar zich toe te trekken, en schakelen
ze hun financi\"ele macht in om de dienst te promoten, vaak binnen de
context van een hegemonisch netwerk van grote, goed gefinancierde
partners. Dit betekent dat bedrijven die niet verworven zijn door
venturekapitaal, financieel op droog zaad raken en het spel
buitengekegeld worden.

In al deze gevallen wordt de waarde van de internetsite niet gecre\"eerd
door de betaalde staf van het bedrijf dat de site runt, maar door de
gebruikers. Met al de nadruk die gelegd wordt op inhoud die door de
gemeenschap gecre\"eerd en gedeeld wordt, wordt de andere kant van de
medaille van de Web 2.0{}-ervaring al snel over het hoofd gezien: 

het eigenaarschap van al deze inhoud en het vermogen haar waarde te
monetariseren. Dit is iets waar de gebruikers niet zo vaak bij
stilstaan; het vertaalt zich slechts in de kleine lettertjes van hun
MySpace{}-gebruiksovereenkomst, of in de Flickr.com in de url van hun
foto's. Gewoonlijk ervaart de gemeenschap dit niet echt als een
knelpunt, het wordt eerder gezien als een kleine prijs, te betalen voor
het gebruik van deze wonderlijke toepassingen en voor het verbluffende
effect van de zoekmachineresultaten wanneer men zijn of haar eigen naam
ingeeft. Gezien de meeste gebruikers geen toegang hebben tot
alternatieve middelen om hun eigen inhoud te produceren en te
publiceren, worden ze aangetrokken tot sites als MySpace en Flickr.

In tussentijd kondigde de {\em corporate world} een volkomen
verschillend idee van de Informatiesnelweg af, door het produceren van
monolithische, gecentraliseerde \quote{online diensten} zoals CompuServe,
Prodigy en AOL. Wat deze van het internet onderscheidden was dat ze
gecentraliseerde systemen waren waarmee alle gebruikers onmiddellijk
konden connecteren, terwijl internet een peer{}-to{}-peer{}-netwerk is,
waarbij elk apparaat met een internetadres onmiddellijk kan
communiceren met eender welk ander apparaat. Dit is wat
peer{}-to{}-peer{}-technologie mogelijk maakt, en dit is ook wat
onafhankelijke internetdienstenproviders mogelijk maakt.

Er dient ook vermeld te worden dat vele opensourceprojecten kunnen
beschouwd worden als de sleutelinnovaties in de ontwikkeling van Web
2.0: vrije software zoals Linux, Apache, PHP, MySQL, Python, etc.
vormen de ruggengraat van Web 2.0, en van het web zelf. Deze projecten
vertonen echter een fundamenteel gebrek, dat zich aftekent in het licht
van wat O'Reilly als de \quote{Kerncompetenties van Web 2.0{}-bedrijven}
aanduidt, namelijk de controle over unieke, moeilijk te vergaren
databronnen die rijker worden naarmate meer mensen hen gebruiken {--}
het aanwenden van de collectieve intelligentie die ze aantrekken. De
gemeenschap toelaten openlijk bij te dragen, en die contributie
gebruiken binnen de context van een eigendomssysteem waarbij de
eigenaar de inhoud bezit, is een karakteristiek van een succesvol Web
2.0{}-bedrijf. De gemeenschap toelaten te bezitten wat ze cre\"eert, is
dat echter niet. Dus, om succesvol te zijn en winst voor investeerders
te cre\"eren, dient een Web 2.0{}-bedrijf mechanismen voor
{\em sharing} en samenwerking te cre\"eren die centraal
gecontroleerd worden. Het ontbreken van die centrale controle zoals bij
Usenet en andere peer{}-gecontroleerde technologie\"en, is de
fundamentele tekortkoming. Enkel de gebruikers halen er voordeel uit,
en niet de afwezige investeerders, want ze zijn immers \quote{bezit} van
niemand. 

Dus, omdat Web 2.0 gefinancierd wordt door Kapitalisme 2006, werd Usenet
grotendeels vergeten. Terwijl iedereen Digg en Flickr gebruikt, en
YouTube een biljoen dollar waard is, is PeerCast, een innovatief
peer{}-to{}-peer{}-livevideostreamingnetwerk dat reeds verscheidene
jaren langer dan YouTube bestaat, virtueel onbekend. 

Vanuit een technologisch standpunt zijn gedistribueerde en
peer{}-to{}-peer{}-technologie\"en (P2P) veel effici\"enter dan de Web
2.0{}-systemen. Daar ze beter gebruik maakt van netwerkbronnen door de
computers en netwerkverbindingen van gebruikers aan te wenden, vermijdt
P2P de knelpunten voortgebracht door gecentraliseerde systemen, en laat
ze toe dat inhoud aan de hand van minder infrastructuur gepubliceerd
kan worden, vaak met niets meer dan een computer en een
consumenteninternetconnectie. P2P{}-systemen vereisen niet de massieve
datacentra van sites zoals YouTube. Het ontbreken van een centrale
infrastructuur gaat ook gepaard met het ontbreken van centrale
controle, wat ook betekent dat censuur vaak een probleem is wanneer men
met privaatrechtelijke \quote{community's} te doen heeft, die buigen naar de
wensen van private en publieke drukkingsgroepen en die beperkingen
opleggen aan de aard van de inhoud. Ook het ontbreken van grote
centrale crossreferenti\"ele databanken van gebruikersinformatie, is
een aanzienlijk voordeel wat privacy betreft. 

Vanuit dit perspectief kan gesteld worden dat Web 2.0 de preventieve
aanval van het kapitalisme op P2P{}-systemen is. Ondanks de talrijke
tekortkomingen van Web 2.0 in vergelijking tot deze laatsten, is het
attractiever voor investeerders, en vergaart het bijgevolg meer geld om
gecentraliseerde oplossingen te financieren en te promoten. Het
eindresultaat hiervan is dat kapitalistische investering vooral in
gecentraliseerde oplossingen gepompt wordt, zodoende hen goedkoop of
zelfs gratis te maken en gemakkelijk toepasbaar door niet technisch
aangelegde informatieproducenten. Zo cre\"eerde deze verhoogde
toegankelijkheid in vergelijking tot de meer techniek vereisende en
kostelijke onderneming van het bezitten van je eigen middelen ter
informatieproductie, een \quote{landloos} informatieproletariaat,
bereidwillig om ali\"enerende inhoudscre\"erende arbeid te verrichten
voor de nieuwe \quote{info{}-landheren} van Web 2.0.

Er wordt vaak gezegd dat het internet, door plots uit het publiek
gefinancierde universitair en militair onderzoek op te duiken, de
bedrijfswereld als het ware uit het niets overviel. Het werd gepromoot
door een {\em cottage industry} van kleine onafhankelijke
internetdienstenproviders, die er in slaagden om munt te slaan uit het
verschaffen van toegang tot het door de staat opgezette en
gefinancierde netwerk. 

Het internet leek een anathema voor de kapitalistische verbeelding. Web
1.0, de oorspronkelijke dotcomboom, werd gekarakteriseerd door een
stormloop om het bezit van de infrastructuur, zodoende de
onafhankelijke internetdienstenproviders te consolideren. Hoewel geld
nogal willekeurig werd rondgestrooid terwijl investeerders het hoofd
braken over waarvoor dit medium dan wel gebruikt zou worden, was de
globale missie grotendeels succesvol. Als je in 1996 een
internetaccount had, werd deze hoogstwaarschijnlijk aangeleverd door
een of ander klein lokaal bedrijf. Tien jaar later verkrijgen de meest
mensen, hoewel sommige van de kleinere bedrijven overeind bleven, hun
internettoegang van gigantische telecommunicatiebedrijven. De missie
van de Internet Investeringsboom 1.0 was precies om de onafhankelijke
dienstenprovider van de kaart te vegen, en grote, rijkelijk
gefinancierde bedrijven terug het stuur te laten overnemen.

De missie van Web 2.0 is om het P2P{}-aspect van het internet uit te
roeien; om jou, je computer, en je internetverbinding te laten afhangen
van een verbinding tot een gecentraliseerde dienst die je
communicatievermogen controleert. Web 2.0 betekent de neergang van open
peer{}-to{}-peer{}-systemen, en de terugkeer van monolithische \quote{online
diensten}. Een sprekend detail hier is, dat de meeste
internetconnecties in de jaren '90 thuis of in het bedrijf, modem{}- en
ISDN{}-connecties, synchroon waren {--} met een evenredig vermogen tot
het verzenden en ontvangen van data. Ze waren zo ontworpen dat je
verbinding je in staat stelde om zowel een producent als een consument
van informatie te zijn. Moderne DSL{}- en kabelmodems zijn daarentegen
asynchroon; ze laten je toe informatie snel te downloaden, maar slechts
traag te uploaden. 

En hierbij laten we dan nog buiten beschouwing dat vele
gebruiksovereenkomsten voor internetdiensten je verbieden servers te
laten draaien op je consumentencircuit, en zelfs de dienst verbreken
als je dit toch doet. 

Kapitalisme, geworteld in het idee dat inkomsten verworven dienen te
worden uit passief gedeeld eigenaarschap, vereist een gecentraliseerde
controle, zoniet zouden producenten geen enkele reden hebben tot het
delen van hun inkomsten met externe aandeelhouders. Kapitalisme valt
daardoor niet te rijmen met vrije P2P{}-netwerken, en bijgevolg zal het
netwerk, zolang als de financiering van internetontwikkeling afkomstig
is van private aandeelhouders die beslag trachten te leggen op waarde
door internetbronnen te bezitten, alleen maar meer ingeperkt en
gecentraliseerd worden. 

In het geval van commonsgebaseerde peer{}-productie moet er zelfs
opgemerkt worden, dat zolang als de commons en het lidmaatschap in de
peergroup gelimiteerd is, en dat input voor de producenten en de
computers die ze gebruiken, verworven wordt van buiten de
commonsgebaseerde peergroup, dat de peer{}-producenten zelf
medeplichtig kunnen zijn aan het exploiterend buit maken van deze
arbeidswaarde. Om bijgevolg de onrechtmatige inbeslagname van
geali\"eneerde arbeidswaarde werkelijk aan te pakken, moeten toegang
tot de commons en lidmaatschap tot de peergroup zover mogelijk
uitgebreid worden naar de inclusie van een totaal systeem van goederen
en diensten. Enkel en alleen wanneer alle productieve goederen
verkrijgbaar zijn bij commonsgebaseerde producenten, kunnen alle
producenten de waarde van het product van hun arbeid behouden.

En hoewel de informatiecommons mogelijk een rol kunnen spelen in een
maatschappelijke verschuiving naar meer inclusieve productiewijzen,
toch is elke re\"ele hoop op een oprechte, gemeenschapsverrijkende,
volgende generatie van internetgebaseerde diensten niet geworteld in de
creatie van private, gecentraliseerde bronnen, maar eerder in de
creatie van co\"operatieve, P2P{}- en commonsgebaseerde systemen {--}
bezit van niemand en iedereen. Ook al is ze naar hedendaagse
standaarden kleinschalig en obscuur, toch was het vroege internet, met
haar focus op peer{}-to{}-peer{}-toepassingen zoals Usenet en e{}-mail,
een uitermate gemeenzame en gedeelde gegevensbron. De commercialisering
van het internet en de opgang van kapitalistische financiering brengt
de inperking van deze informatiecommons met zich mee, waarbij publieke
rijkdom naar privaat profijt wordt vertaald. Web 2.0 moet dus niet
gezien worden als een tweede generatie van de technische of sociale
ontwikkeling van het internet, maar eerder als een tweede golf van
kapitalistische inperking van de informatiecommons. 

Virtueel alle van de meest gebruikte internetbronnen zouden vervangen
kunnen worden door P2P{}-alternatieven. Google zou vervangen kunnen
worden door een P2P{}-zoeksysteem, waarbij elke browser en elke
webserver actieve knooppunten in het zoekproces zouden zijn; Flickr en
YouTube zouden ook vervangen kunnen worden door applicaties als
PeerCast en eDonkey, die toelaten dat gebruikers hun eigen computers en
internetverbindingen gebruiken om collaboratief hun foto's en video's
te delen. Het ontwikkelen van internetbronnen vereist echter het
aanwenden van rijkdom, en zo lang als dat de bron van deze rijkdom
financieel kapitaal is, zal het grote peer{}-to{}-peer{}-potentieel van
het internet ongerealiseerd blijven.}

