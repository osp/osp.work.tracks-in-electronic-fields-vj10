\PlaceImage{ottavi0.png}{D\'erive audio g\'eographique et \'ecoute du spectre \'electro\-ma\-gnétique de Bruxelles}
\PlaceImage{ottavi1.png}{}
\PlaceImage{ottavi2.png}{}
\PlaceImage{ottavi3.png}{}

\AuthorStyle{Julien Ottavi}

\licenseStyle{GNUFDL}

\Fra{\Title{Electromagnetic spectrum Research code 06080}
\SubTitle{(Version 0.1, juin 2008)} {\bf Consid\'erations sur les d\'erives audio{}-g\'eographiques et autres \'ecoutes du spectre \'electromagn\'etique + Manuel de Construction d'antenne \'electromagn\'etique}

\QuoteStyle{L'important n'est plus le dit (contenu) ni le dire (un acte), mais la transformation, et l'invention de dispositifs, encore insou\-p\c{c}onn\'es, qui permettent de multiplier les transformations{\dots} \footnote{Michel De Certeau, {\em L'invention du quotidien}}}

\SubSubTitle{D\'erive urbaine \& spectre \'electromagn\'etique}

Partir d'un point et se laisser mener \`a travers les m\'eandres des rues, croiser par hasard une personne, percevoir une
r\'ealit\'e au del\`a de la r\'ealit\'e, se perdre dans un quartier, retrouver le point de d\'epart{\dots} La d\'erive est une d\'emarche qui
a \'et\'e fortement d\'evelopp\'e par le courant situationniste et notamment \`a partir du livre de Guy Debord {\em Rapport sur la
construction de situations} o\`u Debord propose de \quote{changer le monde} \`a travers le d\'epassement de toutes les formes artistiques par \quotation{un emploi unitaire de tous les moyens de bouleversement de la vie quotidienne.}

La d\'erive est une action qui consiste \`a se laisser porter dans un espace urbain, un quartier, une rue, un immeuble, un parc{\dots} sans
aucune pr\'econception quand \`a ce qui va se passer dans ce parcours. Il s'agit de vous laissez porter au gr\'e d'un courant imaginaire qui vous pousse au del\`a de situations que vous n'avez pas pr\'evue sans avoir calcul\'e un chemin \`a l'avance. 

Depuis le d\'eveloppement de la notion de \quote{d\'erive} men\'e par les situationnistes, de nombreuses autres formes de \quote{d\'erives} sont
apparus utilisant certains caract\`eres originalement conceptualis\'es
et r\'ealis\'es prenant comme point d'entr\'ee certaines contraintes (ligne droite, signes, cartographie{\dots}) ou en
r\'ev\'elant certaines choses qui peuvent se dessiner \`a travers la d\'erive (rencontre, d\'ecouverte de chemin cach\'e ou tout autre manifestation incongrue).

Il se dessine, en d'autre terme, une chose informelle de multiple conjoncture et de contextualisation en
d\'e{}-contextualisation o\`u le d\'eriveur prend part \`a un jeu qui le m\`enera, peut{}-\^etre, \`a son point de non{}-retour, de fait il
se sculpte une forme abstraite o\`u la r\'ealit\'e est renvers\'e en de nombreuses cons\'equences incalculables et infinis. 

La g\'eographie intervient l\`a dedans en forme de continuum de l'action, en proposition de lecture o\`u plut\^ot
d'\'ecriture de la mati\`ere, du mouvement, on cr\'ee le croquis de notre propre mouvement \`a travers les spasmes
impalpables de l'inconscient des gestes sur le goudron. Les cheminements possibles se d\'eroulent sous nos yeux et les
transcriptions mentales et psychologiques se mettent en place sous la forme d'\'ecriture automatique des espaces.

\QuoteStyle{Notre id\'ee centrale est celle de la construction de situations,
c'est{}-\`a{}-dire la construction concr\`ete d'ambiances momentan\'ees
de la vie, et leur transformation en une qualit\'e passionnelle
sup\'erieure. La formule pour renverser le monde, nous ne l'avons pas
cherch\'ee dans les livres mais en errant. C'\'etait une d\'erive \`a
grande journ\'ees, o\`u rien ne ressemblait \`a la veille ({\dots})
surprenantes rencontres, obstacles remarquables, grandioses trahisons,
enchantement p\'erilleux ({\dots}). La psychog\'eographie est l'\'etude
des lois exactes et des effets pr\'ecis du milieu g\'eographique
consciemment am\'enag\'e ou non, agissant directement sur le
comportement affectif des individus. La nouvelle architecture
d\'eterminera une plastique sonore qui s'identifiera au d\'ecor. On
assistera alors \`a la d\'ecouverte de climats bouleversants. \footnote{Guy
Debord}}

Les situationnistes d\'eveloppent aussi le concept d'\quote{ambiance}, \`a
travers lequel ils vont proposer de d\'epasser la logique de
planification urbaine et architecturale de la ville, chercher un autre
mode de lecture, une autre mani\`ere de vivre l'espace urbain.

\QuoteStyle{Entre les divers proc\'ed\'es situationnistes, la d\'erive se
pr\'esente comme une technique du passage h\^atif \`a travers des
ambiances vari\'ees. Le concept de d\'erive est indissolublement li\'e
\`a la reconnaissance d'effets de nature psychog\'eographique, et \`a
l'affirmation d'un comportement ludique{}-constructif, ce qui l'oppose
en tous points aux notions classiques de voyage et de promenade. ({\dots})

Une \'etude approfondie des moyens de cr\'eation d'ambiances et de l'influence psychologique de celles{}-ci, est une des taches que nous entreprenons actuellement. \footnote{Guy Debord}}

L'\quote{ambiance} ce sont donc ces choses qui nous influencent dans un
espace donn\'e : les murs, la lumi\`ere, le son, la mati\`ere, toutes
ces composantes physiques qui sous{}-tendent un lieu et qui nous
agissent, jouant de notre perception, p\'en\'etrant la psych\'e humaine
de mani\`ere subtile. Ce que propose les situationnistes finalement
c'est de d\'ejouer la construction des ambiances tel
qu'elles ont \'et\'e construites par
l'ordre \'etablis, re{}-construire
d'autres ambiances avec les habitants, les
r\^eveurs{\dots} de ces lieux, pas seulement en d\'emolissant et
reconstruisant les architectures ou le paysage urbain tel qu'il est
mais en prenant en compte le facteur de r\'eception et de
compr\'ehension de notre psych\'e face \`a ces \'etats. Se propulser au
del\`a de l'esclavagisme incontr\^olable de notre inconscient ou
plut\^ot lib\'erer notre capacit\'e imaginative, enlever les cha\^ines
de notre inconscient. 

La proposition que nous faisons est celle de prendre comme objet de la
d\'erive l'\'ecoute du ph\'enom\`ene
\'electromagn\'etique. 

L'homme vit depuis qu'il est apparu
sur Terre dans un environnement \'electromagn\'etique naturel issu du
champ magn\'etique terrestre. Depuis plus de quarante ans, de tr\`es
nombreux appareils de consommation courante ont vu le jour. Ils
g\^en\`erent des ondes \'electromagn\'etiques (ou ondes EM).

Une onde \'electromagn\'etique est la combinaison de deux
\quote{perturbations}: l'une est \'electrique,
l'autre est magn\'etique. Ces deux perturbations, qui
oscillent en m\^eme temps mais sur deux plans perpendiculaires, se
d\'eplacent \`a la vitesse de la lumi\`ere. Une onde EM peut donc se
concevoir comme une perturbation \'electrique de la mati\`ere qui se
propage.

Vous pouvez \`a tout moment cr\'eer un champ magn\'etique, utilisant les
principes de base de l'\'electricit\'e, vous pouvez
construire vous{}-m\^eme votre propre g\'en\'erateur \'electrique, vos
amplificateurs et vos haut{}-parleurs, toujours en utilisant le m\^eme
ph\'enom\`ene \'electromagn\'etique, bobine de cuivre, aimant et
surface de r\'eception, de charge ou de vibration.

Les ondes sont entretenues par un champ \'electromagn\'etique, qui
r\'esulte lui aussi de l'association
d'un champ magn\'etique et d'un champ
\'electrique susceptibles de varier dans le temps et de se propager
dans l'espace. 

Pour g\'en\'erer un champ \'electromagn\'etique, il suffit \`a la fois
de produire : \PlaceFramedImage{ottavi14.jpg}{}

\startitemize
\item un champ \'electrique par la pr\'esence de charges \'electriques,
\item et un champ magn\'etique en provoquant le d\'eplacement de ces
m\^emes charges \'electriques. 
\stopitemize

Les ondes EM ne sont alors que la propagation coupl\'ee de ces deux
champs ainsi cr\'e\'es. En d'autres termes l'onde \'electromagn\'etique est une variation p\'eriodique de champ
\'electrique et magn\'etique. Cette onde peut \^etre absorb\'ee par un
r\'ecepteur qui poss\`ede un moment dipolaire. Soumis \`a une
attraction sinuso\"idale un dip\^ole peut se mettre \`a tourner ou \`a
vibrer. Pour les \'energies plus fortes la liaison peut \^etre rompue.
Dans le cadre de l'utilisation d'une
antenne (voir troisi\`eme partie pour la construction) nous allons
op\'erer une boucle qui va recr\'eer ce moment dipolaire. Avec les
essais que vous effectuerez avec l'antenne vous allez
pouvoir constater qu'il y a deux p\^oles dans la
r\'eception. Vous pourrez capter le ph\'enom\`ene sous deux angles
diff\'erents et permettrent parfois une meilleur r\'eception sur
l'un des p\^oles.

Au del\`a du ph\'enom\`ene physique, il y a une r\'ealit\'e que nous
allons utiliser \`a travers les d\'erives, celle de la mise en
vibration des ondes EM et par cons\'equence du spectre
\'electromagn\'etique que nous trouverons sur notre parcours. Une
transposition de l'onde vers une vibration sonore,
utilisant simplement l'amplification et
l'\'electricit\'e, va permettre de rendre audible le
ph\'enom\`ene.

La d\'erive se transforme alors en une \'ecoute du spectre EM, la ville
\'etant un puits sans fin de fr\'equences et de bruits g\'en\'er\'es
par des centaines de machines et d'appareils
\'electriques, toute technologie utilisant
l'\'electricit\'e se trouve de fait dans le champ
\'electromagn\'etique (voir image du spectre EM). Il y a d\'eplacement
par rapport \`a la position des situationnistes, ce
n'est plus seulement les diff\'erentes ambiances qui
vont mener la marche ou leur d\'econstruction mais
l'action de r\'ev\'eler la vie invisible qui travaille
la ville, celles des machines qui travaillent nos corps, qui sont les
m\'ecaniques sous{}-jacentes en activit\'e permanente des espaces
urbains. Les d\'erives audio{}-g\'eographiques et \'ecoutes du spectre
EM peuvent s'aborder comme si vous visitiez une
dimension parall\`ele de notre quotidien, ou d\'ecouvriez une
g\'eographie cach\'e d'un monde invisible, celui des
forces machiniques \`a l'{\oe}uvre, qui depuis plusieurs
ann\'ees se sont d\'evelopp\'es \`a une tr\`es grande vitesse dans
notre environnement, construisant de nouvelles ambiances, de nouvelles
mani\`eres de vivre la ville.

L\`a o\`u les situationnistes soul\`event les rapports sociaux comme
construction de l'espace urbain, les \'ecoutes EM
interrogent le rapport homme{}-machine dans
l'apparition de mati\`eres sonores insoup\c{c}onn\'ees
r\'ev\'elant un nouveau type de compr\'ehension dans le paradigme
machine {--} urbanisme {--} architecture {--} corporalit\'e. Il y a un
nouveau jeu qui s'\'etablit sur la recherche du
ph\'enom\`ene et simultan\'ement, par son \'ecoute, on y actionne des
brisures dans le flux (la machine{}-urbaine). On y entend donc une
nouvelle entit\'e comme forme \quote{vivante}, quelque chose qui nous parle,
un \^etre dou\'e d'ubiquit\'e dont nous faisons
partie, qui nous constituent et nous agit insidieusement, par en
dessous, dans l'ombre de nos propres murs, enferm\'e
dans nos propres certitudes: avons{}-nous r\'eellement le contr\^ole de
la situation?

Arm\'e de notre antenne et d'un syst\`eme
d'amplification portable, nous (un groupe
\quote{spectrorateur}) allons \`a la recherche d'une chose
inaudible devenu audible gr\^ace \`a l'inversion de
son fonctionnement. Le d\'etournement de la fonction m\^eme de
l'antenne comme \'el\'ement de transmission et de
r\'eception, o\`u la ville devient l'\'emetteur. Le
fonctionnement des choses est renvers\'e dans le fait m\^eme de rendre
audible mais aussi dans la cr\'eation de situation que la diffusion des
sons va faire \'emerger \`a travers l'\'ecoute mobile
dans un espace public donn\'e.

\QuoteStyle{{\bf Situation construite}: Moment de la vie, concr\`etement et d\'elib\'er\'ement construit par
l'organisation collective d'une ambiance unitaire et d'un jeu d'\'ev\'enements.}

Les situations ainsi construites se trouvent alors entre deux, il y a
d'une part reconnaissance par l'autre
d'un son inhabituel qui vient d\'eranger son
environnement d'\'ecoute, cela prend un autre
caract\`ere lorsque l'on d\'ecrit le ph\'enom\`ene que
l'on capte et que l'on donne \`a
entendre. Il y a renversement de situation dans le partage
d'une chose qui existe dans notre quotidien mais qui
n'est pas perceptible, qui le devient avec la mise en
son \quote{brut} et \'eruptif d'une singularit\'e. Les sons
capt\'es sont extr\^emement vari\'es et constituent une large palette
de timbres allant de sons aigus au grave plus ou moins pr\'ecis, \`a
diff\'erente sorte de bruits (blanc, marron, rose{\dots} etc.), de nombreux
{\em \quote{patterns}} ou rythmiques r\'ep\'etitifs et asynchrones, ainsi
que des clusters ou grappes de fr\'equences.
L'ensemble se m\'elange dans une organisation
al\'eatoirement distribu\'ee qui d\'epend de machines ou appareils
\'emettant leur champs EM \`a proximit\'e les uns des autres, leur
organisation se font selon des crit\`eres d'utilit\'e
non point en fonction de leur spectre sonore et leur \'eventuel
musicalit\'e.

Il y a par la mise en \'ecoute dans ces espaces (sonores) urbains, une
sorte de mise \`a nu de la ville, une pens\'ee de la dissension, il y a
dialogue et coupure dans le flux. Une sorte de po\'esie de
l'imperceptiblement rugueux, car il y a rugosit\'e
dans le rendu sonore, on y trouve une force sale \`a
l'{\oe}uvre, on y trouve de la po\'esie dans la brisure
avec le r\'eel du quotidien qui s'instaure dans un
espace, un lieu, une rue{\dots} dans le quotidien, habitude, habitus et
m\'ecanique d'utilisation de ces espaces.

Les d\'erives audio{}-g\'eographiques et \'ecoutes EM placent la
perception sensorielle sur un niveau po\'etique dans le sens o\`u
l'auditeur peut potentiellement \^etre actif dans
l'acte de transposition et de mise en \'ecoute, de
d\'eplacements et de renversement de la situation.
L'auditeur re{}-cr\'ee par lui{}-m\^eme la connexion
improbable entre la forme d'o\`u peut provenir les
sons et la mat\'erialit\'e de ces sons. Il n'y a pas
dans le cas de ces recherches et d\'erives d'auteur
\`a proprement parl\'e, il y a un passeur, un initiateur, une personne
r\'ev\'elant une potentialit\'e dormante, une substance en de\c{c}\`a
de notre compr\'ehension imm\'ediate du monde.

\PlaceImage{ottavi7.png}{}
\PlaceImage{ottavi8.png}{}

Prenons l'artiste allemande Christina Kubisch qui
expose ce ph\'enom\`ene depuis plusieurs ann\'ees \`a la fois comme
syst\`eme de transmission de ces pi\`eces sonores et \`a la fois comme
syst\`eme d'\'ecoute du spectre EM dans des espaces
urbains. Dans notre approche, la transmission de la technique de
captation, ainsi que la compr\'ehension du ph\'enom\`ene est
fondamentale, de m\^eme que sur le plan de
l'exploration, il y a diff\'erence avec celle de
Christina qui se fait \`a travers une paire
d'\'ecouteur munis de bobine de cuivre amplifiant le
signal pour l'auditeur, alors que nous diffusons vers
un ext\'erieur, au del\`a de soi.

Son travail est int\'eressant dans le fait
qu'elle participe \`a r\'ev\'eler un domaine peu
connu et \`a y investir une compr\'ehension autre que celle de
l'approche purement scientifique, celle
d'une approche po\'etique et sensorielle, utilisant
aussi l'errance et la d\'erive comme mouvement non
cat\'egorisable de l'\'ecoute des sons environnants.

Le parcours que propose Kubish se fait dans une \'ecoute introspective,
repli\'e vers soi. Ce qui est propos\'e dans les d\'erives
audio{}-g\'eographiques et \'ecoutes du spectre EM,
c'est \`a l'inverse une audition
orient\'ee vers l'ext\'erieur, il y a irruption dans
une r\'ealit\'e commune m\^eme si cela doit cr\'eer une rupture dans le
fonctionnement d'un espace et ainsi que dans les
relations qui peuvent s'y jouer. Le but \'etant de
cr\'eer une situation de d\'etournement par
l'\'ecoute. Pour donner un exemple, dans ces d\'erives
nous avons rencontr\'es plusieurs cas de figure o\`u
l'on peut trouver a la fois une curiosit\'e et un
rejet de l'\'etranger qui se r\'ev\`ele, dans la forme
o\`u s'objective les ondes EM, et surtout dans le fait
de ne pas avoir un cachet scientifique oblit\'er\'e par un cadre
l\'egal. Car dans notre cas, le laboratoire devient
l'espace urbain, il n'y a pas ici de
mod\'elisation et test en laboratoire mais exp\'erimentations sur le
terrain. Dans l'occasion d'une
d\'erive dans la ville de Marseille, nous \'ecoutions un distributeur
de billets (en ext\'erieur) et jouions avec le fait
d'enclencher notre carte bancaire dans le distributeur
et de retirer de l'argent ou consulter nos comptes.
Des employ\'es de l'agence bancaire qui
s'en allaient manger, nous demandent alors ce que nous
\'etions en train de faire avec un ton de curiosit\'e. Dans un premier
temps, nous leur expliquons le but de tout cela et continuons
l'exp\'erience, une deuxi\`eme demande
d'explication nous appara\^it alors avec un cot\'e
incr\'edule et insistant{\dots} Puis prenant conscience de ce que nous
sommes en train de faire ainsi que de notre caract\`ere
d'explorateurs op\'erant dans un cadre purement
artistique, non officiel, jouant l'acte de
d\'etournement, ils nous demandent alors d'arr\^eter
pr\'etextant que nous n'avons pas le droit de faire
\c{c}a, jouant sur les limites du l\'egal avec un ton
d'autorit\'e r\'ev\'elant une peur, celle
d'une anomalie inconnu, comme cet \'etranger qui vient
d\'eranger votre zone de tranquillit\'e, le bon ordre et le
fonctionnement de votre machine d\'esirante, cracheuse de papier.

\SubSubTitle{Lecture et \'ecriture de la ville (R\'eflexion sur la ville
comme manuscrit \`a d\'ecrypter?)}

Dans la continuit\'e de pens\'ee des travaux de ceux de Gordon
Matta{}-Clark qui \'ecrit l'espace architecturale en
op\'erant des d\'ecoupes dans les b\^atiments, ou bien sur les
r\'eflexions de Jacques Derrida dans {\em Diss\'emination} sur
l'\'ecriture de St\'ephane Mallarm\'e, la page comme
architecture \`a habiter, \`a d\'econstruire, \`a re{}-construire,
comme un livre \`a de{}-venir :

\QuoteStyle{Un ouvrage singulier qui fut et ne fut pas un livre,
{\em Un coup de d\'es}{\dots} de Mallarm\'e, autour
duquel Blanchot \'ecrivit tel essai intitul\'e \quote{Le livre \`a venir}
\`a l'int\'erieur duquel se lit l'expression \quote{le livre \`a venir} qui
se trouve \^etre aussi le titre du {\em recueil} {--} mot
qui fait signe, encore, vers la reliure, le rassemblement, la
collection, mais d'abord vers l'{\em accueil} (Mallarm\'e
d\'esigne le lecteur comme un \quote{h\^ote}).

Ce qui me touche ici, et ce qui m\`ene mon propos, c'est que, par un
{\em jeu de mot}, un habile stratag\`eme de votre part,
nous pouvons confondre (mais c'est le fond m\^eme de la diff\'erance)
tous ces textes, intitul\'es {\em livre \`a venir}. Peu
importe, en fait, puisque ce qui nous importe ici, ce n'est pas tant le
texte, mais ce qu'il d\'esigne (en vain, aurait dit Blanchot), c'est
pr\'ecis\'ement cette chose que l'on ne conna\^it pas, qui n'a pas de
forme pour l'instant, qui n'existe pas encore, mais qu'on d\'esigne par
{\em livre \`a venir}. \footnote{{\em Posologie. De Jacques Derrida.} Texte par Beno\^it Vincent \Url{http://www.derrida.ws/index.php?option=com_content&task=view&id=9&Itemid=8}}}

Ces recherches s'inscrivent dans le rapport \`a une
autre forme d'\'ecriture, celle du livre invisible des
ondes EM sur l'espace urbain et les architectures
machiniques.

L'\'ecriture comme la fixation de signes signifiant et
la po\'esie comme \'ecriture de signes rejouant le signifiant en de
multiples autres compr\'ehensions. L'espace urbain et
les architectures comme les pages d'un livre que
l'on lit o\`u que l'on \'ecrit ou les
deux \`a la fois, le mouvement, la d\'erive comme outil
d'\'ecriture, les ondes EM comme forme
d'\'ecriture{\dots} Le spectre EM (spectre comme
fant\^ome et spectre comme gamme) est \`a la fois une \'ecriture, une
succession de signes faisant sens en terme audible et en terme de
fonctionnement machinique, une \'ecriture machinique invisible que
l'on r\'ev\`ele et d\'ecrypte au moyen
d'une sorte de loupe ou d'appareil de
traduction d'un langage encore obscur. La ville
devient en ce sens un livre ouvert \`a travers lequel nous allons
composer notre propre lecture, au fil des rues et des zones \`a fortes
concentrations d'ondes EM. Ce communique alors des
symboles repr\'esentant un langage compos\'e de plusieurs sortes de
signes op\'erant au niveau des timbres, des rythmiques et des
compositions de m\'elodies h\'et\'erog\`enes et composites. Ce langage
pourra \^etre lu de plusieurs fa\c{c}ons, \`a travers une
compr\'ehension de son origine machinique et de sa m\'ecanique
(carcasse) physique lui donnant son champ sonore sp\'ecifique mais
aussi sur son caract\`ere musical aussi bien que sur son inscription
dans le contexte m\^eme: un supermarch\'e, une voiture, une rue de
bijoutier, un carrefour, un panneau publicitaire{\dots} etc.

\PlaceImage{ottavi6.png}{}
\PlaceImage{ottavi4.png}{}

De cet angle nous basculons \`a une forme d'\'ecriture
qui va s'engendrer, dans le cas de nos d\'erives, sous
plusieurs aspects: le premier \'etant sur l'\'ecriture
abrupte de cette manifestation par la mise en \'ecoute, un autre par le
biais d'une forme de d\'eterritorialisation.
L'\'ecoute et la cr\'eation du contexte \'etant le
mode d'\'ecriture de l'exp\'erience.
Comment le contexte change t{}-il et agit{}-il sur un espace? Comment
le fonctionnement de cette espace change au fur et \`a mesure que
l'on y \'ecrit? Le sens des signes pr\'ec\'edemment lu
dans cet espace devient pour un instant quelque chose
d'autre, prend une autre signification, il se produit
autrement. Il s'agit ici d'une
\'ecriture temporaire, une \'ecriture disparaissante{\dots} le son \'etant
la transcription, l'espace urbain ou architecturale la
page, l'antenne EM et l'amplification
comme stylo{}-loupe venant \`a la fois lire et \'ecrire le contexte.
Les lecteurs eux s'en trouvent interpell\'es,
qu'ils soient convi\'es, participants ou
al\'eatoirement attir\'es par l'activit\'e/action se
d\'eroulant devant leurs yeux et leurs oreilles. Nous \'ecrivons le
texte multiple de notre mouvement dans un lieu, la d\'erive ouvre la
lecture al\'eatoire du d\'eplacement, la ville devient
l'espace finalement tangible de notre inconscient
perceptif, passant d'un statut informelle \`a celui
d'une r\'ealit\'e physique r\'eappropriable.

Dans un deuxi\`eme temps, ou dans le temps de
l'\'enonciation du langage, nous \'ecrivons,
litt\'eralement cette fois, sur les emplacements de captation des
champs EM. \`A base de signes annon\c{c}ant le contenu du type de son
capt\'e: fr\'equence, bruit, rythmique, hauteur approximative de la
fr\'equence ainsi que sa repr\'esentation graphique, par la d\'erive
nous laissons ainsi des traces sur les murs, le goudron, le b\'eton,
les panneaux de circulation ou publicitaires, les magasins et autres
m\'etaux forgeant les espaces d'habitations{\dots} etc. La
trace permet de donner un signe de notre d\'erive dans la ville, de
transmettre le fait qu'un objet invisible \`a \'et\'e
d\'ecouvert en ce lieu, que nous y avons r\'ev\'el\'es
l'invisible, l'inaudible, le
revenant{\dots} Donner des signes d'une existence
spectrale, d'une pr\'esence incertaine, possible,
influant \`a l'inconscient la possibilit\'e
d'une entit\'e qui se manifeste au del\`a de nos
perceptions communes.

\page

\framed[offset=1em]{\SubTitle{Manuel de construction d'antenne \'electromagn\'etique (III)}

\startitemize[n,broad]

\item Se munir d'une bobine de fil de cuivre d'environ 0.25mm \InsideImage{ottavi11.jpg}{Une bobine de fil de cuivre}

\item Pr\'eparer un objet sur lequel vous allez faire une centaine de
tour avec le fil de cuivre, d'un diam\`etre
d'environ 40cm. Cette objet doit \^etre suffisamment
pratique pour que vous puissiez retirer la boucle une fois les tours
finis.

\item Faire une centaine de tours en concentrant bien les fils \`a
chaque tour, les fils de la boucle doivent \^etre resserr\'es pour
permettre une meilleure r\'eception. \InsideImage{ottavi10.jpg}{Concentrant bien les fils \`a
chaque tour}

\item Une fois la boucle finis, vous devez avoir deux bouts de c\^able
qui sont le d\'ebut de la boucle et la fin de la boucle.

\item Penser \`a une technique pour solidariser les fils de la boucle
une fois enlev\'e du support pour l'utiliser comme
antenne: gaffer, tuyau, scratch{\dots} etc. \InsideImage{ottavi5.png}{Une technique pour solidariser les fils}

\item Enlever le vernis du cuivre \`a l'aide
d'un briquet et d'un cutter, pour
pouvoir connecter le cuivre \`a un connecteur audio du type jack (3,5
ou 6,5mm).

\item Connecter un des bouts de fil \`a la masse et
l'autre au point chaud du connecteur jack.

\item Plugger le connecteur jack \`a un ampli audio pour faire les
tests, vous devez entendre alors diff\'erents sons en fonction de la
proximit\'e d'appareils du type t\'el\'ephone
portable, ordinateur{\dots} Les points \'electriques \'emettront un 50Hz
bien ronronnant, le \quote{hum} que l'on chasse dans les
salles de concerts et d'enregistrements.

\item Vous munir d'un ampli audio portable et commencer
votre d\'erive dans votre quartier \`a la recherche des sons
inaudibles! \InsideImage{ottavi9.png}{\`A la recherche des sons
inaudibles}

\stopitemize

}

}
