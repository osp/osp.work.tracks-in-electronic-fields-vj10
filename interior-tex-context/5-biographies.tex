\Title{Biographies}
\blank
\def\Bio#1{\noindenting{\startcolumns[n=2,tolerance=verytolerant]{\tfx\setupinterlinespace #1}\stopcolumns}}

\def\English#1{\par{\framed[leftframe=off,rightframe=off,topframe=off]{\ttx\setupinterlinespace EN\par}}{#1}\par}
\def\Francais#1{\par{\framed[leftframe=off,rightframe=off,topframe=off]{\ttx\setupinterlinespace FR\par}}{#1}\par}
\def\Nederlands#1{\par{\framed[leftframe=off,rightframe=off,topframe=off]{\ttx\setupinterlinespace NL\par}}{#1}\par}

\Bio{

\SubSubTitle{Val\'erie Laure Benabou}

\SmallUrl{http://www.juriscom.net/minicv/vlb}

\English{Val\'erie Laure Benabou is associate Professor at the University of
Versailles{}-Saint Quentin and teaches at the Ecole des Mines. She is a
member of the Centre d'Etude et de Recherche en Droit de l'Immat\'eriel
(CERDI), and of the Editorial Board of Propri\'et\'es Intellectuelles.
She also teaches civil law at the University of Barcelona and taught
international commercial law at the Law University in Phnom Penh,
Cambodia. She was a member of the Commission de r\'eflexion du Conseil
d'Etat sur Internet et les r\'eseaux num\'eriques, co{}-ordinated by Ms
Falque{}-Pierrotin, which produced the {\em Rapport du Conseil
d'Etat}, (La Documentation fran\c{c}aise, 1998). She is the author of a
number of works and articles, including \quote{La directive droit d'auteur,
droits voisins et soci\'et\'e de l'information: valse \`a trois temps
avec l'acquis communautaire}, in {\em Europe}, No. 8{}-9, September
2001, p. 3, and in {\em Communication Commerce Electronique},
October 2001, p. 8., and \quote{Vie priv\'ee sur Internet: le tra\c{c}age},
in {\em Les libert\'es individuelles \`a l'\'epreuve des NTIC}, PUL,
2001, p. 89.}

\Nederlands{Professor in priv\'erecht aan de Universit\'e de Versailles Saint
Quentin in Yvelines. Gespecialiseerd in het recht van nieuwe
informatie{}- en communicatietechnologie\"en. Vicevoorzitter van AFPIDA
(Association Fran\c{c}aise pour la Protection Internationale du Droit
d'Auteur). Directrice van het Laboratoire DANTE (Droit des Affaires et
Nouvelles Technologies) van de Universit\'e de Versailles Saint
Quentin. Medeonderzoekster aan het CRDP (Centre de Recherche en Droit
Public) van de Universit\'e de Montr\'eal. Neemt sinds 2001 deel aan
een Internationaal Programma voor Wetenschappelijke Samenwerking rond
het thema \quote{Juridische veiligheid, technische veiligheid}.}

\Francais{Professeur agr\'eg\'ee de droit priv\'e \`a l'Universit\'e de Versailles
Saint Quentin en Yvelines. Sp\'ecialis\'ee en droit des nouvelles
technologies de l'information et de la communication. Elle est
vice{}-pr\'esidente de l'AFPIDA (Association Fran\c{c}aise pour la
Protection Internationale du Droit d'Auteur) et directrice du
Laboratoire DANTE (Droit des Affaires et Nouvelles Technologies) de
l'Universit\'e de Versailles Saint Quentin. Elle est \'egalement
chercheur associ\'e au CRDP (Centre de Recherche en Droit Public) de
l'Universit\'e de Montr\'eal et y participe depuis 2001 \`a un
Programme International de Coop\'eration Scientifique sur le th\`eme
\quote{S\'ecurit\'e Juridique, s\'ecurit\'e technique}.}

\SubSubTitle{Pierre Berthet}

\SmallUrl{http://pierre.berthet.be/}

\English{Studied percussion with Andr\'e Van Belle and Georges-Elie Octors,
improvisation with Garrett List, composition with Frederic Rzewski, and
music theory with Henri Pousseur. Designs and builds sound objects and
installations (composed of steel, plastic, water, magnetic fields
etc.). Presents them in exhibitions and solo or duo performances with
Brigida Romano (CD {\em Continuum asorbus} on the Sub Rosa label) or
Fr\'ed\'eric Le Junter (CD {\em Berthet Le Junter} on the
Vand{\oe}uvres label). Collaborated with 13th tribe (CD {\em Ping
pong anthropology}). Played percussion in Arnold Dreyblatt's Orchestra
of excited strings (CD {\em Animal magnetism}, label Tzadik; CD
{\em The sound of one string}, label Table of the elements).}

\Nederlands{Geluidskunstenaar. Studeerde percussie met Andr\'e Van Belle en
Georges{}-Elie Octors, improvisatie met Garrett List, compositie met
Frederic Rzewski, en muziektheorie met Henri Pousseur. Hij ontwerpt en
bouwt sonore voorwerpen en installaties (in staal, plastiek, water,
magnetische velden etc.). Deze toont hij tijdens tentoonstellingen en
performances, solo of samen met Brigida Romano (cd {\em Continuum
asorbus} bij het label Sub Rosa) en Fr\'ed\'eric Le Junter (cd
{\em Berthet Le Junter} bij het label Vand{\oe}uvres). Berthet
werkte samen met 13th tribe (cd {\em Ping pong anthropology}). Hij
verzorgde de percussie voor Arnold Dreyblatts Orchestra of excited
strings (cd {\em Animal magnetism}, label Tzadik; cd {\em The
sound of one string}, bij het label Table of the elements).}

\Francais{Plasticien sonore. A \'etudi\'e la percussion avec Andr\'e Van Belle et
Georges{}-Elie Octors, l'improvisation avec Garrett List, la
composition avec Frederic Rzewski, et la th\'eorie de la musique avec
Henri Pousseur. Il con\c{c}oit et construit des objets et installations
sonores (en acier, plastique, eau, champs magn\'etiques etc.), et les a
pr\'esent\'es lors d'expositions et de performances en solo ou en duo
avec Brigida Romano (CD {\em Continuum asorbus} sur le label Sub
Rosa) or Fr\'ed\'eric Le Junter (CD {\em Berthet Le Junter} sur le
label Vand{\oe}uvres). A collabor\'e avec 13th tribe (CD {\em Ping
pong anthropology}). A jou\'e de la percussion chez Orchestra of
excited strings d'Arnold Dreyblatt (CD {\em Animal magnetism}, label
Tzadik; CD {\em The sound of one string}, sur le label Table of the
elements).}

\SubSubTitle{Alice Chauchat}

\SmallUrl{http://www.theselection.net/dance/}

\English{Member of the Praticable collective. Alice Chauchat was born in 1977 in
Saint{}-Etienne (France) and lives in Paris. She studied at the
Conservatoire National Sup\'erieur de Lyon and P.A.R.T.S in Brussels.
She is a founding member of the collective B.D.C. With other members
such as Tom Plischke, Martin Nachbar and Hendrik Laevens she created
{\em Events for Television}, {\em Affects} and{\em  (Re)sort},
between 1999 and 2001. In 2001 she presented her first solo
{\em Quotation marks me}. In 2003 she collaborated with Vera Knolle
({\em A Number of Classics in the Age of Performance}). In 2004 she
made {\em J'aime,} together with Anne Juren, and {\em CRYSTALLL},
a collaboration with Alix Eynaudi. She also takes part in other
people's projects, such as {\em Projet}, initiated by Xavier Le Roy,
or {\em Avant{}-garde} by M{\aa}rten Sp{\aa}ngberg.}

\Nederlands{Lid van het collectief Praticable. Alice Chauchat, geboren in 1977 in
Saint{}-Etienne (Frankrijk), woont in Parijs. Ze studeerde aan het
Conservatoire National Sup\'erieur de Lyon en aan P.A.R.T.S, Brussel.
Ze is stichtend lid van het collectief B.D.C.: samen met Tom Plischke,
Martin Nachbar en Hendrik Laevens maakte ze tussen 1999 en 2001
{\em Events for Television}, {\em Affects} en {\em (Re)sort}.
In 2001 brengt ze haar eerste solo {\em Quotation marks me}. Ze
werkte samen met Vera Knolle (2003, {\em A Number of Classics in the
Age of Performance}), Anne Juren (2004, {\em J'aime}), Fr\'ed\'eric
Gies ({\em The breast piece} (Praticable)) en Alix Eynaudi
({\em CRYSTALLL}). Ze neemt ook deel aan projecten van anderen,
zoals {\em Projet} van Xavier Le Roy en {\em Avant{}-garde} van
M{\aa}rten Sp{\aa}ngberg.}

\Francais{Alice Chauchat explore les bases et conditions pour la production, la
pr\'esentation et la r\'eception des spectacles de danse. En mettant en
sc\`ene des corps f\'eminins, elle questionne la relation entre
spectateurs et performeurs. Ses travaux incluent les projets suivants:
{\em Quotation marks me} (2001), solo; le duo
{\em Chor\'egraphies}, dans\'e avec Carlos Pez (2002);
{\em Quelques Classiques \`a l \`Ere de la Performance} en
collaboration avec Vera Knolle (2003); {\em J'aime} en collaboration
avec Anne Juren (2004), {\em Crystalll} (2005) en collaboration avec
Alix Eynaudi, et {\em The breast piece} (Praticable) en
collaboration avec Fr\'ed\'eric Gies. Apr\`es ses \'etudes au
Conservatoire National Sup\'erieur de Lyon et \`a P.A.R.T.S, Alice a
travaill\'e avec Thomas Plischke ({\em Demgegen\"uber Borniertheit})
en 1999. Dans la m\^eme ann\'ee, le collectif B.D.C a \'et\'e fond\'e
avec e. a. Tom Plischke et Martin Nachbar. Ils cr\'eeront
{\em Events for television (again)} (1999), {\em Affects} (2000)
et {\em (Re)sort} (2001). Ses participations \`a d'autres projets
incluent l'interpr\'etation pour d'autres chor\'egraphes (e.a.
M{\aa}rten Sp{\aa}ngberg, Petra Sabisch), la participation \`a
{\em E.X.T.E.N.S.I.O.N.S.} et {\em Projet} initi\'es par Xavier
Le Roy (2003{}-2005), et un travail d'assistante artistique aupr\`es de
divers chor\'egraphes (e.a. Fr\'ed\'eric Gies, Eva Meyer{}-Keller,
Lit\'o Walkey).
}
\SubSubTitle{Michel Cleempoel}

\SmallUrl{http://www.michelcleempoel.be/}

\English{Graduated from the National Superior Art School La Cambre in Brussels.
Author of numerous digital art works and exhibitions. Worked in
collaboration with Nicolas Malev\'e: 

\Url{http://www.deshabillez-vous.be}}

\Nederlands{Multimediakunstenaar, auteur van een reeks werken en tentoonstellingen
die tijd en licht als uitgangspunt hebben. Realiseerde samen met
Nicolas Malev\'e: 

\Url{http://www.deshabillez-vous.be}}

\Francais{Artiste num\'erique avec comme outils le temps et la lumi\`ere,
Cleempoel \'elabore une {\oe}uvre qui questionne le monde et son
apparence. A r\'ealis\'e avec Nicolas Malev\'e :

\Url{http://www.deshabillez-vous.be}}

\SubSubTitle{De Geuzen (a foundation for multi{}-visual research)}

\SmallUrl{http://www.geuzen.org/}

\English{Femke Snelting, Ren\'ee Turner and Riek Sijbring form the art and design
collective De Geuzen (a foundation for multi{}-visual research). De
Geuzen develop various strategies on and off line, to explore their
interests in the female identity, critical resistance, representation
and narrative archives.}

\Nederlands{Femke Snelting, Ren\'ee Turner en Riek Sijbring vormen samen het
kunst{}- en designcollectief De Geuzen (een stichting voor
multi{}-visueel onderzoek). De Geuzen ontwikkelt on{}- en offline
uiteenlopende strategie\"en, aan de hand waarvan ze hun interesses voor
vrouwelijke identiteit, kritisch verzet, representatie en narratieve
archieven, uitdiepen.}

\Francais{Femke Snelting, Ren\'ee Turner et Riek Sijbring, forment le collectif
d'art et de design De Geuzen (une fondation pour la recherche
multi{}-visuelle). De Geuzen d\'eploie une vari\'et\'e de strat\'egies
\`a la fois on et off line pour explorer leurs int\'er\^ets pour
l'identit\'e f\'eminine, la r\'esistance critique, la repr\'esentation
et l'archive narrative. Le groupe a donn\'e des ateliers \'educatifs au
festival Impakt, au Piet Zwart Instituut et \`a La Cambre. Leurs
projets ont \'et\'e montr\'es \`a Manifesta 3 (Slov\'enie), le
K\"unstlerhaus Bremen (Allemagne) et De Appel (Pays{}-Bas). M\^eme si
certains de leurs projets ont fait l'objet de commissions, beaucoup
sont initi\'es de mani\`ere autonome.}

\SubSubTitle{S\'everine Dusollier}

\SmallUrl{http://www.fundp.ac.be/universite/personnes/page_view/01003580/}

\English{Doctor in Law, Professor at the University of Namur (Belgium), Head of
the Department of Intellectual Property Rights at the Research Center
for Computer and Law of the University of Namur, and Project Leader
Creative Commons Belgium, Namur.
}
\Nederlands{Doctor in de Rechten aan de Universit\'e de Namur. Afdelingshoofd van
het Departement van Intellectuele Eigendommen van het Onderzoekscentrum
voor Computer en Recht van de Universit\'e de Namur. Projectleider van
Creative Commons Belgium, Namen.
}
\Francais{Charg\'ee de cours \`a la Fondation Universitaire Notre{}-Dame de la
Paix de Namur. Responsable du D\'epartement Droits Intellectuels au
Centre de Recherche Informatique et Droit de l'Universit\'e de Namur.
Project Leader Creative Commons Belgium, Namur.}

\SubSubTitle{Leif Elggren}

\SmallUrl{http://www.leifelggren.org/}

\English{Leif Elggren (born 1950, Link\"oping, Sweden) is a Swedish artist who
lives and works in Stockholm.

Active since the late 1970s, Leif Elggren has become one of the most
constantly surprising conceptual artists to work in the combined worlds
of audio and visual. A writer, visual artist, stage performer and
composer, he has many albums to his credits, solo and with the Sons of
God, on labels such as Ash International, Touch, Radium and his own
Firework Edition. His music, often conceived as the soundtrack to a
visual installation or experimental stage performance, usually presents
carefully selected sound sources over a long stretch of time and can
range from mesmerising quiet electronics to harsh noise. His
wide{}-ranging and prolific body of art often involves dreams and
subtle absurdities, social hierarchies turned upside{}-down, hidden
actions and events taking on the quality of icons.

Together with artist Carl Michael von Hausswolff, he is a founder of the
Kingdoms of Elgaland{}-Vargaland (KREV), where he enjoys the title of
King.}

\Nederlands{Leif Elggren ({\textdegree}1950) is een Zweeds conceptueel kunstenaar
die in Stockholm woont en werkt. In zijn artistieke praktijk opereert
hij in de verknoopte werelden van de audio en het visuele. Zijn aanpak
is multidisciplinair, gaande van etsen tot performance, het maken van
installaties, video, muziek, en schrijven. Op zijn curriculum als
componist staan zowel soloalbums, als albums gemaakt met the Sons of
God, op labels zoals Ash International, Touch, Radium en zijn eigen
Firework Edition. Zijn muziek, die vaak geconcipieerd wordt als
soundtrack bij een visuele installatie of experimentele performance, is
meestal gebaseerd op zorgvuldig geselecteerde geluidsbronnen, en kan
fluctueren van zachte elektronica tot wrange noise. Zijn eclectische
{\oe}uvre incorporeert vaak dromen en absurditeiten, ondersteboven
gekeerde sociale hi\"erarchie\"en, en verborgen acties en
gebeurtenissen die een iconisch karakter krijgen.

Samen met kunstenaar Carl Michael von Hausswolff stichtte hij de
Koninkrijken van Elgaland{}-Vargaland (KREV), waarvan hijzelf de titel
van Koning draagt.}

\Francais{Leif Elggren est n\'e en 1950 \`a Link\"oping en Su\`ede.
C'est un artiste su\'edois qui vit et travaille \`a
Stockholm. Dans sa pratique artistique, il op\`ere dans les mondes
combin\'es de l'audio et du visuel. Sa production est
multidisciplinaire, elle s'\'etend de la gravure \`a
la performance, de la r\'ealisation d'installations,
de vid\'eos \`a la musique et \`a l'\'ecriture. On
trouve dans son curriculum, comme compositeur, des albums solo ou des
collaborations, comme par exemple avec les Sons of God chez des labels
comme Ash International, Touch, Radium ou son propre label Firework
Edition. Sa musique, souvent con\c{c}ue comme bande{}-son pour une
installation visuelle ou pour une performance exp\'erimentale, est
essentiellement bas\'ee sur des sources soigneusement s\'electionn\'ees
et fluctue entre l'\'electronique douce et le noise
brut. Son {\oe}uvre \'eclectique incorpore souvent des r\^eves et des
absurdit\'es subtiles, des hi\'erarchies sociales renvers\'ees, des
actions cach\'ees et des \'ev\'enements qui prennent un caract\`ere
iconique.

Avec l'artiste Carl Michael von Hausswolff, il a
fond\'e le royaume d'Elgaland{}-Vergaland (KREV), dans
lequel il jouit du titre de Roi.}

\SubSubTitle{elpueblodelchina}

\SmallUrl{http://www.elpueblodechina.org/}

\English{elpueblodechina a.k.a. Alejandra Perez Nu\~nez is a sound artist and
performer working with open source tools, electronic wiring and essay
writing. In collaborative projects with Barcelona based group
Redactiva, she works on psychogeography and social science fiction
projects, developing narratives related to the mapping of collective
imagination. She received an MA in Media Design at the Piet Zwart
Institute in 2005, and has worked with the organization V2\_ in
Rotterdam. She is currently based in Valpara\'iso, Chile, where she is
developing a practice related to appropriation, civil society and
self{}-mediation through electronic media.}

\Nederlands{elpueblodechina a.k.a. Alejandra Perez Nu\~{n}ez is geluidskunstenares en
performer. Ze werkt met opensourcetools, elektronische bekabeling en
schrijft essays. In samenwerking met de groep Redactiva in Barcelona
werkt ze aan projecten rond psychogeografie en sociale sciencefiction,
waarbij ze verhaallijnen ontwikkelt die gelinkt zijn aan
cartografie\"en van collectieve verbeelding. Ze behaalde een MA in
Media Design aan het Piet Zwart Instituut in 2005, en werkte samen met
de organisatie V2\_ in Rotterdam. Op dit moment woont ze in
Valpara\'iso, Chili, waar ze een praktijk ontwikkelt rond
toe{}-eigening, de civiele maatschappij en \quote{zelfmedi\"ering} via
elektronische media.}

\Francais{elpueblodechina a.k.a. Alejandra Perez Nu\~nez est une artiste sonore et
une performeuse qui travaille avec des outils open source et le
c\^ablage \'electronique, et qui \'ecrit des essais. En collaboration
avec le groupe Redactiva, bas\'e \`a Barcelone, elle travaille sur la
psychog\'eographie et sur des projets de science{}-fiction sociale qui
d\'eveloppent des narrations li\'ees \`a la cartographie de
l'imaginaire collectif. Elle est dipl\^om\'ee d'un MA en Media Design
au Piet Zwart Instituut en 2005, et a travaill\'e avec l'organisation
V2\_ \`a Rotterdam. Elle vit actuellement \`a Valparaiso, Chili, o\`u
elle \'elabore une pratique li\'ee \`a l'appropriation, la soci\'et\'e
civile et la m\'ediation personnelle \`a travers les media
\'electroniques.}

\SubSubTitle{Andrea Fiore}

\SmallUrl{http://pzwart.wdka.hro.nl/mdma/alumni/2007/afiore/}

\English{Born in Bari (Italy) in 1980, and graduated in May 2005 in Communication
Sciences at the University of Rome La Sapienza, with a dissertation
thesis on software as cultural and social artefact. His educational
background is mostly theoretical: Humanities and Media Studies. More
recently, he has been focussing on programming and the development of
web based applications, mostly using open source technologies. In 2007
he received an M.A. in Media Design at the Piet Zwart Institute in
Rotterdam.

His areas of interest are: social software, actor network theory,
digital archives, knowledge management, machine readability, semantic
web, data mining, information visualization, profiling, privacy,
ubiquitous computing, locative media.}

\Nederlands{Geboren in Bari (Itali\"e) in 1980. Behaalde zijn diploma in de
Communicatiewetenschappen aan de Universiteit van Rome La Sapienza, met
een thesis over software als cultureel en sociaal artefact. Recent gaat
zijn interesse uit naar profielen, sociale software, de compilatie van
data en de exploratie van numerieke archieven en privacy. In 2007
behaalde hij een M.A. in Media Design aan het Piet Zwart Instituut in
Rotterdam.}

\Francais{A un dipl\^ome en Sciences de la Communication de l'Universit\'e de
Rome, dont un m\'emoire sur le logiciel comme artefact culturel.
R\'ecemment il passe son temps sur les questions de profils, de
software social, de compilation de donn\'ees, d'exploration d'archives
num\'eriques et de vie priv\'ee. Il vient d'obtenir un M.A. en Media
Design du Piet Zwart Instituut, Rotterdam.}

\SubSubTitle{Fr\'ed\'eric Gies}

\English{After studying ballet and contemporary dance, Fr\'ed\'eric Gies worked
with various choreographers such as Daniel Larrieu, Bernard Glandier,
Jean{}-Fran\c{c}ois Duroure, Olivia Grandville and Christophe Haleb. In
1995, he created a duet in collaboration with Odile Seitz
({\em Because I love}). In 1998 he started working with Fr\'ed\'eric
De Carlo. Together they have created various performances such as
{\em Le principal d\'efaut} (CND, Paris), {\em Le principal
d\'efaut{}-solo} (Tipi de Beaubourg, Paris), {\em En corps} (CND,
Paris), {\em Post porn traffic} (Macba, Barcelona), {\em In bed
with Rebecca} (Vooruit, Gent), {\em (don't) Show it!} (Sc\`ene
nationale, Dieppe), {\em Second hand vintage collector (sometimes we
like to mix it up!)} (Ausland, Berlin).

In 2004, he danced in {\em The better you look, the more you see} by
Isabelle Schad and took part in {\em Good Work} ({\em California
Roll}). In the same year, he also started on a series of solos:
{\em Sleeping beauties/Ultra sexy amazons} (1st version in
Tanzfabrik, 2nd in Ausland, Berlin) and {\em The bitch is back under
pressure (reloaded)} (Basso, Berlin). As a memeber of the Praticable
collective, he created {\em Dance} and {\em The breast piece}, in
collaboration with Alice Chauchat. He also collaborated on
{\em Still Lives} ({\em Good Work}: Anderson/ Gies/ Pelmus/
Pocheron/ Schad).}

\Nederlands{Studeerde ballet en hedendaagse dans. Werkte samen met verschillende
choreografen zoals Daniel Larrieu, Bernard Glandier,
Jean{}-Fran\c{c}ois Duroure, Olivia Grandville en Christophe Haleb. In
1995 brengt hij een duet samen met Odile Seitz ({\em Because I
love}). In 1998 begint zijn samenwerking met Fr\'ed\'eric De Carlo:
{\em Le principal d\'efaut} (CND, Parijs), {\em Le principal
d\'efaut{}-solo} (Tipi de Beaubourg, Parijs), {\em En corps} (CND,
Parijs), {\em Post porn traffic} (Macba, Barcelona), {\em In bed
with Rebecca} (Vooruit, Gent), {\em (don't) Show it!} (Sc\`ene
nationale, Dieppe), {\em Second hand vintage collector (sometimes we
like to mix it up!)} (Ausland, Berlijn).

In 2004 danst hij in {\em The better you look, the more you see} van
Isabelle Schad en neemt hij deel aan {\em Good Work}
({\em California Roll}). In hetzelfde jaar begint hij een reeks
solo's: {\em Sleeping beauties/Ultra sexy amazons} (eerste versie in
Tanzfabrik, tweede in Ausland, Berlijn) en {\em The bitch is back
under} {\em pressure (reloaded)} (Basso, Berlijn). Binnen het
collectief {\em Praticable} maakte hij {\em Dance} en {\em The
breast piece}, samen met Alice Chauchat. Hij werkt ook mee aan
{\em Still Lives} ({\em Good Work}: Anderson/ Gies/ Pelmus/
Pocheron/ Schad).}

\Francais{Chor\'egraphe et danseur. Il vit et travaille \`a Berlin. Les pratiques
de corps nourrissent son travail sur les implications politiques des
\'ecritures chor\'egraphiques et des repr\'esentations du corps qui en
\'emergent. Il a \'etudi\'e la danse classique et la danse
contemporaine. Depuis 1999, il suit r\'eguli\`erement des stages de
BMC{\registered}, avec Vera Orlock, Sarita Beraha, Trisha Bauman,
Lambrini Konstantinou et Walburga Glatz. Il d\'ebute sa carri\`ere de
danseur en France en 1992. Il a travaill\'e pour des chor\'egraphes
tels que Daniel Larrieu, Olivia Grandville, Bernard Glandier, Odile
Duboc et Christophe Haleb. En 1995, il cr\'ee avec Odile Seitz la
pi\`ece {\em Because I love}. En 1998, il entame une s\'erie de
collaborations avec Fr\'ed\'eric de Carlo. Ils cr\'eent ensemble
plusieurs pi\`eces et performances: {\em Le principal d\'efau}t
(CND, Paris), {\em Le principal d\'efaut{}-solo} (Tipi de Beaubourg,
Paris), {\em En corps} (CND, Paris), {\em Post porn traffic}
(Macba, Barcelona), {\em In bed with Rebecca} (Vooruit, Gent),
{\em (don't) Show it!} (Sc\`ene nationale, Dieppe), {\em Second
hand, vintage, collector (sometimes we like to mix it up!)} (Ausland,
Berlin). En 2004, il s'installe et commence \`a travailler \`a Berlin.
Il y cr\'ee en 2004 le solo {\em Sleeping Beauties/Ultra sexy
amazons} (Tanzfabrik, et Ausland) et danse pour Isabelle Schad et
Eszter Salamon. En 2006, Il cr\'ee le solo {\em Dance} (Praticable)
(Tanz made in Berlin et Tanznacht) et signe en collaboration avec
Isabelle Schad, Manuel Pelmus et Bruno Pocheron, le projet
{\em Still Lives} (pr\'esent\'e \`a Lille, Essen, Bucarest, Berlin,
Halle et Hannover). La m\^eme ann\'ee, il cr\'ee avec Alice Chauchat,
Fr\'ed\'eric de Carlo, Isabelle Schad et Odile Seitz, le collectif
Praticable. Ils b\'en\'eficient en 2007 d'une r\'esidence \`a
Tanzfabrik Potsdam (Tanzplan Deutschland). En 2007, il cr\'ee en
collaboration avec Alice Chauchat {\em The breast piece}
(Praticable) dans le cadre de Tanz im august, \`a Berlin. Il pr\'epare
actuellement une version de groupe de {\em Dance} (Praticable), qui
sera pr\'esent\'ee en octobre 2008 au Sophiensaele \`a Berlin.}

\SubSubTitle{Dominique Goblet}

\SmallUrl{http://www.dominique-goblet.be/}

\English{Visual artist. She shows her work in galleries and publishes her stories
in magazines and books. In all cases, what she tries to pursue is an
art of the multi{}-faceted narrative. Her exhibitions of paintings {--}
from frame to frame and in the whole space of the gallery {--} could be
\quote{read} as fragmented stories. Her comic books question the deep or thin
relations between human beings. As an author, she has taken part in 
almost all the Frigobox series published by Fr\'eon (Brussels) and to
several Lapin magazines, published by L'Association (Paris). A silent
comic book was published in the gigantic Comix 2000 (L'Association). In
the beginning of 2002, a second book is published by the same editor:
{\em Souvenir d'une journ\'ee parfaite {}- Memories of a perfect day}
{}- a complex story that combines autobiographical facts and fictions.}

\Nederlands{Beeldend kunstenares. Ze toont haar werk in galerijen en publiceert haar
verhalen in tijdschriften en boeken. In alles wat ze doet, streeft ze
een veelzijdige narrativiteit na. De tentoonstellingen van haar
schilderijen {--} kader naast kader over de hele ruimte van een galerij
{--} lezen als gefragmenteerde verhalen. Haar stripverhalen zijn een
onderzoek naar menselijke relaties. Als auteur nam ze deel aan bijna
alle Frigobox{}-series uitgegeven door Fr\'eon (Brussel) en aan
verschillende Lapin{}-magazines, uitgegeven door L'Association
(Parijs). Een woordenloze strip werd uitgegeven in de gigantische Comix
2000 (L'Association). Begin 2002 verscheen haar tweede boek bij
dezelfde uitgever: {\em Souvenir d'une journ\'ee parfaite {}-
Herinneringen aan een perfecte dag} {}- een complexe mix van
autobiografische feiten en fictie.}

\Francais{Dominique Goblet est une artiste plasticienne. Elle expose son travail
dans des galeries et publie ses r\'ecits dans des magazines et des
livres. L'objet de sa recherche est une narration \`a
multiples facettes. Ses expositions de peinture {--}
d'un cadre \`a l'autre et dans
l'espace {--} peuvent \^etre lues comme des r\'ecits
fragment\'es. Ses bandes dessin\'ees questionnent la profondeur ou la
l\'eg\`eret\'e des relations humaines. En tant
qu'auteure, elle a particip\'e \`a peu pr\`es \`a
toutes les s\'eries Frigobox publi\'ees par Fr\'eon (Bruxelles) et \`a
plusieurs Lapin magazines publi\'es par l'Association
(Paris). Elle a r\'ealis\'e une bande dessin\'ee sans parole pour le
gigantesque Comix 2000 (l'Association). D\'ebut 2000,
elle a publi\'e un second livre chez le m\^eme \'editeur:
{\em Souvenir d'une journ\'ee parfaite} {}- une
histoire complexe qui combine des faits autobiographiques et
fictionnels.
}
\SubSubTitle{Tsila Hassine}

\SmallUrl{http://www.missdata.org/}

\English{Tsila Hassine is a media artist / designer. Her interests lie with the
hidden potentialities withheld in the electronic data mines. In her
practice she endeavours to extrude undercurrents of information and
traces of processes that are not easily discerned through regular
consumption of mass networked media. This she accomplishes through
repetitive misuse of available platforms.

She completed a BScs in Mathematics and Computer Science and spent 2003
at the New Media department of the HGK Z\"urich. In 2004 she joined the
Piet Zwart Institute in Rotterdam, where she pursued an MA in Media
Design, until graduating in June 2006 with Google randomizer Shmoogle.
She is currently a researcher at the Design department of the Jan van
Eyck Academie.}

\Nederlands{Tsila Hassine is mediakunstenares en webprogrammeur. Haar interesse gaat
uit naar de verborgen potentialiteiten die verzonken liggen in de
elektronische data mines. In haar praktijk be\-ijvert ze het naar de
oppervlakte brengen van verborgen informatiestromen en sporen van
processen die aan het oog ontsnappen bij de reguliere consumptie van
genetwerkte massamedia. Dit bewerkstelligt ze door het herhaaldelijk
misbruik van reeds beschikbare platformen.

Hassine behaalde een BScs in Wiskunde en Computerwetenschappen en bracht
2003 door aan het departement Nieuwe Media van de HGK Z\"urich. In 2006
behaalde ze een MA in Media Design aan het Piet Zwart Instituut in
Rotterdam met de Google randomizer Shmoogle. Op dit moment is Hassine
onderzoekster aan het Design departement van de Jan Van Eyck Academie.
}
\Francais{Tsila Hassine est une artiste des media et programmeuse web. Elle
s'int\'eresse essentiellement au potentiel cach\'e qui
r\'eside dans la fouille de donn\'ees (data mining). Dans sa pratique,
elle s'emploie \`a extraire des informations
ensevelies et des traces de processus qui ne sont pas faciles \`a
discerner dans la consommation courante des media en r\'eseau. Elle
atteint cet objectif par un usage d\'etourn\'e des plates{}-formes
auxquelles nous avons acc\`es. Elle a \'etudi\'e les math\'ematiques et
l'informatique et a pass\'e l'ann\'ee
2003 au d\'epartement Nouveaux Media de l'HKG
Z\"urich. Elle a achev\'e son parcours au Piet Zwart Institute en 2006
avec la cr\'eation de Shmoogle, une version de Google al\'eatoire.
Tsila Hassine est actuellement chercheuse au d\'epartement de design de
la Jan Van Eyck Academie.}

\SubSubTitle{Simon Hecquet}

\English{Dancer and choreographer. Educated in classical and contemporary dance,
Hecquet has worked with many different dance companies, specialised in
contemporary as well as baroque dance. During this time, he also
studied different notation systems to describe movement, after which he
wrote scores for several dance pieces from the contemporary
choreographic repertory. He also contributed, among others, with the
Quatuor Knust, to projects that restaged important dance pieces of the
20th century. Together with Sabine Prokhoris he made a movie,
{\em Ceci n'est pas une danse chorale} (2004), and a book,
{\em Fabriques de la Danse} (PUF, 2007). He teaches transcription
systems for movement, among others, at the department of Dance at the
Universit\'e de Paris VIII.}

\Nederlands{Danser en choreograaf. Opgeleid in klassieke en hedendaagse dans, was
Hecquet werkzaam bij verschillende gezelschappen, zowel toegespitst op
hedendaagse als op barokke dans. Tegelijkertijd volgde hij een
opleiding in de verschillende notatiesystemen om beweging te
beschrijven, waarna hij partituren realiseerde voor verschillende
stukken uit het hedendaags choreografisch repertoire. Ook droeg hij,
onder andere met het Quatuor Knust, bij tot projecten die belangrijke
dansstukken uit de 20\high{ste} eeuw waarvan choreografische
partituren voorhanden zijn, heropvoerden. Samen met Sabine Prokhoris
realiseerde hij een film, {\em Ceci n'est pas une danse chorale}
(2004), en een boek, {\em Fabriques de la Danse} (PUF, 2007). Hij
onderwijst ook in transcriptiesystemen voor beweging, o.a. aan het
departement Dans van de Universit\'e de Paris VIII.}

\Francais{Danseur et chor\'egraphe. Form\'e en danse classique et contemporaine,
il a \'et\'e interpr\`ete aupr\`es de diff\'erentes compagnies, en
danse contemporaine mais \'egalement en danse baroque. Parall\`element,
il a suivi une formation en diff\'erents syst\`emes d'\'ecriture pour
le mouvement, ce qui l'a conduit \`a r\'ealiser des partitions pour
diff\'erentes pi\`eces du r\'epertoire chor\'egraphique contemporain.
Il est \'egalement, avec le Quatuor Knust entre autres, \`a l'origine
de projets de r\'ecr\'eation de pi\`eces importantes dans l'histoire de
la danse du 20\high{i\`eme} si\`ecle, pour lesquelles
existent des partitions chor\'egraphiques. Avec Sabine Prokhoris, il a
r\'ealis\'e un film, {\em Ceci n'est pas une danse chorale}, en
2004, et un livre, {\em Fabriques de la Danse} (PUF, 2007). Il a
aussi une activit\'e d'enseignement des syst\`emes de transcriptions du
mouvement, entre autres au d\'epartement Danse de l'Universit\'e de
Paris VIII.}

\SubSubTitle{Guy Marc Hinant}

\English{Guy Marc Hinant is a filmmaker of films like {\em The Garden is full
of Metal} (1996), {\em \'El\'ements d'un Merzbau
oubli\'e} (1999), {\em The Pleasure of Regrets {--} a Portrait of
L\'eo Kupper} (2003), {\em Luc Ferrari face to his Tautology} (2006)
and {\em I never promised you a rose garden {--} a portrait of David
Toop through his records collection} (2008), all developed together
with Dominique Lohl\'e. He is the curator of {\em An Anthology of
Noise and Electronic Music CD Series}, and manages the Sub Rosa label.
He writes fragmented fictions and notes on aesthetics (some of his
texts have been published by {\em Editions de
l'Heure}, {\em Luna Park}, {\em Leonardo Music
Journal} etc.).}

\Nederlands{Guy Marc Hinant is een filmmaker van films als {\em The Garden is
full of Metal} (1996), {\em \'El\'ements d'un
Merzbau oubli\'e} (1999), {\em The Pleasure of Regrets {--} un
Portrait de L\'eo Kupper} (2003), {\em Luc Ferrari face \`a sa
Tautology} (2006) en {\em I never promised you a rose garden {--} a
portrait of David Toop through his records collection} (2008), allen
ontwikkeld samen met Dominique Lohl\'e. Hij is de curator van
{\em An Anthology of Noise and Electronic Music CD Series}, en
bestuurt het Sub Rosa label. Hij schrijft korte fictieverhalen en
aantekeningen over esthetiek (sommige van zijn teksten werden
gepubliceerd door {\em Editions de l'Heure},
{\em Luna Park}, {\em Leonardo Music Journal} etc.).}

\Francais{Guy Marc Hinant est cin\'easte. En collaboration avec Dominique Lohl\'e,
il a r\'ealis\'e des films comme {\em The Garden is full of Metal}
(1996), {\em \'El\'ements d'un Merzbau oubli\'e}
(1999), {\em The Pleasure of Regrets {--} un Portrait de L\'eo
Kupper} (2003), {\em Luc Ferrari face \`a sa Tautology} (2006) et
{\em I never promised you a rose garden {--} a portrait of David
Toop through his records collection} (2008). Il est le commissaire de
la s\'erie de Cds {\em An Anthology of Noise and Electronic Music}
et dirige le label Sub Rosa. Il \'ecrit des fictions courtes et des
r\'eflexions sur la question de l'esth\'etique
(certains de ses textes ont \'et\'e publi\'es par les {\em Editions
de l'Heure}, {\em Luna Park}, {\em Leonardo
Music Journal} etc.).}

\SubSubTitle{Dmytri Kleiner}

\SmallUrl{http://www.telekommunisten.net/}

\English{Dmytri Kleiner is a USSR{}-born, Canadian software developer and
cultural producer. In his work, he investigates the intersections of
art, technology and political economy. He is a founder of
Telekommunisten, an anarchist technology collective, and lives in
Berlin with his wife Franziska and his daughter Henriette.}

\Nederlands{Dmytri Kleiner is een an\-ar\-chis\-ti\-sche hacker, medeoprichter van de
Telekommunisten, een co\"operatieve gespecialiseerd in telefonische
systemen. Dmytri werd geboren in de Sovjet{}-Unie en leeft in Berlijn,
samen met zijn vrouw Franziska en hun dochter Henriette.}

\Francais{Dmytri Kleiner est un hacker anarchiste, co{}-fondateur de
Telekommunisten, une coop\'erative sp\'ecialis\'ee dans les syst\`emes
t\'el\'ephoniques. Dmytri est n\'e en Union Sovi\'etique et vit \`a
Berlin avec sa femme Franziska et sa fille Henriette.}

\SubSubTitle{Bettina Knaup}

\English{Cultural producer and curator with a background in theatre and film
studies, political science and gender studies. She is interested in the
interface of live arts, politics and knowledge production, and has
curated and/or produced transnational projects such as the public arts
and science program \quote{open space} of the International Women's
University (Hannover, 1998{}-2000), and the transdisciplinary
performing arts laboratory, IN TRANSIT (Berlin, House of World Cultures
2002{}-2003). Between 2001 and 2004, she has co{}-curated and
co{}-directed the international festival of contemporary arts, CITY OF
WOMEN (Ljubljana). After directing the new European platform for
cultural exchange LabforCulture during its launch phase (Amsterdam,
2004{}-06), Knaup works again as an independent curator with a base in
Berlin.}

\Nederlands{Cultureel producent en curatrice met een achtergrond in theater{}- en
filmstudies, politieke wetenschappen en genderstudies. Ze is
ge\"interesseerd in de interface tussen live performance, politiek en
kennisproductie. Ze nam deel aan de productie en programmatie van de
Internationale Vrouwenuniversiteit (Hannover, 1998{}-2000), en aan het
transdisciplinaire laboratorium voor performance IN TRANSIT (Berlijn,
House of World Cultures 2002{}-2003). Tussen 2001 en 2004 coproduceerde
en coprogrammeerde Knaup het internationale festival voor hedendaagse
kunst CITY OF WOMEN (Ljubljana). Na de directie van de opstartfase van
het nieuw Europees platform voor culturele uitwisseling, LabforCulture
(Amsterdam, 2004{}-06), werkt Knaup opnieuw als een onafhankelijke
curatrice, met Berlijn als uitvalsbasis.}

\Francais{Productrice culturelle et commissaire d'expositions.
Son parcours comprend l'\'etude du th\'e\^atre et du
film ainsi que les sciences politiques et les \'etudes de genre. Elle
s'int\'eresse \`a l'interface entre
la performance, la politique et la production de savoir. Elle a
particip\'e \`a la production et la programmation de l'Universit\'e
Internationale des Femmes (Hannovre, 1998{}-2000) et du laboratoire
transdisciplinaire de performances, IN TRANSIT (Berlin, House of World
Cultures 2002{}-2003). Entre 2001 et 2004 elle a co{}-produit et
co{}-programm\'e le festival des arts contemporains, CITY OF WOMEN
(Ljubljana). Apr\`es avoir dirig\'e la nouvelle plateforme pour
l'\'echange culturel europ\'een LabforCulture
(Amsterdam, 2004{}-06), pendant sa phase de lancement, Bettina Knaup
travaille \`a nouveau comme curatrice ind\'ependante bas\'ee \`a
Berlin.}

\SubSubTitle{Christophe Lazaro}

\SmallUrl{http://www.fundp.ac.be/facultes/droit/recherche/centres/crid/}

\English{Christophe Lazaro is a scientific collaborator at the Law department of
the Facult\'es Notre{}-Dame de la Paix, Namur, and researcher at the
Research Centre for Computer and Law. His interest in legal matters is
complemented by socio{}-anthropological research on virtual communities
(free software community), the human/artefact relationship (prothesis,
implants, RFID chips), transhumanism and posthumanism.}

\Nederlands{Christophe Lazaro is wetenschappelijk medewerker aan het departement
rechten van de Facult\'es Notre{}-Dame de la Paix, Namen, en
onderzoeker bij het Centre de Recherche Informatique et Droit van de
Universit\'e de Namur. Zijn interesse voor rechten vult hij aan met
socio{}-antropologisch onderzoek naar virtuele gemeenschappen (vrije
software), de relatie mens/artefact (prothesen, implantaten,
RFID{}-chips), transhumanisme en posthumanisme...}

\Francais{Christophe Lazaro est collaborateur scientifique au d\'epartement de
droit des Facult\'es Notre{}-Dame de la Paix, Namur, et chercheur au
Centre de Recherche Informatique et Droit de l'Universit\'e de Namur.
Son int\'er\^et pour le droit s'enrichit de ses recherches
socio{}-anthropologiques sur les communaut\'es virtuelles (logiciels
libres), la relation homme/artefact (proth\`eses, implants, puces
RFID), le transhumanisme et le post{}-humain.}

\SubSubTitle{Manu Luksch}

\SmallUrl{http://www.ambienttv.net/}

\English{Manu Luksch, founder of ambientTV.NET, is a filmmaker who works outside
the frame. The \quote{moving image}, and in particular the evolution of film
in the digital or networked age, has been a core theme of her works.
Characteristic is the blurring of boundaries between linear and
hypertextual narrative, directed work and multiple authorship, and
post{}-produced and self{}-generative pieces. Expanding the idea of the
viewing environment is also of importance; recent works have been shown
on electronic billboards in public urban spaces and in open{}-air
cinemas in remote rural places.}

\Nederlands{Manu Luksch, medeoprichtster van ambientTV.NET, is een webkunstenares en
cineaste uit Wenen, met Londen als uitvalsbasis. Het {\quote{bewegend beeld},}
en meer bepaald de evolutie van film in het digitale en \quote{vernetwerkte}
tijdperk, is een centraal thema in haar werk. Karakteristiek zijn ook
het vervagen van grenzen tussen lineair en hypertekstueel narratief,
tussen geregisseerd werk en \quote{meervoudig auteurschap}, tussen
geproducete en zelfgeneratieve werken. Ook de verruiming van het idee
van de vertoningscontext is belangrijk; recente werken werden in de
publieke stedelijke ruimte op elektronische billboards getoond en in
openluchtcinema's op afgelegen rurale plaatsen.}

\Francais{Net{}-artiste et cin\'easte viennoise bas\'ee \`a Londres,
co{}-fondatrice d'ambientTV.NET. L'
\quote{image en mouvement}, et plus pr\'ecis\'ement
l'\'evolution du film \`a l'\`ere du
r\'eseau (digital), est un th\`eme central de son {\oe}uvre. Son travail
estompe les fronti\`eres entre la narration lin\'eaire et
hypertextuelle, entre la r\'ealisation personnelle et collective, entre
les {\oe}uvres produites et g\'en\'eratives. Manu Luksch
s'int\'eresse aussi \`a
l'environnement dans lequel sont vues ses cr\'eations;
ses travaux r\'ecents ont \'et\'e montr\'es sur des affichages
\'electroniques dans l'espace urbain ainsi que dans
des cin\'emas en plein air dans des espaces ruraux.}

\SubSubTitle{Adrian Mackenzie}

\SmallUrl{http://www.lancs.ac.uk/staff/mackenza/}

\English{Adrian Mackenzie (Centre for Social and Economic Aspects of Genomics,
Lancaster University) does social research in areas of new media,
wireless and genomic cultures. He has studied the cultural life of
software, and how software has taken on cultural value ({\em Cutting
code: software and sociality}, New York: Peter Lang, 2006). He has
recently been working on signal processing, looking at how artists,
activists, development projects, and community groups are making
alternate or competing communication infrastructures.}

\Nederlands{Adrian Mackenzie (Centre for Social and Economic Aspects of Genomics,
Lancaster University) doet sociaal onderzoek in de domeinen van nieuwe
media, draadloze netwerken en genomische culturen. Hij bestudeerde het
culturele leven van software en de manier waarop software een culturele
waarde heeft gekregen ({\em Cutting code: software and sociality},
New York: Peter Lang, 2006). Recent bestudeerde hij signal processing,
door te observeren hoe kunstenaars, activisten, ontwikkelingsprojecten
en gemeenschappen alternatieve infrastructuren voor communicatie
opzetten.}

\Francais{Adrian Mackenzie (Centre for Social and Economic Aspects of Genomics,
Lancaster University) conduit des recherches sociales dans les champs
des nouveaux m\'edias, des r\'eseaux sans fil et les cultures
g\'enomiques. Il a \'etudi\'e la vie culturelle du logiciel et comment
le logiciel a acquis de la valeur culturelle ({\em Cutting code:
software and sociality}, New York : Peter Lang, 2006). Il a travaill\'e
r\'ecemment sur le processus du signal, observant comment les artistes,
les activistes, les projets de d\'eveloppement et les communaut\'es
cr\'eent des infrastructures de communication alternatives.}

\SubSubTitle{Nicolas Malev\'e}

\SmallUrl{http://copycult.constantvzw.org/}

\English{Since 1998 multimedia artist Nicolas Malev\'e has been an active member
of the organization of Constant. As such, he has taken part in
organizing various activities connected with alternatives to
copyrights, such as \quote{Copy.cult \& The Original Si(g)n} in 2000. He has
also been developing multimedia projects and web applications for
cultural organizations. His research work is currently focused on
information structures, metadata and the semantic web, and how they are
visually represented.}

\Nederlands{Multimedia{}-maker. Sinds 1998 actief lid van de vereniging Constant.
Hij werkte aan verschillende activiteiten rond alternatieven voor
copyright, zoals \quote{Copy.cult \& The Original Si(g)n} in 2000, en
ontwikkelt multimediaprojecten en webapplicaties voor culturele
organisaties. Op dit moment houdt hij zich in zijn onderzoekswerk bezig
met visuele en artistieke vormen voor informatiestructuren, metadata en
het semantische web.}

\Francais{Depuis 1998, Nicolas Malev\'e, r\'ealisateur multimedia, est un membre
actif de l'association Constant. Il a pris part \`a l'organisation de
diff\'erentes activit\'es li\'ees aux alternatives au droit d'auteur,
comme \quote{Copy.cult \& The Original Si(g)n}, en 2000. Il a d\'evelopp\'e
des projets multim\'edias et des applications web pour des
organisations culturelles. Sa recherche actuelle est orient\'ee vers
les structures de l'information, les metadata, le web s\'emantique et
les formes visuelles et artistiques qu'elles g\'en\`erent.
}

\SubSubTitle{M\'eTAmorphoZ}

\SmallUrl{http://www.metamorphoz.be/}

\English{Born in September 2001, represented here by Val\'erie Cordy and Natalia
De Mello, the M\'eTAmorphoZ collective is a multidisciplinary
association that create installations, spectacles and transdisciplinary
performances that mix artistic experiments and digital practices.}

\Nederlands{Het collectief M\'eTAmorphoZ (Val\'erie Cordy, Natalia de Mello) werd
opgericht in september 2001 als een multidisciplinaire vereniging voor
de creatie van installaties, voorstellingen en transdisciplinaire
performances waarbij artistiek experiment samengaat met digitale
praktijken.}

\Francais{N\'e en septembre 2001, le collectif M\'eTAmorphoZ (Val\'erie Cordy,
Natalia de Mello) est une association multidisciplinaire qui cr\'ee des
installations, des spectacles et des performances transdisciplinaires
m\^elant propositions artistiques et pratiques num\'eriques. Avec le
projet Doppelg\"anger, le collectif s'int\'eresse \`a la th\'ematique
du double \'electronique dans la soci\'et\'e de contr\^ole et de
surveillance. {\em Notre identit\'e \'electronique,
embl\`eme de cette nouvelle soci\'et\'e de contr\^ole, redouble
d\'esormais notre identit\'e organique et sociale. Mais l'obligation
l\'egale de se voir assigner une identit\'e unique, stable et
infalsifiable n'est{}-elle pas, en d\'efinitive, un danger pour notre
libert\'e fondamentale de revendiquer des identit\'es qui sont
forc\'ement et irr\'em\'ediablement multiples pour chacun d'entre nous
?}}

\SubSubTitle{Michael Murtaugh}

\SmallUrl{http://automatist.org/}

\English{Freelance developer of (tools for) online documentaries and other forms
of digital archives. He works and lives in the Netherlands and online
at automatist.org. He teaches at the MA Media Design program at the
Piet Zwart Institute in Rotterdam.}

\Nederlands{Ontwikkelt als freelancer (tools voor) online documentaires en andere
vormen van digitale archieven. Hij werkt en woont in Nederland en
online op automatist.org. Hij is docent aan het Piet Zwart Instituut in
Rotterdam, voor het programma MA Media Design.}

\Francais{Producteur ind\'ependant d'outils pour le documentaire en ligne et tout
autre forme d'archives digitales. Il vit et travaille en Hollande et en
ligne sur automatist.org. Il enseigne au Piet Zwart Institute de
Rotterdam, dans le cadre du programme du MA en Media Design.}

\SubSubTitle{Julien Ottavi}

\SmallUrl{http://www.noiser.org/}

\English{Ottavi is the founder, artistic programmer, audio computer researcher
(networks and audio research) and sound artist of the experimental
music organization Apo33. Founded in 1997, Apo33 is a collective of
artists, musicians, sound artists, philosophers and computer
scientists, who aim to promote new types of music and sound practices
that do not receive large media coverage. The purpose of Apo33 is to
create the conditions for the development of all of the kinds of music
and sound practices that contribute to the advancement of sound
creation, including electronic music, concrete music, contemporary
written music, sound poetry, sound art and other practices which as yet
have no name. Apo33 refers to all of these practices as \quote{Audio Art}.}

\Nederlands{Ottavi is de oprichter, artistieke programmator, onderzoeker (netwerken
en audio{}-onderzoek) en geluidskunstenaar van de experimentele
muziekorganisatie Apo33. Apo33 is een collectief van kunstenaars,
muzikanten, geluidskunstenaars, filosofen en computerwetenschappers,
opgericht in 1997. Hun doel is om nieuwe vormen van muziek en
geluidspraktijken te promoten die geen grote mediabelangstelling
genieten. Ze cre\"eren een context voor de ontwikkeling van alle
mogelijke vormen van muziek en geluidspraktijken, die bijdragen aan de
evolutie van de geluidscreatie: elektronische muziek, concrete muziek,
hedendaagse muziek, geluidspo\"ezie, geluidskunst, en allerlei
praktijken waarvan de naam nog moet worden uitgevonden. Apo33 verwijst
naar al deze verschillende vormen als \quote{Audio Art}.}

\Francais{M\'ediactiviste, artiste{}-chercheur, musicien, performeur. Membre
fondateur d'Apo33 et activateur des associations Noise Mutation \&
Ecos. D\'eveloppe un travail de recherche et de cr\'eation croisant art
sonore, po\'esie sonore, nouvelles technologies, bricolage de
dispositifs \'electroniques et performance. Actif dans le mouvement du
libre, il a particip\'e au d\'eveloppement de la distribution
multim\'edia libre APODIO, qu'il utilise dans toutes ses cr\'eations,
processus et ateliers. Il organise de nombreux \'ev\'enements allant de
rencontres entre chercheurs, artistes et activistes, des festivals
d'art sonore ou de cr\'eation multim\'edia, de performance, \`a des
ateliers sur les logiciels libres et la r\'eappropriation des
dispositifs technologiques. Au{}-del\`a des m\'ediums et des
cat\'egories, l'activation et la mise en ab\^ime d'\'energies, de
concepts et de forces par l'exp\'erience est une de ses mani\`eres de
pratiquer la cr\'eation dans ses r\'ealit\'es mat\'erielles, sociales,
sensorielles, sensibles et conceptuelles.}

\SubSubTitle{Jussi Parikka}

\SmallUrl{http://users.utu.fi/juspar/}

\English{Jussi Parikka teaches and writes on the cultural theory and history of
new media. He has a PhD in Cultural History from the University of
Turku, Finland, and is Senior Lecturer in Media Studies at Anglia
Ruskin University, Cambridge, UK. Parikka has published a book on
\quote{cultural theory in the age of digital machines} (Koneoppi, in Finnish)
and his {\em Digital Contagions: A Media Archaeology of Computer
Viruses} has been published by Peter Lang, New York, Digital
Formations{}-series (2007). Parikka is currently working on a book on
\quote{Insect Media}, which focuses on the media theoretical and historical
interconnections of biology and technology.}

\Nederlands{Jussi Parikka doceert en schrijft over cultuurtheorie en de geschiedenis
van nieuwe media. Hij doctoreerde in Cultural History aan de
Universiteit van Turku in Finland, en is Senior Lecturer in Media
Studies aan de Anglia Ruskin University in Cambridge. Parikka
publiceerde een boek over \quote{cultuurtheorie in het tijdperk van de
digitale machines} (Koneoppi, in het Fins). Zijn boek {\em Digital
Contagions: A Media Archaeology of Computer Viruses} werd uitgegeven
door Peter Lang, New York, Digital Formations{}-series (2007). Parikka
werkt op dit moment aan een boek over \quote{Insect Media}, dat focust op de
mediatheoretische en historische interconnecties tussen biologie en
technologie.}

\Francais{Jussi Parikka enseigne et \'ecrit sur la th\'eorie culturelle et
l'histoire de nouveaux m\'edias. Il a un doctorat en histoire
culturelle de l'Universit\'e de Turku, Finlande, et est ma\^itre de
conf\'erences dans d'\'etudes des Media \`a l'Universit\'e Anglia
Ruskin, Cambridge. Parikka a publi\'e un livre sur \quote{la th\'eorie
culturelle à l'\^age des machines num\'eriques} (Koneoppi, en
finnois) et son livre {\em Digital Contagions: A Media Archaeology
of Computer Viruses} est publi\'e par Peter Lang, New York, Digital
Formations{}-series (2007). Parikka travaille actuellement sur un livre
sur les \quote{Insect Media}, qui se concentre sur les intercommunications
m\'edia{}-th\'eoriques et historiques de la bio\-logie et de la
technologie.}

\SubSubTitle{Sadie Plant}

\English{Sadie Plant is the author of {\em The Most Radical Gesture, Zeros and
Ones,} and {\em Writing on Drugs}. She has taught in the Department
of Cultural Studies, University of Birmingham, and the Department of
Philosophy, University of Warwick. For the last ten years she has been
working independently and living in Birmingham, where she is involved
with the Ikon Gallery, Stan's Cafe Theatre Company, and the Birmingham
Institute of Art and Design.}

\Nederlands{Sadie Plant is de auteur van {\em The Most Radical Gesture},
{\em Zeros and Ones}, en {\em Writing on Drugs}. Ze gaf les aan
het Departement Culturele Studies, University of Birmingham, en aan het
Departement Filosofie, University of Warwick. Sinds tien jaar woont en
werkt ze als zelfstandige in Birmingham, onder meer voor de Ikon
Gallery, Stan's Cafe Theatre Company, en het Birmingham Institute of
Art and Design.
}
\Francais{Sadie Plant est l'auteur de {\em The Most Radical Gesture : The
Situationist International in a Postmodern Age} (1992) et de
{\em Zeros and Ones : Di\-gi\-tal Women and the New Technoculture}
(1997). Son livre le plus r\'ecent, {\em Writing on Drugs}, a
\'et\'e publi\'e en 1999. Elle a enseign\'e dans le D\'epartement
d'\'etudes culturel \`a l'Universit\'e de Birmingham et dans le
D\'epartement de philosophie de l'Universit\'e de Warwick. Depuis une
dizaine d'ann\'ees, elle est ind\'ependante et vit \`a Birgmingham,
o\`u elle \ travaille notamment en collaboration avec l'Ikon Gallery,
la Stan's Cafe Theatre Company, et l'Institut de Birmingham d'Art et
Design.}

\SubSubTitle{Praticable}

\SmallUrl{http://www.theselection.net/dance/praticable/presentation.html}

\English{Praticable proposes itself as a horizontal work structure, which brings
into relation research, creation, transmission and production
structure. This structure is the basis for the creation of many
performances that will be signed by one or more participants in the
project. These performances are grounded, in one way or another, in the
exploration of body practices to approach representation. Concretely,
the form of Praticable is periods of common research of /on physical
practices which will be the soil for the various creations. The
creation periods will be part of the research periods. Thus, each
specific project implies the involvement of all participants in the
practice, the research and the elaboration of the practice from which
the piece will ensue.}

\Nederlands{Praticable is een onderzoeks{}- en samenwerkingsproject van
verschillende kunstenaars (op dit moment: Alice Chauchat, Fr\'ed\'eric
de Carlo, Fr\'ed\'eric Gies, Isabelle Schad en Odile Seitz). Praticable
is een horizontale werkstructuur die onderzoek, creatie,
overdracht/transmissie en productie{}-structuur met elkaar verbindt.
Deze structuur biedt de basis voor de creatie van een hele reeks
voorstellingen, van de hand van een of meerder leden van het project.
De voorstellingen vertrekken altijd op een of andere manier vanuit het
onderzoek naar lichaamspraktijken. Concreet vormen tijd voor onderzoek
en fysieke praktijk de basis van de verschillende creaties van
Praticable. De onderzoeksperiodes zijn vaak onafhankelijk van de
verschillende creaties, maar kunnen er ook deel van uitmaken.}

\Francais{Praticable est un projet de recherche et de collaboration entre
plusieurs artistes (\`a ce jour : Alice Chauchat, Fr\'ed\'eric de
Carlo, Fr\'ed\'eric Gies, Isabelle Schad et Odile Seitz). Praticable
est une structure horizontale de travail qui met en relation recherche,
cr\'eation, transmission et structure de production. Cette structure
est la base pour la cr\'eation de plusieurs pi\`eces, sign\'ees par un
ou plusieurs participants au projet. Ces pi\`eces s'attachent, d'une
mani\`ere ou d'une autre, \`a partir de l'exploration de pratiques de
corps pour aller vers la repr\'esentation. Concr\`etement, Praticable
prend la forme de temps de recherche et de pratique physique en commun,
servant de terreau pour les diff\'erentes cr\'eations. Ces temps de
recherches peuvent se d\'erouler de mani\`ere ind\'ependante par
rapports aux diff\'erents projets de cr\'eations, aussi bien que
s'int\'egrer \`a ceux{}-ci.}

\SubSubTitle{Sabine Prokhoris}

\English{Psychoanalyst and author of, among others, {\em Witch's Kitchen:
Freud, Faust, and the Transference} (Cornell University Press, 1995),
and co{}-author with Simon Hecquet of {\em Fabriques de la Danse}
(PUF, 2007). She is also active in contemporary dance, as a critic and
a choreographer. In 2004 she made the film {\em Ceci n'est pas une
danse chorale} together with Simon Hecquet.}

\Nederlands{Psychoanalytica. Publiceerde {\em La cuisine de la sorci\`ere}
(Aubier, 1988), en {\em Le sexe prescrit {}- La diff\'erence
sexuelle en question} (Aubier, 2000. Heruitgave Champs{}-Flammarion,
2002). Ze is samen met Simon Hecquet coauteur van {\em Fabriques de
la Danse} (PUF, 2007). Daarnaast is ze, zowel als critica als
choreografe, ook actief in de hedendaagse dans. In 2004 maakte ze samen
met Simon Hecquet de film {\em Ceci n'est pas une danse chorale}.}

\Francais{Ancienne \'el\`eve de l'\'Ecole Normale Sup\'erieure, agr\'eg\'ee de
philosophie, Sabine Prokhoris est psychanalyste. Elle a publi\'e
{\em La cuisine de la sorci\`ere} (Aubier, 1988), et {\em Le sexe
prescrit {}- La diff\'erence sexuelle en question} (Aubier, 2000.
R\'e\'ed. Champs{}-Flammarion, 2002). Avec Simon Hecquet, elle est
coauteur de {\em Fabriques de la Danse} (PUF, 2007). Elle
d\'eveloppe, parall\`element \`a son activit\'e de psychanalyste, un
travail dans le champ chor\'egraphique, de critique essentiellement,
mais \'egalement de cr\'eation. Elle a dans ce cadre{}-l\`a r\'ealis\'e
un film en 2004 avec Simon Hecquet, {\em Ceci n'est pas une danse
chorale}.}

\SubSubTitle{In\`es Rabadan}

\English{After obtaining a master's degree in Philosophy and Letters, In\`es
Rabadan studied film at the IAD. Her short films ({\em Vacance},
{\em Surveiller les Tortues}, {\em Maintenant}, {\em Si
j'avais dix doigts}, {\em Le jour du soleil}), were shown at about
sixty festivals. {\em Surveiller les tortues~}and
{\em Maintenant} were awarded at the festivals of Clermont,
Vend\^ome, Chicago, Aix, Grenoble, Brest and Namur. Occasionally she
supervises scenario workshops. Her first feature film,
{\em Belhorizon}, was selected for the festivals of Montr\'eal,
Namur, Cr\'eteil, Buenos Aires, Santiago de Chile, Santo Domingo and
Mannheim{}-Heidelberg. At the end of 2006, it was released in Belgium,
France and Switzerland.}

\Nederlands{Na een licentie in Filosofie \& Letteren aan de ULB, studeerde In\`es
Rabadan voor cineaste aan het IAD. Haar kortfilms ({\em Vacance},
{\em Surveiller les Tortues}, {\em Maintenant}, {\em Si
j'avais dix doigts}, {\em Le jour du} {\em soleil}), werden op
een zestigtal festivals vertoond. {\em Surveiller les tortues~}en
{\em Maintenant} werden bekroond in Clermont, Vend\^ome, Chicago,
Aix, Grenoble, Brest en Namen. Tevens begeleidt ze occasionaal
scenario{}-ateliers. Haar eerste langspeelfilm, {\em Belhorizon},
werd geselecteerd voor de festivals van Montr\'eal, Namen, Cr\'eteil,
Buenos Aires, Santiago de Chile, Santo Domingo en
Mannheim{}-Heidelberg. Eind 2006 kwam hij uit in Belgi\"e, Frankrijk en
Zwitserland.}

\Francais{Apr\`es une license en Philosophie \& Lettres \`a l'Universit\'e de
Bruxelles, In\`es Rabadan a \'etudi\'e le cin\'ema \`a l'IAD.
R\'ealisatrice de courts{}-m\'etrages film et vid\'eo
({\em Vacance}, {\em Surveiller les Tortues},
{\em Maintenant}, {\em Si j'avais dix doigts}, {\em Le jour du
soleil}), elle a montr\'e ses films dans une soixantaine de festivals.
{\em Surveiller les tortues~}et {\em Maintenant} ont \'et\'e
prim\'es \`a Clermont, Vend\^ome, Chicago, Aix, Grenoble, Brest et
Namur. Elle anime aussi \`a l'occasion des ateliers de sc\'enario. Son
premier long{}-m\'etrage, {\em Belhorizon}, a \'et\'e
s\'electionn\'e dans les festivals de Montr\'eal, Namur, Cr\'eteil,
Buenos Aires, Santiago de Chile, Santo Domingo et
Mannheim{}-Heidelberg. Il est sorti en Belgique, en France et en Suisse
\`a la fin de l'ann\'ee 2006.}

\SubSubTitle{Antoinette Rouvroy}

\SmallUrl{http://www.fundp.ac.be/facultes/droit/recherche/centres/crid/}

\English{Antoinette Rouvroy is researcher at the Law department of the Facult\'es
Notre{}-Dame de la Paix in Namur, and at the Research Centre for
Computer and Law. Her domains of expertise range from rights and ethics
of biotechnologies, philosophy of Law and \quote{critical legal studies} to
interdisciplinary questions related to privacy and
non{}-discrimination, science and technology studies, law and language.}

\Nederlands{Antoinette Rouvroy is onderzoekster aan het departement Rechten van de
Facult\'es Notre{}-Dame de la Paix in Namen, en aan het Centre de
Recherche Informatique et Droit van de Universiteit van Namen. Zij is
gespecialiseerd in het recht en de ethiek van biotechnologie, in
filosofie van het recht, in \quote{\em critical legal studies}, alsook
in \ interdisciplinaire vraagstukken aangaande privacy en
non{}-discriminatie,  \quote{\em science and technology studies}, recht
en taal.
}
\Francais{Antoinette Rouvroy est chercheuse au d\'epartement de droit des
Facult\'es Notre{}-Dame de la Paix, Namur, et au Centre de Recherche
Informatique et Droit de l'Universit\'e de Namur. Ses domaines de
comp\'etence sont le droit et \'ethique des biotechnologies, la
philosophie du droit et \quote{\em critical legal studies}, ainsi que
les questions interdisciplinaires relatives \`a la vie priv\'ee et \`a
la non{}-discrimination, \quote{\em science and technology studies}, et
droit et langage.}

\SubSubTitle{Femke Snelting}

\SmallUrl{http://ospublish.constantvzw.org/}

\English{Femke Snelting is a member of the art and design collective De Geuzen
and of the experimental design agency OSP.}

\Nederlands{Femke Snelting is lid van kunst{}- en ontwerpcollectief De Geuzen. Ze is
lid van Constant en actief in de experimentele ontwerpgroep OSP.}

\Francais{Femke Snelting est une artiste du collectif De Geuzen, et active dans le
groupe design experimentale OSP.}

\SubSubTitle{Michael Terry}

\SmallUrl{http://www.ingimp.org/}

\English{Computer Scientist, University of Waterloo, Canada.}

\Nederlands{Professor in de Computerwetenschappen, University of Waterloo, Canada.}

\Francais{Professeur, informaticien de l'Universit\'e de Waterloo au Canada.}

\SubSubTitle{Carl Michael von Hausswolff}

\English{Von Hausswolff was born in 1956 in Link\v{s}ping, Sweden. He lives and
works in Stockholm. Since the end of the 70s, von Hausswolff has been
working as a composer using the tape recorder as his main instrument
and as a conceptual visual artist working with performance art,
light{}- and sound installations and photography. His audio
compositions from 1979 to 1992, constructed almost exclusively from
basic material taken from earlier audiovisual installations and
performance works, essentially consist of complex macromal drones with
a surface of aesthetic elegance and beauty. In later works, von
Hausswolff retained the aesthetic elegance and the drone, and added a
purely isolationistic sonic condition to composing.}

\Nederlands{Zweeds kunstenaar en curator Carl Michael von Hausswolff
({\textdegree}1956) heeft een fascinatie voor schaduwzones, zowel in
zijn dagelijkse omgeving als in het culturele domein. Sinds eind jaren
'70 bestudeert hij geluiden, voornamelijk aan de hand van een
bandopnemer. In zijn audiowerk en zijn installaties onderzoekt hij
vooral fysieke werelden die op de grens liggen van de menselijke
perceptie. Hij doet een beroep op diverse elektronische media voor de
captatie en bewerking van data, energetische velden, visuele en
auditieve fenomenen. In de wereld van de beeldende kunst maakte hij
naam dankzij tentoonstellingen tijdens de Bi\"ennale van Istanbul,
Documenta X in Kassel, en de tweede Bi\"ennale van Johannesburg
(allemaal in 1997), als curator van het Noors paviljoen op de
Bi\"ennale van Veneti\"e, en als een van de stichters van de virtuele
natie, de \quote{Koninkrijken van Elgaland{}-Vargaland}.}

\Francais{L'artiste et curateur su\'edois Carl Michael von Hausswolff
({\textdegree}1956) est surtout fascin\'e par les zones d'ombre, tant
dans son environnement quotidien que dans le domaine culturel. Depuis
la fin des ann\'ees 1970 il effectue des \'etudes sur les sonorit\'es,
principalement \`a l'aide d'un magn\'etophone. Cependant, dans ses
{\oe}uvres sonores et ses installations, il s'applique surtout \`a
explorer les r\'ealit\'es physiques \`a la limite de la perception
humaine. Il fait appel \`a divers m\'edias \'electroniques pour
exploiter et traiter les flux de donn\'ees, les champs
\'energ\'etiques, les ph\'enom\`enes visuels et acoustiques. Dans le
monde des arts plastiques, il s'est fait conna\^itre par des
expositions lors de la Biennale d'Istanbul, de la Documenta X \`a
Kassel, et de la deuxi\`eme Biennale de Johannesburg (toutes en 1997),
en tant que curateur du pavillon Nordique \`a la Biennale de Venise, et
comme l'un des fondateurs d'une nation virtuelle, les \quote{Royaumes
d'Elgaland{}-Vargaland}.
}
\SubSubTitle{Marc Wathieu}

\SmallUrl{http://www.erg.be/sdr/blog/}

\English{Marc Wathieu teaches at Erg (digital arts) and HEAJ (visual
communication). He is a digital artist (he works with the Brussels
based collective LAB[au]) and sound designer. He is also an official
representative of the Robots Trade Union with the human institutions.
During V/J10 he presented the Robots Trade Union's Chart and ambitions.}

\Nederlands{Marc Wathieu geeft les aan het Erg (digitale kunsten) en aan het HEAJ
(visuele communicatie). Bovendien is hij digitaal kunstenaar (bij het
Brusselse collectief LAB[au]) en geluidsontwerper. Hij is ook een
offici\"ele vertegenwoordiger van de Vakbond van de Robots bij de
menselijke instellingen. Tijdens V/J10 stelde hij het Charter en de
ambities van de Robotvakbond voor.}

\Francais{Marc Wathieu enseigne \`a l'Erg (arts num\'eriques) et \`a la HEAJ
(communication visuelle). Il poursuit en pa\-rall\`ele des activit\'es
dans le domaine de l'art num\'erique (avec le collectif bru\-xellois
LAB[au]) ou du sound design (identit\'e sonore de La Une{}-RTBF). Il
est aussi un agent mandat\'e par le Syndicat des Robots pour le
repr\'esenter aupr\`es des institutions humaines.}

\SubSubTitle{Peter Westenberg}

\SmallUrl{http://www.videomagazijn.org/}

\English{Peter Westenberg is an artist and film and video maker, and member of
Constant. His projects evolve from an interest in social cartography,
urban anomalies and the relationships between locative identity and
cultural geography. His work was shown at exhibitions and festivals
such as Portobello Film festival London, Argosfestival Brussel, Impakt
Utrecht, International Film Festival Rotterdam and Videoex Z\"urich.}

\Nederlands{Peter Westenberg is beeldend kunstenaar en film{}- en videomaker, en lid
van Constant. Zijn projecten ontstaan uit een interesse voor sociale
cartografie, urbane anomalie\"en, meervoudige identiteiten en fricties
tussen wens en werkelijkheid. Zijn werk werd vertoond op
tentoonstellingen en festivals waaronder Portobello Filmfestival
Londen, Impakt Utrecht, Argos Brussel, International Film Festival
Rotterdam en Videoex Z\"urich.}

\Francais{Peter Westenberg est artiste visuel et r\'ealisateur de vid\'eos et de
courts{}-m\'etrages, et membre de Constant. Il s'int\'eresse \`a la
cartographie sociale, aux anomalies urbaines, aux identit\'es
multiples, aux tensions entre le r\^eve et la r\'ealit\'e. Son travail
a \'et\'e montr\'e lors d'expositions et de festivals tels que le
Portobello Flmfestival de Londres, Impakt Utrecht, le Festival Argos
Brussel, le Festival International du film de Rotterdam et Videoex \`a
Z\"urich.}

\SubSubTitle{Brian Wyrick}

\SmallUrl{http://www.pseudoscope.com}

\English{Brian Wyrick is an artist, filmmaker and web developer working in Berlin
and Chicago. He is also co{}-founder of Group 312 Films, a
Chicago{}-based film group.}

\Nederlands{Brian Wyrick is een kunstenaar, filmmaker en webontwikkelaar werkzaam in
Berlijn en Chicago. Hij is medeoprichter van Group 312 Films, een
filmgroep met Chicago als uitvalsbasis.}

\Francais{Brian Wyrick est un artiste, cin\'easte et d\'eveloppeur web qui
travaille \`a Berlin et Chicago. Il est aussi le co{}-fondateur du
Group 312 Films, un collectif de cin\'eastes bas\'e \`a Chicago.}


\SubSubTitle{Simon Yuill}

\SmallUrl{http://www.spring-alpha.org/}

\English{Artist and programmer based in Glasgow, Scotland. He is a developer in
the spring\_alpha and Social Versioning System (SVS) projects. He has
helped to set up and run a number of hacklabs and free media labs in
Scotland including the Chateau Institute of Technology (ChIT) and
Electron Club, as well as the Glasgow branch of OpenLab. He has written
on aspects of Free Software and cultural praxis, and has contributed to
publications such as {\em Software Studies} (MIT Press, 2008),
{\em the FLOSS Manuals} and {\em Digital Artists Handbook
project} (GOTO10 and Folly).}

\Nederlands{Kunstenaar en programmeur uit Glasgow, Schotland. Hij werkt aan
projecten als the spring\_alpha en Social Versioning System (SVS). Hij
was medeoprichter en medewerker van een hele reeks hacklabs en free
media labs in Schotland, waaronder de Chateau Institute of Technology
(ChIT), de Electron Club, en de Glasgowse afdeling van OpenLab. Hij
schreef over aspecten van Vrije Software en culturele praxis, en droeg
bij aan publicaties als {\em Software Studies} (MIT Press, 2008),
{\em the FLOSS Manuals} en {\em Digital Artists Handbook project}
(GOTO10 en Folly).
}
\Francais{Artiste/programmeur bas\'e \`a Glasgow en Ecosse. Son travail aborde la
programmation, \`a la fois sur ses m\'ecanismes formels de bas niveau
et sur ses relations de haut niveau aux structures sociales et aux
syst\`emes. Ces probl\`emes sont explor\'es \`a travers des projets de
logiciels, des \'ecrits, des discussions, et des workshops intera\-ctifs.
Les projets en cours comprennent \quote{Spring Alpha}, b\^atissant
l'investigation sociale au travers d'un design de jeu vid\'eo, et \quote{Your
Machines}, une s\'erie d'\'ev\'enements et ateliers pratiques de
cr\'eation bas\'es sur les technologies des logiciels libres et open
source et leurs cons\'equences sociopolitiques. Il est le membre
fondateur de ChIT, le Ch\^ateau Institute of Technology, un laboratoire
sur les m\'edias, tenu par des artistes dans le c{\oe}ur de Glasgow.}
}
