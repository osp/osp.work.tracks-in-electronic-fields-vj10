\PlaceImage{plant.jpg}{Sadie Plant reports at V/J10}

\AuthorStyle{Sadie Plant}

\licenseStyle{Creative Commons Attribution{}-NonCommercial{}-ShareAlike}

\Remark{Interwoven with her own thoughts and experiences, Sadie Plant gave
a situated report on the Mutual Motions track, and responded to the issues
discussed during the week{}-end.\par}

\Eng{\Title{A Situated Report}

I have to begin with many thanks to Femke and Laurence, because it
really has been a great pleasure for me to have been here this weekend.
It's nearly five years since I came to an event like this, believe it
or not, and I really cannot say enough how much I have enjoyed it, and
how stimulating I have found it. So yes, a big thank you to both for
getting me here. And as you say, it's ten years since I wrote
{\em Zeros + Ones}, and you are marking ten years of this festival
too, so it's an interesting moment to think about a lot of the issues
that have come up over the weekend. This is a more or less spontaneous
report, very much an \quote{open performance}, to use Simon Yuill's words,
and not to be taken as any kind of definitive account of what has
happened this weekend. But still I hope it can bring a few of the many
and varied strands of this event together, not to form a true
conclusion, but perhaps to provide some kind of {\em digestif} after
a wonderful meal. 

I thought I should begin as Femke very wisely began, with the theme of
cooking. Femke gave us a recipe at the beginning of the weekend, really
a kind of recipe for the whole event, with cooking as an example of the
fact that there are many models, many activities, many things that we
do in our everyday lives, which might inform and expand our ideas about
technology and how we work with them. So, I too will begin with this
idea of cooking, which is as Femke said a very magical, transformative
experience. Femke's clip from the Cath\'erine Deneuve film was a really
lovely instance of the kind of deep elemental, magical chemistry which
goes on in cooking. It is this that makes it such an instructive and
interesting candidate, for a model to illuminate the work of
programming, which itself obviously has this same kind of potential to
bring something into effect in a very direct and immediate sense. And
cooking is also the work behind the scene, the often forgotten work,
again a little bit like programming, that results in something which
{--} again like a lot of technology {--} can operate on many different
scales. Cooking is in one sense the most basic kind of activity, a
simple matter of survival, but it can also work on a gourmet level too,
where it becomes the most refined {--} and well paid {--} kind of work.
It can be the most detailed, fiddly, sort of decorative work; it can be
the most backbreaking, heavy industrial work {--} bread making for
example as well. So it really covers the whole panoply of these
extremes. 

If we think about a recipe, and ask ourselves about the machine that the
recipe requires, it's obviously running on an incredibly complex
assemblage: you have the kitchen, you have all the ingredients, you
have machines for cooling things, machines for heating things, you have
the person doing the cooking, the tools in question. We really are
talking here about a complex process, and not just an end result. The
process is also, again, a very \quote{open} activity. Simon Yuill defined an
`open performance' as a partial composition completed in the
performance. 

Cooking is always about experimentation and the kitchen really is a kind
of lab. The instructions may be exact, the conditions may be more or
less precise but the results are never the same twice. There are just
too many variables, too many contingencies involved. Of course like any
experimental work, it can go completely wrong, it often does go wrong:
sometimes it really is all about process, and not about eating at all!
But as Simon again said today, quoting Sun Ra: there are no real
mistakes, there are no truly wrong things. This was certainly the case
with the fantastic cooking process that we had throughout the whole day
yesterday, which ended with us eating these fantastic mussels, which I
am sure elpueblodechina thought in fact were not as they should have
been. But only she knew what she was aiming at: for the people who ate
them they were delicious, their flavour enhanced by the whole
experience of their production. elpueblodechina's meal made us ask:
what does it mean for something to go wrong? She was using a cooking
technique which has come out of generations and generations of errors,
mistakes, probings, fallings backs, not just simply a continuous kind
of story of progress, success, and forward movement. So the mistakes
are clearly always a very big part of how things work in life, in any
context in life, but especially of course in the context of programming
and working with software and working with technologies, which we often
still tend to assume are incredibly reliable, logical systems, but in
fact are full of glitches and errors. As thinkers and activists
resistant to and critical of mainstream methods and cultures, this is
something that we need to keep encouraging. 

I have for a long time been interested in textiles, and I can't resist
mentioning the fact that the word \quote{recipe} was the old word for
knitting patterns: people didn't talk about knitting patterns, but
\quote{recipes} for knitting. This brings us to another interesting junction
with another set of very basic, repetitive kinds of domestic and often
overlooked activities, which are nevertheless absolutely basic to human
existence. Just as we all eat food, so we all wear clothes. As with
cooking, the production of textiles again has this same kind of sense
of being very basic to our survival, very elemental in that sense, but
it can also function at a high level of detailed, refined activity as
well. With a piece of knitting it is difficult to see the ways in which
a single thread becomes looped into a continuous textile. But if you
look at a woven pattern, the program that has led to the pattern is
right there in front of you, as you see the textile itself. This makes
weaving a very nice, basic and early example of how this kind of
immediacy can be brought into operation. What you look at in a piece of
woven cloth is not just a representation of something that can happen
somewhere else, but the actual instructions for producing and
reproducing that piece of woven cloth as well. So that's the kind of
deep {\em intuitive} connection that it has with computer
programming, as well as the more linear historical connections of which
I have often spoken. 

There are some other nice connections between textiles, cooking and
programming as well. Several times yesterday there was a lot of talk
about both experts and amateurs, and developers and users. These are
divisions which constantly, and often perhaps with good reason,
reassert themselves, and often carry gendered connotations too. In the
realm of cooking, you have the chef on the one hand, who is often male
and enjoys the high status of the inventive, creative expert, and the
cook on the other, who is more likely to be female and works under
quite a different rubric. In reality, it might be said that the
distinction is far from precise: the very practise of using computers,
of cooking, of knitting, is almost inevitably one of constantly
contributing to their development, because they are all relatively open
systems and they all evolve through people's constant, repetitive use
of them. So it is ultimately very difficult to distinguish between the
user and the developer, or the expert and the amateur. The experiment,
the research, the development is always happening in the kitchen, in
the bedroom, on the bus, using your mobile or using your computer.
Fernand Braudel speaks about this kind of \quote{micro{}-histories}, this
sense of repetitive activity, which is done in many trades and many
lines, and that really is the deep unconscious history of human
activity. And arguably that's where the most interesting developments
happen, albeit in a very unsung, unseen, often almost hidden way. It is
this kind of deep collectivity, this profound sense of
micro{}-collaboration, which has often been tapped into this weekend. 

Still, of course, the social and conceptual divisions persist, and
still, just as we have our celebrity chefs, so we have our celebrity
programmers and dominant corporate software developers. And just as we
have our forgotten and overlooked cooks, so we have people who are
dismissed, or even dismiss themselves, as \quote{just computer users}. The
technological realities are such that people are often forced into this
role, with programmes that really are so fixed and closed that almost
nothing remains for the user to contribute. The structural and social
divisions remain, and are reproduced on gendered lines as well. 

In the 1940s, computer programming was considered to be extremely
menial, and not at all a glamorous or powerful activity. Then of
course, the business of dealing with the software was strictly women's
work, and it was with the hardware of the system that the most powerful
activity lay. That was where the real solid development was done, and
that was where the men were working, with what were then the real nuts
and bolts of the machines. Now of course, it has all turned around. It
is women who are building the chips and putting the hardware {--} such
as it is these days {--} together, while the male expertise has shifted
to the writing of software. In only half a century, the evolution of
the technology has shifted the whole notion of where the power lies. No
doubt {--} and not least through weekends like this {--} the story will
keep moving on. 

But as the world of computing does move more and more into software and
leave the hardware behind, it is accompanied by the perceived danger
that the technology and, by extension, the cultures around it, tend to
become more and more disembodied and intangible. This has long been
seen as a danger because it tends to reinforce what have historically,
in the Western world at least, been some of the more oppressive
tendencies to affect women and all the other bodies that haven't quite
fitted the philosophical ideal. Both the Platonic and Christian
traditions have tended to dismissing or repress the body, and with it
all the kind of messy, gritty, tangible stuff of culture, as transient,
difficult, and flawed. And what has been elevated is of course the much
more formal, idealist, disembodied kind of activities and processes.
This is a site of continual struggle, and I guess part of the purpose
of a weekend like this is to keep working away, re{}-injecting some
sense of materiality, of physicality, of the body, of geography, into
what are always in danger of becoming much more formal and disembodied
worlds. What Femke and Laurence have striven to remind us this weekend
is that however elevated and removed our work appears to be from the
matter of bodies and physical techniques, we remain bodies, complex
material processes, working in a complex material work. 

Once again, there still tends to be something of a gendered divide. The
dance workshop organised this morning by Alice Chauchat and
Fr\'ed\'eric Gies was an inspiring but also difficult experience for
many of us, unused as we are to using our bodies in such literally
physical and public ways. It was not until we came out of the workshop
into a space which was suddenly mixed in terms of gender, that I
realised that the participants in the workshop had been almost
exclusively female. It was only the women who had gone to this kind of
more physical, embodied, and indeed personally challenging part of the
weekend. But we all need to continually re{}-engage with this sense of
the body, all this messiness and grittiness, which it is in many vested
interests to constantly cleanse from the world. We have to make
ourselves deal with all the embarrassment, the awkwardness, and the
problematic side of this more tangible and physical world. For that
reason it has been fantastic that we have had such strong input from
people involved in dance and physical movement, people working with
bodies and the real sense of space. Sabine Prokhoris and Simon Hecquet
made us think about what it means to transcribe the movements of the
body; S\'everine Dusollier and Val\'erie Laure Benabou got us to
question the legal status of such movements too. And what we have
gained from all of this is this sense that we are {\em all always}
working with our bodies, we are always using our bodies, with more or
less awareness and talent, of course, whether we are dancing or baking
or knitting or slumped over our keyboards. In some ways we shouldn't
even need to say it, but the fact that we do need to remind ourselves
of our embodiment shows just how easy it is for us to forget our
physicality. This morning's dance workshop really showed some of the
virtues of being able to turn off one's self{}-consciousness, to
dismiss the constantly controlling part of one's self and to function
on a different, slightly more automatic level. Or perhaps one might say
just to prioritise a level of bodily activity, of bodily awareness, of
a sense of spatiality that is so easy to forget in our very cerebral
society. 

What Fr\'ed\'eric and Alice showed us was not simply about using the
body, but rather how to overcome the old dualism of thinking of the
body as a kind of servant of the mind. Perhaps this is how we should
think about our relationships to our technologies as well, not just to
see them as our servants, and ourselves as the authors or subjects of
the activity, but rather to perceive the interactivity, the sense of an
{\em interplay}, not between two dualistic things, the body and the
mind, or the agent and the tool, the producer and the user, but to try
and see much more of a continuum of different levels and different
kinds and different speeds of material activity, some very big and
clunky, others at extremely complex micro{}-levels. During the dance
workshop, Fr\'ed\'eric talked about all the synaptic connections that
are happening as one moves one's body, in order to instil in us this
awareness of ourselves as physical, material, thinking machines,
assemblages of many different kinds of activity. And again, I think
this idea of bringing together dance, food, software, and brainpower,
to see ourselves operating at all these different levels, has been
extremely rewarding. 

Femke asked a question of Sabine and Simon yesterday, which perhaps
never quite got answered, but expressed something about how as people
living in this especially wireless world, we are now carrying more and
more technical devices, just as I am now holding this microphone, and
how these additional machines might be changing our awarenesses of
ourselves. Again it came up this morning in the workshop when we were
asked to imagine that we might have different parts of our bodies,
another head, or our feet may have mirrors in them, or in one brilliant
example that we might have magnets, so that we were forced to have
parts of our bodies drawn together in unlikely combinations, just to
imagine a different kind of sense of self that you get from that
experience, or a different way of moving through space. But in many
ways, because of our technologies now, we don't need to imagine such
shifts: we are most of us now carrying some kind of telecommunicating
device, for example, and while we are not physically attached to our
machines {--} not yet anyway {--}, we are at least emotionally attached
to them. Often they are very much with us and part of us: the mobile
phone in your pocket is to hand, it is almost a part of us. And I too
am very interested in how that has changed not only our more
intellectual conceptions of ourselves, but also our physical selves.
The fact that I am holding this thing [the microphone] obviously does
change my body, its capacities, and its awareness of itself. We are all
aware of this to some extent: everyone knows that if you put on very
formal clothes, for example, you behave in different ways, your body
and your whole experience of its movement and spatiality changes.
Living in a very conservative part of Pakistan a few years ago, where I
had to really be completely covered up and just show my eyes, gave me
an acute sense of this kind of change: I had to sit, stand, walk and
turn to look at things in an entirely new set of ways. In a less
dramatic but equally affective way, wirelessness obviously introduces a
new sense of our bodies, of what we can do with our bodies, of what we
carry with us on our bodies, and consequently of who we are and how we
interact with our environment. And in this sense wirelessness has also
brought the body back into play, rescuing us from what only ten years
ago seemed to be the very real dangers of a more formal and disembodied
sense of a virtual world, which was then imagined as some kind of
\quote{other place}, a notion of cyberspace, up there somehow, in an almost
heavenly conception. Wirelessness has made it possible for computer
devices to operate in an actual, geographical environment: they can now
come with us. We can almost start to talk more realistically about a
much more interesting notion of the cyborg, rather than some big clunky
thing trailing wires. It really {\em can} start to function as a
more interesting idea, and I am very interested in the political and
philosophical implications of this development as well, and in that it
does reintroduce the body to as I say what was in danger of becoming a
very kind of abstract and formal kind of cyberspace. It brings us back
into touch with ourselves and our geographies. 

The interaction between actual space and virtual space, has been another
theme of this weekend; this ability to translate, to move between
different kinds of spaces, to move from the analogue to the digital, to
negotiate the interface between bodies and machines. Yesterday we heard
from Adrian Mackenzie about digital signal processing, the possibility
of moving between that real sort of analogue world of human experience
and the coding necessary to computing. Sabine and Simon talked about
the possibilities of translating movement into dance, and this also has
come up several times today, and also with Simon's work in relation to
music and notation. Simon and Sabine made the point that with the
transcription and reading of a dance, one is offered {--} rather as
with a recipe {--} the same ingredients, the same list of instructions,
but once again as with cooking, you will never get the same dance, or
you will never get the same food as a consequence. They were interested
in the idea of notation, not to preserve or to conserve, but rather to
be able to send food or dance off into the future, to make it possible
in the future. And Simon referred to these fantastic diagrams from The
Scratch Orchestra, as an entirely different way of conceiving and
perceiving music, not as a score, a notation in this prescriptive,
conserving sense of the word, but as the opportunity to take something
forward into the future. And to do so not by writing down the sounds,
or trying to capture the sounds, but rather as a way of describing the
actions necessary to produce those sounds, is almost to conceive the
production of music as a kind of dance, and again to emphasise its
embodiment and physicality.

This sense of performance brings into play the idea of \quote{play} itself,
whether \quote{playing} a musical instrument, \quote{playing} a musical score, or \quote{playing} the body in an effort to dance. I think in some dance
traditions one speaks about \quote{playing the body}; in Tai Chi it is
certainly said that one plays the body, as though it was an instrument.
And when I think about what I have been doing for the last five years,
it's involved having children, it's involved learning languages, it's
involved doing lots of cooking, and lots of playing, funny enough. And
what has been lovely for me about this weekend is that all of these
things have been discussed, but they haven't been just discussed, they
have actually been {\em done} as well. So we have not only thought
about cooking, but cooking has happened, not only with the mussels, but
also with the fantastic food that has been provided all weekend. We
haven't just thought about dancing, but dancing has actually been done.
We haven't just thought about translating, but with great thanks to the
translators {--} who I think have often had a very difficult job {--}
translating has also happened as well. And in all of these cases we
have seen what might so easily have been a simply theoretical
discussion, has itself been translated into real bodily activity: they
have all been, literally, brought into play. And this term {\quote{play},}
which spans a kind of mathematical play of numbers, in relation to
software and programming, and also the world of music and dance, has
enormous potential for us all: Simon talked about \quote{playing free} as an
alternative term to \quote{improvisation}, and this notion of \quote{playing free}
might well prove very useful in relation to all these questions of
making music, using the body, and even playing the system in terms of
subverting or hacking into the mainstream cultural and technical
programs with which we presented.

This weekend was inspired by several desires and impulses to which I
feel very sympathetic, and which remain very urgent in all our debates
about technology. As we have seen, one of the most important of those
desires is to reinsert the body into what is always in danger of
becoming a disembodied realm of computing and technology. And to
reinsert that body not as a kind of Chaplinesque cog in the wheel that
we saw when In\`es Rabad\'an introduced {\em Modern Times} last
night, but as something more problematic, something more complex and
more interesting. And also not to do so nostalgically, with some idea
of some kind of lost natural activity that we need to regain, or to
reassert, or to reintroduce. There is no true body, there is no natural
body, that we can recapture from some mythical past and bring back into
play. At the same time we need to find a way of moving forward, and
inserting our senses of bodies and physicality into the future, to
insist that there is something lively and responsive and messy and
awkward always at work in what could have the tendency otherwise to be
a world of closed systems and dead loops. 

One of the ways of doing this is to constantly problematise both
individualised conceptions of the body and orthodox notions of
communities and groups. Michael Terry's presentation about ingimp,
developed in order to imagine the community of people who are using his
image manipulation software, raised some very problematic issues about
the notion of community, which were also brought up again by Simon
today, with this ideas about collaboration and collectivity, and what
exactly it means to come together and try to escape an individualised
notion of one's own work. Femke's point to Michael exemplified the ways
in which the notion of community has some real dangers: Michael or his
team had done the representations of the community themselves {--} so
if people told them they were graphic artists, they had found their own
kind of symbols for what a graphic artist would look like {--}, and
when Femke suggested that people {--} especially if they were graphic
artists {--} might be capable of producing their own representations
and giving their own way of imagining themselves, Michael's response
was to the effect that people might then come up with what he and his
team would consider to be \quote{undesirable images} of themselves. And this
of course is the age old problem with the idea of a community: an open,
democratic grouping is great when you're {\em in} it and you all
agree what's desirable, but what happens to all the people that don't
quite fit the picture? How open can one afford to be? We need some
broader, different senses of how to come together which, as Alice and
Fr\'ed\'eric were discussed, are ways of collaborating without becoming
a new fixed totality. If we go back to the practices of cooking,
weaving, knitting, and dancing, these long histories of very everyday
activities that people have performed for generation after generation,
in every culture in the world {--} it is at this level that we can see
a kind of collective activity, which is way beyond anything one might
call a \quote{community} in the self{}-conscious sense of the term. And it's
also way beyond any simple notion of a distributed collection of
individuals: it is perhaps somewhere at the junction of these modes, an
in{}-between way of working which has come together in its own
unconscious ways over long periods of time. 

This weekend has provided a rich menu of questions and themes to feed in
and out of the writing and use of software, as well as all our other
ways of dealing with our machines, ourselves, and each other. To keep
the body and all its flows and complexities in play, in a lively and
productive sense; to keep all the interruptive possibilities alive; to
stop things closing down; to keep or to foster the sense of
collectivity in a highly individualised and totalising world; to find
new ways {--} {\em constantly} find new ways {--} of collaborating
and distributing information: these are all crucial and ongoing
struggles in which we must all remain continually engaged. And I notice
even now that I used this term \quote{to keep}, as though there was something
to conserve and preserve, as though the point of making the recipes and
writing the programs is to preserve something. But the \quote{keeping} in
question here is much more a matter of \quote{keeping on}, of constantly
inventing and producing without, as Simon said earlier, leaving
ourselves too vulnerable to all the new kinds of exploitation, the new
kinds of territorialisation, which are always waiting around the corner
to capture even the most fluid and radical moves we make. This whole
weekend has been an energising reminder, a stimulating and inspiriting
call to keep problematising things, to keep inventing and to keep
reinventing, to keep on keeping on. And I thank you very much for
giving me the chance to be here and share it all. Thank you. 

A quick postscript. After this \quote{spontaneous report} was made, the
audience moved upstairs to watch a performance by the dancer
Fr\'ed\'eric Gies, who had co{}-hosted the morning's workshop. I found
the energy, the vulnerability, and the emotion with which he danced
quite overwhelming. The Madonna track {}- {\em Hung Up (Time Goes by
so Slowly)} {--} to which he danced ran through my head for the whole
train journey back to Birmingham, and when I got home and checked out
the Madonna video on YouTube I was even more moved to see what a
beautiful commentary and continuation of her choreography Fr\'ed\'eric
had achieved. This really was an example not only of playing the body,
the music, and the culture, but also of effecting the kind of \quote{free
play} and \quote{open performance}, which had resonated through the whole
weekend and inspired us all to keep our work and ourselves in motion.
So here's an extra thank you to Fr\'ed\'eric Gies. Madonna will never
sound the same to me.}
