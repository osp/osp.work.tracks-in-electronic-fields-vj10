\setupcaptions[width=.9\textwidth,number=yes,style=\ttxx,numberconversion=Romannumerals,location=top]
\setupindenting[no]

%SETTING UP WHITESPACE CREATES A PROBLEM
%\setupwhitespace[medium]

\AuthorStyle{Femke Snelting}

\licenseStyle{Free Art License}

\Ned{\Title{Encoderen en decoderen}

\blank

\startcolumns[n=2,balance=no]

\DecodeFigure{000-snelting}

\Legend{Met een toverspreuk {--} zo begint dit beeldverhaal over het omzetten
van handelingen in code en weer terug.
\blank
Catherine zingt een magische formule, een recept voor een
liefdescake die haar prins moet betoveren. Ze is Ezelsvel, de
keukenprinses uit de klassieke vertelling van Charles Perrault.

\column

\midaligned{{\em Steek uw oven aan.}}
\midaligned{{\em Neem wat bloem,}}
\midaligned{{\em giet het in de kom}}
\midaligned{{\em vier handen vol}}
\midaligned{{\em en maak er een kuiltje in...}}
\blank
Vanuit haar sprookjeskeuken, waar conversie de dagelijkse routine bepaalt, vertelt ze over de feedbackloop tussen programma en bewegingen. \footnote{{\em Peau d'Ane}. Jacques Demy, 1970. Musical met Catherine Deneuve in de hoofdrol; in deze sc\`ene zingt Deneuve haar (succesvolle!) recept voor \quote{Cake d'Amour}.}\footnote{\quotation{Als ik weet hoe ik een taart maak, weet ik ook hoe een bom te maken}. Affiche, Bang Bang, 2005}}
\DecodeFigure{001-snelting}

\page

\Legend{Lang geleden vroeg de professor welk object haar interesse in programmeren had gewekt. Catherine hoefde niet lang na te denken: De {\em \quote{Easy{}-Bake Oven}}!
\blank
\quotation{Als kind leek koken magisch, een mysterieus ritueel, vol exotische ingredi\"enten, met z'n eigen boeken vol expliciete en gecompliceerde procedures, geschreven in een vreemde taal. Koken bracht mijn familie samen. Door mezelf te leren koken, tartte ik die band, en tegelijkertijd demythologiseerde ik de magie. Zo kwam de toverkracht binnen mijn bereik.} \footnote{Jeugdfoto van Dana McCauley (recepten{}-schrijfster). \quotation{My generation is divided into two groups: women who had Easy{}-Bake Ovens as children and those who didn't.} \Url{http://www.homemakers.com/Food{\char38}Nutrition/cookscorner/november-2007-n240005p4.html} 2008.} \footnote{Tekst gebaseerd op: Michael Murtaugh, \quote{The Easy{}-Bake Oven}, in: Sherry Turkle, {\em Objects in Mind: Falling for Science, Technology and Design}, MIT Press, 2008.}}

\DecodeFigure{003-snelting}
\Legend{Het had haar voor altijd nieuwsgierig gemaakt naar verhoudingen en ingredi\"enten. \footnote{Sinds 2003 werken Kayle Brandon en Kate Rich aan
de receptuur voor Cube{}-Cola, gebaseerd op een opensourcerecept. Cube{}-Cola is verkrijgbaar aan de bar van Cube Microplex in Bristol,
of in geconcentreerde vorm te bestellen bij: \Url{http://sparror.cubecinema.com/cube/cola/kit/}}}

\DecodeFigure{004-snelting}
\Legend{Nu probeerde ze het recept van coca cola te achterhalen.}

\DecodeFigure{005-snelting}
\Legend{De samenstelling was gemakkelijk gevonden: karamel, cafe\"ine, suiker, spuitwater, citroenzuur en acht verschillende essenti\"ele oli\"en.}

\DecodeFigure{006-snelting}
\Legend{Maar om daarvan een smakelijke drank te brouwen, was toch gecompliceerder dan ze in eerste instantie had gedacht.}

\DecodeFigure{007-snelting}
\Legend{Terwijl ze doe{}-het{}-zelf{}-technologie inventief combineerde met wetenschappelijke methodes, vergat ze rekening te
houden met de specifieke omstandigheden in haar thuislab. \footnote{\quotation{In het methodologisch onderzoek naar de panificatie van Frans brood, zijn de karakteristieken voor textuur gelieerd aan de aanraking: soepelheid, elasticiteit en plakkerigheid.}, {\em Les Pains Fran\c{c}ais, Evolution, qualit\'e, production}, MAE{}-ERTI\'editeurs, 2002}}

\DecodeFigure{008-snelting}
\Legend{Het recept impliceerde een uiterste abstractie waarbij elke onvoorziene gebeurtenis, context, plaats, vuil, ambigu\"iteit, wederzijdse afhankelijkheid, plaats en tijd, was weggedacht. \footnote{\quotation{Classic puff pastry begins with a basic dough called a d\'etrempe (pronounced day{}-trahmp) that is rolled out and wrapped around a slab of butter. The dough is then repeatedly rolled, folded, and turned.}, Molly Stevens, {\em A Shortcut to Flaky Puff Pastry}. \Url{http://www.taunton.com/finecooking/articles/how-to/rough-puff-pastry.aspx} 2008}}

\DecodeFigure{009-snelting}
\Legend{Het veronderstelde herhaalbaarheid en reproduceerbaarheid, maar vroeg in feite een constante encodering en decodering. }

\DecodeFigure{010-snelting}
\Legend{Een voorspelbare grammatica was het resultaat: ononderbroken handelingen moesten worden opgedeeld in afzonderlijke
stappen.}

\DecodeFigure{011-snelting}
\Legend{Een titel, een lijst ingredi\"enten, benodigd gereedschap, bereidingstijd, bereidingswijze en tot besluit nog een tip voor een
variatie...}

\DecodeFigure{012-snelting}
\Legend{Mix, zeef, roer, verwarm de oven voor, bak totdat de randen goudbruin kleuren, beboter de vorm, vier volle handen bloem alstublieft!}

\DecodeFigure{013-snelting}\column\column
\Legend{Catherine zocht tussen de vele kookblogs en receptencollecties online naar precieze instructies voor het maken van
bladerdeeg.}

\DecodeFigure{014-snelting}
\Legend{Dit kwetsbare en complexe gebak vormde de {\em 'rite de passage'} van liefhebber naar expert.}

\DecodeFigure{015-snelting}
\Legend{Het deeg vergde concentratie, handigheid en veel geduld, maar het resultaat ...}

\DecodeFigure{016-snelting}
\Legend{... duizend luchtige, flinterdunne en boterachtige laagjes, was de heilige graal voor een amateurpatissier zoals zij.}

\DecodeFigure{017-snelting}
\Legend{Telde de temperatuur in haar keuken, het merk van haar oven, de vochtigheid van de boter? De textuur van haar werkblad? De
warmte van haar hand? 
\blank
Terwijl ze het deeg voorzichtig in plastic folie rolde, klaar voor een laatste 'tour', dacht ze terug aan de Quatre Quarts \footnote{Quatre Quarts en Poundcake zijn twee verwante recepten die teruggaan naar het begin van de 18e eeuw. Terwijl de Bretonse Quatre Quarts is gebaseerd op vier gelijke delen, afgemeten aan de hand van het gewicht van de eieren, bevat de traditionele Engelse versie 5 ingredi\"enten: naast een pond boter, een pond meel, een pond suiker, 8 eieren (ongeveer een pond), wordt ook nog een pond gedroogde vruchten toegevoegd.} die haar vader voor haar bakte.}

\DecodeFramed{018-snelting}
\Legend{Zijn robuuste cake was lichtjaren verwijderd van Catherines eigen elegante Mille Feuille. \footnote{Screenshots van Gourmet Recipe Manager, Open Source software waarmee je receptencollecties kunt aanleggen, automatisch boodschappenlijsten kunt genereren en voedingsinformatie per recept of maaltijd kunt aflezen op basis van de USDA food database.}}

\DecodeFramed{019-snelting}
\Legend{De doeltreffende vierkwartsmaat van de Quatre Quarts (gelijke delen boter, eieren, bloem en suiker) had zich als een virus
verspreid, van generatie tot generatie, zelfs van land tot land.}

\DecodeFigure{020-snelting}
\Legend{\footnote{Recept voor Pound Cake. Uit: John Murray, {\em A Lady}. {\em A New System of Domestic Cookery; Formed Upon Principles of Economy; Adapted to the Use of Private Families}, London, 1824. De tekst is een parafrase op Luce Giards beschrijving van de manier waarop ze leerde koken. Hoewel ze altijd geweigerd had in de voetstappen van haar moeder te treden, merkte ze tot haar eigen verbazing toen ze eindelijk een eigen appartement had, dat ze ongemerkt geleerd had over het geluid van deeg, de geur van brandende boter, het sissen van vlees in de pan. In: Luce Giard, Pierre Mayol, Michel De Certeau, {\em The Practice of Everyday Life: Vol. 2}, University of Minnesota Press, 1998.}}

\DecodeFigure{021-snelting}
\Legend{Overal waar Catherine kwam, ontdekte ze locale varianten: Poundcakes en Pondskoeken, Panqu\'e of Ponque. \footnote{{\em Mrs Beeton's Cookery For All}, Pan, 1984.} \footnote{{\em Isabelle Beeton's Book of Household Management} verscheen voor het eerst in 1836. Vijftig jaar na de dood van de auteur is de tekst vrij van auteursrecht. Sindsdien zijn er oneindig veel verschillende versies in omloop; ook werd de integrale tekst in 2003 toegevoegd aan het Gutenbergproject \Url{http://www.gutenberg.org/etext/10136}}}

\DecodeFigure{022-snelting}
\Legend{Ze had zich pas gerealiseerd hoeveel ze eigenlijk van hem had geleerd, toen ze zelf was gaan koken. De herinneringen hadden zich
ondertussen vermengd met die van anderen, maar het geluid van deeg op het aanrecht, de scherpe geur van brandende boter, het sissen van vlees in de pan, waren haar altijd bijgebleven. \footnote{Mrs Isabella Beeton, {\em Mrs Beeton's Cookery Book {--} Diamond Jubilee Edition}, Impala, 2006.}}

\DecodeFigure{023-snelting}
\Legend{{\em Mrs. Beeton's Cookery For All} was het eerste kookboek dat ze kocht, en af en toe bladerde ze er nog in
terug. \footnote{Mrs Beeton, {\em Mrs Beeton's Household Management}, Wordsworth Editions Ltd, 2006.}}

\DecodeFigure{024-snelting}
\Legend{De Victoriaanse bestseller van Mrs. Isabelle Beeton had iets voor iedereen {--} recepten, medische geschiedenis, juridische
adviezen en (natuurlijk) instructies voor hoe de bedienden op te leiden. \footnote{Isabelle Beeton, {\em The Best of Mrs Beeton's Easy Everyday Cooking}, Cassell, 2006.}}

\DecodeFramed{025-snelting}
\Legend{De tekst van Isabelle Beeton deelt genereus kennis, in de taal van een commando:
\blank
Doe eerst dit en dan dat en vergeet vooral niet zus en
later zo toe te voegen! \footnote{\quotation{The Linux Cookbook's tested techniques distill years of hard{}-won experience into practical cut{}-and{}-paste solutions to everyday Linux dilemmas.}, O'Reilly, 2004.}}

\DecodeFigure{026-snelting}
\Legend{Haar belevenissen als huishoudster in dienst van een rijke familie transformeerde Isabelle tot een handleiding die nog steeds door iedereen kan worden toegepast. \footnote{Peter Lavin, {\em Object{}-Oriented PHP: Concepts, Techniques, and Code}, No Starch Press, 2006.}}

\DecodeFigure{027-snelting}
\Legend{Isabelle was pas 21 toen ze aan het boek begon, en kon niet erg goed koken. De meeste recepten in haar collectie
nam ze over uit de kookboeken van collega{}-huisvrouwen.
\blank
Het was Catherine opgevallen dat de beste kookboekauteurs vaak ver weg van huis hun werk deden. Waren hun recepten beter omdat ze over hun
grenzen schreven? \footnote{Beeld: Uitgeefster Judith Jones met de kookboeken die ze in haar leven publiceerde. Tekst gebaseerd op: \quote{Judith Jones: A Life in Food} (Radio{}-interview met Judith Jones). \Url{http://www.onpointradio.org/shows/2007/11/20071119_b_main.asp} 2008.}}
\blank
\Legend{Ze las:
\blank
\quotation{Nullen en enen zijn niet kieskeurig: ze erkennen geen van de oude grenzen tussen de diverse communicatiekanalen en gaan op in de
verrijzende, volkomen nieuwe, zintuiglijke ambiance waarin geleidelijk aan duidelijk wordt dat \quote{voeling} niets met huid uitstaande heeft en
\quote{voeling houden} of \quote{voeling zoeken} kwesties zijn van een vruchtbaar samenkomen van de zintuigen, van gezicht vertaald in geluid en geluid in bewegingen en smaak en geur.} \footnote{Tekst gebaseerd op: Sadie Plant, {\em Nullen en Enen/De ondergang van het patriarchaat}, Contact Amsterdam, 1998.}}
\blank
\Legend{Ze dacht:
\blank
\quotation{Zou het kunnen dat technologie zo dicht op onze huid is komen zitten,
dat we er geen afstand meer van kunnen nemen?}
\blank
\quotation{Hoe kunnen we andere vormen van begrijpen introduceren, niet gebaseerd
op kritische distantie, maar op bewegingen die de normen en waarden
ingeschreven in onze omgeving voelbaar maken?}
\blank
En neuriede zachtjes voor zich uit:
\blank
\midaligned{{\em Steek uw oven aan.}}
\midaligned{{\em Neem wat bloem,}}
\midaligned{{\em giet het in de kom}}
\midaligned{{\em vier handen vol}}
\midaligned{{\em en maak er een kuiltje in.}}}

\stopcolumns

}

%RESTORE DEFAULTS
\setupcaptions[style=\tfxx\setupinterlinespace,number=no,location=bottom,width=1.45cm,align=middle]
\setupindenting[yes, 1em, next]

