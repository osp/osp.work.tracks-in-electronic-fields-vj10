\PlaceImage{ghost1.jpg}{Drawings by Dominique Goblet, EVP sounds by Carl Michael von Hausswolff, images by Guy{}-Marc Hinant}
\PlaceImage{ghost0.jpg}{EVP could be the result of psychic echoes from the past, psychokinesis, or the thoughts of aliens or nature spirits.}

\AuthorStyle{CM von Hausswolff, Guy{}-Marc Hinant}

\licenseStyle{Creative Commons Attribution{}-NonCommercial{}-ShareAlike}

\Remark{For more information on EVP, see: \Url{http://en.wikipedia.org/wiki/Electronic_voice_phenomenon#_note-fontana1}

}

\Eng{\Title{Ghost Machinery}

During V/J10 we showed an audiovisual installation entitled {\em Ghost
Machinery}, with drawings by Dominique Goblet, EVP sounds by Carl
Michael von Hausswolff, and images by Guy{}-Marc Hinant, based on Dr. 
Stempnicks Electronic Voice Phenomena recordings.

EVP has been studied primarily by paranormal researchers since the
1950s, who have concluded that the most likely explanation for the
phenomena is that they are produced by the spirits of the deceased. In
1959, Attila Von Szalay first claimed to have recorded the \quote{voices of
the dead}, which led to the experiments of Friedrich J\"urgenson. The
1970s brought increased interest and research including the work of
Konstantine Raudive. In 1980, William O'Neill backed by industrialist
George Meek built a \quote{Spiricom} device, which was said to facilitate
very clear communication between this world and the spirit world. 

Investigation of EVP continues today through the work of many
experimenters, including Sarah Estep and Alexander McRae. In addition
to spirits, paranormal researchers have claimed that EVP could be due
to psychic echoes from the past, psychokinesis unconsciously produced
by living people, or the thoughts of aliens or nature spirits.
Paranormal investigators have used EVP in various ways, including as a
tool in an attempt to contact the souls of dead loved ones and in ghost
hunting. Organizations dedicated to EVP include the American
Association of Electronic Voice Phenomena, the International Ghost
Hunters Society, as well as the skeptical Rorschach Audio project.}

\page

\Ned{\Title{Ghost Machinery}

Tijdens V/J10 toonden we een audiovisuele installatie getiteld {\em Ghost
Machinery}, met tekeningen van Dominique Goblet, EVP{}-opnamen door Carl
Michael von Hausswolff, en beelden van Guy{}-Marc Hinant, gebaseerd op
Dr. Stempnicks EVP{}-opnamen.

EVP werd voornamelijk tijdens de jaren '50 bestudeerd door onderzoekers
van het paranormale, die tot de conclusie kwamen dat de meest
waarschijnlijke verklaring voor deze fenomenen is, dat ze voortgebracht
worden door de geesten van overledenen. Het was Attila Von Szalay die
in 1959 voor het eerst verklaarde \quote{stemmen van de doden} te hebben
opgenomen; een verklaring die onder andere de aanzet gaf tot verder
experiment door Friedrich J\"urgenson. In de jaren '70 geraakte men nog
meer in de ban van deze fenomenen en ontwikkelde Konstantin Raudive
zijn vermaard onderzoek. In 1980 fabriceerde William O'Neill,
geruggensteund door industrieel George Meek, een \quote{Spiricom}{}-toestel,
dat geacht werd de communicatie tussen deze en gene wereld, waar de
geesten huizen, te faciliteren. Vandaag de dag wordt het
EVP{}-onderzoek voortgezet doorheen het werk van tal van
experimenteerders, zoals Sarah Estep en Alexander McRae.

Onderzoekers van het paranormale menen dat EVP niet alleen afkomstig is
van geesten, maar ook psychische echo's uit het verleden, psychokinese
onbewust voortgebracht door levende mensen, of gedachten van
buitenaardse wezens of natuurgeesten, kunnen zijn. EVP werd door deze onderzoekers dan ook op verschillende manieren aangewend, onder andere als een manier om in contact te treden met de
zielen van geliefde overledenen of om spoken te verjagen. Organisaties
gewijd aan EVP zijn o.a. de American Association of Electronic Voice
Phenomena, de International Ghost Hunters Society, en het skeptische
Rorschach Audio Project.}
